%%%%%%%%%%%%%%%%%%%%%%%%%%%%%%%%%%%%%%%%%%%%%%%%%%%%%%%%%%%%%%%%%%%%%%%%
% Template for a Master's Thesis or Ph.D. dissertation
% at the University of Wisconsin-Milwaukee.
%
% Designed for LaTeX version 2e
%
% Updated by Adam J. Smith
% December, 2007
%
% This thesis template requires the file "UWMthesis.sty", available
% on the UWM's Atmospheric Science Club website, and possible in other
% locations.
%
% A LaTeX primer is not provided here.  For instructions on how to use commands for
% figures, table, equations, and bibliography citations, please see the documentation
% listed later in this document.
% 
% This template follows the "Fall 2007" version of the 
% University of Wisconsin-Milwaukee standards for the Master's thesis
% and Ph.D. dissertation.  Please feel free to update this template
% as needed to comply with these standards.
%
% For current thesis and dissertation formatting information, visit the UWM graduate school
% website at the following address:
% http://www.graduateschool.uwm.edu/students/current/thesis-and-dissertation-formatting/
%
% IMPORTANT: Be sure to meet with the appropriate Graduate School personnel to verify
% whether your final document meets the UWM standards.  If not, the document may not
% be accepted until any problems are corrected.
%%%%%%%%%%%%%%%%%%%%%%%%%%%%%%%%%%%%%%%%%%%%%%%%%%%%%%%%%%%%%%%%%%%%%%%%

% If you are writing a master's thesis, use the first option (master).
% If you are writing a phd dissertation, use the second option (phd).
%\documentclass[master]{UWMThesis}
\documentclass[phd]{UWMThesis}

\usepackage{fancyhdr}
\usepackage{xcolor}
\usepackage{amsmath}
\usepackage{amsthm}
\usepackage{bbm}
\usepackage{amssymb}
\usepackage{enumerate}
\usepackage{kantlipsum}
\usepackage{tabularx}

%\usepackage{todonotes}
%\usepackage{todo}
\usepackage{fixmetodonotes}
\defnote{Comment}{inline}{\hspace{10pt}}
\defnote{Note}{inline}{\hspace{10pt}\fbox}
%------------------------------------

% Uncomment the following "usepackage" line if you wish to use BiBTeX to create the
% bibliography.  This package is necessary to use the \citep or \citet commands, which are
% commonly used in publications like the Journal of Geophysical Research.
% If the style file "natbib.sty" is not provided with your release of MiKTeX, it is available
% on the Internet.  One example web source is:
% http://ads.harvard.edu/pubs/bibtex/astronat/natbib.sty
\usepackage{natbib}
\usepackage[ngerman, english]{babel} 
%------------------------------------

%Absolute value nice definition
\usepackage{mathtools}

\DeclarePairedDelimiter\abs{\lvert}{\rvert}%
\DeclarePairedDelimiter\norm{\lVert}{\rVert}%

% Swap the definition of \abs* and \norm*, so that \abs
% and \norm resizes the size of the brackets, and the 
% starred version does not.
\makeatletter
\let\oldabs\abs
\def\abs{\@ifstar{\oldabs}{\oldabs*}}
%
\let\oldnorm\norm
\def\norm{\@ifstar{\oldnorm}{\oldnorm*}}
\makeatother
%------------------------------------

% Uncomment this package if you want to use graphics files, such as .eps files
\usepackage{graphicx}

% Other packages go here as needed...
%\usepackage{mathabx}
\usepackage[colorlinks=true, linkcolor=blue, citecolor=blue]{hyperref}

% make todo note lines fancy af and display at left margin
%\let\tmptodo\todo
%\renewcommand{\todo}[1]{\tmptodo[fancyline]{#1}}
%\reversemarginpar

% Why not..
\allowdisplaybreaks 
%------------------------------------

%%%%%%%%%%%%%%%%%%%%%%%%%%%%%%%%%%%%%%%
%%%%%%%%%%%% Customization %%%%%%%%%%%%
%%%%%%%%%%%%%%%%%%%%%%%%%%%%%%%%%%%%%%%

% ----- my commands -----
\newcommand{\comment}[1]{\fbox{\begin{minipage}{\textwidth}\Comment{\textcolor{blue}{#1}}\end{minipage}}\\}
\newcommand\numberthis{\addtocounter{equation}{1}\tag{\theequation}}
\newcommand{\intrange}[3]{$#1 = #2, \dots, #3$}
\newcommand{\itref}[2]{(#1\ref{#2})}

\newenvironment{myarray}{\begin{center}$\begin{array}{ll} }{\end{array}$ \end{center}}

\renewcommand{\P}{\mathbb{P}}
\newcommand{\E}{\mathbb{E}}
\newcommand{\R}{\mathbb{R}}
\newcommand{\F}{\mathcal{F}}
\newcommand{\A}{\mathcal{A}}
\newcommand{\G}{\mathcal{G}}
\newcommand{\N}{\mathcal{N}}

\newcommand{\fnkm}{F_n^{km}}
\newcommand{\fnse}{F_n^{se}}
\newcommand{\wn}[2]{W_{#1:#2}}
\newcommand{\wnkm}[2]{W_{#1:#2}^{km}}
\newcommand{\wnse}[2]{W_{#1:#2}^{se}}
\newcommand{\wnseb}[2]{\bar{W}_{#1:#2}^{se}}
\newcommand{\sn}[1]{S_{#1}}
\newcommand{\snkm}[2]{S_{#1,#2}^{km}}
\newcommand{\snse}[2]{S_{#1,#2}^{se}}
\newcommand{\stnse}[1]{S_{2,#1}^{se}}
\newcommand{\snseb}[2]{\bar{S}_{#1,#2}^{se}}
\newcommand{\unkm}[2]{U_{#1,#2}^{km}}
\newcommand{\unse}[2]{U_{#1,#2}^{se}}

\newcommand{\StN}[1]{\tilde{S}_{#1}^N}
\newcommand{\FtN}[1]{\tilde{\F}_{#1}^N}
\newcommand{\YN}[1]{Y_{#1}^N}
\newcommand{\xitN}[1]{\tilde{\xi}_{#1}^N}
\newcommand{\UNab}[1]{U_{#1}^N[a,b]}

\newcommand{\I}[1]{{\mathbbm{I}\{#1\}}}
%\newcommand{\I}[1]{{\mathbbm{1}\{#1\}}}

\newcommand{\df}{d.\,f.}
\newcommand{\ie}{i.\,e.}
\newcommand{\iid}{i.\,i.\,d.}
\newcommand{\rv}{r.\,v.}
%\newcommand{\st}{s.\,t.}
\newcommand{\wpo}{w.\,p.\,1}
\newcommand{\as}{a.\,s.}
\newcommand{\st}{s.\,t.}
\newcommand{\wrt}{w.\,r.\,t.}

\newcommand{\cfbox}[2]{%
	\colorlet{currentcolor}{.}%
	{\color{#1}%
		\fbox{\color{currentcolor}#2}}%
}
\renewcommand{\.}{\textrm{ .}}


% \newtheorem{lemma}{Lemma}
% \newtheorem{theorem}{Theorem}

% ----- my mathoperators -----
\newcommand{\doublesum}{\mathop{\sum\sum}}
\newcommand{\qeq}{\mathop{\stackrel{?}{=}}}
\newcommand{\qleq}{\mathop{\stackrel{?}{\leq}}}
\newcommand{\qgeq}{\mathop{\stackrel{?}{\geq}}}

\newcommand{\cf}{c.\,f.}
\newcommand{\mdot}{\textrm{ .}}
\newcommand{\mcomma}{\textrm{ ,}}

% ----- UWM stylings -----
%\renewcommand{\evensidemargin}{.875in}  
%\renewcommand{\oddsidemargin}{.875in}   
%\renewcommand{\topmargin}{-.3in}
%\renewcommand{\headheight}{0.2in}
%\renewcommand{\marginparwidth}{1.4in}
%\renewcommand{\marginparsep}{1.4pt}
%\renewcommand{\headsep}{.65in}
%\renewcommand{\footskip}{0.3in}
%\renewcommand{\textheight}{8in}
%\renewcommand{\textwidth}{5.9in} % crashes todonotes

\newcommand{\ls}{\vspace{.1in}}

\newtheorem{thm}{Theorem}
\newtheorem{lemma}[thm]{Lemma}
\newtheorem{cor}[thm]{Corollary}
\newtheorem{prop}[thm]{Proposition}
\theoremstyle{definition}
\newtheorem{example}[thm]{Example}
\newtheorem{remark}[thm]{Remark}
\newtheorem{defn}[thm]{Definition}
\newtheorem{ques}[thm]{Question}
\newtheorem{exer}[thm]{Exercise}
\numberwithin{thm}{chapter}

\newcommand{\todo}{\TODO}

% \renewcommand{\phi}{\varphi}

%%%%%%%%%%%%%%%%%%%%%%%%%%%%%%%%%%%%%%%%%
%%%%%%%%%%%% Personalization %%%%%%%%%%%%
%%%%%%%%%%%%%%%%%%%%%%%%%%%%%%%%%%%%%%%%%

% Insert your full name in the brackets.
\renewcommand{\ThesisAuthor}{Jan Hoft}

% If you are graduating in Spring, insert May.  If you are graduating in Fall, insert December.
\renewcommand{\ThesisMonth}{May}

% Insert the year of your graduation here
\renewcommand{\ThesisYear}{2018}

% Your thesis title goes here.  It will automatically be formatted to use multiple lines
% (if needed)
\renewcommand{\ThesisTitle}{Large sample properties of U-Statistics under semiparametric Random Censorship}

% Insert your advising professor's name here.  DO NOT include a prefix of "Prof." or "Dr."
% here!  The prefix will be inserted automatically.
\renewcommand{\ThesisAdvisor}{Professor Jay Beder}

%------------------------------------

% If your thesis or dissertation has multiple volumes, set the argument to true.
% If not, set the argument to false.
\setboolean{multvolumes}{false}

% If your thesis or dissertation has multiple appendices, set the argument to true.
% If not, set the argument to false.
\setboolean{singleappendix}{false}

%---------------------------------------

% Creating new commands for displaying derivatives
% Add additional new commands as needed...
\newcommand{\ptlder}[2]{\frac{\partial #1}{\partial #2}}
\newcommand{\totder}[2]{\frac{d #1}{d #2}}

%---------------------------------------

%%%%%%%%%%%%%%%%%%%%%%%%%%%%%%%%%%%%%%%%%%%%%%%%%%%%%%%%%%%%%%%%%%%%%%%%%%%%%%%%%%%%%%%%%%%
%%%%%%%%%%%%%%%%%%%%%%%%%%%%%%%%%% BEGINNING OF DOCUMENT %%%%%%%%%%%%%%%%%%%%%%%%%%%%%%%%%%
%%%%%%%%%%%%%%%%%%%%%%%%%%%%%%%%%%%%%%%%%%%%%%%%%%%%%%%%%%%%%%%%%%%%%%%%%%%%%%%%%%%%%%%%%%%
\begin{document}
	
	% \listofnotes
	
	\tableofcontents

	\chapter{Notation and assumptions} \label{ch:notation}
In this chapter we will state the main definitions and assumptions used throughout this work. We will start by defining the estimator to be considered and introduce all necessary notation for the remaining chapters.
%
\section{Definitions and notation}
Recall the following definitions for $n\geq2$
$$\wnse{i}{n} = \frac{m(Z_{i:n}, \hat{\theta}_n)}{n-i+1} \prod\limits_{j=1}^{i-1}\left(1-\frac{m(Z_{j:n}, \hat{\theta}_n)}{n-j+1}\right)$$
%
and
$$S_{2,n}^{se} = \doublesum\limits_{1\leq i<j\leq n}\phi(Z_{i:n}, Z_{j:n})W_{i:n}^{se}W_{j:n}^{se}$$
%
Furthermore define
$$W_{i:n}(q) = \frac{q(Z_{i:n})}{n-i+1}\prod_{k=1}^{i-1}\left[1-\frac{q(Z_{k:n})}{n-k+1}\right]$$
and 
$$S_{n}(q) = \doublesum\limits_{1\leq i<j\leq n}\phi(Z_{i:n}, Z_{j:n})W_{i:n}(q)W_{j:n}(q)$$
for some measurable function $q$ \st\ $q(t)\in[0,1]$ for all $t\in\R_+$.
%
Next define
$$\F_n = \sigma\{Z_{1:n}, \dots, Z_{n:n}, Z_{n+1}, Z_{n+2}, \dots\}$$
%
The following quantities will be needed in section \ref{sec:supermart}. Define for $n\geq 2$ and $s < t$
\begin{align*}
B_n(s,q) &:= \prod_{k=1}^{n}\left[1+\frac{1-q(Z_{k})}{n-R_{k,n}}\right]^{\I{Z_{k} < s}}\\
C_n(s,q) &:= \sum_{i=1}^{n+1}\left[\frac{1-q(s)}{n-i+2}\right]\I{Z_{i-1:n} < s \leq Z_{i:n}}\\
D_n(s,t,q) &:= \prod_{k=1}^{n} \left[1+\frac{1-q(Z_k)}{n-R_{k,n} +2}\right]^{2\I{Z_k<s}} \prod_{k=1}^{n}\left[1+\frac{1-q(Z_k)}{n-R_ {k,n}+1}\right]^{\I{s < Z_k < t}}\\
\Delta_n(s,t,q) &:= \E\left[D_n(s,t,q) \right]\\
\bar{\Delta}_n(s,t,q) &:= \E\left[C_n(s,q)D_n(s,t,q) \right]
\end{align*}
and
$$D(s,t,q) := \exp\left(2\int_{0}^{s} \frac{1-q(z)}{1-H(z)} H(dz) + \int_{s}^{t} \frac{1-q(z)}{1-H(z)} H(dz)\right)\mdot$$
We will write $B_n(s) \equiv B_n(s,q)$, $C_n(s) \equiv C_n(s,q)$, $D_n(s,t) \equiv D_n(s,t,q)$, $\Delta_n(s,t) \equiv \Delta_n(s,t,q)$, $\bar\Delta_n(s,t) \equiv \bar\Delta_n(s,t,q)$ and $D(s,t) \equiv D(s,t,q)$. Next let
$$\bar{S}_n(q) := \doublesum\limits_{1\leq i < j \leq n} \phi(Z_{i:n},Z_{j:n}) \bar{W}_{i:n}(q) \bar{W}_{j:n}(q)$$
where 
$$\bar{W}_{i:n}(q) := \frac{1}{n-i+1}\prod_{k=1}^{n}\left(1-\frac{q(Z_{k:n})}{n-k+1}\right)\mdot$$
Moreover define for $s<t$
\begin{align*}
S(q) &:= \frac{1}{2}\int_{0}^{\infty} \int_{0}^{\infty} \phi(s,t) q(s)q(t) \exp\left(\int_{0}^{s} \frac{1-q(x)}{1-H(x)} H(dx)\right)\\
&\qquad\qquad\qquad \times \exp\left(\int_{0}^{t} \frac{1-q(x)}{1-H(x)} H(dx) \right)H(ds)H(dt)
\end{align*}
and 
\begin{align*}
\bar{S}(q) &:= \frac{1}{2}\int_{0}^{\infty} \int_{0}^{\infty} \phi(s,t)  \exp\left(\int_{0}^{s} \frac{1-q(x)}{1-H(x)} H(dx)\right)\\
&\qquad\qquad\qquad \times \exp\left(\int_{0}^{t} \frac{1-q(x)}{1-H(x)} H(dx) \right)H(ds)H(dt)\mdot
\end{align*}
We will write $\sn{n} \equiv \sn{n}(q)$, $\wn{i}{n} \equiv \wn{i}{n}(q)$, $S\equiv S(q)$ and $\bar S\equiv \bar S(q)$ throughout this thesis.
%
\section{Assumptions}
The following assumptions will be needed throughout this thesis:
\begin{enumerate}[({A}1)]
	\item \label{ass:kernel_gen} The kernel $\phi: \R^2 \longrightarrow \R$ is measurable, non-negative and symmetric in its arguments. In effect $\phi(s,t) = \phi(t,s)$ for all $s,t \in \R_+$. 
	\item \label{ass:H_nonneg} $H$ is continuous and concentrated on the non-negative real line.
	\item \label{ass:intgral_phi_q} The following statement holds true
	$$\int_{0}^{\tau_H} \int_{0}^{\tau_H} \frac{\phi(s,t)}{m(s, \theta_0)m(t,\theta_0)(1-H(s))^\epsilon(1-H(t))^{\epsilon}} F(dt)F(ds) < \infty$$
	for some $0<\epsilon\leq 1$.
	\item $m(z,\theta)$ is non-decreasing in $z$. \label{ass:m_increas}
\end{enumerate}
Here condition (A\ref{ass:kernel_gen}) is a the standard assumption for U-Statistics (\cf\ \cite{lee1990u}). Assumptions (A\ref{ass:H_nonneg}) is the same as in \cite{dikta2000strong}. (A\ref{ass:intgral_phi_q}) is here the 2-dimensional equivalent to the condition in Theorem 1.1 of \cite{dikta2000strong}. Condition (A\ref{ass:m_increas}) poses an additional restriction on the censoring model $m$ here. We will discuss the restrictions imposed by (A\ref{ass:m_increas}) and see examples of different models for $m$, which satisfy this condition in Chapter \ref{ch:model}. Moreover, Chapter \ref{ch:simulation} shows simulation studies under different choices for $m$.\\
\\
%
%\clearpage
%
We will need the following assumptions about the Censoring Model $m$ and the Maximum Likelihood estimate $\hat\theta_n$:
\begin{enumerate}[({M}1)]
	\item \label{ass:m_consistency} $\hat{\theta}_n$ is measurable and tends to $\theta_0$
	\item \label{ass:m_nbhd} For any $\epsilon>0$ there exists a neighborhood $V(\epsilon, \theta_0)\subset \Theta$ of $\theta_0$ \st\ for all $\theta\in V(\epsilon, \theta_0)$ 
	$$\sup\limits_{x\geq 0} |m(x, \theta) - m(x, \theta_0)| < \epsilon$$
\end{enumerate}
Condition (M\ref{ass:m_consistency}) above guarantees the strong consistency of the MLE. (M\ref{ass:m_consistency}) and (M\ref{ass:m_nbhd}) are identical to (A1) and (A2) in \cite{dikta2000strong}.

	
	\chapter{Uniform Almost Sure Convergence}
	Recall the following quantities
$$H^1(x) = \int_{0}^{x} m(z,\theta_0) H(dz)$$
and 
$$H^1_n(x) = \int_{0}^{x} m(z,\theta_0) H_n(dz) = \frac{1}{n}\sum_{i=1}^{n}\I{Z_{i:n} \leq x} m(Z_{i:n}, \theta_0)\mcomma$$
\cf\ \cite{dikta1998semiparametric}, Lemma 3.12.
%
\section{New results}
The following lemma contains an integration by parts result, which will be useful in order to prove Lemma \ref{lem:uniform_s_t}.
\begin{lemma} \label{lem:int_by_parts}
	For any $0\leq s<t\leq T$ we have
	\begin{align*}
	&\int_s^{t-} \frac{1}{1-H(z)} H_n(dz) - \int_s^{t} \frac{1}{1-H(z)} H(dz)\\
	&= \frac{H_n(t) - H(t)}{1-H(t)} - \frac{H_n(s-) - H(s)}{1-H(s)} - \int_s^{t} \frac{H_n(z-) - H(z)}{(1-H(z))^2} H(dz) - \gamma_n(t) \numberthis\label{eq:int_by_parts_1}
	\end{align*}
	and 
	\begin{align*}
	&\int_s^{t-} \frac{1}{1-H(z)} H_n^1(dz) - \int_s^{t} \frac{1}{1-H(z)} H^1(dz)\\
	&= \frac{H^1_n(t) - H^1(t)}{1-H(t)} - \frac{H^1_n(s-) - H^1(s)}{1-H(s)} - \int_s^{t} \frac{H^1_n(z-) - H^1(z)}{(1-H(z))^2} H(dz) - \gamma^1_n(t) \numberthis\label{eq:int_by_parts_2}
	\end{align*}
	where
	$$\gamma_n(t) = \frac{H_n(t)-H_n(t-)}{1-H(t)} \quad\textrm{ and }\quad \gamma_n^1(t) = \frac{H^1_n(t)-H^1_n(t-)}{1-H(t)}\mdot$$	
	\begin{proof}
		First consider that we can write 
		\begin{align*}
		\int_{s}^{t} \frac{1}{1-H(z)} H_n(dz) = \int_{s}^{t-} \frac{1}{1-H(z)} H_n(dz) + \gamma_n(s)\mdot
		\end{align*}
		Thus we have 
		\begin{align*}
		\int_{s}^{t-} \frac{1}{1-H(z)} H_n(dz) &=\int_{s}^{t} \frac{1}{1-H(z)} H_n(dz) - \gamma_n(s)\\
		&=\int_{s}^{t} \left(\frac{1}{1-H(z)} - 1\right) H_n(dz) + \int_{s}^{t} 1 H_n(dz) - \gamma_n(s)\\
		&= \int_{s}^{t} \frac{H(z)}{1-H(z)} H_n(dz) + H_n(t) - H_n(s-) - \gamma_n(s)
		\end{align*}
		since we have
		$$\int_{s}^{t}1 H_n(dz) = \int_{0}^{t}1 H_n(dz) - \int_{0}^{s-}1 H_n(dz) = H_n(t) - H_n(s-) \mdot$$
		%
		We will now use a version of integration by parts (see \cite{cohn2013measure}, p. 164) to show % p. 189
		\begin{align*}
		&\int_{s}^{t} \frac{H(z)}{1-H(z)} H_n(dz) + H_n(t) - H_n(s-)\\
		&=\frac{H_n(t)}{1-H(t)} - \frac{H_n(s-)}{1-H(s)} - \int_{s}^{t} \frac{H_n(z)}{(1-H(z))^2} H(dz)
		\end{align*}
		First let's define $\tilde{G}(x):=H_n(x)$ and
		$$\tilde{F}(x) := \frac{H(x)}{1-H(x)}$$
		%
		Moreover denote $\mu_{\tilde F}$ and $\mu_{\tilde G}$ the measures induced by $\tilde F$ and $\tilde G$ respectively. Note that we have
		\begin{equation*}
		\mu_{\tilde F}(]s,t]) =\tilde  F(t) - \tilde F(s)  \numberthis\label{eq:mu_F}
		\end{equation*}
		%
		Next consider that we can write
		$$\tilde F(x) = \int_{0}^{x} \frac{1}{(1-H(z))^2} H(dz)$$ 
		since we have
		\begin{align*}
		\int_{0}^{x} \frac{1}{(1-H(z))^2} H(dz) &= \int_{0}^{H(x)} \frac{1}{(1-u)^2} du\\
		&= \int_{0}^{H(x)} \frac{1}{(1-u)^2} du\\
		&= \frac{1}{1-H(x)} - 1\\
		&= \frac{H(x)}{1-H(x)}\mdot
		\end{align*}
		%
		Now combining the above with \eqref{eq:mu_F} yields
		\begin{align*}
		\mu_{\tilde F}(]s,t]) = \tilde  F(t) - \tilde F(s)  = \int_{s}^{t} \frac{1}{(1-H(z))^2} H(dz)\mdot
		\end{align*}	
		%
		Therefore the Radon Nikodym derivative of $\mu_{\tilde F}$ \wrt\ $H$ is given by
		\begin{equation*}
		\frac{\mu_{\tilde F}(dx)}{H(dx)} = \frac{1}{(1-H(x))^2} \numberthis\label{eq:RN}\mdot
		\end{equation*}
		%
		Note that $\tilde{F}$ and $\tilde{G}$ are bounded, right-continuous and vanish at $-\infty$. Thus we can apply \cite{cohn2013measure}, p. 164, to obtain
		\begin{align*}
		\int_{s}^{t} \tilde{F}(z) \mu_{\tilde G}(dz) &=\tilde F(t) \tilde G(t) - \tilde F(s-) \tilde G(s-) - \int_{s}^{t} \tilde G(z-) \mu_{\tilde F}(dz)\mdot
		\end{align*}
		%
		Now we get by \eqref{eq:RN} and by definition of $\tilde F$ and $\tilde G$ that
		\begin{align*}
		\int_{0}^{s} \frac{H(z)}{1-H(z)} H_n(dz) &=\frac{H_n(t)H(t)}{1-H(t)} - \frac{H_n(s-)H(s)}{1-H(s)} - \int_{s}^{t} H_n(z-) \mu_{\tilde F}(dz)\\
		&=\frac{H_n(t)H(t)}{1-H(t)} - \frac{H_n(s-)H(s)}{1-H(s)} - \int_{s}^{t} \frac{H_n(z-)}{(1-H(z))^2} H(dz)\mdot
		\end{align*}	 
		%
		Therefore we have
		\begin{align*}
		\int_{s}^{t-} \frac{1}{1-H(z)} H_n(dz) &= \int_{s}^{t} \frac{H(z)}{1-H(z)} H_n(dz) + H_n(t) - H_n(s-) - \gamma_n(s)\\ 
		&=\frac{H_n(t)H(t)}{1-H(t)} - \frac{H_n(s-)H(s)}{1-H(s)} - \int_{0}^{s} \frac{H_n(z-)}{(1-H(z))^2} H(dz)\\
		&\qquad + H_n(t) - H_n(s-) - \gamma_n(s)\\
		&=\frac{H_n(t)}{1-H(t)} - \frac{H_n(s-)}{1-H(s)} - \int_{0}^{s} \frac{H_n(z-)}{(1-H(z))^2} H(dz)\\
		&\qquad - \gamma_n(s)\mdot
		\numberthis\label{eq:int_hn_ab}
		\end{align*}
		The latter equality holds, since 
		\begin{align*}
		\frac{H_n(t)H(t)}{1-H(t)} + H_n(t) =\frac{H_n(t)}{1-H(t)}
		\end{align*}
		and 
		\begin{align*}
		\frac{H_n(s-)H(s)}{1-H(s)} + H_n(s-) = \frac{H_n(s-)}{1-H(s)}\mdot
		\end{align*}
		%
		Now consider the following
		\begin{align*}
		\int_{s}^{t} \frac{1}{1-H(z)} H(dz) &=\int_{s}^{t} \frac{H(z)}{1-H(z)} H(dz) + H(t) - H(s)
		\end{align*}
		Define $\bar{G}(x):=H(x)$ and note that $\bar{G}(x)$ is bounded, right-continuous and vanishes at $-\infty$. Therefore applying \cite{cohn2013measure}, p. 164, to $\tilde{F}$ and $\bar G$ yields
		\begin{align*}
		\int_{s}^{t} \frac{H(z)}{1-H(z)} H(dz) =\frac{H^2(t)}{1-H(t)} - \frac{H^2(s)}{1-H(s)} - \int_{s}^{t} \frac{H(z)}{(1-H(z))^2} H(dz)\mdot
		\end{align*}
		Hence we have
		\begin{align*}
		\int_{s}^{t} \frac{1}{1-H(z)} H(dz) &= \frac{H^2(t)}{1-H(t)} - \frac{H^2(s)}{1-H(s)} - \int_{s}^{t} \frac{H(z)}{(1-H(z))^2} H(dz)\\
		&\qquad + H(t) - H(s)\\
		&= \frac{H(t)}{1-H(t)} - \frac{H(s)}{1-H(s)} - \int_{s}^{t} \frac{H(z)}{(1-H(z))^2} H(dz)\numberthis\label{eq:int_h_ab}\mdot
		\end{align*}
		%
		Combining \eqref{eq:int_hn_ab} and \eqref{eq:int_h_ab} yields
		\begin{align*}
		&\int_s^{t-} \frac{1}{1-H(z)} H_n(dz) - \int_s^{t} \frac{1}{1-H(z)} H(dz)\\
		&= \frac{H_n(t) - H(t)}{1-H(t)} - \frac{H_n(s-) - H(s)}{1-H(s)} - \int_s^{t} \frac{H_n(z-) - H(z)}{1-H(z)} H(dz) - \gamma_n(t)\mdot
		\end{align*}
		Thus equation \eqref{eq:int_by_parts_1} from the statement of the lemma has been established. Next define $\tilde{G}^1(x):=H^1_n(x)$ and apply \cite{cohn2013measure}, p. 164, to $\tilde{F}$ and $\tilde G^1$ to obtain
		\begin{align*}
		\int_{s}^{t} \frac{H(z)}{1-H(z)} H_n^1(dz) =\frac{H^1_n(t)H(t)}{1-H(t)} - \frac{H^1_n(s-)H(s)}{1-H(s)} - \int_{s}^{t} \frac{H^1_n(z)}{(1-H(z))^2} H(dz)\numberthis\label{eq:int_hn1_ab}
		\end{align*}
		Next define $\bar{G}^1(x):=H^1(x)$ and apply \cite{cohn2013measure}, p. 164, to $\tilde{F}$ and $\bar G^1$ to obtain
		\begin{align*}
		\int_{s}^{t} \frac{H(z)}{1-H(z)} H^1(dz) =\frac{H^1(t)H(t)}{1-H(t)} - \frac{H^1(s-)H(s)}{1-H(s)} - \int_{s}^{t} \frac{H^1(z)}{(1-H(z))^2} H(dz)\numberthis\label{eq:int_h1_ab}
		\end{align*}
		Finally consider the following
		\begin{align*}
		&\int_s^{t-} \frac{1}{1-H(z)} H^1_n(dz) - \int_s^{t} \frac{1}{1-H(z)} H^1(dz)\\
		&= \int_s^{t} \frac{1}{1-H(z)} H^1_n(dz) - \int_s^{t} \frac{1}{1-H(z)} H^1(dz) - \gamma_n^1(t)\\
		&= \int_s^{t} \frac{H(z)}{1-H(z)} H^1_n(dz) + H_n^1(t) - H_n^1(s-)\\
		&\qquad - \int_s^{t} \frac{1}{1-H(z)} H(dz) + H^1(t) - H^1(s-) - \gamma_n^1(t)\mdot
		\end{align*}
		Now combining the above with equations \eqref{eq:int_hn1_ab} and \eqref{eq:int_h1_ab} yields the second part of the lemma.
	\end{proof}
\end{lemma}
%
The lemma below contains a statement about uniform convergence of processes considered in the proof of Lemma \ref{lem:dn_limit}. It will be used to establish Corollary \ref{cor:indep_s_t}. 
\begin{lemma} \label{lem:uniform_s_t}
	The following holds for any $T<\tau_H$.
	\begin{align*}
	\sup\limits_{0\leq s<t\leq T}\left|\int_{s}^{t-} \frac{1-m(z,\theta_0)}{1-H(z)} H_n(dz) - \int_{s}^{t} \frac{1-m(z,\theta_0)}{1-H(z)} H(dz)\right|\to 0
	\end{align*}
	almost surely as $n\to\infty$.
	% 
	\begin{proof}
		First consider the following
		\begin{align*}
		&\sup\limits_{0\leq s<t\leq T}\left|\int_{s}^{t-} \frac{1-m(z,\theta_0)}{1-H(z)} H_n(dz) - \int_{s}^{t} \frac{1-m(z,\theta_0)}{1-H(z)} H(dz)\right|\\
		&= \sup\limits_{0\leq s<t\leq T}\left|\int_{s}^{t-} \frac{1}{1-H(z)} H_n(dz) - \int_{s}^{t-} \frac{1}{1-H(z)} H(dz)\right.\\
		&\qquad\qquad + \left.\int_{s}^{t-} \frac{m(z,\theta_0)}{1-H(z)} H(dz) - \int_{s}^{t-} \frac{m(z,\theta_0)}{1-H(z)} H_n(dz)\right|\\
		&= \sup\limits_{0\leq s<t\leq T}\left|\int_{s}^{t-} \frac{1}{1-H(z)} H_n(dz) - \int_{s}^{t-} \frac{1}{1-H(z)} H(dz)\right.\\
		&\qquad\qquad + \left. \int_{s}^{t-} \frac{1}{1-H(z)} H^1(dz) - \int_{s}^{t-} \frac{1}{1-H(z)} H_n^1(dz)\right|\\
		&\leq \sup\limits_{0\leq s<t\leq T}\left| \int_{s}^{t-} \frac{1}{1-H(z)} H_n(dz) - \int_{s}^{t-} \frac{1}{1-H(z)} H(dz)\right|\\
		&\qquad +  \sup\limits_{0\leq s<t\leq T}\left| \int_{s}^{t-} \frac{1}{1-H(z)} H^1(dz) - \int_{s}^{t-} \frac{1}{1-H(z)} H_n^1(dz)\right| \mdot\numberthis\label{eq:hn_h}\mdot
		\end{align*}
		Applying Lemma \ref{lem:int_by_parts} equation \eqref{eq:int_by_parts_1} to the first term above yields
		\begin{align*}
		&\sup\limits_{0\leq s<t\leq T}\left| \int_{s}^{t-} \frac{1}{1-H(z)} H_n(dz) - \int_{s}^{t-} \frac{1}{1-H(z)} H(dz)\right|\\
		&= \sup\limits_{0\leq s<t\leq T}\left| \frac{H_n(t) - H(t)}{1-H(t)} - \frac{H_n(s-) - H(s)}{1-H(s)} \right.\\
		&\qquad\qquad\qquad \left. - \int_s^{t} \frac{H_n(z-) - H(z)}{(1-H(z))^2} H(dz) - \frac{H_n(t-) - H_n(t)}{1-H(t)}\right|\\
		&\leq \sup\limits_{0\leq s<t\leq T}\left| \frac{H_n(t) - H(t)}{1-H(t)}\right| + \sup\limits_{0\leq s<t\leq T}\left| \frac{H_n(s-) - H(s)}{1-H(s)}\right|\\
		&\qquad + \sup\limits_{0\leq s<t\leq T}\left|\int_s^{t} \frac{H_n(z-) - H(z)}{(1-H(z))^2} H(dz)\right| + \sup\limits_{0\leq s<t\leq T}\left|\frac{H_n(t-) - H_n(t)}{1-H(t)}\right|\mdot
		\end{align*}
		Next consider that we have
		$$\sup\limits_{0\leq s<t\leq T}\left| \frac{H_n(t) - H(t)}{1-H(t)}\right| \leq \frac{\sup\limits_{x\leq T}\left| H_n(x) - H(x)\right|}{1-H(T)}$$
		and
		$$\sup\limits_{0\leq s<t\leq T}\left| \frac{H_n(s-) - H(s)}{1-H(s)}\right| \leq \frac{\sup\limits_{x\leq T}\left| H_n(x) - H(x)\right| + \frac{1}{n}}{1-H(T)}\mdot$$
		Furthermore consider
		\begin{align*}
			\sup\limits_{0\leq s<t\leq T}\left|\int_s^{t} \frac{H_n(z-) - H(z)}{(1-H(z))^2} H(dz)\right| &\leq \sup\limits_{0\leq s<t\leq T}\left|\int_0^{t} \frac{H_n(z-) - H(z)}{(1-H(z))^2} H(dz)\right|\\
			&\qquad + \sup\limits_{0\leq s<t\leq T}\left|\int_0^{s} \frac{H_n(z-) - H(z)}{(1-H(z))^2} H(dz)\right|\\
			&\leq 2\cdot \frac{\sup\limits_{x\leq T}\left|H_n(x) - H(x)\right| + \frac{1}{n}}{(1-H(T))^2}\mcomma
		\end{align*}
		since we have for $t\leq T$
		$$\left|\int_0^{t} \frac{H_n(z-) - H(z)}{(1-H(z))^2} H(dz)\right| \leq \int_0^{t} \frac{\left|H_n(z-) - H(z)\right|}{(1-H(T))^2} H(dz)\leq \frac{\sup\limits_{x\leq T}\left|H_n(x) - H(x)\right| + \frac{1}{n}}{(1-H(T))^2}$$
		using Jensen's inequality. Moreover note that	$H_n(s-) - H_n(s) \leq n^{-1}$ for any $0\leq s \leq T$ and hence
		$$\sup\limits_{0\leq s<t\leq T}\left|\frac{H_n(s-) - H_n(s)}{1-H(s)}\right| \leq \frac{1}{n(1-H(T))}\mdot $$
		Therefore we obtain
		\begin{align*}
		&\sup\limits_{0\leq s<t\leq T}\left| \int_{s}^{t-} \frac{1}{1-H(z)} H_n(dz) - \int_{s}^{t-} \frac{1}{1-H(z)} H(dz)\right|\\
		&\leq \frac{\sup\limits_{x\leq T}\left| H_n(x) - H(x)\right|}{1-H(T)} + \frac{\sup\limits_{x\leq T}\left| H_n(x) - H(x)\right| + \frac{1}{n}}{1-H(T)} \\
		&\qquad + 2\cdot\frac{\sup\limits_{x\leq T}\left|H_n(x) - H(x)\right| + \frac{1}{n}}{(1-H(T))^2} + \frac{1}{n(1-H(T))}\\
		&\to 0
		\end{align*}
		almost surely as $n\to\infty$ by the Glivenko-Cantelli Theorem and since $H(T)<1$.
		%
		Now let's consider the latter term in \eqref{eq:hn_h}. Applying Lemma \ref{lem:int_by_parts} equation \eqref{eq:int_by_parts_2} yields
		\begin{align*}
		&\sup\limits_{0\leq s<t\leq T}\left| \int_{s}^{t-} \frac{1}{1-H(z)} H^1_n(dz) - \int_{s}^{t-} \frac{1}{1-H(z)} H^1(dz)\right|\\
		&= \sup\limits_{0\leq s<t\leq T}\left| \frac{H^1_n(t) - H^1(t)}{1-H(t)} -  \frac{H^1_n(s-) - H^1(s)}{1-H(s)}\right.\\
		&\qquad\qquad\qquad \left. - \int_s^{t} \frac{H^1_n(z-) - H^1(z)}{(1-H(z))^2} H(dz) - \frac{H^1_n(t-) - H^1_n(t)}{1-H(t)}\right|\\
		&\leq \sup\limits_{0\leq s<t\leq T}\left| \frac{H^1_n(t) - H^1(t)}{1-H(t)}\right| + \sup\limits_{0\leq s<t\leq T}\left| \frac{H^1_n(s-) - H^1(s)}{1-H(s)}\right|\\
		&\qquad  + \sup\limits_{0\leq s<t\leq T}\left|\int_s^{t} \frac{H^1_n(z-) - H^1(z)}{(1-H(z))^2} H(dz)\right| + \sup\limits_{0\leq s<t\leq T}\left|\frac{H^1_n(t-) - H^1_n(t)}{1-H(t)}\right|\\
		&\leq \frac{\sup\limits_{x\leq T}\left| H^1_n(x) - H^1(x)\right|}{1-H(T)} + \frac{\sup\limits_{x\leq T}\left| H^1_n(x) - H^1(x)\right| + \frac{1}{n}}{1-H(T)}\\ 
		&\qquad + 2\cdot\frac{\sup\limits_{x\leq T}\left|H^1_n(x) - H^1(x)\right| + \frac{1}{n}}{(1-H(T))^2} + \frac{1}{n(1-H(T))}\\
		&\to 0
		\end{align*}
		almost surely as $n\to\infty$ by the Glivenko Cantelli Theorem and since $H(T)<1$.
		%	In Lemma \ref{lem:dn_limit} we have seen that
		%	\begin{align*}
		%		\tilde\Lambda_n(s) := \int_{0}^{s-} \frac{1-q(z)}{1-H_n(z)+\frac{2}{n}} H_n(dz) \to \int_{0}^{s} \frac{1-q(z)}{1-H(z)} H(dz) =: \tilde\Lambda(s)
		%	\end{align*}
		%	almost surely as $n\to\infty$ by Glivenko-Cantelli and the SLLN. Moreover note that $\tilde{\Lambda}_n(s)$ and $\tilde{\Lambda}(s)$ are increasing in $s$. Hence we get
		%	$$\sup\limits_{s\leq T} \left| \int_{0}^{s-} \frac{1-q(z)}{1-H_n(z)+\frac{2}{n}} H_n(dz) - \int_{0}^{s} \frac{1-q(z)}{1-H(z)} H(dz) \right| \to 0$$
		%	almost surely, according to \cite{shorack2009empirical}, p. 304f. 
	\end{proof}
\end{lemma}
%
The following Corollary is important for the proof of Theorem \ref{thm:snmn_limit}.
\begin{cor} \label{cor:indep_s_t}
	The measure zero sets $\{\omega| C_n(s,m;\omega) \nrightarrow C(s,m) \textrm{ as } n\to\infty\}$ and $\{\omega| D_n(s,t,m;\omega) \nrightarrow D(s,t,m) \textrm{ as } n\to\infty\}$ are independent of $s$ and $t$.
	\begin{proof}
		In Lemma \ref{lem:dn_limit} we have seen that $D_n(s,t,q)$ converges almost surely to $D(s,t,q)$ by Glivenko Cantelli and the SLLN. In order to show the statement of the corollary we need to show that this convergence is uniform in $s$ and $t$. Let $q\equiv m(\cdot, \theta_0)$ and recall from the proof of Lemma \ref{lem:dn_limit} that we have 
		\begin{align*}
		& \left|\int_{0}^{s-} \frac{(1-q(z))(H_n(z)-H(z)-\frac{2}{n})}{(1-H_n(z)+\frac{2}{n})(1-H(z))} H_n(dz)\right|\\
		&\leq \frac{\sup_{z\leq T}\abs{H_n(z)- H(z) -\frac{2}{n}}}{1-H(T)} \int_{0}^{T-}\frac{1}{1-H_n(z)} H_n(dz) \longrightarrow 0 
		\end{align*}
		almost surely as $n\to\infty$. Note that the right hand side above converges to zero independent of $s$ and $t$. Next recall that
		\begin{align*}
		\int_{0}^{s-} \frac{1-q(z)}{1-H(z)} H_n(dz) \longrightarrow \int_{0}^{s} \frac{1-q(z)}{1-H(z)} H(dz) \numberthis\label{eq:slln_dn1}
		\end{align*}		
		by the SLLN. Note that this means pointwise convergence. But according to Lemma \ref{lem:uniform_s_t} we also have
		\begin{align*}
		\sup\limits_{0\leq s\leq T}\left|\int_{0}^{s-} \frac{1-m(z,\theta_0)}{1-H(z)} H_n(dz) - \int_{0}^{s} \frac{1-m(z,\theta_0)}{1-H(z)} H(dz)\right|\to 0
		\end{align*}
		almost surely as $n\to\infty$. Thus we can show that the convergence in \eqref{eq:slln_dn1} is indeed uniform in $s$ and $t$. For the last part of the proof, we need  
		\begin{align*}
		\sup\limits_{0\leq s<t\leq T}\left|\int_{s}^{t-} \frac{1-m(z,\theta_0)}{1-H(z)} H_n(dz) - \int_{s}^{t} \frac{1-m(z,\theta_0)}{1-H(z)} H(dz)\right|\to 0
		\end{align*}
		almost surely as $n\to\infty$, which is provided by Lemma \ref{lem:uniform_s_t} as well. Hence $D_n(s,t,m) \to D(s,t,m)$ almost surely, uniformly in $s$ and $t$ as $n\to\infty$. By similar arguments we get that $C_n(s,m) \to C(s,m)$ almost surely, uniformly in $s$ and $t$ as $n\to\infty$, considering the proof of Lemma \ref{lem:Cn_bounds_and_limit}.
	\end{proof}
\end{cor}

\section{Unchanged from my thesis}
Recall the following  definition from the beginning of this chapter:
$$D_n(s,t) := \prod_{k=1}^{n} \left[1+\frac{1-q(Z_k)}{n-R_{k,n} +2}\right]^{2\I{Z_k<s}} \prod_{k=1}^{n}\left[1+\frac{1-q(Z_k)}{n-R_ {k,n}+1}\right]^{\I{s < Z_k < t}}\\$$
The next lemma identifies the almost sure limit of $D_n$ for $n\to\infty$. Define for $s<t$
$$D(s,t) := \exp\left(2\int_{0}^{s} \frac{1-q(z)}{1-H(z)} H(dz) + \int_{s}^{t} \frac{1-q(z)}{1-H(z)} H(dz)\right)$$
\begin{lemma} \label{lem:dn_limit}
	For any $s < t \leq T$ \st\ $H(T)<1$, we have
	$$\lim\limits_{n\to\infty}D_n(s,t) = D(s,t)\mdot$$
	%	
	\begin{proof}
		First recall the following definition
		\begin{align*}
		D_n(s,t) &:= \prod_{k=1}^{n} \left[1+\frac{1-q(Z_k)}{n-R_{k,n} +2}\right]^{2\I{Z_k<s}} \prod_{k=1}^{n}\left[1+\frac{1-q(Z_k)}{n-R_ {k,n}+1}\right]^{\I{s < Z_k < t}} \mdot
		\end{align*}
		%
		Next let 
		\begin{align*}
		x_k &:= \frac{1-q(Z_k)}{n(1- H_n(Z_k) + 2/n)}\\
		y_k &:= \frac{1-q(Z_k)}{n(1- H_n(Z_k) + 1/n)}\\
		s_k &:= \I{Z_k < s} \\
		t_k &:= \I{s < Z_k < t}
		\end{align*}
		for $s<t$ and $k=1,\dots,n$.
		%
		Now consider 
		\begin{align*}
		D_n(s,t) &= \prod_{k=1}^{n} \left[1+\frac{1-q(Z_k)}{n(1- H_n(Z_k) + 2/n)}\I{Z_k<s}\right]^{2}\\ 
		&\qquad \times \prod_{k=1}^{n}\left[1+\frac{1-q(Z_k)}{n(1-H_n(Z_k)+1/n)}\I{s < Z_k < t}\right]\\
		&= \prod_{k=1}^{n} \left[1+x_k s_k\right]^{2} \prod_{k=1}^{n}\left[1+y_k t_k\right]\\
		&= \exp\left(2\sum_{k=1}^{n}\ln\left[1+x_k s_k\right] + \sum_{k=1}^{n}\ln\left[1+y_k t_k\right]\right)\mdot
		\end{align*}
		%
		Note that $0 \leq x_k s_k \leq 1$ and $0 \leq y_k t_k \leq 1$. Consider that the following inequality holds  
		$$-\frac{x^2}{2} \leq \ln(1+x) - x \leq 0$$ 
		for any $x \geq 0$ (cf.  \cite{stute1993strong}, p. 1603). This implies 
		$$-\frac{1}{2}\sum_{k=1}^{n}x_k^2 s_k \leq \sum_{k=1}^{n}\ln(1+x_k s_k) - \sum_{k=1}^{n}x_k s_k \leq 0\mdot$$ 
		But now 
		\begin{align*}
		\sum_{k=1}^{n} x_k^2 s_k &= \frac{1}{n^2} \sum_{k=1}^{n} \left(\frac{1-q(Z_k)}{1-H_n(Z_k)+\frac{2}{n}}\right)^2\I{Z_k<s}\\
		&\leq \frac{1}{n^2} \sum_{k=1}^{n} \left(\frac{1}{1-H_n(s)+\frac{1}{n}}\right)^2\\
		&= \frac{1}{n(1-H_n(s)+n^{-1})^2} \longrightarrow 0
		\end{align*}
		almost surely as $n\to\infty$, since $H(s)<H(t)<1$ (\cf\ \cite{stute1993strong}, p. 1603). Therefore we have
		$$\abs{\sum_{k=1}^{n}\ln(1+x_k s_k) - \sum_{k=1}^{n}x_k s_k} \longrightarrow 0$$
		with probability 1 as $n\to\infty$. 
		%
		Similarly we obtain
		$$\abs{\sum_{k=1}^{n}\ln(1+y_k t_k) - \sum_{k=1}^{n}y_k t_k} \longrightarrow 0$$
		with probability 1 as $n\to\infty$. Hence 
		$$\lim\limits_{n\to\infty} D_n(s,t) = \lim\limits_{n\to\infty} \exp\left(2\sum_{k=1}^{n} x_k s_k + \sum_{k=1}^{n}y_k t_k\right)\mdot$$
		%
		Now consider 
		\begin{align*}
		\sum_{k=1}^{n} x_k s_k &= \frac{1}{n}\sum_{k=1}^{n} \frac{1-q(Z_k)}{1-H_n(Z_k)+\frac{2}{n}}\I{Z_k<s}\\
		&= \int_{0}^{s-} \frac{1-q(z)}{1-H_n(z)+\frac{2}{n}} H_n(dz)\\
		&= \int_{0}^{s-} \frac{1-q(z)}{1-H(z)} H_n(dz) + \int_{0}^{s-} \frac{1-q(z)}{1-H_n(z)+\frac{2}{n}} - \frac{1-q(z)}{1-H(z)} H_n(dz)\\
		&= \int_{0}^{s-} \frac{1-q(z)}{1-H(z)} H_n(dz) + \int_{0}^{s-} \frac{(1-q(z))(H_n(z)-H(z)-\frac{2}{n})}{(1-H_n(z)+\frac{2}{n})(1-H(z))} H_n(dz)\mdot \numberthis \label{eq:xksk_int}
		\end{align*}
		%
		Note that the second term on the right hand side of the latter equation above tends to zero for  $n\to\infty$, because
		\begin{align*}
		& \left|\int_{0}^{s-} \frac{(1-q(z))(H_n(z)-H(z)-\frac{2}{n})}{(1-H_n(z)+\frac{2}{n})(1-H(z))} H_n(dz)\right|\\
		&\leq \frac{\sup_{z\leq T}\abs{H_n(z)- H(z) -\frac{2}{n}}}{1-H(T)} \int_{0}^{T-}\frac{1}{1-H_n(z)} H_n(dz) \longrightarrow 0 \numberthis\label{eq:gc}
		\end{align*}
		%
		almost surely as $n\to\infty$, by the Glivenko-Cantelli Theorem and since $H(T)<1$. Moreover we have
		\begin{align*}
		\int_{0}^{s-} \frac{1-q(z)}{1-H(z)} H_n(dz) \longrightarrow \int_{0}^{s} \frac{1-q(z)}{1-H(z)} H(dz)
		\end{align*}		
		by the SLLN. Therefore we obtain 
		$$\lim\limits_{n\to\infty} \sum_{k=1}^{n} x_k s_k = \int_{0}^{s} \frac{1-q(z)}{1-H(z)} H(dz)\mdot$$
		By the same arguments, we can show that 
		$$\lim\limits_{n\to\infty} \sum_{k=1}^{n} y_k t_k = \int_{s}^{t} \frac{1-q(z)}{1-H(z)} H(dz)\mdot$$
		Thus we finally conclude
		$$\lim\limits_{n\to\infty} D_n(s,t) = \exp\left(2\int_{0}^{s} \frac{1-q(z)}{1-H(z)} H(dz) + \int_{s}^{t} \frac{1-q(z)}{1-H(z)} H(dz)\right)$$
		almost surely.
	\end{proof}
\end{lemma}
%
\begin{lemma}
	For continuous $H$ and $t\leq T<\tau_H$, we have $C_n(t) \to 0$ as $n \to \infty$ \wpo, and $C_n(t) \in [0,1]$ for all $n\geq 1$ and $t\geq 0$.
	\label{lem:Cn_bounds_and_limit}
	%
	\begin{proof}
		It is easy to see that $0\leq C_n(t) \leq 1$ for any $t\geq 0$ and $n\geq 2$, since $0\leq q(t)\leq 1$ and $\I{Z_{i-1:n} < t \leq Z_{i:n}} = 1$ for exactly one $i \in \{1,\dots,n+1\}$. Let's now consider 
		\begin{align*}
		C_n(t) &= \sum_{i=1}^{n+1} \frac{1-q(t)}{n-i+2} [\I{Z_{i-1:n}<t} - \I{Z_{i:n}<t}]\\
		&= \sum_{i=1}^{n+1} \frac{1-q(t)}{n-i+2} \I{Z_{i-1:n}<t} - \sum_{i=1}^{n+1} \frac{1-q(t)}{n-i+2} \I{Z_{i:n}<t}\\
		&= \sum_{i=0}^{n} \frac{1-q(t)}{n-i+1} \I{Z_{i:n}<t} - \sum_{i=1}^{n} \frac{1-q(t)}{n-i+2} \I{Z_{i:n}<t}\\
		&= \sum_{i=1}^{n} \frac{1-q(t)}{n-i+1} \I{Z_{i:n}<t} + \frac{(1-q(t))}{n+1}  - \sum_{i=1}^{n} \frac{1-q(t)}{n-i+2} \I{Z_{i:n}<t}\\
		&= (1-q(t))\left\{\frac{1}{n+1} + \sum_{i=1}^{n} \left[\frac{1}{n-i+1} - \frac{1}{n-i+2}\right]\I{Z_{i:n}<t}\right\}\\
		&= (1-q(t))\sum_{i=1}^{n} \left[\frac{1}{n-nH_n(Z_{i:n})+1} \frac{1}{n-nH_n(Z_{i:n})+2}\right]\I{Z_{i:n}<t}\\
		&\qquad + \frac{1-q(t)}{n+1}\\
		&= (1-q(t))\int_{0}^{t} \left[\frac{1}{1-H_n(x)+\frac{1}{n}} - \frac{1}{1-H_n(x)+\frac{2}{n}}\right]H_n(dx)\\
		&\qquad + \frac{1-q(t)}{n+1}\mdot \numberthis \label{eq:cn}
		\end{align*}
		In Lemma \ref{lem:dn_limit} we have seen that
		$$\int_{0}^{t} \frac{1}{1-H_n(x)+\frac{2}{n}} H_n(dx) \to \int_{0}^{t} \frac{1}{1-H(x)} H(dx)\mdot$$
		By the same arguments we obtain 
		$$\int_{0}^{t} \frac{1}{1-H_n(x)+\frac{1}{n}}H_n(dx) \to \int_{0}^{t} \frac{1}{1-H(x)} H(dx)\mdot$$
		Therefore the right hand side of \eqref{eq:cn} converges to zero. 
	\end{proof}
\end{lemma}


	\bibliographystyle{./ametsoc}
	%\bibliographystyle{alphanum}
	
	% Generate the bibliography using entries from the following .BIB file
	\bibliography{./thesis}
\end{document}