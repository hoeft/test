\chapter{Existence of the limit} \label{ch:ex_limit}

As we have seen during the previous chapter, we were not able to establish that $(S_n, \F_n)_ {n\geq 1}$ is a reverse time supermartingale, and hence we can not establish the almost sure existence of the limit of $S_n$ in the same way as e.\,g. in \cite{dikta2000strong} and \cite{bose1999strong}. The purpose of this chapter is to generalize Doob's Upcrossing Theorem for our estimator $S_n$ in order to show that $S=\lim_{n\to\infty} S_n$ exists $\P$-almost surely.\\
\\
%
Figure \ref{fig:structure_ex} below shows the interdependence structure of the proofs within this chapter.
\setlength{\fboxsep}{5pt}
\begin{figure}[h!]
	\begin{center}
			\fbox{\begin{forest}
				for tree={
					font=\sffamily\bfseries,
					line width=1pt,
					draw=black,
					rounded corners,
					align=center,
					child anchor=north,
					parent anchor=south,
					drop shadow,
					grow = south,
					l sep+=5pt,
					edge path={
						\noexpand\path[color=black, rounded corners=5pt, >={Stealth[length=1pt]}, line width=1pt, \forestoption{edge}]
						(!u.parent anchor) -- +(0,-5pt) -|
						(.child anchor)\forestoption{edge label};
					},
					where level={3}{tier=tier3}{},
					where level={0}{
						l sep-=170pt,
						child anchor=north,
						parent anchor=south,
						line width=1pt,
						align=center,
						grow = south,
					}{},
					where level={1}{
						child anchor=north,
						parent anchor=south,
						align=center,
						edge path={
							\noexpand\path[color=black, rounded corners=5pt, >={Stealth[length=1pt]}, line width=1pt, \forestoption{edge}]
							(!u.parent anchor) -- 
							(.child anchor)\forestoption{edge label};
						}
					}{},
				}
				[Theorem \ref{thm:existence_limit}, inner color=col1in, outer color=col1out
					[Theorem \ref{thm:upcrossing}, inner color=white, outer color=white
						[Lemma \ref{lem:upcrossings_yn}, inner color=white, outer color=white]
						[Lemma \ref{lem:cs}, inner color=white, outer color=white
							[L \ref{lem:qi}, inner color=white, outer color=white]
							[L \ref{lem:optional_skipping}, inner color=white, outer color=white]
						]
						[Lemma \ref{lem:expectation_sq}, inner color=white, outer color=white
							[L \ref{lem:zizone}, inner color=white, outer color=white
								[L \ref{lem:bnbn_change_order}, inner color=white, outer color=white]
							]
							[L \ref{lem:representation_bn}, inner color=white, outer color=white]							
							[L \ref{lem:neveu}, inner color=white, outer color=white
								[L \ref{lem:dn_limit}, inner color=white, outer color=white]
								[L \ref{lem:dn_supermart}, inner color=white, outer color=white]
								[L \ref{lem:hewitt_savage}, inner color=white, outer color=white]
							]
						]
						[Lemma \ref{lem:qisquare_upper_bound}, inner color=white, outer color=white
							[L \ref{lem:qi_increas}, inner color=white, outer color=white]
							[L \ref{lem:q_spacings}, inner color=white, outer color=white]
							[L \ref{lem:bounds}, inner color=white, outer color=white]
						]
					]
				]]
		\end{forest}}
	\end{center}
	\caption{Interdependence Structure of the lemmas and theorems within this chapter.}
	\label{fig:structure_ex}
\end{figure}
\clearpage
%
The following assumptions on $q$ will be needed in Section \ref{sec:upper_bound}. 
\begin{enumerate}[({Q}1)]
	\item \label{ass:sup_qprime} There exists $c_1 < \infty$ \st\ $\sup_{x} (q\circ H^{-1})'(x) \leq c_1$. 
	\item \label{ass:q_H_one} We have $q\circ H^{-1}(1) = 1$.
\end{enumerate}
%
%
%
\section{Modifying Doob's Upcrossing Theorem}
During this section we will generalize Doob's Upcrossing Theorem to our framework. To get closer to the situation of Doob's Upcrossing Theorem, we define the following quantities. Let $N<\infty$ and define for $1\leq n\leq N$ 
\begin{equation*}
\StN{n} := \sn{N-n+1} \textrm{, } \FtN{n} := \F_{N-n+1} \textrm{ \ and \ } \xitN{n} := \xi_{N-n+1}\mdot
\end{equation*}
Note that $\{\FtN{n}\}_{1\leq n\leq N}$ is now an increasing $\sigma$-field in $n$.
%
Below we will define everything needed, in order to generalize Doob's Upcrossing Theorem.
\begin{defn}
	Let $N\geq 2$. For $1\leq n\leq N$ and $a, b \in \mathbb{R}$ with $a < b$, let 
	\begin{align*}
	T_0 &:= 0\\
	T_1 &:= \begin{cases} 
	\min\{1\leq n\leq N | \StN{n}\leq a\} & \textrm{ if } \{1\leq n\leq N | \StN{n}\leq a\}\neq \emptyset\\
	N & \textrm{ if } \{1\leq n\leq N | \StN{n}\leq a\}= \emptyset
	\end{cases}\\	
	T_2 &:= \begin{cases} 
	\min\{T_1\leq n\leq N | \StN{n}\geq b\} & \textrm{ if } \{T_1\leq n\leq N | \StN{n}\leq a\}\neq \emptyset\\
	N & \textrm{ if } \{T_1\leq n\leq N | \StN{n}\geq b\} = \emptyset
	\end{cases}\\	
	\vdots &\quad \vdots \quad \vdots	\\
	T_{2m-1} &:= \begin{cases} 
	\min\{T_{2m-2}\leq n\leq N | \StN{n}\leq a\} & \textrm{ if } \{T_{2m-2}\leq n\leq N | \StN{n}\leq a\}\neq \emptyset\\
	N & \textrm{ if } \{T_{2m-2}\leq n\leq N | \StN{n}\leq a\}= \emptyset
	\end{cases}\\	
	T_{2m} &:= \begin{cases} 
	\min\{T_{2m-1}\leq n\leq N | \StN{n}\geq b\} & \textrm{ if } \{T_{2m-1}\leq n\leq N | \StN{n}\leq a\}\neq \emptyset\\
	N & \textrm{ if } \{T_{2m-1}\leq n\leq N | \StN{n}\geq b\} = \emptyset
	\end{cases}\mdot
	\end{align*}
	%
	Now we can define the number of upcrossings of $[a, b]$ by $\StN{1}, ..., \StN{n}$ as follows:
	\[
	\UNab{n} :=\begin{cases}  
	\max\{1\leq m \leq N | T_{2m} < N\} & \textrm{ if } \{1\leq m \leq N | T_{2m} < N\}\neq \emptyset\\
	0 &  \textrm{ if } \{1\leq m \leq N | T_{2m} < N\} = \emptyset
	\end{cases}
	\]
	%
	Furthermore let for $1\leq k\leq n-1$
	\[ \epsilon_k := \begin{cases} 
	0 & \textrm{ if } k < T_1 \\
	1 & \textrm{ if } T_1 \leq k < T_2\\
	0 & \textrm{ if } T_2 \leq k < T_3\\
	1 & \textrm{ if } T_3 \leq k < T_4\\
	\dots & \textrm{ if } \dots
	\end{cases}
	\]
	and define
	$$\YN{n} := \StN{1} + \sum\limits_{k=1}^{n-1} \epsilon_k (\StN{k+1}-\StN{k})$$
	for $1\leq n\leq N$. 	
\end{defn}
%
The following lemma shows that the expected number of upcrossings of $[a,b]$ is bounded above by the expected value of $Y_n^N$. Thus afterwards it will remain to show that the limit $\E[Y_N^N]$ has a finite limit, which will be done in Section \ref{sec:dn}.
\begin{lemma}
	For $1\leq n\leq N$ we have
	$$\E[\UNab{n}]\leq \frac{\E[\YN{n}]}{b-a}\mdot$$
	\label{lem:upcrossings_yn}
\end{lemma}
%
\begin{proof}
	Consider for $1\leq n\leq N$ and $N\geq 2$
	\begin{align*}
	Y_n^N &= \StN{1} + \sum_{k=1}^{n-1}\epsilon_k(\StN{k+1}-\StN{k})\\
	&= \StN{1} + \sum_{k=1}^{n}(\StN{T_{2k}}-\StN{T_{2k-1}})\\
	&\geq \sum_{k=1}^{n}(\StN{T_{2k}}-\StN{T_{2k-1}})
	\end{align*}
	by definition of $\epsilon_k$. The latter inequality above holds, since $\StN{1}\geq 0$. Note that by definition of $T_1, T_2, \dots$ we have
	$$\sum_{k=1}^{n}(\StN{T_{2k}}-\StN{T_{2k-1}}) \geq (b-a)\UNab{n}\mdot$$
	From here the assertion follows directly.
\end{proof}
%
The following lemma provides a representation for the expectation of the process $Y_N^N$.
\begin{lemma}
	\label{lem:optional_skipping}
	For $1\leq n\leq N$ let
	$$\YN{n} := \StN{1} + \sum_{k=1}^{n-1} \epsilon_k (\StN{k+1} - \StN{k}) $$
	with
	\[ \epsilon_k := \begin{cases} 
		1 & (\StN{1},\dots,\StN{k})\in B_k \\
		0 & otherwise 
	\end{cases}
	\]
	for $k=1,\dots, n-1$. Here $B_k$ is an arbitrary set in $\mathfrak{B}(\mathbb{R}^k)$. Then we have
	\begin{equation}
		\E[\YN{n}] = \E[\StN{n}] - \sum_{k=1}^{n-1}\E\left[(1-\epsilon_k)\left(\E[\StN{k+1}|\FtN{k}]  - \StN{k} \right)\right]\mdot
		\label{eq:ineq_yk}
	\end{equation}
\end{lemma}
%
\begin{proof}
	Consider for $1\leq n\leq N$ and $N\geq 2$
	\begin{align*}
	&  \StN{n+1} - \YN{n+1} \\ 
	&= (1-\epsilon_1)(\StN{2} - \StN{1}) + (1-\epsilon_2)(\StN{3} - \StN{2}) + ... + (1-\epsilon_k)(\StN{n+1} - \StN{n})\\ 
	&= (\StN{n} - \YN{n}) + (1-\epsilon_n)(\StN{n+1} - \StN{n}) \mdot
	\end{align*}
	%
	Conditioning on $\FtN{n}$ on both sides yields
	\begin{align*}
		\E[\StN{n+1} - \YN{n+1}|\FtN{n}] &= \StN{n} - \YN{n} + (1-\epsilon_n)\left(\E[(\StN{n+1})|\FtN{n}]  - \StN{n} \right)\mdot
	\end{align*}
	\\
	Now taking expectations on both sides yields
	$$\E[\StN{n+1} - \YN{n+1}] \geq \E[\StN{n} - \YN{n}] + \E\left[(1-\epsilon_n)\left(\E[\StN{n+1}|\FtN{n}]  - \StN{n} \right)\right]\mdot$$
	Note that 
	\begin{align*}
		\E[\StN{2} - \YN{2}] &= \E[\StN{1} - \YN{1}] + \E\left[(1-\epsilon_1)\left(\E[\StN{2}|\FtN{1}]  - \StN{1} \right)\right]\\ 
		&= \E\left[(1-\epsilon_1)\left(\E[\StN{2}|\FtN{1}]  - \StN{1} \right)\right]\\ 
	\end{align*}
	since $\YN{1} = \StN{1}$. Moreover we have
	\begin{align*}
		\E[\StN{3} - \YN{3}] &= \E[\StN{2} - \YN{2}] + \E\left[(1-\epsilon_2)\left(\E[\StN{3}|\FtN{2}]  - \StN{2} \right)\right]\\  
		&= \E\left[(1-\epsilon_1)\left(\E[\StN{2}|\FtN{1}]  - \StN{1} \right)\right]\\
		&\qquad + \E\left[(1-\epsilon_2)\left(\E[\StN{3}|\FtN{2}]  - \StN{2} \right)\right]\\  
		&\cdots& \\ 
		\E[\StN{n} - \YN{n}] &= \sum_{k=1}^{n-1}\E\left[(1-\epsilon_k)\left(\E[\StN{k+1}|\FtN{k}]  - \StN{k} \right)\right]\mdot
	\end{align*}
	%
	Hence we get
	\begin{equation*}
		\E[\YN{n}] = \E[\StN{n}] - \sum_{k=1}^{n-1}\E\left[(1-\epsilon_k)\left(\E[\StN{k+1}|\FtN{k}]  - \StN{k} \right)\right]\mdot
	\end{equation*}	
\end{proof}
%
\begin{remark}
	Note that we have $\YN{1}=\StN{1}$, as the sum in the definition above is in this case empty and hence treated as zero. Moreover note that we have $\YN{n+1} = \StN{n+1}$ if $\epsilon_k=1$ for all $1\leq k \leq n$. 
\end{remark}
%
The Lemma below establishes an upper bound for $\E[Y_N^N]$. 
\begin{lemma} \label{lem:cs}
	We have for $N\geq 2$
	\begin{align*}
		\E[Y_N^N] &\leq \E[\StN{N}] + \sum_{k=1}^{N-1} \alpha_{N-k+1} \numberthis\label{eq:yn}
	\end{align*}
	where
	\begin{align*}
		\alpha_{N-k+1} &:= \doublesum\limits_{1\leq i<j\leq N-k+1}\E\left[\phi^2(Z_{i:N-k+1}, Z_{j:N-k+1}) W^2_{i:N-k+1} W^2_{j:N-k+1}\right]^{\frac{1}{2}}\\
		&\qquad\qquad\qquad\qquad \times \E\left[(Q_{i,j}^{N-k+1} - 1)^2\right]^{\frac{1}{2}}\mdot
	\end{align*}
	%
	\begin{proof}
		Combining Lemmas \ref{lem:optional_skipping} and \ref{lem:upcrossings_yn} yields the following for $n\leq N$
		$$(b-a) \E[U_n[a,b]] \leq \E[Y_n^N] \leq \E[\StN{n}] - \sum_{k=1}^{n-1} \E[(1-\epsilon_k) \left(\E[\StN{k+1} | \F_k^N] - \StN{k}\right)]\mdot$$
		%
		Moreover we get from Lemma \ref{lem:qi}
		\begin{align*}
			\E[\StN{k+1}|\FtN{k}] &= \E[\sn{N-k} | \F_{N-k+1}]\\
			&= \doublesum\limits_{1\leq i<j\leq N-k+1} \phi(Z_{i:N-k+1}, Z_{j:N-k+1}) W_{i:N-k+1} W_{j:N-k+1} Q_{i,j}^{N-k+1}\mdot
		\end{align*}
		%
		Therefore we obtain
		\begin{align*}
			\E[Y_N^N] &\leq \E[\StN{N}] - \sum_{k=1}^{N-1} \E[(1-\epsilon_k) \E[\StN{k+1} | \F_k^N] - \StN{k} ]\\
			&=  \E[\StN{N}] - \sum_{k=1}^{N-1} \doublesum\limits_{1\leq i<j\leq N-k+1} \E\left[(1-\epsilon_k)\phi(Z_{i:N-k+1}, Z_{j:N-k+1}) \right. \\
			&\qquad\qquad\qquad\qquad\qquad\qquad \times \left. W_{i:N-k+1} W_{j:N-k+1}(Q_{i,j}^{N-k+1} - 1) \right]\\
			&\leq  \E[\StN{N}] + \left|\sum_{k=1}^{N-1} \doublesum\limits_{1\leq i<j\leq N-k+1} \E\left[(1-\epsilon_k)\phi(Z_{i:N-k+1}, Z_{j:N-k+1}) \right.\right. \\
			&\qquad\qquad\qquad\qquad\qquad\qquad \times \left.\left. W_{i:N-k+1} W_{j:N-k+1}(Q_{i,j}^{N-k+1} - 1) \right]\vphantom{\doublesum\limits_{1\leq i<j\leq N-k+1}}\right|\\
			&\leq  \E[\StN{N}] + \sum_{k=1}^{N-1} \doublesum\limits_{1\leq i<j\leq N-k+1} \left|\E\left[(1-\epsilon_k)\phi(Z_{i:N-k+1}, Z_{j:N-k+1}) \right.\right. \\
			&\qquad\qquad\qquad\qquad\qquad\qquad \times \left.\left. W_{i:N-k+1} W_{j:N-k+1}(Q_{i,j}^{N-k+1} - 1) \right]\right|\mdot
		\end{align*}
		%
		Now using Jensen's inequality yields
		\begin{align*}
			\E[Y_N^N] &\leq \E[\StN{N}] + \sum_{k=1}^{N-1} \doublesum\limits_{1\leq i<j\leq N-k+1}\E\left[(1-\epsilon_k) \phi(Z_{i:N-k+1}, Z_{j:N-k+1}) \right.\\
			&\qquad\qquad\qquad\qquad\qquad\qquad \times \left. W_{i:N-k+1} W_{j:N-k+1} \cdot \abs{(Q_{i,j}^{N-k+1} - 1)}\right]\\
			&\leq \E[\StN{N}] + \sum_{k=1}^{N-1} \doublesum\limits_{1\leq i<j\leq N-k+1}\E\left[\phi(Z_{i:N-k+1}, Z_{j:N-k+1}) \right. \\
			&\qquad\qquad\qquad\qquad\qquad\qquad \times \left. W_{i:N-k+1} W_{j:N-k+1} \cdot \abs{(Q_{i,j}^{N-k+1} - 1)}\right]\mdot
		\end{align*}
		The latter inequality above holds, because $1-\epsilon_k \leq 1$ for all $k\leq N-1$. 
		%
		By applying the Cauchy-Schwarz inequality on the expectation above, we obtain
		\begin{align*}
			\E[Y_N^N] &\leq \E[\StN{N}] + \sum_{k=1}^{N-1} \doublesum\limits_{1\leq i<j\leq N-k+1}\E\left[\phi^2(Z_{i:N-k+1}, Z_{j:N-k+1}) W^2_{i:N-k+1} W^2_{j:N-k+1}\right]^{\frac{1}{2}}\\
			&\qquad\qquad\qquad\qquad\qquad\qquad \times \E\left[(Q_{i,j}^{N-k+1} - 1)^2\right]^{\frac{1}{2}}\mdot \numberthis\label{eq:yn}
		\end{align*}
	\end{proof}
\end{lemma}
%
%
%
\section{The reverse supermartingale $D_n$} \label{sec:dn}
Let's first define the following quantities for $n\geq 1$ and $s < t$:
\begin{align*}
	B_n(s) &:= \prod_{k=1}^{n}\left[1+\frac{1-q(Z_{k})}{n-R_{k,n}}\right]^{\I{Z_{k} < s}}\\
	C_n(s) &:= \sum_{i=1}^{n+1}\left[\frac{1-q(s)}{n-i+2}\right]\I{Z_{i-1:n} < s \leq Z_{i:n}}\\
	D_n(s,t) &:= \prod_{k=1}^{n} \left[1+\frac{1-q(Z_k)}{n-R_{k,n} +2}\right]^{2\I{Z_k<s}} \prod_{k=1}^{n}\left[1+\frac{1-q(Z_k)}{n-R_ {k,n}+1}\right]^{\I{s < Z_k < t}}\\
	\Delta_n(s,t) &:= \E\left[D_n(s,t) \right]\\
	\bar{\Delta}_n(s,t) &:= \E\left[C_n(s)D_n(s,t) \right]\mdot
\end{align*}
Here $Z_{0:n} := -\infty$ and $Z_{n+1:n} := \infty$.\\
\\
During this section, we will derive a representation of $\E[S_n]$ which involves the process $D_n$. This will be done in Lemma \ref{lem:zizone} and Lemma \ref{lem:representation_bn}. We will then show that $\{D_n,\F_n\}$ is a reverse supermartingale in Lemma \ref{lem:dn_supermart} and identify the limit of $D_n$ in Lemma \ref{lem:dn_limit}. Those results will lead to Lemma \ref{lem:neveu}, which will be central to develop certain bounds in Section \ref{sec:upper_bound}, and hence to show the almost sure existence of the limit $S$. Moreover Lemmas \ref{lem:zizone}, \ref{lem:representation_bn} and \ref{lem:neveu} will play a central role in identifying the limit $S$ in Chapter \ref{ch:identify_limit}.\\
\\
The lemma below contains a basic result needed to prove Lemma \ref{lem:representation_bn}.
\begin{lemma}
	\label{lem:bnbn_change_order}
	Let $s\neq t$. Then the conditional expectation 
	$$\E[B_n(s)B_n(t) | Z_i=s, Z_j=t]$$ 
	is independent of $i,j$ and hence
	$$\E[B_n(s)B_n(t) | Z_i=s, Z_j=t] = \E[B_n(s)B_n(t) | Z_1=s, Z_2=t]$$
	holds almost surely.
	%
	\begin{proof}
	    For the sake of notational simplicity denote for $s<t$ $s_k^n := \I{Z_{k:n} < s}$ and $t_k^n := \I{s\leq Z_{k:n} < t}$. Consider on $\{s<t\}$ 
		\begin{align*}
			&\E\left[B_n(s)B_n(t)|Z_i=s, Z_j=t\right]\\
			&= \E\left[\prod_{k=1}^{n}\left(1+\frac{1-q(Z_{k:n})}{n-k}\right)^{2s_k^n + t_k^n} | Z_i=s, Z_j=t\right]\\
			&= \E\left[\sum_{k_1=1}^{n-1} \sum_{k_2=2}^{n} \I{Z_{k_1:n} = s} \I{Z_{k_2:n} = t}\left(1+\frac{1-q(s)}{n-k_1}\right)\right.\\ 
			&\qquad\qquad \times \prod_{k=1}^{k_1-1}\left(1+\frac{1-q(Z_{k:n})}{n-k}\right)^{2s_k^n + t_k^n} \\
			&\qquad\qquad \times \prod_{k=k_1+1}^{k_2-1}\left(1+\frac{1-q(Z_{k:n})}{n-k}\right)^{2s_k^n + t_k^n} \\
			&\qquad\qquad \times  \left.\prod_{k=k_2+1}^{n}\left(1+\frac{1-q(Z_{k:n})}{n-k}\right)^{2s_k^n + t_k^n} | Z_i=s, Z_j=t\right]
		\end{align*}
		since $s_{k_1}^n = 0$, $t_{k_1}^n=1$, $s_{k_2}^n = 0$ and $t_{k_2}^n=0$. Moreover we have 
		\[ \begin{cases} 
			s_{k}^n = 1 \textrm{ and } t_{k}^n = 0 & \textrm{ if } k < k_1  \\
			s_{k}^n = 0 \textrm{ and } t_{k}^n = 1 & \textrm{ if } k_1 < k < k_2\\
			s_{k}^n = 0 \textrm{ and } t_{k}^n = 0 & \textrm{ if } k_2 < k \\
			\end{cases}\mdot
		\]		
		Therefore we obtain
		\begin{align*}
			&\E\left[B_n(s)B_n(t)|Z_i=s, Z_j=t\right]\\
			&= \E\left[\sum_{k_1=1}^{n-1} \sum_{k_2=2}^{n} \I{Z_{k_1:n} = s} \I{Z_{k_2:n} = t}\left(1+\frac{1-q(s)}{n-k_1}\right)\right.\\ 
			&\qquad\qquad \times \prod_{k=1}^{k_1-1}\left(1+\frac{1-q(Z_{k:n})}{n-k}\right)^{2s_k^n} \\
			&\qquad\qquad \times \left. \prod_{k=k_1+1}^{k_2-1}\left(1+\frac{1-q(Z_{k:n})}{n-k}\right)^{t_k^n} | Z_i=s, Z_j=t\right]\mdot
		\end{align*}
		%
		Next we need to introduce some more notation. For $1\leq i,j\leq n$ and $n\geq 2$, let $\{Z_{k:n-2}\}_{k\leq n-2}$ denote the ordered $Z$-values among $Z_1,\dots, Z_n$ with $Z_i$ and $Z_j$ removed from the sample. Note that
		\[ Z_{k:n} = \begin{cases} 
			Z_{k:n-2} & k < k_1  \\
			Z_{k-1:n-2} & k_1 < k < k_2
		\end{cases}\numberthis\label{eq:zkn_zknminustwo}\mdot
		\]	
		Thus we have
		\begin{align*}
			&\E\left[B_n(s)B_n(t)|Z_i=s, Z_j=t\right]\\
			&= \E\left[\sum_{k_1=1}^{n} \sum_{k_2=1}^{n} \I{Z_{k_1-1:n-2} < s \leq Z_{k_1:n-2}} \I{Z_{k_2-2:n-2} < t \leq Z_{k_2-1:n-2}}\right.\\ 
			&\qquad\qquad \times \left(1+\frac{1-q(s)}{n-k_1}\right)\prod_{k=1}^{k_1-1}\left(1+\frac{1-q(Z_{k:n-2})}{n-k}\right)^{2s_k^{n-2}} \\
			&\qquad\qquad \times \left. \prod_{k=k_1+1}^{k_2-1}\left(1+\frac{1-q(Z_{k-1:n-2})}{n-k}\right)^{t_{k-1}^{n-2}} | Z_i=s, Z_j=t\right]\\
			&= \E\left[\sum_{k_1=1}^{n} \sum_{k_2=1}^{n} \I{Z_{k_1-1:n-2} < s \leq Z_{k_1:n-2}} \I{Z_{k_2-2:n-2} < t \leq Z_{k_2-1:n-2}}\right.\\ 
			&\qquad\qquad \times \left(1+\frac{1-q(s)}{n-k_1}\right)\prod_{k=1}^{k_1-1}\left(1+\frac{1-q(Z_{k:n-2})}{n-k}\right)^{2s_k^{n-2}} \\
			&\qquad\qquad \times \left. \prod_{k=k_1}^{k_2-2}\left(1+\frac{1-q(Z_{k:n-2})}{n-k-1}\right)^{t_{k}^{n-2}}\right]\\
			&= \E\left[\sum_{k_1=1}^{n} \I{Z_{k_1-1:n-2} < s \leq Z_{k_1:n-2}} \left(1+\frac{1-q(s)}{n-k_1}\right) \right.\\ 
			&\qquad\qquad \times \prod_{k=1}^{n-2}\left(1+\frac{1-q(Z_{k:n-2})}{n-k}\right)^{2s_k^{n-2}} \\
			&\qquad\qquad \times \left. \prod_{k=k_1}^{n-2}\left(1+\frac{1-q(Z_{k:n-2})}{n-k-1}\right)^{t_{k}^{n-2}}\right]
		\end{align*}		
		which is independent of $i,j$. Next consider the case $t<s$. Define $\tilde{t}_k^n := \I{Z_{k:n} < t}$ and $\tilde{s}_k^n := \I{t\leq Z_{k:n} < s}$. Using similar arguments we can show that in this case 
		\begin{align*}
			&\E\left[B_n(s)B_n(t)|Z_i=s, Z_j=t\right]\\
			&= \E\left[\sum_{k_1=1}^{n} \I{Z_{k_1-1:n-2} < t \leq Z_{k_1:n-2}} \left(1+\frac{1-q(t)}{n-k_1}\right) \right.\\ 
			&\qquad\qquad \times \prod_{k=1}^{n-2}\left(1+\frac{1-q(Z_{k:n-2})}{n-k}\right)^{2\tilde{t}_k^{n-2}} \\
			&\qquad\qquad \times \left. \prod_{k=k_1}^{n-2}\left(1+\frac{1-q(Z_{k:n-2})}{n-k-1}\right)^{\tilde{s}_{k}^{n-2}}\right]
		\end{align*}	
		which is independent of $i,j$ as well. Thus we have on $\{s\neq t\}$ that $\E\left[B_n(s)B_n(t)|Z_i=s, Z_j=t\right]$ is independent of $i,j$ and hence
		$$\E\left[B_n(s)B_n(t)|Z_i=s, Z_j=t\right] = \E\left[B_n(s)B_n(t)|Z_1=s, Z_2=t\right]\mdot$$
	\end{proof}
\end{lemma}
%
\begin{lemma} \label{lem:zizone}
	Let $\tilde{\phi}: \R^2_+ \longrightarrow \R_+$ be a Borel-measurable function. Then we have for any $n\geq 2$ 
	\begin{align*}
	&\E[\tilde{\phi}(Z_{i},Z_{j}) B_n(Z_{i}) B_n(Z_{j})]\\
	& = \E[\tilde{\phi}(Z_1,Z_2) B_n(Z_1) B_n(Z_2)]\mdot
	\end{align*}
	%
	\begin{proof}
		Consider that $\{Z_i=Z_j\}$ is a measure zero set, since $H$ is continuous. Therefore the following holds for $1\leq i,j \leq n$ 
		\begin{align*}
		& \E\left[\tilde{\phi}(Z_{i},Z_{j}) \E\left[B_n(Z_{i}) B_n(Z_{j})| Z_i,Z_j\right]\right]\\
		&= \E\left[\I{Z_i<Z_j}\tilde{\phi}(Z_{i},Z_{j}) \E\left[B_n(Z_{i}) B_n(Z_{j})| Z_i,Z_j\right]\right]\\
		&\qquad + \E\left[\I{Z_i>Z_j}\tilde{\phi}(Z_{i},Z_{j}) \E\left[B_n(Z_{i}) B_n(Z_{j})| Z_i,Z_j\right]\right]\\
		&= \int_{0}^{\infty}\int_{0}^{\infty}\I{s<t}\tilde{\phi}(s,t) \E\left[B_n(s) B_n(t)| Z_i=s,Z_j=t\right]H(ds)H(dt)\\
		&\qquad + \int_{0}^{\infty}\int_{0}^{\infty}\I{s>t}\tilde{\phi}(s,t) \E\left[B_n(s) B_n(t)| Z_i=s,Z_j=t\right]H(ds)H(dt)\\
		&=: \int_{0}^{\infty}\int_{0}^{\infty}\tilde{\phi}(s,t) I_1(s,t) H(ds)H(dt) + \int_{0}^{\infty}\int_{0}^{\infty}\tilde{\phi}(s,t) I_2(s,t)H(ds)H(dt)\mdot \numberthis\label{eq:ioneitwo}
		\end{align*}
		%
		Now let's consider $I_1$ above. Using Lemma \ref{lem:bnbn_change_order} we obtain for $1\leq i,j\leq n$
		\begin{align*}
			I_1(s,t) &= \I{s<t} \E[B_n(s)B_n(t)|Z_i=s, Z_j=t]\\
			&= \I{s<t} \E[B_n(s)B_n(t)|Z_1=s, Z_2=t]
		\end{align*}
		and 
		\begin{align*}
			I_2(s,t) &= \I{s>t} \E[B_n(s)B_n(t)|Z_i=s, Z_j=t]\\
			&= \I{s>t} \E[B_n(s)B_n(t)|Z_1=s, Z_2=t]\mdot
		\end{align*}
		%
		Therefore we obtain, according to \eqref{eq:ioneitwo}, that 
		\begin{align*}
			\E\left[\tilde{\phi}(Z_{i},Z_{j}) B_n(Z_{i}) B_n(Z_{j})\right] &= \E\left[\tilde{\phi}(Z_{i},Z_{j}) \E\left[B_n(Z_{i}) B_n(Z_{j})| Z_i,Z_j\right]\right]\\
			&= \E\left[\tilde{\phi}(Z_{1},Z_{2}) B_n(Z_{1}) B_n(Z_{2})\right]\mdot
		\end{align*}
	\end{proof}
\end{lemma}
%
\begin{lemma} \label{lem:representation_bn}
	Let $\tilde{\phi}: \R^2_+ \longrightarrow \R_+$ be a Borel-measurable function. Then we have for any $s<t$ and $n\geq 2$ 
	\begin{align*}
		&\E[\tilde{\phi}(Z_1,Z_2) B_n(Z_1) B_n(Z_2)]\\
		& = \E[2\tilde{\phi}(Z_1,Z_2) \{\Delta_{n-2}(Z_1,Z_2) + \bar{\Delta}_{n-2}(Z_1,Z_2)\}\I{Z_1<Z_2}]\mdot
	\end{align*}
	%
	\begin{proof}
		Consider the following
		\begin{align*}
			B_n(Z_1)B_n(Z_2) &= \prod_{k=1}^{n}\left[1+\frac{1-q(Z_{k})}{n-R_{k,n}}\right]^{\I{Z_{k} < Z_1}+\I{Z_{k} < Z_2}}\\
			&= \left[1+\frac{1-q(Z_{1})}{n-R_{1,n}}\right]^{\I{Z_{1} < Z_2}} \left[1+\frac{1-q(Z_{2})}{n-R_{2,n}}\right]^{\I{Z_{2} < Z_1}}\\
			&\qquad \times \prod_{k=3}^{n}\left[1+\frac{1-q(Z_{k})}{n-R_{k,n}}\right]^{\I{Z_{k} < Z_1}+\I{Z_{k} < Z_2}}\\
			&= \I{Z_1<Z_2}\left[1+\frac{1-q(Z_{1})}{n-R_{1,n}}\right] \\
			&\qquad\qquad \times \prod_{k=1}^{n-2}\left[1+\frac{1-q(Z_{k+2})}{n-R_{k+2,n}}\right]^{\I{Z_{k+2} < Z_1}+\I{Z_{k+2} < Z_2}}\\
			&\quad + \I{Z_1>Z_2}\left[1+\frac{1-q(Z_{2})}{n-R_{2,n}}\right] \\
			&\qquad\qquad \times \prod_{k=1}^{n-2}\left[1+\frac{1-q(Z_{k+2})}{n-R_{k+2,n}}\right]^{\I{Z_{k+2} < Z_1}+\I{Z_{k+2} < Z_2}}\\
			&\quad + \I{Z_1=Z_2}\prod_{k=1}^{n-2}\left[1+\frac{1-q(Z_{k+2})}{n-R_{k+2,n}}\right]^{2\I{Z_{k+2} < Z_1}}\mdot \numberthis\label{eq:bnbn1}
		\end{align*}
		%
		On $\{Z_1<Z_2\}$ we have 
		\begin{align*}
			\prod_{k=1}^{n-2}\left[1+\frac{1-q(Z_{k+2})}{n-R_{k+2,n}}\right]^{\I{Z_{k+2} < Z_2}} &= \prod_{k=1}^{n-2}\left[1+\frac{1-q(Z_{k+2})}{n-\tilde{R}_{k,n-2}}\right]^{\I{Z_{k+2} < Z_1}}\\
			&\quad \times  \prod_{k=1}^{n-2}\left[1+\frac{1-q(Z_{k+2})}{n-\tilde{R}_{k,n-2}-1}\right]^{\I{Z_1 < Z_{k+2} < Z_2}}
		\end{align*}
		where $\tilde{R}_{k,n-2}$ denotes the rank of the $Z_k$, $k=3,\dots, n$ among themselves. The above holds since 
		\[ R_{k+2,n} = \begin{cases} 
			\tilde{R}_{k,n-2} & \textrm{ if } Z_{k+2} < Z_1 \\
			\tilde{R}_{k, n-2} + 1 & \textrm{ if } Z_1 < Z_{k+2} < Z_2 
		\end{cases}
		\]
		for $k=1,\dots,n-2$. 
		% 
		Therefore \eqref{eq:bnbn1} yields
		\begin{align*}
			B_n(Z_1)B_n(Z_2) &= \I{Z_1<Z_2}\left[1+\frac{1-q(Z_{1})}{n-R_{1,n}}\right] \\
			&\qquad\qquad \times \prod_{k=1}^{n-2}\left[1+\frac{1-q(Z_{k+2})}{n-\tilde{R}_{k,n-2}}\right]^{2\I{Z_{k+2} < Z_1}}\\
			&\qquad\qquad \times \prod_{k=1}^{n-2}\left[1+\frac{1-q(Z_{k+2})}{n-\tilde{R}_{k,n-2}-1}\right]^{\I{Z_1 < Z_{k+2} < Z_2}}\\
			&\quad + \I{Z_2<Z_1}\left[1+\frac{1-q(Z_{2})}{n-R_{2,n}}\right] \\
			&\qquad\qquad \times \prod_{k=1}^{n-2}\left[1+\frac{1-q(Z_{k+2})}{n-\tilde{R}_{k,n-2}}\right]^{2\I{Z_{k+2} < Z_2}}\\
			&\qquad\qquad \times \prod_{k=1}^{n-2}\left[1+\frac{1-q(Z_{k+2})}{n-\tilde{R}_{k,n-2}-1}\right]^{\I{Z_2 < Z_{k+2} < Z_1}}\\
			&\quad + \I{Z_1=Z_2}\prod_{k=1}^{n-2}\left[1+\frac{1-q(Z_{k+2})}{n-\tilde{R}_{k,n-2}}\right]^{2\I{Z_{k+2} < Z_1}}\mdot \numberthis\label{eq:bnbn_rank}
		\end{align*}
		%
		Now let's denote $Z_{k:n-2}$ the ordered $Z$-values among $Z_3,\dots, Z_n$ for $k=1,\dots,n-2$. Consider that we can write 
		\begin{align*}
			\left[1+\frac{1-q(Z_{1})}{n-R_{1,n}}\right] = \sum_{i=1}^{n-1}\left[1+\frac{1-q(s)}{n-i}\right]\I{Z_{i-1:n-2} < Z_1 \leq Z_{i:n-2}}\mdot
		\end{align*}
		%
		Note that $Z_{k:n-2}$ is independent of $Z_1$ and $Z_2$ for $k=1,\dots,n-2$. Therefore we obtain the following, by conditioning \eqref{eq:bnbn_rank} on $Z_1,Z_2$:
		\begin{align*}
			&\E[B_n(Z_1)B_n(Z_2)|Z_1 = s, Z_2 = t]\\
			&= \I{s<t}\E\left[\left(\sum_{i=1}^{n-1}\left[1+\frac{1-q(s)}{n-i}\right]\I{Z_{i-1:n-2} < s \leq Z_{i:n-2}}\right)\right.\\
			&\qquad\qquad\qquad \times \prod_{k=1}^{n-2}\left[1+\frac{1-q(Z_{k:n-2})}{n-k}\right]^{2\I{Z_{k:n-2} < s}}\\
			&\qquad\qquad\qquad \times \left. \prod_{k=1}^{n-2}\left[1+\frac{1-q(Z_{k:n-2})}{n-k-1}\right]^{\I{s < Z_{k:n-2} < t}} \right]\\
			&\quad + \I{t<s}\E\left[\left(\sum_{i=1}^{n-1}\left[1+\frac{1-q(t)}{n-i}\right]\I{Z_{i-1:n-2} < t \leq Z_{i:n-2}}\right)\right. \\
			&\qquad\qquad\qquad \times \prod_{k=1}^{n-2}\left[1+\frac{1-q(Z_{k:n-2})}{n-k}\right]^{2\I{Z_{k:n-2} < t}}\\
			&\qquad\qquad\qquad \times \left. \prod_{k=1}^{n-2}\left[1+\frac{1-q(Z_{k:n-2})}{n-k-1}\right]^{\I{t < Z_{k:n-2} < s}}\right]\\
			&\quad + \I{s=t}\E\left[\prod_{k=1}^{n-2}\left[1+\frac{1-q(Z_{k:n-2})}{n-k}\right]^{2\I{Z_{k:n-2} < s}} \right]\\
			&= \alpha(s,t) + \alpha(t,s) + \beta(s,t)
		\end{align*}
		where 
		\begin{align*}
			\alpha(s,t) &:=\I{s<t}\E\left[\left(\sum_{i=1}^{n-1}\left[1+\frac{1-q(s)}{n-i}\right]\I{Z_{i-1:n-2} < s \leq Z_{i:n-2}}\right)\right.\\
			&\qquad\qquad\qquad \times \prod_{k=1}^{n-2}\left[1+\frac{1-q(Z_{k:n-2})}{n-k}\right]^{2\I{Z_{k:n-2} < s}}\\
			&\qquad\qquad\qquad \times \left. \prod_{k=1}^{n-2}\left[1+\frac{1-q(Z_{k:n-2})}{n-k-1}\right]^{\I{s < Z_{k:n-2} < t}} \right]
		\end{align*}
		and 
		\begin{align*}
			\beta(s,t) &:=\I{s=t}\E\left[\prod_{k=1}^{n-2}\left[1+\frac{1-q(Z_{k:n-2})}{n-k}\right]^{2\I{Z_{k:n-2} < s}} \right]\mdot
		\end{align*}		
		%
		Consider that we have
		$$\E[\alpha(Z_1,Z_2)] = \E[\alpha(Z_2,Z_1)]$$
		under (A\ref{ass:kernel_gen}), because $Z_1$ and $Z_2$ are \iid\ and 
		$$\E[\beta(Z_1,Z_2)] = 0$$
		since $H$ is continuous. Therefore we get 
		\begin{align*}
			&\E[\tilde{\phi}(Z_1,Z_2)B_n(Z_1)B_n(Z_2)]\\
			&= \E[\tilde{\phi}(Z_1,Z_2)(\alpha(Z_1,Z_2) + \alpha(Z_2,Z_1) + \beta(Z_1,Z_2))]\\
			&= \E[2\tilde{\phi}(Z_1,Z_2)\alpha(Z_1,Z_2)]\mdot \numberthis\label{eq:expectationalpha}
		\end{align*}
		%
		Next consider that 
		\begin{align*}
			\alpha(s,t) &=\I{s<t}\E\left[(1+C_n(s)) D_{n-2}(s,t) \right]\\
			&= \I{s<t}(\Delta_{n-2}(s,t) + \bar{\Delta}_{n-2}(s,t))\mdot
		\end{align*}
		The latter equality holds, since
		\begin{align*}
			&\sum_{i=1}^{n-1}\left[1+\frac{1-q(s)}{n-i}\right]\I{Z_{i-1:n-2} < s \leq Z_{i:n-2}}\\
			&=\sum_{i=1}^{n-1}\I{Z_{i-1:n-2} < s \leq Z_{i:n-2}} + \sum_{i=1}^{n-1}\left[\frac{1-q(s)}{n-i}\right]\I{Z_{i-1:n-2} < s \leq Z_{i:n-2}}\\
			&= 1 + C_n(s)\mdot
		\end{align*}
		Now the statement of the lemma follows directly from \eqref{eq:expectationalpha}.
	\end{proof}
\end{lemma}
%
The next lemma identifies the almost sure limit of $D_n$ for $n\to\infty$. Define for $s<t$
$$D(s,t) := \exp\left(2\int_{0}^{s} \frac{1-q(z)}{1-H(z)} H(dz) + \int_{s}^{t} \frac{1-q(z)}{1-H(z)} H(dz)\right)$$
\begin{lemma} \label{lem:dn_limit}
	For any $s < t$ \st\ $H(t)<1$, we have
	$$\lim\limits_{n\to\infty}D_n(s,t) = D(s,t)\mdot$$
	%	
	\begin{proof}
		First recall the following definition
		\begin{align*}
			D_n(s,t) &:= \prod_{k=1}^{n} \left[1+\frac{1-q(Z_k)}{n-R_{k,n} +2}\right]^{2\I{Z_k<s}} \prod_{k=1}^{n}\left[1+\frac{1-q(Z_k)}{n-R_ {k,n}+1}\right]^{\I{s < Z_k < t}} \mdot
		\end{align*}
		%
		Next let 
		\begin{align*}
		x_k &:= \frac{1-q(Z_k)}{n(1- H_n(Z_k) + 2/n)}\\
		y_k &:= \frac{1-q(Z_k)}{n(1- H_n(Z_k) + 1/n)}\\
		s_k &:= \I{Z_k < s} \\
		t_k &:= \I{s < Z_k < t}
		\end{align*}
		for $s<t$ and $k=1,\dots,n$.
		%
		Now consider 
		\begin{align*}
			D_n(s,t) &= \prod_{k=1}^{n} \left[1+\frac{1-q(Z_k)}{n(1- H_n(Z_k) + 2/n)}\I{Z_k<s}\right]^{2}\\ 
			&\qquad \times \prod_{k=1}^{n}\left[1+\frac{1-q(Z_k)}{n(1-H_n(Z_k)+1/n)}\I{s < Z_k < t}\right]\\
			&= \prod_{k=1}^{n} \left[1+x_k s_k\right]^{2} \prod_{k=1}^{n}\left[1+y_k t_k\right]\\
			&= \exp\left(2\sum_{k=1}^{n}\ln\left[1+x_k s_k\right] + \sum_{k=1}^{n}\ln\left[1+y_k t_k\right]\right)\mdot
		\end{align*}
		%
		Note that $0 \leq x_k s_k \leq 1$ and $0 \leq y_k t_k \leq 1$. Consider that the following inequality holds  
		$$-\frac{x^2}{2} \leq \ln(1+x) - x \leq 0$$ 
		for any $x \geq 0$ (cf.  \cite{stute1993strong}, p. 1603). This implies 
		$$-\frac{1}{2}\sum_{k=1}^{n}x_k^2 s_k \leq \sum_{k=1}^{n}\ln(1+x_k s_k) - \sum_{k=1}^{n}x_k s_k \leq 0\mdot$$ 
		But now 
		\begin{align*}
			\sum_{k=1}^{n} x_k^2 s_k &= \frac{1}{n^2} \sum_{k=1}^{n} \left(\frac{1-q(Z_k)}{1-H_n(Z_k)+\frac{2}{n}}\right)^2\I{Z_k<s}\\
			&\leq \frac{1}{n^2} \sum_{k=1}^{n} \left(\frac{1}{1-H_n(s)+\frac{1}{n}}\right)^2\\
			&= \frac{1}{n(1-H_n(s)+n^{-1})^2} \longrightarrow 0
		\end{align*}
		almost surely as $n\to\infty$, since $H(s)<H(t)<1$ (\cf\ \cite{stute1993strong}, p. 1603). Therefore we have
		$$\abs{\sum_{k=1}^{n}\ln(1+x_k s_k) - \sum_{k=1}^{n}x_k s_k} \longrightarrow 0$$
		with probability 1 as $n\to\infty$. 
		%
		Similarly we obtain
		$$\abs{\sum_{k=1}^{n}\ln(1+y_k t_k) - \sum_{k=1}^{n}y_k t_k} \longrightarrow 0$$
		with probability 1 as $n\to\infty$. Hence 
		$$\lim\limits_{n\to\infty} D_n(s) = \lim\limits_{n\to\infty} \exp\left(2\sum_{k=1}^{n} x_k s_k + \sum_{k=1}^{n}y_k t_k\right)\mdot$$
		%
		Now consider 
		\begin{align*}
			\sum_{k=1}^{n} x_k s_k &= \frac{1}{n}\sum_{k=1}^{n} \frac{1-q(Z_k)}{1-H_n(Z_k)+\frac{2}{n}}\I{Z_k<s}\\
			&= \int_{0}^{s-} \frac{1-q(z)}{1-H_n(z)+\frac{2}{n}} H_n(dz)\\
			&= \int_{0}^{s-} \frac{1-q(z)}{1-H(z)} H_n(dz) + \int_{0}^{s-} \frac{1-q(z)}{1-H_n(z)+\frac{2}{n}} - \frac{1-q(z)}{1-H(z)} H_n(dz)\\
			&= \int_{0}^{s-} \frac{1-q(z)}{1-H(z)} H_n(dz) + \int_{0}^{s-} \frac{(1-q(z))(H_n(z)-H(z)-\frac{2}{n})}{(1-H_n(z)+\frac{2}{n})(1-H(z))} H_n(dz)\mdot \numberthis \label{eq:xksk_int}
		\end{align*}
		%
		Note that the second term on the right hand side of the latter equation above tends to zero for  $n\to\infty$, because
		\begin{align*}
			& \int_{0}^{s-} \frac{(1-q(z))(H_n(z)-H(z)-\frac{2}{n})}{(1-H_n(z)+\frac{2}{n})(1-H(z))} H_n(dz)\\
			&\leq \frac{\sup_{z}|H_n(z)- H(z) -\frac{2}{n}|}{1-H(s)} \int_{0}^{s-}\frac{1}{1-H_n(z)} H_n(dz) \longrightarrow 0
		\end{align*}
		%
		almost surely as $n\to\infty$, by the Glivenko-Cantelli Theorem and since $H(s)<1$. Moreover we have
		\begin{align*}
			\int_{0}^{s-} \frac{1-q(z)}{1-H(z)} H_n(dz) \longrightarrow \int_{0}^{s} \frac{1-q(z)}{1-H(z)} H(dz)
		\end{align*}		
		by the SLLN. Therefore we obtain 
		$$\lim\limits_{n\to\infty} \sum_{k=1}^{n} x_k s_k = \int_{0}^{s} \frac{1-q(z)}{1-H(z)} H(dz)\mdot$$
		By the same arguments, we can show that 
		$$\lim\limits_{n\to\infty} \sum_{k=1}^{n} y_k t_k = \int_{s}^{t} \frac{1-q(z)}{1-H(z)} H(dz)\mdot$$
		Thus we finally conclude
		$$\lim\limits_{n\to\infty} D_n(s,t) = \exp\left(2\int_{0}^{s} \frac{1-q(z)}{1-H(z)} H(dz) + \int_{s}^{t} \frac{1-q(z)}{1-H(z)} H(dz)\right)$$
		almost surely.
	\end{proof}
\end{lemma}
%
\begin{lemma} \label{lem:dn_supermart}
	$\{D_n, \F_n\}_{n\geq 1}$ is a non-negative reverse supermartingale.
	%
	\begin{proof}
		Consider that for $s<t$ and $n\geq 1$
		\begin{align*}
			\E[D_n(s,t)| \F_{n+1}] &= \E\left[\prod_{k=1}^{n}\left(1+\frac{1-q(Z_{k:n})}{n-k+2}\right)^{2\I{Z_{k:n} <s}}\right.\\
			&\qquad \left. \times \prod_{k=1}^{n}\left(1+\frac{1-q(Z_{k:n})}{n-k+1}\right)^\I{s < Z_{k:n} < t} | \F_{n+1}\right]\\
			&= \sum_{i=1}^{n+1}\E\left[\I{Z_{n+1} = Z_{i:n+1}} \prod_{k=1}^{n}\dots | \F_{n+1}\right]\\
			&= \sum_{i=1}^{n+1}\E\left[\I{Z_{n+1} = Z_{i:n+1}} \prod_{k=1}^{i-1}\left(1+\frac{1-q(Z_{k:n+1})}{n-k+2}\right)^{2\I{Z_{k:n+1} <s}} \right.\\
			&\qquad\qquad \times \prod_{k=i}^{n}\left(1+\frac{1-q(Z_{k+1:n+1})}{n-k+2}\right)^{2\I{Z_{k+1:n+1} <s}}\\
			&\qquad\qquad \times \prod_{k=1}^{i-1}\left(1+\frac{1-q(Z_{k:n+1})}{n-k+1}\right)^\I{s < Z_{k:n+1} < t}\\
			&\qquad\qquad \left. \times \prod_{k=i}^{n}\left(1+\frac{1-q(Z_{k+1:n+1})}{n-k+1}\right)^\I{s < Z_{k+1:n+1} < t}| \F_{n+1}\right]\\
			&= \sum_{i=1}^{n+1}\E\left[\I{Z_{n+1} = Z_{i:n+1}} \prod_{k=1}^{i-1}\left(1+\frac{1-q(Z_{k:n+1})}{n-k+2}\right)^{2\I{Z_{k:n+1} <s}} \right.\\
			&\qquad\qquad \times \prod_{k=i+1}^{n+1}\left(1+\frac{1-q(Z_{k:n+1})}{n-k+3}\right)^{2\I{Z_{k:n+1} <s}}\\
			&\qquad\qquad \times \prod_{k=1}^{i-1}\left(1+\frac{1-q(Z_{k:n+1})}{n-k+1}\right)^\I{s < Z_{k:n+1} < t}\\
			&\qquad\qquad \left. \times \prod_{k=i+1}^{n+1}\left(1+\frac{1-q(Z_{k:n+1})}{n-k+2}\right)^\I{s < Z_{k:n+1} < t}| \F_{n+1}\right]\mdot
		\end{align*}
		%
		Now each product within the conditional expectation is measurable \wrt\ $\F_{n+1}$. Moreover we have for $i=1,\dots,n$ 
		\begin{align*}
			\E[\I{Z_{n+1}=Z_{i:n+1}}|\F_n+1] &= \P(Z_{n+1}=Z_{i:n+1}|\F_{n+1})\\
			&= \P(R_{n+1,n+1} = i)\\
			&= \frac{1}{n+1}\mdot
		\end{align*}
		%
		Thus we obtain
		\begin{align*}
			\E[D_n(s,t)| \F_{n+1}] &= \frac{1}{n+1} \sum_{i=1}^{n+1} \prod_{k=1}^{i-1}\left(1+\frac{1-q(Z_{k:n+1})}{n-k+2}\right)^{2\I{Z_{k:n+1} <s}}\\
			&\qquad\qquad\qquad \times \left(1+\frac{1-q(Z_{k:n+1})}{n-k+1}\right)^\I{s < Z_{k:n+1} < t}\\
			&\qquad\qquad \times \prod_{k=i+1}^{n+1}\left(1+\frac{1-q(Z_{k:n+1})}{n-k+3}\right)^{2\I{Z_{k:n+1} <s}}\\ &\qquad\qquad\qquad \times \left(1+\frac{1-q(Z_{k:n+1})}{n-k+2}\right)^\I{s < Z_{k:n+1} < t}\mdot \numberthis \label{eq:cond_exp_dnp1}
		\end{align*}
		%
		We will now proceed by induction on $n$. First let 
		$$x_k := 1-q(Z_{k:2}) \textrm{, } s_k := \I{Z_{k:2} < s} \textrm{ and } t_k := \I{s < Z_{k:2} < t}$$
		for $k=1,2$. Note that that $x_k$ and $y_k$ are different, compared to the corresponding definitions in lemma \ref{lem:dn_limit}, as they involves the ordered $Z$-values here. 
		%
		Next consider
		\begin{align*}
			\E[D_1(s,t) | \F_2] &= \frac{1}{2}\left[\left(1+\frac{1-q(Z_{2:2})}{2}\right)^{2\I{Z_{2:2}<s}} \times \left(1+(1-q(Z_{2:2}))\right)^{\I{s < Z_{2:2} < t}} \right.\\
			&\qquad \left. + \left(1+\frac{1-q(Z_{1:2})}{2}\right)^{2\I{Z_{1:2}<s}} \times \left(1+(1-q(Z_{1:2}))\right)^{\I{s < Z_{1:2} < t}}\right]\\
			&= \frac{1}{2}\left[\left(1+\frac{x_2}{2}s_2\right)^{2} \times \left(1+x_2t_2\right) + \left(1+\frac{x_1}{2}s_1\right)^{2} \times \left(1+x_1t_1\right)\right]\mdot
		\end{align*}
		%
		Moreover we have
		\begin{align*}
			D_2(s,t) &= \prod_{k=1}^{2} \left[1+\frac{1-q(Z_{k:2})}{4-k}\right]^{2\I{Z_{k:2}<s}} \prod_{k=1}^{2}\left[1+\frac{1-q(Z_{k:2})}{3-k}\right]^{\I{s < Z_{k:2} < t}}\\
			&= \left[1 + \frac{x_1}{3}s_1\right]^2 \times \left[1+\frac{x_1}{2}t_1\right] \times \left[1+\frac{x_2}{2}s_2\right]^2 \times \left[1+x_2t_2\right]\\
			&= \left[1 + \frac{x_1}{2}t_1 + \left(\frac{x_1^2}{9} + \frac{2}{3}x_1\right)s_1\right] \times \left[1 + x_2t_2 + \left(\frac{x_2^2}{4} + x_2\right)s_2\right]\mdot
		\end{align*}		
		%
		Therefore we obtain 
		\begin{align*}
			\E[D_1(s,t) | \F_2] - D_2(s,t) \leq \frac{x_1^2}{72} - \frac{x_1}{6} \leq 0\mdot 
		\end{align*}
		since $0 \leq x_1 \leq 1$. Thus $\E[D_1(s,t) | \F_2] \leq D_2(s,t)$ for any $s<t$, as needed. Now assume that 
		$$\E[D_n(s,t) | \F_{n+1}] \leq D_{n+1}(s,t)$$
		holds for any $n\geq 1$. 
		%
		Note that the latter is equivalent to assuming
		\begin{align*}
			& \frac{1}{n+1} \sum_{i=1}^{n+1} \prod_{k=1}^{i-1}\left(1+\frac{1-q(y_k)}{n-k+2}\right)^{2\I{y_k <s}}  \left(1+\frac{1-q(y_k)}{n-k+1}\right)^\I{s < y_k < t}\\
			&\qquad\qquad \times \prod_{k=i+1}^{n+1}\left(1+\frac{1-q(y_k)}{n-k+3}\right)^{2\I{y_k <s}} \left(1+\frac{1-q(y_k)}{n-k+2}\right)^\I{s < y_k < t}\\
			&\leq \prod_{k=1}^{n+1}\left(1+\frac{1-q(y_k)}{n-k+3}\right)^{2\I{y_k <s}} \prod_{k=1}^{n+1}\left(1+\frac{1-q(y_k)}{n-k+2}\right)^\I{s < y_k < t} \numberthis \label{eq:supermart_yk}
		\end{align*}
		holds for arbitrary $y_k \geq 0$. Next define for $s<t$ and $n\geq 1$
		$$A_{n+2}(s,t) := \prod_{k=2}^{n+2} \left[1+\frac{1-q(Z_{k:n+2})}{n-k+4}\right]^{2\I{Z_{k:n+2} < s}} \times \left[1+\frac{1-q(Z_{k:n+2})}{n-k+3}\right]^\I{s < Z_{k:n+2} < t}\mdot $$
		%
		Now consider that we get from \eqref{eq:cond_exp_dnp1}
		\begin{align*}
			&\E[D_{n+1}(s,t)| \F_{n+2}]	\\
			&= \frac{1}{n+2} \sum_{i=1}^{n+2} \prod_{k=1}^{i-1}\left(1+\frac{1-q(Z_{k:n+2})}{n-k+3}\right)^{2\I{Z_{k:n+2} <s}}  \left(1+\frac{1-q(Z_{k:n+2})}{n-k+2}\right)^\I{s < Z_{k:n+2} < t}\\
			&\qquad\qquad\quad \times \prod_{k=i+1}^{n+2}\left(1+\frac{1-q(Z_{k:n+2})}{n-k+4}\right)^{2\I{Z_{k:n+2} <s}} \left(1+\frac{1-q(Z_{k:n+2})}{n-k+3}\right)^\I{s < Z_{k:n+2} < t}\\
			&= \frac{A_{n+2}}{n+2} + \frac{1}{n+2}\sum_{i=2}^{n+2} \prod_{k=1}^{i-1} \dots \times \prod_{k=i+1}^{n+2}\dots \\
			&= \frac{A_{n+2}}{n+2} + \frac{1}{n+2}\sum_{i=1}^{n+1} \prod_{k=1}^{i} \dots \times \prod_{k=i+2}^{n+2}\dots\\
			&= \frac{A_{n+2}}{n+2} + \frac{1}{n+2}\left(1+\frac{1-q(Z_{1:n+2})}{n+2}\right)^{2\I{Z_{1:n+2} < s}} \left(1+\frac{1-q(Z_{1:n+2})}{n+1}\right)^\I{s < Z_{1:n+2} < t}\\
			&\qquad\qquad\qquad\times \sum_{i=1}^{n+1} \prod_{k=1}^{i-1} \left(1+\frac{1-q(Z_{k+1:n+2})}{n-k+2}\right)^{2\I{Z_{k+1:n+2} <s}}\\
			&\qquad\qquad\qquad\qquad\qquad \times \left(1+\frac{1-q(Z_{k+1:n+2})}{n-k+1}\right)^\I{s < Z_{k+1:n+2} < t}\\
			&\qquad\qquad\qquad\qquad \times \prod_{k=i+1}^{n+1}\left(1+\frac{1-q(Z_{k+1:n+2})}{n-k+3}\right)^{2\I{Z_{k+1:n+2} <s}}\\
			&\qquad\qquad\qquad\qquad\qquad \times \left(1+\frac{1-q(Z_{k+1:n+2})}{n-k+2}\right)^\I{s < Z_{k+1:n+2} < t} \mdot 
		\end{align*}
		%
		Using \eqref{eq:supermart_yk} on the right hand side of the equation above yields
		\begin{align*}
			&\E[D_{n+1}(s,t)| \F_{n+2}]	\\
			&\leq \frac{A_{n+2}}{n+2} + \frac{n+1}{n+2}\left(1+\frac{1-q(Z_{1:n+2})}{n+2}\right)^{2\I{Z_{1:n+2} < s}} \left(1+\frac{1-q(Z_{1:n+2})}{n+1}\right)^\I{s < Z_{1:n+2} < t}\\
			&\qquad\qquad\qquad\times \prod_{k=1}^{n+1} \left(1+\frac{1-q(Z_{k+1:n+2})}{n-k+3}\right)^{2\I{Z_{k+1:n+2} <s}}\\
			&\qquad\qquad\qquad\qquad\qquad \times \left(1+\frac{1-q(Z_{k+1:n+2})}{n-k+2}\right)^\I{s < Z_{k+1:n+2} < t}\\
			&= A_{n+2} \left[\frac{1}{n+2} + \frac{n+1}{n+2}\left(1+\frac{1-q(Z_{1:n+2})}{n+2}\right)^{2\I{Z_{1:n+2} < s}} \right. \\
			&\qquad\qquad\qquad\qquad\qquad \left. \times \left(1+\frac{1-q(Z_{1:n+2})}{n+1}\right)^\I{s < Z_{1:n+2} < t}\right] \mdot 
		\end{align*}
		%
		For the moment, let
		$$x_1 := 1-q(Z_{1:n+2}) \textrm{, } s_1 := \I{Z_{1:n+2} < s} \textrm{ and } t_1 := \I{s < Z_{1:n+2} < t} $$
		%
		Now we can rewrite the above as
		\begin{align*}
			\E[D_{n+1}(s,t)| \F_{n+2}]	&\leq A_{n+2} \left[\frac{1}{n+2} + \frac{n+1}{n+2}\left(1+\frac{x_1s_1}{n+2}\right)^{2} \left(1+\frac{x_1t_1}{n+1}\right)\right]\mdot  \numberthis\label{eq:dn_supermart_an}
		\end{align*}
		%
		Next consider 
		\begin{align*}
			\left(1+\frac{x_1t_1}{n+1}\right) &= \left(1+\frac{x_1t_1}{n+2}-\frac{1}{n+2}\right) \left(1+\frac{1}{n+1}\right)\\
			&=  \left(1+\frac{x_1t_1}{n+2}\right)+\frac{1}{n+1}\left(1+\frac{x_1t_1}{(n+2)}\right) - \frac{1}{n+1}\\
			&= \left(1+\frac{x_1t_1}{n+2}\right)+\frac{x_1t_1}{(n+1)(n+2)}\mdot 
		\end{align*}
		%
		Thus we get
		\begin{align*}
			&\frac{n+1}{n+2}\left(1+\frac{x_1s_1}{n+2}\right)^{2} \left(1+\frac{x_1t_1}{n+1}\right) \\
			&= \frac{n+1}{n+2}\left(1+\frac{x_1s_1}{n+2}\right)^{2}\left(1+\frac{x_1t_1}{n+2}\right) + \left(1+\frac{x_1s_1}{n+2}\right)^{2}\frac{x_1t_1}{(n+2)^2}\mdot 
		\end{align*}
		%
		But now 
		\begin{align*}
			\left(1+\frac{x_1s_1}{n+2}\right)^{2}\frac{x_1t_1}{(n+2)^2} &= \left(1+2\frac{x_1s_1}{n+2}+\frac{x^2_1s_1}{(n+2)^2}\right)\frac{x_1t_1}{(n+2)^2}\\
			&= \frac{x_1t_1}{(n+2)^2}
		\end{align*}
		since $s_1\cdot t_1=0$ for all $s<t$. Hence we can rewrite the term in brackets in \eqref{eq:dn_supermart_an} as 
		\begin{align*}
			&\frac{1}{n+2} + \frac{n+1}{n+2}\left(1+\frac{x_1s_1}{n+2}\right)^{2} \left(1+\frac{x_1t_1}{n+1}\right) \\
			&=\frac{1}{n+2} + \frac{x_1t_1}{(n+2)^2} + \frac{n+1}{n+2}\left(1+\frac{x_1s_1}{n+2}\right)^{2}\left(1+\frac{x_1t_1}{n+2}\right)\\
			&=\frac{1}{n+2}\left(1+\frac{x_1t_1}{n+2}\right) + \frac{n+1}{n+2}\left(1+\frac{x_1s_1}{n+2}\right)^{2}\left(1+\frac{x_1t_1}{n+2}\right)\\
			&=\left[\frac{1}{n+2} + \frac{n+1}{n+2}\left(1+\frac{x_1}{n+2}\right)^{2s_1}\right]\left(1+\frac{x_1}{n+2}\right)^{t_1}\\
			&\leq \left(1+\frac{x_1}{n+3}\right)^{2s_1}\left(1+\frac{x_1}{n+2}\right)^{t_1}\mdot 
		\end{align*}
		The latter inequality above holds, since 
		$$\left[\frac{1}{n+2} + \frac{n+1}{n+2}\left(1+\frac{x}{n+2}\right)^{2}\right] \leq \left(1+\frac{x}{n+3}\right)^{2}$$
		for any $0\leq x\leq 1$. (\cf\ \cite{bose1999strong}, page 197). Therefore we can rewrite \eqref{eq:dn_supermart_an} as
		\begin{align*}
			\E[D_{n+1}(s,t)| \F_{n+2}]	&\leq A_{n+2} \left(1+\frac{1-q(Z_{1:n+2})}{n+3}\right)^{2\I{Z_{1:n+2}<s}} \\
			&\qquad\quad \times \left(1+\frac{1-q(Z_{1:n+2})}{n+2}\right)^{\I{s<Z_{1:n+2} <t}}\\
			&= D_{n+2}(s,t)\mdot
		\end{align*}		
		This concludes the proof.
	\end{proof}
\end{lemma}
%
\begin{lemma}For any $s<t$ \st\ $H(t)<1$ the following statement holds true
	$$\Delta_{n}(s,t) = \E[D_n(s,t)] = \E[D_n(s,t)|\F_\infty] \nearrow D(s,t)\mdot$$
	\label{lem:neveu}
	\begin{proof}
		Consider that we have for $n\geq 2$
		$$\Delta_{n}(s,t) = \E[D_n(s,t)] = \E[D_n(s,t)|\F_\infty] $$
		by definition of $\Delta_{n}(s,t)$ and Lemma \ref{lem:hewitt_savage}. Next note that we have $D_n(s,t) \to D(s,t)$ almost surely, according to Lemma \ref{lem:dn_limit}. Moreover we get from Lemma \ref{lem:dn_supermart}, that $\{D_n,\F_n\}_{n\geq 1}$ is a reverse supermartingale. Now this together with Proposition 5-3-11 of \cite{neveu1975discrete} yields
		$$\E[D_n(s,t)|\F_\infty] \nearrow D(s,t)\mdot$$
		This proves the lemma.
	\end{proof}
\end{lemma}
%
%
%
%
\section{Upper bound for $\alpha_n$} \label{sec:upper_bound}
%
In Lemma \ref{lem:representation_bn} the previous section we derived a representation of $S_n$, which contains a reverse supermartingale $D_n$. Moreover we derived different properties of $D_n$. We will now use the results of the previous section, to show that 
$$\lim\limits_{n\to\infty}\doublesum\limits_{1\leq i<j\leq n+1}\E\left[\phi^2(Z_{i:n}, Z_{j:n}) W^2_{i:n} W^2_{j:n}\right]^{\frac{1}{2}}$$	
is finite. Moreover we will establish an upper bound for $\E[(Q_i^{n+1} - 1)^2]$. Those results will be used in Section \ref{sec:finite_limit} to prove the almost sure existence of the limit of $S_n$ as $n\to\infty$. 
%
\begin{lemma} \label{lem:expectation_sq}
	Suppose condition (A\ref{ass:intgral_phi_q}) is satisfied. Then the following statement holds true
	$$\lim\limits_{n\to\infty}\doublesum\limits_{1\leq i<j\leq n+1}\E\left[\phi^2(Z_{i:n}, Z_{j:n}) W^2_{i:n} W^2_{j:n}\right]^{\frac{1}{2}} < \infty $$	
	%
	\begin{proof} %[\textbf{Proof of lemma \ref{lem:expectation_sq}}]
		Let (A\ref{ass:intgral_phi_q}) be satisfied. Consider the following
		\begin{align*}
			& \doublesum\limits_{1\leq i<j\leq n}\E\left[\phi^2(Z_{i:n}, Z_{j:n}) W^2_{i:n} W^2_{j:n}\right]^{\frac{1}{2}}\\
			&= \doublesum\limits_{1\leq i<j\leq n}\E\left[\phi^2(Z_{i:n}, Z_{j:n}) \frac{q^2(Z_{i:n})}{(n-i+1)^2}\prod_{k=1}^{i-1}\left[1-\frac{q(Z_{k:n})}{n-k+1}\right]^2\right.\\
			&\qquad\qquad\qquad \times \left. \frac{q^2(Z_{j:n})}{(n-j+1)^2}\prod_{l=1}^{j-1}\left[1-\frac{q(Z_{l:n})}{n-l+1}\right]^2\right]^\frac{1}{2}\\
			&\leq \doublesum\limits_{1\leq i<j\leq n}\E\left[\phi^2(Z_{i:n}, Z_{j:n}) \frac{q^2(Z_{i:n})}{(n-i+1)^2}\prod_{k=1}^{i-1}\left[1-\frac{q(Z_{k:n})}{n-k+1}\right]\right.\\
			&\qquad\qquad\qquad \times \left. \frac{q^2(Z_{j:n})}{(n-j+1)^2}\prod_{l=1}^{j-1}\left[1-\frac{q(Z_{l:n})}{n-l+1}\right]\right]^\frac{1}{2}\mdot \numberthis\label{eq:E_na}
		\end{align*}
		%
		Next we will modify the products above. Recall the following definition 
		$$B_n(s) := \prod_{k=1}^{n}\left[1+\frac{1-q(Z_{k})}{n-R_{k,n}}\right]^{\I{Z_{k} < s}}$$
		and note that for $i=1,\dots,n$
		\begin{align*}
			B_n(Z_{i:n}) &= \prod_{k=1}^{n}\left[1+\frac{1-q(Z_{k})}{n-R_{k,n}}\right]^{\I{Z_{k} < Z_{i:n}}}\\
			&= \prod_{k=1}^{n}\left[1+\frac{1-q(Z_{k:n})}{n-k}\right]^{\I{Z_{k:n} < Z_{i:n}}}\\
			&=  \prod_{k=1}^{i-1}\left[1+\frac{1-q(Z_{k:n})}{n-k}\right]\mdot
		\end{align*}
		%
		Moreover consider that for $i=1,\dots,n$
		\begin{align*}
			\frac{1}{n-i+1}\prod_{k=1}^{i-1}\left[1-\frac{q(Z_{k:n})}{n-k+1}\right]
			&=  \frac{1}{n-i+1}\prod_{k=1}^{i-1}\left[\frac{n-k+1-q(Z_{k:n})}{n-k+1}\right] \\
			&=  \frac{1}{n-i+1}\prod_{k=1}^{i-1}\left[\frac{n-k+1-q(Z_{k:n})}{n-k} \cdot \frac{n-k}{n-k+1}\right]\\
			&=  \frac{1}{n}\prod_{k=1}^{i-1}\left[1+\frac{1-q(Z_{k:n})}{n-k}\right]\\
			&=  \frac{B_n(Z_{i:n})}{n}\mdot
		\end{align*}
		%
		Now combining the above with \eqref{eq:E_na} yields 
		\begin{align*}
			&\doublesum\limits_{1\leq i<j\leq n}\E\left[\phi^2(Z_{i:n}, Z_{j:n}) W^2_{i:n} W^2_{j:n}\right]^{\frac{1}{2}}\\
			&\leq \doublesum\limits_{1\leq i<j\leq n}\E\left[\phi^2(Z_{i:n}, Z_{j:n}) \frac{q^2(Z_{i:n})}{n(n-i+1)}\frac{q^2(Z_{j:n})}{n(n-j+1)} B_n(Z_{i:n})B_n(Z_{j:n})\right]^\frac{1}{2}\\
			&\leq \frac{1}{n}\sum_{i=1}^{n}\sum_{j=1}^{n}\frac{1}{(n-i+1)^\frac{1}{2}(n-j+1)^\frac{1}{2}}\\
			&\qquad\qquad\times \E\left[\phi^2(Z_{i:n}, Z_{j:n}) q^2(Z_{i:n})q^2(Z_{j:n}) B_n(Z_{i:n})B_n(Z_{j:n})\right]^\frac{1}{2}\\
			&= \frac{1}{n}\sum_{i=1}^{n}\sum_{j=1}^{n} \frac{1}{(n-R_{i,n}+1)^{\frac{1}{2}} (n-R_{j,n}+1)^{\frac{1}{2}}}\\
			&\qquad\qquad \times \E\left[\phi^2(Z_{i}, Z_{j}) q^2(Z_{i})q^2(Z_{j})B_n(Z_{i})B_n(Z_{j})\right]^{\frac{1}{2}} \mdot
		\end{align*}
		%
		According to Lemma \ref{lem:zizone}, we have
		\begin{align*}
			&\E\left[\phi^2(Z_{i}, Z_{j}) q^2(Z_{i})q^2(Z_{j})B_n(Z_{i})B_n(Z_{j})\right]\\
			&= \E\left[\phi^2(Z_{1}, Z_{2}) q^2(Z_{1})q^2(Z_{2})B_n(Z_{1})B_n(Z_{2})\right]\mdot
		\end{align*}
		%
		Therefore we get
		\begin{align*}
			&\doublesum\limits_{1\leq i<j\leq n}\E\left[\phi^2(Z_{i:n}, Z_{j:n}) W^2_{i:n} W^2_{j:n}\right]^{\frac{1}{2}}\\
			&\leq \frac{1}{n}\sum_{i=1}^{n}\sum_{j=1}^{n} \frac{1}{(n-R_{i,n}+1)^{\frac{1}{2}} (n-R_{j,n}+1)^{\frac{1}{2}}}\\
			&\qquad\qquad \times \E\left[\phi^2(Z_{1}, Z_{2}) q^2(Z_{1})q^2(Z_{2})B_n(Z_{1})B_n(Z_{2})\right]^{\frac{1}{2}}\mdot
		\end{align*}
		%
		Next consider that we have
		\begin{align*}
			\sum_{j=1}^{n} \frac{1}{(n-R_{j,n}+1)^{\frac{1}{2}}} &= \sum_{j=1}^{n} \frac{1}{j^{\frac{1}{2}}}\\
			&= 1 + \sum_{j=2}^{n} \int_{j-1}^{j} \frac{1}{\sqrt{j}} dx\\
			&\leq 1 + \sum_{j=2}^{n} \int_{j-1}^{j} \frac{1}{\sqrt{x}} dx\\
			&\leq 2\sqrt{n}
		\end{align*}
		for all $n\geq 1$.
		%
		We therefore obtain
		\begin{align*}
			&\doublesum\limits_{1\leq i<j\leq n}\E\left[\phi^2(Z_{i:n}, Z_{j:n}) W^2_{i:n} W^2_{j:n}\right]^{\frac{1}{2}}\\
			&\leq 4\cdot\E\left[\phi^2(Z_{1}, Z_{2}) q^2(Z_{1})q^2(Z_{2})B_n(Z_{1})B_n(Z_{2})\right]^{\frac{1}{2}}\mdot \numberthis \label{eq:expectation_phi_sq}
		\end{align*}
		%
		Since $q$ and $\phi$ are Borel-measurable, we can apply Lemma \ref{lem:representation_bn} to obtain
		\begin{align*}
			&\doublesum\limits_{1\leq i<j\leq n}\E\left[\phi^2(Z_{i:n}, Z_{j:n}) W^2_{i:n} W^2_{j:n}\right]^{\frac{1}{2}}\\
			&\leq 8\cdot\E\left[\phi^2(Z_{1}, Z_{2}) q^2(Z_{1})q^2(Z_{2})(\Delta_{n-2}(Z_1,Z_2) + \bar{\Delta}_{n-2}(Z_1,Z_2))\right]^{\frac{1}{2}}\mdot
		\end{align*}
		%
		Note that $0\leq C_n(s)\leq 1$ for all $n\geq 1$ and $s\in \R_+$. Thus 
		$$\bar\Delta_n(s,t) = \E[C_n(s)D_n(s,t)] \leq \Delta_{n}(s,t)$$ 
		for all $n\geq 1$ and $s<t$.
		%
		Therefore we get 
		\begin{align*}
			&\doublesum\limits_{1\leq i<j\leq n}\E\left[\phi^2(Z_{i:n}, Z_{j:n}) W^2_{i:n} W^2_{j:n}\right]^{\frac{1}{2}}\\
			&\leq 16\cdot\E\left[\phi^2(Z_{1}, Z_{2}) q^2(Z_{1})q^2(Z_{2})\Delta_{n-2}(Z_1,Z_2)\right]^{\frac{1}{2}}\mdot
		\end{align*}	
		%
		By virtue of Lemma \ref{lem:neveu}, we have
		$$\Delta_{n}(s,t) = \E[D_n(s,t)] = \E[D_n(s,t)|\F_\infty] \nearrow D(s,t)\mdot$$
		%
		But this implies in particular that $\E[D_n(s,t)] \leq D(s,t)$ for all $n\geq 1$. Hence 
		\begin{align*}
			&\doublesum\limits_{1\leq i<j\leq n}\E\left[\phi^2(Z_{i:n}, Z_{j:n}) W^2_{i:n} W^2_{j:n}\right]^{\frac{1}{2}}\\
			&\leq 16\cdot\E\left[\phi^2(Z_{1}, Z_{2}) q^2(Z_{1})q^2(Z_{2})D(Z_1,Z_2)\right]^{\frac{1}{2}}\mdot
		\end{align*}		
		%
		Next consider that for each $s<t$ \st\ $H(t)<1$
		\begin{align*}
			D(s,t) &= \exp\left(2\int_{0}^{s} \frac{1-q(z)}{1-H(z)} H(dz) + \int_{s}^{t} \frac{1-q(z)}{1-H(z)} H(dz)\right)\\
			&\leq \exp\left(2\int_{0}^{s} \frac{1}{1-H(z)} H(dz) + \int_{s}^{t} \frac{1}{1-H(z)} H(dz)\right)\\
			&= \exp\left(\int_{0}^{s} \frac{1}{1-H(z)} H(dz) + \int_{0}^{t} \frac{1}{1-H(z)} H(dz)\right)\\
			&= \exp\left(-\ln(1-H(s)) -\ln(1-H(t))\right)\\
			&= \frac{1}{(1-H(s))(1-H(t))}\mdot
		\end{align*}
		%
		Therefore we have
		\begin{align*}
			&\doublesum\limits_{1\leq i<j\leq n}\E\left[\phi^2(Z_{i:n}, Z_{j:n}) W^2_{i:n} W^2_{j:n}\right]^{\frac{1}{2}}\\
			&\leq 16\cdot\E\left[\frac{\phi^2(Z_{1}, Z_{2})}{(1-H(Z_1))(1-H(Z_2))}\right]^{\frac{1}{2}}\\
			&\leq 16\cdot\left\{\int_{0}^{Z_1} \int_{0}^{Z_2}\frac{\phi^2(s, t)}{(1-H(s))(1-H(t))} H(ds)H(dt)\right\}^{\frac{1}{2}}\mdot
		\end{align*}		
		%
		Now taking into consideration the Radon-Nikodym derivatives (\cf\ \cite{dikta2000strong}, page 8)
		$$\frac{H^1(dt)}{H(dt)} = m(t,\theta_0) \textrm{ and } \frac{H^1(dt)}{F(dt)} = 1-G(t) \textrm{,}$$
		yields
		\begin{align*}
			&\doublesum\limits_{1\leq i<j\leq n}\E\left[\phi^2(Z_{i:n}, Z_{j:n}) W^2_{i:n} W^2_{j:n}\right]^{\frac{1}{2}}\\
			&\leq 16\cdot\left\{\int_{0}^{Z_1} \int_{0}^{Z_2}\frac{\phi^2(s, t)}{m(s,\theta_0)m(t, \theta_0)(1-H(s))(1-H(t))} H^1(ds)H^1(dt)\right\}^{\frac{1}{2}}\\
			&= 16\cdot\left\{\int_{0}^{Z_1} \int_{0}^{Z_2}\frac{\phi^2(s, t)}{m(s,\theta_0)m(t, \theta_0)(1-F(s))(1-F(t))} F(ds)F(dt)\right\}^{\frac{1}{2}}
		\end{align*}	
		since $1-H(x) = (1-F(x))(1-G(x))$ for all $x\in \R_+$. But now the integral above is finite under (A\ref{ass:intgral_phi_q}), since we have for $0\leq \epsilon\leq 1$
		$$(1-H(x))^\epsilon = (1-F(x))^\epsilon(1-G(x))^\epsilon \leq (1-F(x))^\epsilon \leq 1-F(x) \mdot$$
		Therefore we finally obtain
		$$\lim\limits_{n\to\infty}\doublesum\limits_{1\leq i<j\leq n}\E\left[\phi^2(Z_{i:n}, Z_{j:n}) W^2_{i:n} W^2_{j:n}\right]^{\frac{1}{2}} < \infty\mdot$$
	\end{proof}
\end{lemma}	
%
The following results about $q$ in  combination with order statistics are necessary for the proof of Lemma \ref{lem:qisquare_upper_bound} below. 
\begin{lemma} \label{lem:q_spacings}
	Let (Q\ref{ass:sup_qprime}) be satisfied. Then the following statements hold true for $k\leq n-1$
	\begin{enumerate}[(i)]
		\item We have
		\begin{equation}
			\E[|q(Z_{k:n})-q(Z_{k+1:n})|] \leq \frac{c_1}{n+1}\mdot
			\label{eq:q_spacings_a}
		\end{equation}
		\item Furthermore assume that (Q\ref{ass:q_H_one}) holds. Then
		\begin{equation}
			\E[1-q(Z_{k:n})] \leq \frac{c_1(n-k+1)}{n+1}\mdot
			\label{eq:q_spacings_lastk}
		\end{equation}
	\end{enumerate}
	%
	\begin{proof}
		Let $q_H := q\circ H^{-1}$ and consider that we can write
		\begin{equation}
			q(H^{-1}(x)) = q(H^{-1}(x_0)) + q_H'(\hat{x})(x-x_0)
			\label{eq:taylor_q}
		\end{equation}
		using Taylor's expansion for some $\hat{x}$ in between $x$ and $x_0$. Therefore we have 
		$$q(H^{-1}(x)) - q(H^{-1}(x_0)) = q_H'(\hat{x})(x-x_0)$$
		and hence
		\begin{equation}
			|q(H^{-1}(x)) - q(H^{-1}(x_0))| = |q_H'(\hat{x})|\cdot |x-x_0|\mdot
			\label{eq:q_taylor}
		\end{equation}
		%
		Now let $U_1, \dots,U_n$ be \iid\ $Uni[0,1]$ and set $x=U_{k:n}$ and $x_0=U_{k+1:n}$ to get
		\begin{equation*}
			\E[|q(H^{-1}(U_{k:n})) - q(H^{-1}(U_{k+1:n}))|] = \E[|q(Z_{k:n}) - q(Z_{k+1:n})|]\mdot
		\end{equation*}
		%
		Thus we get from \eqref{eq:q_taylor}
		\begin{equation*}
			\E[|q(Z_{k:n}) - q(Z_{k+1:n})|] = E[|q_H'(\hat{x})|\cdot (U_{k+1:n} - U_{k:n})]
		\end{equation*}
		where $\hat{x} \in [U_{k:n}, U_{k+1:n}]$.
		%
		Assumption (Q\ref{ass:sup_qprime}) directly implies that $\abs{q_H'(x)}\leq c_1$ for all $x \in [0,1]$. Hence we have
		$$\E[|q(Z_{k:n}) - q(Z_{k+1:n})|] = c_1 \E[U_{k+1:n} - U_{k:n}]\mdot$$
		%											 271
		According to \cite{shorack2009empirical} (p. 131), we have
		\begin{equation}
			\E[U_{k+1:n} - U_{k:n}] = \frac{1}{n+1}\mdot
			\label{eq:u_spacings}
		\end{equation}
		%
		Therefore we may conclude
		\begin{align*}
			\E[|q(Z_{k:n}) - q(Z_{k+1:n})|] &\leq c_1\E[U_{k+1:n} - U_{k:n}]\\
			&= \frac{c_1}{n+1}\mdot \numberthis \label{eq:q_diff_a}
		\end{align*}
		The proof of part (i) is hereby complete. 
		%
		We will now continue with the proof of part (ii). Consider 
		\begin{align*}
			1-q(Z_{k:n}) &=  1 - q(Z_{n:n}) + \sum_{l=k}^{n-1}(q(Z_{l+1:n}) - q(Z_{l:n}))\\
			&\leq 1 - q(Z_{n:n}) + \sum_{l=k}^{n-1}\abs{q(Z_{l+1:n}) - q(Z_{l:n})}\mdot
		\end{align*}
		%
		Taking expectations on each side yields
		\begin{equation*}
			1 - \E[q(Z_{k:n})] \leq 1 - \E[q(Z_{n:n})] + \sum_{l=k}^{n-1}\E[\abs{q(Z_{l+1:n}) - q(Z_{l:n})}]\mdot
		\end{equation*}	
		%
		Now we apply inequality \eqref{eq:q_diff_a} to the expectation under the sum to get 
		\begin{equation}
			1 - \E[q(Z_{k:n})] \leq 1 - \E[q(Z_{n:n})] + \frac{c_1(n-k)}{n+1}\mdot
			\label{eq:q_k_upperbnd_alt}
		\end{equation}
		%
		Recall the Taylor expansion from above
		\begin{equation*}
		q(H^{-1}(x)) = q(H^{-1}(x_0)) + q_H'(\hat{x})(x-x_0)\mdot
		\end{equation*}	
		%
		Setting $x = 1$ and $x_0 = U_{n:n}$ and taking expectations on both sides yields
		\begin{equation*}
			\E[q(H^{-1}(1))] = \E[q(Z_{n:n})] + \E[q_H'(\hat{x})(1 - U_{n:n})]
		\end{equation*}
		where $\hat{x} \in [U_{n:n}, 1]$ 
		%
		Now we get from assumption (A\ref{ass:sup_qprime}) that
		\begin{align*}
			\E[q(Z_{n:n})] &= \E[q(H^{-1}(1))] - \E[q_H'(\hat{x})(1 - U_{n:n})]\\
			&\geq \E[q(H^{-1}(1))] - c_1\E[1 - U_{n:n}]\mdot
		\end{align*}
		%									  271
		Using \cite{shorack2009empirical} (p. 131) again, we obtain
		\begin{align*}
			\E[q(Z_{n:n})] &= \E[q(H^{-1}(1))] - \frac{c_1}{n+1}\mdot
		\end{align*}
		%
		Applying (Q\ref{ass:q_H_one}) yields
		\begin{equation*}
			\E[q(Z_{n:n})]  \geq 1 - \frac{c_1}{n+1}\mdot
		\end{equation*}
		%
		By combining the above with \eqref{eq:q_k_upperbnd_alt} we get
		\begin{equation*}
			1 - \E[q(Z_{k:n})] \leq 1 - 1 + \frac{c_1}{n+1} + \frac{c_1(n-k)}{n+1} = \frac{c_1(n-k+1)}{n+1} \mdot
		\end{equation*}	
		%
		This concludes the proof of part (ii). 
	\end{proof}
\end{lemma}
%
The lemma below identifies an upper bound for the second term in \eqref{eq:yn}.
\begin{lemma} \label{lem:qisquare_upper_bound} 
	Let (Q\ref{ass:sup_qprime})  and (Q\ref{ass:q_H_one}) be satisfied. Then the following holds for $n\geq 2$ and $1\leq i\leq n+1$ 
	\begin{align*}
		\E[(Q_i^{n+1} - 1)^2] &\leq \frac{c_2}{n^\frac{7}{3}}
	\end{align*}
	where $c_2:= 560 c_1 + 1$ and $c_1$ as defined in (Q\ref{ass:sup_qprime}). 
	%
	\begin{proof}
		Consider for $n\geq 2$ and $1\leq i\leq n+1$
		\begin{equation*}
			Q_i^{n+1} - 1 = Q_1^{n+1} + \sum_{k_1=1}^{i-1} (Q_{k_1+1}^{n+1} - Q_{k_1}^{n+1}) - 1 \numberthis \label{eq:qi_sum}
		\end{equation*}
		and recall the following definition
		$$Q_i^{n+1} := (n+1)\left\{\sum_{r=1}^{i-1}\left[\frac{\pi_r}{n-r+2-q(Z_{r:n+1})}\right]^2 + \frac{\pi_i \pi_{i+1}}{n-i+1} \right\}$$
		where
		$$\pi_i := \prod_{k=1}^{i-1} \left[\frac{n-k+1-q(Z_{k:n+1})}{n-k+2-q(Z_{k:n+1})}\right]\mdot$$
		%
		We have $\pi_1 = 1$, since the product above is empty for $i=1$ and 
		$$\pi_2 = \frac{n-q(Z_{1:n+1})}{n+1-q(Z_{1:n+1})}\mdot$$ 
		Thus we get
		\begin{align*}
			Q_1^{n+1} - 1 &= (n+1)\frac{\pi_1 \pi_2}{n} - 1\\
			&= \frac{(n+1)(n-q(Z_{1:n+1}))}{n(n+1-q(Z_{1:n+1}))} - 1\\
			&= \frac{n(n+1-q(Z_{1:n+1}))-q(Z_{1:n+1})}{n(n+1-q(Z_{1:n+1}))} - 1\\
			&= 1 - \frac{q(Z_{1:n+1})}{n(n+1-q(Z_{1:n+1}))} - 1\\
			&= -\frac{q(Z_{1:n+1})}{n(n+1-q(Z_{1:n+1}))}\mdot
		\end{align*}
		%
		Therefore we get from \eqref{eq:qi_sum} 
		\begin{equation*}
			Q_i^{n+1} - 1 = \sum_{k_1=1}^{i-1} (Q_{k_1+1}^{n+1} - Q_{k_1}^{n+1}) - \frac{q(Z_{1:n+1})}{n(n+1-q(Z_{1:n+1}))}\mdot
		\end{equation*}
		%
		Moreover we have 
		\begin{align*}
			(Q_i^{n+1} - 1)^2 &= \sum_{k_1=1}^{i-1}\sum_{k_2=1}^{i-1}(Q_{k_1+1}^{n+1} - Q_{k_1}^{n+1})(Q_{k_2+1}^{n+1} - Q_{k_2}^{n+1})\\
			&\qquad - \frac{2q(Z_{1:n+1})}{n(n+1-q(Z_{1:n+1}))} \sum_{k=1}^{i-1}(Q_{k_1+1}^{n+1} - Q_{k_1}^{n+1})\\
			&\qquad + \frac{q^2(Z_{1:n+1})}{n^2(n+1-q(Z_{1:n+1}))^2}\\
			&\leq \sum_{k_1=1}^{i-1}\sum_{k_2=1}^{i-1}|Q_{k_1+1}^{n+1} - Q_{k_1}^{n+1}|\cdot|Q_{k_2+1}^{n+1} - Q_{k_2}^{n+1}|\\
			&\qquad + \frac{2q(Z_{1:n+1})}{n(n+1-q(Z_{1:n+1}))} \sum_{k_1=1}^{i-1}|Q_{k_1+1}^{n+1} - Q_{k_1}^{n+1}|\\
			&\qquad + \frac{q^2(Z_{1:n+1})}{n^2(n+1-q(Z_{1:n+1}))^2}\\
			&\leq \sum_{k_1=1}^{i-1}\sum_{k_2=1}^{i-1}|Q_{k_1+1}^{n+1} - Q_{k_1}^{n+1}|\cdot|Q_{k_2+1}^{n+1} - Q_{k_2}^{n+1}|\\
			&\qquad + \frac{2}{n^2} \sum_{k_1=1}^{i-1}|Q_{k_1+1}^{n+1} - Q_{k_1}^{n+1}| + \frac{1}{n^4}\mdot \numberthis\label{eq:qiminusonesq}
		\end{align*}
		%
		Recall that we set $q_i := q(Z_{i:n+1})$. According to Lemma \ref{lem:qi_increas} we have
		\begin{align*}
		&|Q_{i+1}^{n+1} - Q_i^{n+1}| \\
		&= \frac{\tilde{\pi}_i^2(n-i+2)^2}{n+1} \cdot \left|\frac{(q_i-q_{i+1})(n-i)(n-i+1) - q_{i+1}(1-q_i)(n-i+1-q_{i})}{(n-i)(n-i+1)(n-i+2-q_i)^2(n-i+1-q_{i+1})}\right|\\
		&\leq \frac{\tilde{\pi}_i^2(n-i+2)^2}{n+1} \cdot \frac{|q_i-q_{i+1}|(n-i)(n-i+1) + q_{i+1}(1-q_i)(n-i+1-q_{i})}{(n-i)(n-i+1)(n-i+2-q_i)^2(n-i+1-q_{i+1})}\\
		&\leq \frac{(n-i+2)^2}{n+1} \left\{\frac{\abs{q_i-q_{i+1}}(n-i)(n-i+1)+ q_{i+1}(1-q_i)(n-i+1)}{(n-i)(n-i+1)(n-i+1)^2(n-i)}\right\}\\
		&= \frac{(n-i+2)^2}{n+1} \left\{\frac{\abs{q_i-q_{i+1}}(n-i)+ q_{i+1}(1-q_i)}{(n-i)^2(n-i+1)^2}\right\}\\
		&\leq \frac{4\abs{q_i-q_{i+1}}}{(n+1)(n-i)} + \frac{4(1-q_i)}{(n+1)(n-i)^2}\mdot \numberthis\label{eq:qi_diff}
		\end{align*}
		The latter inequality above holds since 
		$$\frac{n-i+2}{n-i+1} = 1 + \frac{1}{n-i+1} \leq 2$$
		and $q_{i+1} \leq 1$.
		%
		Thus we have
		\begin{align*}
		&|Q_{k_1+1}^{n+1} - Q_{k_1}^{n+1}|\cdot|Q_{k_2+1}^{n+1} - Q_{k_2}^{n+1}|\\ 
		&\leq \left[\frac{4\abs{q_{k_1}-q_{k_1+1}}}{(n+1)(n-k_1)} + \frac{4(1-q_{k_1})}{(n+1)(n-k_1)^2}\right] \\
		&\qquad \times \left[\frac{4\abs{q_{k_2}-q_{k_2+1}}}{(n+1)(n-k_2)} + \frac{4(1-q_{k_2})}{(n+1)(n-k_2)^2}\right]\\
		&= \frac{16\abs{q_{k_1}-q_{k_1+1}}\abs{q_{k_2}-q_{k_2+1}} }{(n+1)^2(n-k_1)(n-k_2)} + \frac{16\abs{q_{k_1}-q_{k_1+1}}(1-q_{k_2})}{(n+1)^2(n-k_1)(n-k_2)^2}\\
		&\qquad + \frac{16(1-q_{k_1})\abs{q_{k_2}-q_{k_2+1}}}{(n+1)^2(n-k_1)^2(n-k_2)} + \frac{16(1-q_{k_1})(1-q_{k_2})}{(n+1)^2(n-k_1)^2(n-k_2)^2}\\
		&\leq \frac{16\abs{q_{k_1}-q_{k_1+1}}}{(n+1)^2(n-k_1)(n-k_2)} + \frac{16\abs{q_{k_1}-q_{k_1+1}}}{(n+1)^2(n-k_1)(n-k_2)^2}\\
		&\qquad + \frac{16\abs{q_{k_2}-q_{k_2+1}}}{(n+1)^2(n-k_1)^2(n-k_2)} + \frac{16(1-q_{k_1})}{(n+1)^2(n-k_1)^2(n-k_2)^2}\mdot
		\end{align*}
		Here the latter inequality holds, since we have $\abs{q_{k}-q_{k+1}} \leq 1$ and $1-q_{k} \leq 1$ for all $k\leq n-1$. \\
		\\
		Next recall that 
		\begin{align*}
		(Q_i^{n+1} - 1)^2 &\leq \sum_{k_1=1}^{i-1}\sum_{k_2=1}^{i-1}\abs{Q_{k_1+1}^{n+1} - Q_{k_1}^{n+1}}\abs{Q_{k_2+1}^{n+1} - Q_{k_2}^{n+1}}\\
		&\qquad + \frac{2}{n^2} \sum_{k_1=1}^{i-1}\abs{Q_{k_1+1}^{n+1} - Q_{k_1}^{n+1}} + \frac{1}{n^4} \mdot
		\end{align*}
		%
		Taking expectations on each side yields
		\begin{align*}
		\E[(Q_i^{n+1} - 1)^2] &\leq \sum_{k_1=1}^{i-1}\sum_{k_2=1}^{i-1}\E[\abs{Q_{k_1+1}^{n+1} - Q_{k_1}^{n+1}}\abs{Q_{k_2+1}^{n+1} - Q_{k_2}^{n+1}}]\\
		&\qquad + \frac{2}{n^2} \sum_{k_1=1}^{i-1}\E[|Q_{k_1+1}^{n+1} - Q_{k_1}^{n+1}|] + \frac{1}{n^4}\mdot \numberthis\label{eq:qiminusonesq_exp}
		\end{align*}
		%
		We will now regard the two sums above individually. Consider the expectation under the double sum above. We have 
		\begin{align*}
		&\E\left[\abs{Q_{k_1+1}^{n+1} - Q_{k_1}^{n+1}}\abs{Q_{k_2+1}^{n+1} - Q_{k_2}^{n+1}}\right]\\ 
		&\leq \frac{16\E\left[\abs{q_{k_1}-q_{k_1+1}}\right]}{(n+1)^2(n-k_1)(n-k_2)} + \frac{16\E\left[\abs{q_{k_1}-q_{k_1+1}}\right]}{(n+1)^2(n-k_1)(n-k_2)^2}\\
		&\qquad + \frac{16\E\left[\abs{q_{k_2}-q_{k_2+1}}\right]}{(n+1)^2(n-k_1)^2(n-k_2)} + \frac{16\E\left[(1-q_{k_1})\right]}{(n+1)^2(n-k_1)^2(n-k_2)^2}\mdot \numberthis\label{eq:qiminusonesq}
		\end{align*}
		%
		According to Lemma \ref{lem:q_spacings}, we have 
		$$\E[|q(Z_{k:n})-q(Z_{k+1:n})|] \leq \frac{c_1}{n+1}$$ 
		and 
		$$\E[1-q(Z_{k:n})] \leq \frac{c_1(n-k+1)}{n+1}\mdot$$
		%
		Therefore
		\begin{align*}
			&\E[\abs{Q_{k_1+1}^{n+1} - Q_{k_1}^{n+1}}\abs{Q_{k_2+1}^{n+1} - Q_{k_2}^{n+1}}] \\
			&\leq \frac{16c_1}{(n+1)^3(n-k_1)(n-k_2)} + \frac{16c_1}{(n+1)^3(n-k_1)(n-k_2)^2}\\
			&\qquad + \frac{16c_1}{(n+1)^3(n-k_1)^2(n-k_2)} + \frac{16c_1(n-k_1) + 16c_1}{(n+1)^3(n-k_1)^2(n-k_2)^2}\mdot
		\end{align*}
		%
		Thus we obtain
		\begin{align*}
		&\sum_{k_1=1}^{i-1}\sum_{k_2=1}^{i-1}\E[\abs{Q_{k_1+1}^{n+1} - Q_{k_1}^{n+1}}\abs{Q_{k_2+1}^{n+1} - Q_{k_2}^{n+1}}]\\
		&\leq \sum_{k_1=1}^{i-1}\sum_{k_2=1}^{i-1}\frac{16c_1}{(n+1)^3(n-k_1)(n-k_2)} + \sum_{k_1=1}^{i-1}\sum_{k_2=1}^{i-1}\frac{16c_1}{(n+1)^3(n-k_1)(n-k_2)^2}\\
		&\qquad + \sum_{k_1=1}^{i-1}\sum_{k_2=1}^{i-1}\frac{16c_1}{(n+1)^3(n-k_1)^2(n-k_2)} + \sum_{k_1=1}^{i-1}\sum_{k_2=1}^{i-1}\frac{16c_1(n-k_1) }{(n+1)^3(n-k_1)^2(n-k_2)^2}\\
		&\qquad + \sum_{k_1=1}^{i-1}\sum_{k_2=1}^{i-1}\frac{16c_1}{(n+1)^3(n-k_1)^2(n-k_2)^2}\\
		&= \frac{16c_1}{(n+1)^3}\sum_{k_1=1}^{i-1}\frac{1}{(n-k_1)}\sum_{k_2=1}^{i-1}\frac{1}{(n-k_2)} + \frac{32c_1}{(n+1)^3}\sum_{k_1=1}^{i-1}\frac{1}{n-k_1}\sum_{k_2=1}^{i-1}\frac{1}{(n-k_2)^2}\\
		&\qquad + \frac{16c_1}{(n+1)^3}\sum_{k_1=1}^{i-1}\frac{1}{(n-k_1)^2}\sum_{k_2=1}^{i-1}\frac{1}{n-k_2} + \frac{16c_1}{(n+1)^3}\sum_{k_1=1}^{i-1}\frac{1}{(n-k_1)^2}\sum_{k_2=1}^{i-1}\frac{1}{(n-k_2)^2}\\
		&= \frac{16c_1}{(n+1)^3}\sum_{k_1=n-i+1}^{n-1}\frac{1}{k_1}\sum_{k_2=n-i+1}^{n-1}\frac{1}{k_2} + \frac{32c_1}{(n+1)^3}\sum_{k_1=n-i+1}^{n-1}\frac{1}{k_1}\sum_{k_2=n-i+1}^{n-1}\frac{1}{k_2^2}\\
		&\qquad + \frac{16c_1}{(n+1)^3}\sum_{k_1=n-i+1}^{n-1}\frac{1}{k_1^2}\sum_{k_2=n-i+1}^{n-1}\frac{1}{k_2} + \frac{16c_1}{(n+1)^3}\sum_{k_1=n-i+1}^{n-1}\frac{1}{k_1^2}\sum_{k_2=n-i+1}^{n-1}\frac{1}{k_2^2}\mdot \numberthis \label{eq:sumqk}
		\end{align*}
		%
		Now using inequalities \eqref{eq:sum_ln} and \eqref{eq:ln_over_n_upperb} from Lemma \ref{lem:bounds} on inequality \eqref{eq:sumqk} yields 
		\begin{align*}
		&\sum_{k_1=1}^{i-1}\sum_{k_2=1}^{i-1}\E[\abs{Q_{k_1+1}^{n+1} - Q_{k_1}^{n+1}}\abs{Q_{k_2+1}^{n+1} - Q_{k_2}^{n+1}}]\\
		&\leq \frac{16c_1}{(n+1)^3}(\ln(n-1)+1)^2 + \frac{64c_1}{(n+1)^3}(\ln{(n-1)}+1)\\
		&\qquad + \frac{32c_1}{(n+1)^3}(\ln{(n-1)}+1) + \frac{64c_1}{(n+1)^3}\\
		&\leq \frac{144c_1}{(n+1)^\frac{7}{3}}+ \frac{288c_1}{(n+1)^\frac{8}{3}} + \frac{64c_1}{(n+1)^3}\\
		&\leq \frac{496c_1}{(n+1)^\frac{7}{3}}\mdot \numberthis \label{eq:first_sum}
		\end{align*}
		%
		We will now proceed with the second sum in \eqref{eq:qiminusonesq_exp}. According to \eqref{eq:qi_diff} we have
		\begin{equation*}
		\E[\abs{Q_{i+1}^{n+1} - Q_i^{n+1}}] \leq \frac{4\E[\abs{q_i-q_{i+1}}]}{(n+1)(n-i)} + \frac{4\E[1-q_i]}{(n+1)(n-i)^2}\mdot
		\end{equation*}
		%
		Therefore we obtain
		\begin{equation*}
		\frac{2}{n^2} \sum_{k_1=1}^{i-1}\E[|Q_{k_1+1}^{n+1} - Q_{k_1}^{n+1}|] \leq \frac{8}{n^2(n+1)}\sum_{k_1=1}^{i-1}\frac{\E[\abs{q_{k_1}-q_{k_1+1}}]}{n-k_1} + \frac{\E[1-q_{k_1}]}{(n-k_1)^2}\mdot
		\end{equation*}
		%
		Using \eqref{eq:q_spacings_a} and \eqref{eq:q_spacings_lastk} again reveals
		\begin{align*}
		\frac{2}{n^2} \sum_{k_1=1}^{i-1}\E[|Q_{k_1+1}^{n+1} - Q_{k_1}^{n+1}|] &\leq  \frac{8}{n^2(n+1)^2}\left\{\sum_{k_1=1}^{i-1}\frac{c_1}{(n-k_1)} + \sum_{k_1=1}^{i-1}\frac{c_1(n-k_1+1)}{(n-k_1)^2}\right\}\\
		&=  \frac{8}{n^2(n+1)^2}\left\{2\sum_{k_1=1}^{i-1}\frac{c_1}{(n-k_1)} + \sum_{k_1=1}^{i-1}\frac{c_1}{(n-k_1)^2}\right\}\\
		&=  \frac{8}{n^2(n+1)^2}\left\{2\cdot\sum_{k_1=n-i+1}^{n-1}\frac{c_1}{k_1} + \sum_{k_1=n-i+1}^{n-1}\frac{c_1}{k_1^2}\right\}\mdot
		\end{align*}
		%
		According to \eqref{eq:sum_ln} and \eqref{eq:ln_over_n_upperb} of Lemma \ref{lem:bounds} we have
		\begin{align*}
		\frac{2}{n^2} \sum_{k_1=1}^{i-1}\E[|Q_{k_1+1}^{n+1} - Q_{k_1}^{n+1}|] &\leq  \frac{8\cdot\{2c_1(\ln(n-1)+1)+2c_1\}}{n^2(n+1)^2}\\
		&=  \frac{16c_1(\ln(n-1)+1)}{n^2(n+1)^2} + \frac{16c_1}{n^2(n+1)^2}\\
		&\leq  \frac{48c_1}{n^2(n+1)^\frac{5}{3}} + \frac{16c_1}{n^2(n+1)^2}\\
		&\leq  \frac{64c_1}{n^2(n+1)^\frac{5}{3}} \mdot \numberthis\label{eq:second_sum}
		\end{align*}
		%
		Now recall the following fact from \eqref{eq:qiminusonesq_exp}
		\begin{align*}
		\E[(Q_i^{n+1} - 1)^2] &= \sum_{k_1=1}^{i-1}\sum_{k_2=1}^{i-1}\E[\abs{Q_{k_1+1}^{n+1} - Q_{k_1}^{n+1}}\abs{Q_{k_2+1}^{n+1} - Q_{k_2}^{n+1}}]\\
		&\qquad + \frac{2}{n^2} \sum_{k_1=1}^{i-1}\E[|Q_{k_1+1}^{n+1} - Q_{k_1}^{n+1}|] + \frac{1}{n^4}\mdot
		\end{align*}
		%
		Combining the above with \eqref{eq:first_sum} and \eqref{eq:second_sum} yields
		\begin{align*}
		\E[(Q_i^{n+1} - 1)^2] &\leq \frac{496c_1}{(n+1)^\frac{7}{3}}+ \frac{64c_1}{n^2(n+1)^\frac{5}{3}} + \frac{1}{n^4}\\
		&\leq \frac{496c_1}{n^\frac{7}{3}} + \frac{64c_1}{n^\frac{11}{3}} + \frac{1}{n^4}\\
		&\leq \frac{1}{n^\frac{7}{3}}\left[496c_1 + \frac{64c_1}{n^\frac{4}{3}} + \frac{1}{n^\frac{5}{3}}\right]\\
		&\leq \frac{560c_1+1}{n^\frac{7}{3}}\\
		&= \frac{c_2}{n^\frac{7}{3}}\mdot
		\end{align*}
		This concludes the proof.
	\end{proof}
\end{lemma}
%
%
%
\section{Finite Limit} \label{sec:finite_limit}
During the preceding sections of this chapter, we established everything we need to prove the almost sure existence of the limit $S=\lim_{n\to\infty}S_n$. In this section we will first show, that the expected number of Upcrossings of $\tilde{S}_1^N,\dots,\tilde{S}_N^N$ is finite in Lemma \ref{thm:upcrossing}. Afterwards we will show how this implies the almost sure existence of $S=\lim_{n\to\infty}S_n(q)$ in Lemma \ref{thm:existence_limit}. 
\begin{thm}	\label{thm:upcrossing}
	Assume that (A\ref{ass:kernel_gen}) through (A\ref{ass:intgral_phi_q}), (Q\ref{ass:sup_qprime}) and (Q\ref{ass:q_H_one}) hold. Then we have
	$$\lim\limits_{N\to\infty} \E[\UNab{N}] < \infty\mdot$$
	%
	\textbf{Note:} For $N\geq 2$ we have $\{\StN{1},\dots,\StN{N}\} = \{S_N,\dots, S_1\}$. Thus $\UNab{N}$ is the number of upcrossings of $S_N,\dots, S_1$, which coincides with the number of downcrossings of $S_1,\dots, S_N$ (\cf\ \cite{neveu1975discrete}, p. 116).
	%
	\begin{proof}
		Let (A\ref{ass:kernel_gen}) through (A\ref{ass:q_H_one}) be satisfied. According to Lemma \ref{lem:cs}, we have 
		$$\E[\UNab{N}] \leq \frac{\E[\YN{N}]}{b-a}$$
		and furthermore
		\begin{align*}
		\E[Y_N^N] &\leq \E[\StN{N}] + \sum_{k=1}^{N-1} \doublesum\limits_{1\leq i<j\leq N-k+1}\E\left[\phi^2(Z_{i:N-k+1}, Z_{j:N-k+1}) W^2_{i:N-k+1} W^2_{j:N-k+1}\right]^{\frac{1}{2}}\\
		&\qquad\qquad\qquad\qquad\qquad\qquad \times \E\left[(Q_{i,j}^{N-k+1} - 1)^2\right]^{\frac{1}{2}}\mdot \numberthis\label{eq:yn}
		\end{align*}
		%
		First note that $\E[\StN{N}]=\E[S_2]$ and that we have
		\begin{align*}
			\E[S_2] &= \int_{0}^{\infty}\int_{0}^{\infty} \phi(s,t) \frac{q(s)q(t)}{2}\left[1-\frac{q(s)}{2}\right]H(ds)H(dt)\\
			&\leq \frac{1}{2}\int_{0}^{\infty}\int_{0}^{\infty}\phi(s,t) H(ds)H(dt)\\
			&\leq \frac{1}{2}\int_{0}^{\infty}\int_{0}^{\infty}\frac{\phi(s,t)}{m(s,\theta_0)m(t,\theta_0)(1-H(s))(1-H(t))} H(ds)H(dt)\\
			&< \infty\mdot
		\end{align*}
		Next consider that Lemma \ref{lem:qisquare_upper_bound} identifies an upper bound for the expectation above as
		\begin{equation*}
		\E[(Q_i^{N-k+1} - 1)^2]^\frac{1}{2} \leq \frac{\sqrt{c_2}}{(N-k)^\frac{7}{6}}\mdot
		\end{equation*}
		%
		Now combining the latter with \eqref{eq:yn} yields 
		\begin{align*}
		\E[Y_N^N] &\leq  \E[\StN{N}] + \sum_{k=1}^{N-1} \doublesum\limits_{1\leq i<j\leq N-k+1}\E\left[\phi^2(Z_{i:N-k+1}, Z_{j:N-k+1}) W^2_{i:N-k+1} W^2_{j:N-k+1}\right]^{\frac{1}{2}}\\
		&\qquad\qquad\qquad\qquad\qquad\qquad \times \E\left[(Q_{i,j}^{N-k+1} - 1)^2\right]^{\frac{1}{2}}\\
		&\leq \left\{\sup_{N}\E[\StN{N}]\right\} + \sum_{k=1}^{N-1} \doublesum\limits_{1\leq i<j\leq N-k+1}\E\left[\phi^2(Z_{i:N-k+1}, Z_{j:N-k+1}) W^2_{i:N-k+1} W^2_{j:N-k+1}\right]^{\frac{1}{2}}\\
		&\qquad\qquad\qquad\qquad\qquad\qquad \times \frac{\sqrt{c_2}}{(N-k)^\frac{7}{6}}\mdot
		\end{align*}
		%
		But, according to Lemma \ref{lem:expectation_sq}, the limit of the double sum above is finite. Moreover 
		$$ \sum_{k=1}^{N-1}\frac{\sqrt{c_2}}{(N-k)^\frac{7}{6}}$$
		converges to a finite limit as $N\to\infty$. Furthermore note, that $\sup_{N}\E[\StN{N}]$ is finite under (A\ref{ass:intgral_phi_q}). Hence we have seen that
		$$\lim\limits_{N\to\infty}\E[\UNab{N}]\leq \lim\limits_{N\to\infty}\frac{\E[\YN{N}]}{b-a} <\infty\mdot$$
	\end{proof}
\end{thm}

%Now we will apply theorem \ref{thm:upcrossing} in order to prove the existence of the limit
%$$S_\infty = \lim\limits_{N\to\infty} S_N\.$$
The following theorem gives the main statement of this chapter. 
\begin{thm}
	\label{thm:existence_limit}
	Suppose (A\ref{ass:kernel_gen}) through (A\ref{ass:intgral_phi_q}), (Q\ref{ass:sup_qprime}) and (Q\ref{ass:q_H_one}) are satisfied. Then the limit
	$S = \lim_{n\to\infty} \sn{n}$
	exists $\P$-almost surely.
	%
	\begin{proof}
		Let's first define the set of all $\omega$ for which $S_n$ does not converge as
		$$\Lambda := \{\omega | S_n(\omega) \textrm{ does not converge}\}\mdot$$
		%
		Consider that can write
		\begin{align*}
			\Lambda &= \{\omega | \liminf_{n}S_n(\omega) < \limsup_{n} S_n(\omega)\}\\
			&= \bigcup_{a,b\in\mathbb{Q}}\{\omega | \liminf_{n}S_n(\omega) < a < b < \limsup_{n} S_n(\omega)\}\mdot
		\end{align*}
		%  
		Recall that we have $\UNab{N}$, the number of upcrossings of $[a,b]$ by $\StN{1}, \dots, \StN{N}$. But this is equal to the number of upcrossings of $[a,b]$ by $S_N, \dots, S_1$. Furthermore recall that 
		$$U_\infty[a,b] = \lim\limits_{N\to\infty} \UNab{N}\mdot$$
		%
		Consider that for each $\omega \in \{\omega | \liminf_{n}S_n(\omega) < a < b < \limsup_{n} S_n(\omega)\}$ we must have $U_{\infty}[a,b](\omega) = \infty$. This follows directly from the definitions of $\liminf$ and $\limsup$. Thus we can write
		\begin{equation*}
		\Lambda = \bigcup_{a,b\in\mathbb{Q}}\{ \omega | U_\infty[a,b](\omega) = \infty\} = \bigcup_{a,b\in\mathbb{Q}} \Lambda_{a,b}
		\end{equation*}
		where $\Lambda_{a,b} := \{ \omega | U_\infty[a,b](\omega) = \infty\}$.
		%
		Consequently we get that
		\begin{equation}
			\E[\I{\Lambda_{a,b}} U_\infty[a,b]] = \begin{cases}
			\infty &\textrm{ if } \P(\Lambda_{a,b})>0\\
			0 &\textrm{ if } \P(\Lambda_{a,b})=0\\
			\end{cases}\mdot
			\label{eq:lambda_u}
		\end{equation}
		%
		But, according to Theorem \ref{thm:upcrossing}, we have
		$$\lim\limits_{N\to\infty}\E[\UNab{N}]<\infty\mdot$$
		%
		Clearly $\UNab{N}$ is non-decreasing in $N$. Hence we get by virtue of the Monotone Convergence Theorem 
		$$\lim\limits_{N\to\infty}\E[\UNab{N}] = \E[U_\infty[a,b]] <\infty$$	
		%	
		and hence that
		$$\E[\I{\Lambda_{a,b}} U_\infty[a,b]] \leq \E[U_\infty[a,b]] < \infty\mdot$$
		%
		Now the latter together with \eqref{eq:lambda_u} implies that $\P(\Lambda_{a,b}) = 0$. Therefore we have
		\begin{equation*}
			\P(\Lambda) = \P\left(\bigcup_{a,b\in\mathbb{Q}}\Lambda_{a,b}\right) = \sum_{a,b \in \mathbb{Q}} \P(\Lambda_{a,b}) = 0 \mdot
		\end{equation*}
		Hereby the proof is concluded.
	\end{proof}
\end{thm}


