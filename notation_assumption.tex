\chapter{Notation and assumptions}
In this chapter we will state the main definitions and assumptions used throughout this work. We will start by defining the estimator to be considered and introduce all necessary notation for the remaining chapters.\\
\\
Recall the following definition
$$\wnse{i}{n} = \frac{m(Z_{i:n}, \hat{\theta}_n)}{n-i+1} \prod\limits_{j=1}^{i-1}\left(1-\frac{m(Z_{j:n}, \hat{\theta}_n)}{n-j+1}\right)$$
%
Now we define for $n\geq2$
$$S_{2,n}^{se} = \doublesum\limits_{1\leq i<j\leq n}\phi(Z_{i:n}, Z_{j:n})W_{i:n}^{se}W_{j:n}^{se}$$
This process will be called semiparametric U-Statistic of degree $2$ throughout this thesis.
%
Furthermore define
$$W_{i:n}(q) = \frac{q(Z_{i:n})}{n-i+1}\prod_{k=1}^{i-1}\left[1-\frac{q(Z_{k:n})}{n-k+1}\right]$$
and 
$$S_{n}(q) = \doublesum\limits_{1\leq i<j\leq n}\phi(Z_{i:n}, Z_{j:n})W_{i:n}(q)W_{j:n}(q)$$
%
\begin{example}
	Let $q(Z_{i:n}) = \delta_{[i:n]}$ for $1\leq i\leq n$. Then $W_{i:n}(q) = \wnkm{i}{n}$ and therefore 
	$$S_n(q) = \snkm{2}{n}$$
\end{example}
%
\begin{example}
	Let $q(t) = m(t, \hat\theta_n)$ for $t\in\R^+$. Then $W_{i:n}(q) = \wnse{i}{n}$ and therefore 
	$$S_n(q) = \snse{2}{n}$$
\end{example}
%
\noindent Moreover define
$$\F_n = \sigma\{Z_{1:n}, \dots, Z_{n:n}, Z_{n+1}, Z_{n+2}, \dots\}$$
%
Throughout this work we will write $\sn{n} := \sn{n}(q)$ and $\wn{i}{n} := \wn{i}{n}(q)$.
for $1\leq i\leq n$.\\
\\
%
The following assumptions will be needed throughout this work, in order to prove the SLLN for $\sn{n}$.
\begin{enumerate}[({A}1)]
	\item \label{ass:kernel_gen} The kernel $\phi: \R^2 \longrightarrow \R$ is measurable, non-negative and symmetric in its arguments. In effect $\phi(s,t) = \phi(t,s)$ for all $s,t \in \R_+$. 
	\item \label{ass:H_nonneg} The \df\ $H$ is continuous and concentrated on the non-negative real line.
	\item \label{ass:intgral_phi_q} For $s,t\in \R_+$ the following statement holds true
	$$\int_{0}^{s} \int_{0}^{t} \frac{\phi(s,t)}{m(s, \theta_0)m(t,\theta_0)(1-H(s))^\epsilon(1-H(t))^{\epsilon}} F(dt)F(ds) < \infty$$
	for some $0<\epsilon\leq 1$.
	\item \label{ass:sup_mprime} There exists $c_1 < \infty$ \st\ $\sup_{x} (m\circ H^{-1})'(x) \leq c_1$. 
	\item \label{ass:m_H_one} We have $m\circ H^{-1}(1) = 1$.
\end{enumerate}
%
\vspace{1cm}
We will need the following assumptions about the Censoring Model $m$ and the Maximum Likelihood estimate $\hat\theta_n$:
\begin{enumerate}[({M}1)]
	\item \label{ass:m_consistency} $\hat{\theta}_n$ is measurable and tends to $\theta_0$
	\item \label{ass:m_nbhd} For any $\epsilon>0$ there exists a neighborhood $V(\epsilon, \theta_0)\subset \Theta$ of $\theta_0$ \st\ for all $\theta\in V(\epsilon, \theta_0)$ 
	$$\sup\limits_{x\geq 0} |m(x, \theta) - m(x, \theta_0)| < \epsilon$$
\end{enumerate}
