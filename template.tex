%%%%%%%%%%%%%%%%%%%%%%%%%%%%%%%%%%%%%%%%%%%%%%%%%%%%%%%%%%%%%%%%%%%%%%%%
% Template for a Master's Thesis or Ph.D. dissertation
% at the University of Wisconsin-Milwaukee.
%
% Designed for LaTeX version 2e
%
% Updated by Adam J. Smith
% December, 2007
%
% This thesis template requires the file "UWMthesis.sty", available
% on the UWM's Atmospheric Science Club website, and possible in other
% locations.
%
% A LaTeX primer is not provided here.  For instructions on how to use commands for
% figures, table, equations, and bibliography citations, please see the documentation
% listed later in this document.
% 
% This template follows the "Fall 2007" version of the 
% University of Wisconsin-Milwaukee standards for the Master's thesis
% and Ph.D. dissertation.  Please feel free to update this template
% as needed to comply with these standards.
%
% For current thesis and dissertation formatting information, visit the UWM graduate school
% website at the following address:
% http://www.graduateschool.uwm.edu/students/current/thesis-and-dissertation-formatting/
%
% IMPORTANT: Be sure to meet with the appropriate Graduate School personnel to verify
% whether your final document meets the UWM standards.  If not, the document may not
% be accepted until any problems are corrected.
%%%%%%%%%%%%%%%%%%%%%%%%%%%%%%%%%%%%%%%%%%%%%%%%%%%%%%%%%%%%%%%%%%%%%%%%

% If you are writing a master's thesis, use the first option (master).
% If you are writing a phd dissertation, use the second option (phd).
%\documentclass[master]{UWMThesis}
\documentclass[phd]{UWMThesis}

\usepackage{fancyhdr}
\usepackage{xcolor}
\usepackage{amsmath}
\usepackage{amsthm}
\usepackage{bbm}
\usepackage{amssymb}
\usepackage{enumerate}
\usepackage{kantlipsum}
\usepackage{tabularx}

%\usepackage{todonotes}
%\usepackage{todo}
\usepackage{fixmetodonotes}
\defnote{Comment}{inline}{\hspace{10pt}}
\defnote{Note}{inline}{\hspace{10pt}\fbox}
%------------------------------------

% Uncomment the following "usepackage" line if you wish to use BiBTeX to create the
% bibliography.  This package is necessary to use the \citep or \citet commands, which are
% commonly used in publications like the Journal of Geophysical Research.
% If the style file "natbib.sty" is not provided with your release of MiKTeX, it is available
% on the Internet.  One example web source is:
% http://ads.harvard.edu/pubs/bibtex/astronat/natbib.sty
\usepackage{natbib}
\usepackage[ngerman, english]{babel} 
%------------------------------------

%Absolute value nice definition
\usepackage{mathtools}

\DeclarePairedDelimiter\abs{\lvert}{\rvert}%
\DeclarePairedDelimiter\norm{\lVert}{\rVert}%

% Swap the definition of \abs* and \norm*, so that \abs
% and \norm resizes the size of the brackets, and the 
% starred version does not.
\makeatletter
\let\oldabs\abs
\def\abs{\@ifstar{\oldabs}{\oldabs*}}
%
\let\oldnorm\norm
\def\norm{\@ifstar{\oldnorm}{\oldnorm*}}
\makeatother
%------------------------------------

% Uncomment this package if you want to use graphics files, such as .eps files
\usepackage{graphicx}

% Other packages go here as needed...
%\usepackage{mathabx}
\usepackage[colorlinks=true, linkcolor=blue, citecolor=blue]{hyperref}

% make todo note lines fancy af and display at left margin
%\let\tmptodo\todo
%\renewcommand{\todo}[1]{\tmptodo[fancyline]{#1}}
%\reversemarginpar

% Why not..
\allowdisplaybreaks 
%------------------------------------

%%%%%%%%%%%%%%%%%%%%%%%%%%%%%%%%%%%%%%%
%%%%%%%%%%%% Customization %%%%%%%%%%%%
%%%%%%%%%%%%%%%%%%%%%%%%%%%%%%%%%%%%%%%

% ----- my commands -----
\newcommand{\comment}[1]{\fbox{\begin{minipage}{\textwidth}\Comment{\textcolor{blue}{#1}}\end{minipage}}\\}
\newcommand\numberthis{\addtocounter{equation}{1}\tag{\theequation}}
\newcommand{\intrange}[3]{$#1 = #2, \dots, #3$}
\newcommand{\itref}[2]{(#1\ref{#2})}

\newenvironment{myarray}{\begin{center}$\begin{array}{ll} }{\end{array}$ \end{center}}

\renewcommand{\P}{\mathbb{P}}
\newcommand{\E}{\mathbb{E}}
\newcommand{\R}{\mathbb{R}}
\newcommand{\F}{\mathcal{F}}
\newcommand{\A}{\mathcal{A}}
\newcommand{\G}{\mathcal{G}}
\newcommand{\N}{\mathcal{N}}

\newcommand{\fnkm}{F_n^{km}}
\newcommand{\fnse}{F_n^{se}}
\newcommand{\wn}[2]{W_{#1:#2}}
\newcommand{\wnkm}[2]{W_{#1:#2}^{km}}
\newcommand{\wnse}[2]{W_{#1:#2}^{se}}
\newcommand{\wnseb}[2]{\bar{W}_{#1:#2}^{se}}
\newcommand{\sn}[1]{S_{#1}}
\newcommand{\snkm}[2]{S_{#1,#2}^{km}}
\newcommand{\snse}[2]{S_{#1,#2}^{se}}
\newcommand{\stnse}[1]{S_{2,#1}^{se}}
\newcommand{\snseb}[2]{\bar{S}_{#1,#2}^{se}}
\newcommand{\unkm}[2]{U_{#1,#2}^{km}}
\newcommand{\unse}[2]{U_{#1,#2}^{se}}

\newcommand{\StN}[1]{\tilde{S}_{#1}^N}
\newcommand{\FtN}[1]{\tilde{\F}_{#1}^N}
\newcommand{\YN}[1]{Y_{#1}^N}
\newcommand{\xitN}[1]{\tilde{\xi}_{#1}^N}
\newcommand{\UNab}[1]{U_{#1}^N[a,b]}

\newcommand{\I}[1]{{\mathbbm{I}\{#1\}}}
%\newcommand{\I}[1]{{\mathbbm{1}\{#1\}}}

\newcommand{\df}{d.\,f.}
\newcommand{\ie}{i.\,e.}
\newcommand{\iid}{i.\,i.\,d.}
\newcommand{\rv}{r.\,v.}
%\newcommand{\st}{s.\,t.}
\newcommand{\wpo}{w.\,p.\,1}
\newcommand{\as}{a.\,s.}
\newcommand{\st}{s.\,t.}
\newcommand{\wrt}{w.\,r.\,t.}

\newcommand{\cfbox}[2]{%
	\colorlet{currentcolor}{.}%
	{\color{#1}%
		\fbox{\color{currentcolor}#2}}%
}
\renewcommand{\.}{\textrm{ .}}


% \newtheorem{lemma}{Lemma}
% \newtheorem{theorem}{Theorem}

% ----- my mathoperators -----
\newcommand{\doublesum}{\mathop{\sum\sum}}
\newcommand{\qeq}{\mathop{\stackrel{?}{=}}}
\newcommand{\qleq}{\mathop{\stackrel{?}{\leq}}}
\newcommand{\qgeq}{\mathop{\stackrel{?}{\geq}}}


% ----- UWM stylings -----
%\renewcommand{\evensidemargin}{.875in}  
%\renewcommand{\oddsidemargin}{.875in}   
%\renewcommand{\topmargin}{-.3in}
%\renewcommand{\headheight}{0.2in}
%\renewcommand{\marginparwidth}{1.4in}
%\renewcommand{\marginparsep}{1.4pt}
%\renewcommand{\headsep}{.65in}
%\renewcommand{\footskip}{0.3in}
%\renewcommand{\textheight}{8in}
%\renewcommand{\textwidth}{5.9in} % crashes todonotes

\newcommand{\ls}{\vspace{.1in}}

\newtheorem{thm}{Theorem}
\newtheorem{lemma}[thm]{Lemma}
\newtheorem{cor}[thm]{Corollary}
\newtheorem{prop}[thm]{Proposition}
\theoremstyle{definition}
\newtheorem{example}[thm]{Example}
\newtheorem{remark}[thm]{Remark}
\newtheorem{defn}[thm]{Definition}
\newtheorem{ques}[thm]{Question}
\newtheorem{exer}[thm]{Exercise}
\numberwithin{thm}{chapter}

\newcommand{\todo}{\TODO}

% \renewcommand{\phi}{\varphi}

%%%%%%%%%%%%%%%%%%%%%%%%%%%%%%%%%%%%%%%%%
%%%%%%%%%%%% Personalization %%%%%%%%%%%%
%%%%%%%%%%%%%%%%%%%%%%%%%%%%%%%%%%%%%%%%%

% Insert your full name in the brackets.
\renewcommand{\ThesisAuthor}{Jan Hoft}

% If you are graduating in Spring, insert May.  If you are graduating in Fall, insert December.
\renewcommand{\ThesisMonth}{May}

% Insert the year of your graduation here
\renewcommand{\ThesisYear}{2018}

% Your thesis title goes here.  It will automatically be formatted to use multiple lines
% (if needed)
\renewcommand{\ThesisTitle}{Large sample properties of U-Statistics under semiparametric Random Censorship}

% Insert your advising professor's name here.  DO NOT include a prefix of "Prof." or "Dr."
% here!  The prefix will be inserted automatically.
\renewcommand{\ThesisAdvisor}{Gerhard Dikta and Professor Jugal Ghorai}

%------------------------------------

% If your thesis or dissertation has multiple volumes, set the argument to true.
% If not, set the argument to false.
\setboolean{multvolumes}{false}

% If your thesis or dissertation has multiple appendices, set the argument to true.
% If not, set the argument to false.
\setboolean{singleappendix}{false}

%---------------------------------------

% Creating new commands for displaying derivatives
% Add additional new commands as needed...
\newcommand{\ptlder}[2]{\frac{\partial #1}{\partial #2}}
\newcommand{\totder}[2]{\frac{d #1}{d #2}}

%---------------------------------------

%%%%%%%%%%%%%%%%%%%%%%%%%%%%%%%%%%%%%%%%%%%%%%%%%%%%%%%%%%%%%%%%%%%%%%%%%%%%%%%%%%%%%%%%%%%
%%%%%%%%%%%%%%%%%%%%%%%%%%%%%%%%%% BEGINNING OF DOCUMENT %%%%%%%%%%%%%%%%%%%%%%%%%%%%%%%%%%
%%%%%%%%%%%%%%%%%%%%%%%%%%%%%%%%%%%%%%%%%%%%%%%%%%%%%%%%%%%%%%%%%%%%%%%%%%%%%%%%%%%%%%%%%%%
\begin{document}
	
	% \listofnotes
	
	\tableofcontents

	\chapter{Modifying the Martingale Convergence Theorem}
	\section{Definitions and Assumptions}
%
We're considering the estimator
$$\sn{n} = \doublesum\limits_{1\leq i<j\leq n} \phi(Z_{i:n}, Z_{j:n}) W_{i:n} W_{j:n}$$
where 
$$W_{i:n} = \frac{q(Z_{i:n})}{n-i+1}\prod_{k=1}^{i-1}\left[1-\frac{q(Z_{k:n})}{n-k+1}\right]$$
Define $\F_n := \sigma\{Z_{1:n}, \dots, Z_{n:n}, Z_{n+1}, Z_{n+2}, \dots\}$. Furthermore we will need the following definitions in order to get into a framework that is more similar to that of (forward) sub-martingales. Define
$$\tilde{S}^N_n := S_{N-n+1} \textrm{, } \F_n^N := \F_{N-n+1}$$
%
Let $U_n[a,b]$ denote the number of upcrossings of $\tilde{S}_1^N, \dots, \tilde{S}_n^N$ and define 
$$Y_n^N := \StN{1} + \sum_{i=1}^{n-1} \epsilon_i (\StN{i+1} - \StN{i}) $$
with 
\[ \epsilon_i := \begin{cases} 
1 & (\StN{1},\dots, \StN{i}) \in B\\
0 & \textrm{o.w.}
\end{cases}
\]
for some Borel set $B \in  \mathcal{B}(\mathbb{R}^i)$.
%
We can show that 
$$(b-a) \E[U_n[a,b]] \leq \E[Y_n^N] \leq \E[\StN{n}] - \sum_{k=1}^{n-1} \E[(1-\epsilon_k) \E[\StN{k+1} - \StN{k} | \F_k^N]]$$
%
We need to show 
\begin{align}
&\lim\limits_{N\to\infty}(b-a) \E[U_N[a,b]]\nonumber\\
&\leq \lim\limits_{N\to\infty}\E[Y_N^N]\nonumber\\
&\leq \lim\limits_{N\to\infty}\E[\StN{N}] - \sum_{k=1}^{N-1} \E[(1-\epsilon_k) \E[\StN{k+1} - \StN{k} | \F_k^N]]\nonumber\\
&\leq \lim\limits_{N\to\infty}\E[\StN{N}] - \sum_{k=1}^{N-1} \E[(1-\epsilon_k) \E[\StN{k+1} | \F_k^N]  - \StN{k}]\nonumber\\
&<\infty\nonumber
\end{align}
%
So the main concern is to show that the sum of increases of $\StN{k}$ on the right hand side converges. We will need the following assumptions in order to prove the above:
\begin{enumerate}[({A}1)]
	\item The following holds
	$$\int_{0}^{\infty}\int_{0}^{\infty} \phi^2(s,t) H(ds)H(dt) < \infty$$ \label{as:sup_sn}\\
	\item There exists $c_1 \in \R^+$ \st\ $\sup_{x} (q\circ H^{-1})'(x) \leq c_1$. \label{as:sup_qprime}
	\item We have $q\circ H^{-1}(1) = 1$. \label{as:q_H_one}
%	\item There exists $c_3 \in \R^+$ \st\ \label{as:phi_sq}
%	$$\doublesum\limits_{1\leq i<j\leq N-k+1}\E\left[\phi^2(Z_{i:N-k+1}, Z_{j:N-k+1}) W^2_{i:N-k+1} W^2_{j:N-k+1}\right]^{\frac{1}{2}} \leq c_3$$ 
%	\todo{This assumption is not final yet. I need to complete the proof of Lemma \ref{lem:expectation_sq} (last section of this document) in order to formulate the assumption properly.}
\end{enumerate}
%
%Note that (A\ref{as:sup_sn}) implies that 
%$$\int_{0}^{\infty}\int_{0}^{\infty} \phi^2(s,t) H(ds)H(dt) < \infty$$

%\todo{Show the following:}
%\begin{itemize}
%	\item For all $t\in \R^+$ we have $H(t) \leq F(t)$.
%	\item Then
%	$$\int_{0}^{\infty} \phi^2(s,t) H(ds)H(dt) \leq \int_{0}^{\infty} \phi^2(s,t) F(ds)F(dt) < \infty$$
%	and assume 
%	$$\int_{0}^{\infty} \phi^2(s,t) F(ds)F(dt) < \infty$$
%	\item Thus by  Jensen's inequality
%	\begin{align*}
%		\left(\int_{0}^{\infty} \phi(s,t) H(ds)H(dt)\right)^2 &\leq \int_{0}^{\infty} \phi^2(s,t) H(ds)H(dt)\\
%		&\leq \int_{0}^{\infty} \phi^2(s,t) F(ds)F(dt)\\
%		&< \infty
%	\end{align*}
%	\item .. which implies
%	$$\int_{0}^{\infty} \phi(s,t) H(ds)H(dt) < \infty$$
%	and 
%	$$\int_{0}^{\infty} \phi(s,t) F(ds)F(dt) < \infty$$
%\end{itemize}
	\section{Generalized Upcrossing Theorem}
\begin{thm}
	Assume that (A\ref{as:sup_sn}) through (A\ref{as:q_H_one}) hold. Then we have
	\begin{align}
		&\lim\limits_{N\to\infty}(b-a) \E[U_N[a,b]]\nonumber\\
		&\leq \lim\limits_{N\to\infty}\E[Y_N^N]\nonumber\\
		&\leq \lim\limits_{N\to\infty}\E[\StN{N}] - \sum_{k=1}^{N-1} \E[(1-\epsilon_k) \E[\StN{k+1} | \F_k^N]  - \StN{k}]\nonumber\\
		&<\infty\nonumber
	\end{align}
	\label{thm:upcrossing}
\end{thm}
%
\noindent We will first establish all necessary lemmas and then continue with the proof of Theorem \ref{thm:upcrossing} at the end of this section. The following lemma establishes a representation for the conditional expectation under the sum above, that is similar to \cite{dikta2000strong}.
\begin{lemma}
	Define
	\[ Q_{ij}^{n+1} := \begin{cases} 
	Q_i^{n+1} & j\leq n\\
	Q_i^{n+1} - \frac{(n+1)\pi_i \pi_n (1-q(Z_{n:n+1}))}{(n-i+1)(2-q(Z_{n:n+1}))} & j=n+1
	\end{cases}
	\]
	with 
	$$Q_i^{n+1} := (n+1)\left\{\sum_{r=1}^{i-1}\left[\frac{\pi_r}{n-r+2-q(Z_{r:n+1})}\right]^2 + \frac{\pi_i \pi_{i+1}}{n-i+1} \right\}$$
	and 
	$$\pi_i := \prod_{k=1}^{i-1} \left[\frac{n-k+1-q(Z_{k:n+1})}{n-k+2-q(Z_{k:n+1})}\right]$$
	Then 
	$$\E[\sn{n}|\F_{n+1}] = \doublesum\limits_{1\leq i<j\leq n+1} \phi(Z_{i:n+1}, Z_{j:n+1}) W_{i:n+1} W_{j:n+1} Q_{i,j}^{n+1}$$
	\label{lem:qi}
	
	\begin{proof}
		This lemma has been proven in my thesis. We already checked the calculations.
	\end{proof}
\end{lemma}
%
\noindent We will need the following result on the increases of the $Q_i^{n+1}$'s later in the proof of Theorem \ref{thm:upcrossing}. 
\begin{lemma}
	Let $Q_i^{n+1}$ be defined as above. Then
	$$Q_{i+1}^{n+1} - Q_i^{n+1} = \frac{\tilde{\pi}_i^2(n-i+2)^2}{n+1} \left\{\frac{(q_i-q_{i+1})(n-i)(n-i+1) - q_{i+1}(1-q_i)(n-i+1-q_{i})}{(n-i)(n-i+1)(n-i+2-q_i)^2(n-i+1-q_{i+1})}\right\}$$
	where $q_i := q(Z_{i:n+1})$ and 
	$$\tilde{\pi}_i := \pi_i\frac{n+1}{n-i+2}$$
	\label{lem:qi_diff}
	%
	Note that $\tilde{\pi}_i \leq 1$ for all $i\leq n+1$. 
	\begin{proof}
		I proved this lemma in my thesis. 
	\end{proof}
\end{lemma}
%
\begin{lemma}
	Let (A\ref{as:sup_qprime}) be satisfied. Then the following statements hold true for $k\leq n-1$
	\begin{enumerate}[(i)]
		\item We have
		\begin{equation}
			\E[|q(Z_{k:n})-q(Z_{k+1:n})|] \leq \frac{c_1}{n+1}
			\label{eq:q_spacings_a}
		\end{equation}
		\item Furthermore assume that (A\ref{as:q_H_one}) holds. Then
		\begin{equation}
			\E[1-q(Z_{k:n})] \leq \frac{c_1(n-k+1)}{n+1}
			\label{eq:q_spacings_lastk}
		\end{equation}
	\end{enumerate}
	\label{lem:q_spacings}
	
	\begin{proof}
		Let $q_H := q\circ H^{-1}$ and consider that we can write
		\begin{equation}
			q(H^{-1}(x)) = q(H^{-1}(x_0)) + q_H'(\hat{x})(x-x_0)
			\label{eq:taylor_q}
		\end{equation}
		using Taylor expansion for some $\hat{x}$ in between $x$ and $x_0$. Therefore we have 
		$$q(H^{-1}(x)) - q(H^{-1}(x_0)) = q_H'(\hat{x})(x-x_0)$$
		and hence
		\begin{equation}
			|q(H^{-1}(x)) - q(H^{-1}(x_0))| = |q_H'(\hat{x})|\cdot |x-x_0|
			\label{eq:q_taylor}
		\end{equation}
		%
		Now let $U_1, \dots,U_n$ be \iid\ $Uni[0,1]$ and set $x=U_{k:n}$ and $x_0=U_{k+1:n}$ to get
		\begin{equation*}
			\E[|q(H^{-1}(U_{k:n})) - q(H^{-1}(U_{k+1:n}))|] = \E[|q(Z_{k:n}) - q(Z_{k+1:n})|]
		\end{equation*}
		%
		Thus we get from \eqref{eq:q_taylor}
		\begin{equation*}
			\E[|q(Z_{k:n}) - q(Z_{k+1:n})|] = E[|q_H'(\hat{x})|\cdot (U_{k+1:n} - U_{k:n})]
		\end{equation*}
		where $\hat{x} \in [U_{k:n}, U_{k+1:n}]$.
		%
		From assumption (A\ref{as:sup_qprime}) directly follows that $\abs{q_H'(x)}\leq c_1$ for all $x \in [0,1]$. Hence we have
		$$\E[|q(Z_{k:n}) - q(Z_{k+1:n})|] = c_1 \E[U_{k+1:n} - U_{k:n}]$$
		%
		According to \cite{shorack2009empirical} (p. 271), we have
		\begin{equation}
			\E[U_{k+1:n} - U_{k:n}] = \frac{1}{n+1}
			\label{eq:u_spacings}
		\end{equation}
		%
		Therefore we may conclude
		\begin{align*}
			\E[|q(Z_{k:n}) - q(Z_{k+1:n})|] &\leq c_1\E[U_{k+1:n} - U_{k:n}]\\
			&= \frac{c_1}{n+1} \numberthis \label{eq:q_diff_a}
		\end{align*}
		This completes the proof part (i). 
		%
		We will now continue with the proof of part (ii). Consider 
		\begin{align*}
			1-q(Z_{k:n}) &=  1 - q(Z_{n:n}) + \sum_{l=k}^{n-1}(q(Z_{l+1:n}) - q(Z_{l:n}))\\
			&\leq 1 - q(Z_{n:n}) + \sum_{l=k}^{n-1}\abs{q(Z_{l+1:n}) - q(Z_{l:n})}
		\end{align*}
		%
		Taking expectations on each side yields
		\begin{equation*}
			1 - \E[q(Z_{k:n})] \leq 1 - \E[q(Z_{n:n})] + \sum_{l=k}^{n-1}\E[\abs{q(Z_{l+1:n}) - q(Z_{l:n})}]
		\end{equation*}	
		%
		Now we apply inequality \eqref{eq:q_diff_a} to the expectation under the sum to get 
		\begin{equation}
			1 - \E[q(Z_{k:n})] \leq 1 - \E[q(Z_{n:n})] + \frac{c_1(n-k)}{n+1}
			\label{eq:q_k_upperbnd_alt}
		\end{equation}
		%
		Recall the Taylor expansion from above
		\begin{equation*}
			q(H^{-1}(x)) = q(H^{-1}(x_0)) + q_H'(\hat{x})(x-x_0)
		\end{equation*}	
		%
		Setting $x = 1$ and $x_0 = U_{n:n}$ and taking expectations on both sides yields
		\begin{equation*}
			\E[q(H^{-1}(1))] = \E[q(Z_{n:n})] + \E[q_H'(\hat{x})(1 - U_{n:n})]
		\end{equation*}
		where $\hat{x} \in [U_{n:n}, 1]$ 
		%
		Now we get from assumption (A\ref{as:sup_qprime}) that
		\begin{align*}
			\E[q(Z_{n:n})] &= \E[q(H^{-1}(1))] - \E[q_H'(\hat{x})(1 - U_{n:n})]\\
			&\geq \E[q(H^{-1}(1))] - c_1\E[1 - U_{n:n}]\\
		\end{align*}
		%
		Using \cite{shorack2009empirical} (p. 271) again, we obtain
		\begin{align*}
			\E[q(Z_{n:n})]
			&= \E[q(H^{-1}(1))] - \frac{c_1}{n+1}\\
		\end{align*}
		%
		Applying (A\ref{as:q_H_one}) yields
		\begin{equation*}
			\E[q(Z_{n:n})]  \geq 1 - \frac{c_1}{n+1}
		\end{equation*}
		%
		By combining the above with \eqref{eq:q_k_upperbnd_alt} we get
		\begin{equation*}
			1 - \E[q(Z_{k:n})] \leq 1 - 1 + \frac{c_1}{n+1} + \frac{c_1(n-k)}{n+1} = \frac{c_1(n-k+1)}{n+1} 
		\end{equation*}	
		%
		This concludes the proof of part (ii). 
	\end{proof}
\end{lemma}
%
\noindent The following lemma contains some upper bounds that will be needed later in the proof of Theorem \ref{thm:upcrossing}.
\begin{lemma}
	For $n\geq 2$ the following statements hold true
	\begin{enumerate}[(i)]
		\item \begin{equation}
			\sum_{k=1}^{n-1} \frac{1}{k} \leq \ln(n-1) + 1
			\label{eq:sum_ln}
		\end{equation}
		%
		\item \begin{equation}
			\frac{\ln(n-1)+1}{(n+1)^{\frac{1}{3}}} \leq 3
			\label{eq:ln_over_n_upperb}
		\end{equation}
	\end{enumerate}
	\label{lem:bounds}
	
	\begin{proof}
		We will start with the proof of part (i). Consider 
		\begin{align*}
			& \quad\sum_{k=1}^{n-1} \frac{1}{k} \quad \leq \quad  \ln(n-1) + 1 \\
			\Leftrightarrow & \quad \sum_{k=1}^{n-1} \frac{1}{k} - 1 \quad \leq \quad  \ln(n-1)\\
			\Leftrightarrow & \quad \sum_{k=2}^{n-1} \frac{1}{k} \quad \leq \quad  \ln(n-1)
			\numberthis\label{eq:prodexp}
		\end{align*}
		%
		Moreover we have
		\begin{align*}
			\sum_{k=2}^{n-1} \frac{1}{k} &= \sum_{k=2}^{n-1} \int_{k-1}^{k}\frac{1}{k} dx\\
			&\leq \sum_{k=2}^{n-1} \int_{k-1}^{k}\frac{1}{x} dx\\
			&\leq \sum_{k=2}^{n-1} \ln(k) - \ln(k-1)\\
			&\leq \ln(n-1) - \ln(1)\\
			&= \ln(n-1)
		\end{align*}
		Thus proving part (i). 
		%
		We will continue with the proof of part (ii). Note that \eqref{eq:ln_over_n_upperb} is equivalent to showing 
		\begin{equation*}
			\ln(n-1)+1 \leq 3(n+1)^{\frac{1}{3}}
		\end{equation*}
		%
		Since $\ln(n-1)\leq \ln(n+1)$, this will be implied by the following
		\begin{equation}
			\ln(n+1)+1 \leq 3(n+1)^{\frac{1}{3}}
			\label{eq:ln_root}
		\end{equation}
		It is easy to check that inequality \eqref{eq:ln_root} holds for $n=2$. Now consider that 
		$$\frac{d}{dn}(\ln(n+1)+1) = \frac{1}{n+1}$$
		and
		$$\frac{d}{dn} 3(n+1)^\frac{1}{3} = \frac{1}{(n+1)^\frac{2}{3}}$$
		to get
		\begin{equation}
		\frac{d}{dn}(\ln(n+1)+1) \leq \frac{d}{dn} 3(n+1)^\frac{1}{3}
		\label{eq:derivs}
		\end{equation}	
		for all $n\geq 2$. Now the result in (ii) follows directly from \eqref{eq:ln_root} and \eqref{eq:derivs} .
	\end{proof}
\end{lemma}
%
\noindent Now we established everything we need in order to proceed with the proof of Theorem \ref{thm:upcrossing}. Recall that we need to show 
\begin{equation*}
	\lim\limits_{N\to\infty}(b-a) \E[U_N[a,b]] <\infty\nonumber
\end{equation*}
%
\begin{proof}[\textbf{Proof of Theorem 1}]
	Let (A\ref{as:sup_sn}) through (A\ref{as:q_H_one}) be satisfied. Recall the following inequality (proven in my thesis). We have for $n\leq N$
	$$(b-a) \E[U_n[a,b]] \leq \E[Y_n^N] \leq \E[\StN{n}] - \sum_{k=1}^{n-1} \E[(1-\epsilon_k) \E[\StN{k+1} | \F_k^N] - \StN{k}]$$
	%
	Moreover we get from Lemma \ref{lem:qi}
	\begin{align*}
	\E[\StN{k+1}|\FtN{k}] &= \E[\sn{N-k} | \F_{N-k+1}]\\
	&= \doublesum\limits_{1\leq i<j\leq N-k+1} \phi(Z_{i:N-k+1}, Z_{j:N-k+1}) W_{i:N-k+1} W_{j:N-k+1} Q_{i,j}^{N-k+1}
	\end{align*}
	%
	Therefore we get
	\begin{align*}
		\E[Y_N^N] &\leq \E[\StN{N}] - \sum_{k=1}^{N-1} \E[(1-\epsilon_k) \E[\StN{k+1} | \F_k^N] - \StN{k} ]\\
		&=  \E[\StN{N}] - \sum_{k=1}^{N-1} \E\left[(1-\epsilon_k) \doublesum\limits_{1\leq i<j\leq N-k+1} \phi(Z_{i:N-k+1}, Z_{j:N-k+1}) \right. \\
		&\qquad\qquad\qquad\qquad\qquad\qquad \times \left. W_{i:N-k+1} W_{j:N-k+1}(Q_{i,j}^{N-k+1} - 1) \vphantom{\doublesum\limits_{1\leq i<j\leq N-k+1}}\right]\\
		&=  \E[\StN{N}] - \sum_{k=1}^{N-1} \doublesum\limits_{1\leq i<j\leq N-k+1} \E\left[(1-\epsilon_k)\phi(Z_{i:N-k+1}, Z_{j:N-k+1}) \right. \\
		&\qquad\qquad\qquad\qquad\qquad\qquad \times \left. W_{i:N-k+1} W_{j:N-k+1}(Q_{i,j}^{N-k+1} - 1) \right]\\
		&\leq  \E[\StN{N}] + \left|\sum_{k=1}^{N-1} \doublesum\limits_{1\leq i<j\leq N-k+1} \E\left[(1-\epsilon_k)\phi(Z_{i:N-k+1}, Z_{j:N-k+1}) \right.\right. \\
		&\qquad\qquad\qquad\qquad\qquad\qquad \times \left.\left. W_{i:N-k+1} W_{j:N-k+1}(Q_{i,j}^{N-k+1} - 1) \right]\vphantom{\doublesum\limits_{1\leq i<j\leq N-k+1}}\right|\\
		&\leq  \E[\StN{N}] + \sum_{k=1}^{N-1} \doublesum\limits_{1\leq i<j\leq N-k+1} \left|\E\left[(1-\epsilon_k)\phi(Z_{i:N-k+1}, Z_{j:N-k+1}) \right.\right. \\
		&\qquad\qquad\qquad\qquad\qquad\qquad \times \left.\left. W_{i:N-k+1} W_{j:N-k+1}(Q_{i,j}^{N-k+1} - 1) \right]\right|
	\end{align*}
	%
	Now using Jensen's inequality yields
	\begin{align*}
		\E[Y_N^N] &\leq \E[\StN{N}] + \sum_{k=1}^{N-1} \doublesum\limits_{1\leq i<j\leq N-k+1}\E\left[(1-\epsilon_k) \phi(Z_{i:N-k+1}, Z_{j:N-k+1}) \right.\\
		&\qquad\qquad\qquad\qquad\qquad\qquad \times \left. W_{i:N-k+1} W_{j:N-k+1} \abs{(Q_{i,j}^{N-k+1} - 1)}\right]\\
		&\leq \E[\StN{N}] + \sum_{k=1}^{N-1} \doublesum\limits_{1\leq i<j\leq N-k+1}\E\left[\phi(Z_{i:N-k+1}, Z_{j:N-k+1}) \right. \\
		&\qquad\qquad\qquad\qquad\qquad\qquad \times \left. W_{i:N-k+1} W_{j:N-k+1} \abs{(Q_{i,j}^{N-k+1} - 1)}\right]\\
	\end{align*}
	The latter inequality above holds, because $1-\epsilon_k \leq 1$ for all $k\leq N-1$. 
	%
	By applying the Cauchy-Schwarz inequality on the expectation above, we obtain
	\begin{align*}
		\E[Y_N^N] &\leq \E[\StN{N}] + \sum_{k=1}^{N-1} \doublesum\limits_{1\leq i<j\leq N-k+1}\E\left[\phi^2(Z_{i:N-k+1}, Z_{j:N-k+1}) W^2_{i:N-k+1} W^2_{j:N-k+1}\right]^{\frac{1}{2}}\\
		&\qquad\qquad\qquad\qquad\qquad\qquad \times \E\left[(Q_{i,j}^{N-k+1} - 1)^2\right]^{\frac{1}{2}}\numberthis\label{eq:yn}
	\end{align*}
	%
	We will now proceed to find an upper bound for $\E\left[(Q_{i,j}^{N-k+1} - 1)^2\right]^{\frac{1}{2}}$. For the purpose of simpler notation we set $n := n(k,N) = N-k$. The inequality above can now be written as 
	\begin{align*}
		\E[Y_N^N] &\leq \E[S_{1}] + \sum_{k=1}^{N-1} \doublesum\limits_{1\leq i<j\leq n+1}\E\left[\phi^2(Z_{i:n+1}, Z_{j:n+1}) W^2_{i:n+1} W^2_{j:n+1}\right]^{\frac{1}{2}}\\
		&\qquad\qquad\qquad\qquad\qquad \times \E\left[(Q_{i,j}^{n+1} - 1)^2\right]^{\frac{1}{2}}
	\end{align*}
	%
	\textbf{Note} $k_1$ and $k_2$ below do \textbf{not} correspond to $k$ above in any way. Consider 
	\begin{equation*}
		Q_i^{n+1} - 1 = Q_1^{n+1} + \sum_{k_1=1}^{i-1} (Q_{k_1+1}^{n+1} - Q_{k_1}^{n+1}) - 1 \numberthis \label{eq:qi_sum}
	\end{equation*}
	and recall the following definition
	$$Q_i^{n+1} := (n+1)\left\{\sum_{r=1}^{i-1}\left[\frac{\pi_r}{n-r+2-q(Z_{r:n+1})}\right]^2 + \frac{\pi_i \pi_{i+1}}{n-i+1} \right\}$$
	where
	$$\pi_i := \prod_{k=1}^{i-1} \left[\frac{n-k+1-q(Z_{k:n+1})}{n-k+2-q(Z_{k:n+1})}\right]$$
	%
	We have $\pi_1 = 1$, since the product above is empty for $i=1$ and 
	$$\pi_2 = \frac{n-q(Z_{1:n+1})}{n+1-q(Z_{1:n+1})}$$ 
	Thus we get
	\begin{align*}
		Q_1^{n+1} - 1 &= (n+1)\frac{\pi_1 \pi_2}{n} - 1\\
		&= \frac{(n+1)(n-q(Z_{1:n+1}))}{n(n+1-q(Z_{1:n+1}))} - 1\\
		&= \frac{n(n+1-q(Z_{1:n+1}))-q(Z_{1:n+1})}{n(n+1-q(Z_{1:n+1}))} - 1\\
		&= 1 - \frac{q(Z_{1:n+1})}{n(n+1-q(Z_{1:n+1}))} - 1\\
		&= -\frac{q(Z_{1:n+1})}{n(n+1-q(Z_{1:n+1}))}\\
	\end{align*}
	%
	Therefore we get from \eqref{eq:qi_sum} 
	\begin{equation*}
		Q_i^{n+1} - 1 = \sum_{k_1=1}^{i-1} (Q_{k_1+1}^{n+1} - Q_{k_1}^{n+1}) - \frac{q(Z_{1:n+1})}{n(n+1-q(Z_{1:n+1}))}
	\end{equation*}
	%
	Moreover we have 
	\begin{align*}
		(Q_i^{n+1} - 1)^2 &= \sum_{k_1=1}^{i-1}\sum_{k_2=1}^{i-1}(Q_{k_1+1}^{n+1} - Q_{k_1}^{n+1})(Q_{k_2+1}^{n+1} - Q_{k_2}^{n+1})\\
		 &\qquad - \frac{2q(Z_{1:n+1})}{n(n+1-q(Z_{1:n+1}))} \sum_{k=1}^{i-1}(Q_{k_1+1}^{n+1} - Q_{k_1}^{n+1})\\
		 &\qquad + \frac{q^2(Z_{1:n+1})}{n^2(n+1-q(Z_{1:n+1}))^2}\\
		 &\leq \sum_{k_1=1}^{i-1}\sum_{k_2=1}^{i-1}|Q_{k_1+1}^{n+1} - Q_{k_1}^{n+1}|\cdot|Q_{k_2+1}^{n+1} - Q_{k_2}^{n+1}|\\
		 &\qquad + \frac{2q(Z_{1:n+1})}{n(n+1-q(Z_{1:n+1}))} \sum_{k_1=1}^{i-1}|Q_{k_1+1}^{n+1} - Q_{k_1}^{n+1}|\\
		 &\qquad + \frac{q^2(Z_{1:n+1})}{n^2(n+1-q(Z_{1:n+1}))^2}\\
	 	 &\leq \sum_{k_1=1}^{i-1}\sum_{k_2=1}^{i-1}|Q_{k_1+1}^{n+1} - Q_{k_1}^{n+1}|\cdot|Q_{k_2+1}^{n+1} - Q_{k_2}^{n+1}|\\
		 &\qquad + \frac{2}{n^2} \sum_{k_1=1}^{i-1}|Q_{k_1+1}^{n+1} - Q_{k_1}^{n+1}| + \frac{1}{n^4} \numberthis\label{eq:qiminusonesq}
	\end{align*}
	%
	Remember that we set $q_i := q(Z_{i:n+1})$. We get from Lemma \ref{lem:qi_diff} that
	\begin{align*}
		&|Q_{i+1}^{n+1} - Q_i^{n+1}| \\
		&= \frac{\tilde{\pi}_i^2(n-i+2)^2}{n+1} \cdot \left|\frac{(q_i-q_{i+1})(n-i)(n-i+1) - q_{i+1}(1-q_i)(n-i+1-q_{i})}{(n-i)(n-i+1)(n-i+2-q_i)^2(n-i+1-q_{i+1})}\right|\\
		&\leq \frac{\tilde{\pi}_i^2(n-i+2)^2}{n+1} \cdot \frac{|q_i-q_{i+1}|(n-i)(n-i+1) + q_{i+1}(1-q_i)(n-i+1-q_{i})}{(n-i)(n-i+1)(n-i+2-q_i)^2(n-i+1-q_{i+1})}\\
		&\leq \frac{(n-i+2)^2}{n+1} \left\{\frac{\abs{q_i-q_{i+1}}(n-i)(n-i+1)+ q_{i+1}(1-q_i)(n-i+1)}{(n-i)(n-i+1)(n-i+1)^2(n-i)}\right\}\\
		&= \frac{(n-i+2)^2}{n+1} \left\{\frac{\abs{q_i-q_{i+1}}(n-i)+ q_{i+1}(1-q_i)}{(n-i)^2(n-i+1)^2}\right\}\\
		&\leq \frac{4\abs{q_i-q_{i+1}}}{(n+1)(n-i)} + \frac{4(1-q_i)}{(n+1)(n-i)^2}\numberthis\label{eq:qi_diff}
	\end{align*}
	The latter inequality above holds since 
	$$\frac{n-i+2}{n-i+1} = 1 + \frac{1}{n-i+1} \leq 2$$
	and $q_{i+1} \leq 1$.
	%
	Thus we have
	\begin{align*}
		&|Q_{k_1+1}^{n+1} - Q_{k_1}^{n+1}|\cdot|Q_{k_2+1}^{n+1} - Q_{k_2}^{n+1}|\\ 
		&\leq \left[\frac{4\abs{q_{k_1}-q_{k_1+1}}}{(n+1)(n-k_1)} + \frac{4(1-q_{k_1})}{(n+1)(n-k_1)^2}\right] \\
		&\qquad \times \left[\frac{4\abs{q_{k_2}-q_{k_2+1}}}{(n+1)(n-k_2)} + \frac{4(1-q_{k_2})}{(n+1)(n-k_2)^2}\right]\\
		&= \frac{16\abs{q_{k_1}-q_{k_1+1}}\abs{q_{k_2}-q_{k_2+1}} }{(n+1)^2(n-k_1)(n-k_2)} + \frac{16\abs{q_{k_1}-q_{k_1+1}}(1-q_{k_2})}{(n+1)^2(n-k_1)(n-k_2)^2}\\
		&\qquad + \frac{16(1-q_{k_1})\abs{q_{k_2}-q_{k_2+1}}}{(n+1)^2(n-k_1)^2(n-k_2)} + \frac{16(1-q_{k_1})(1-q_{k_2})}{(n+1)^2(n-k_1)^2(n-k_2)^2}\\
		&\leq \frac{16\abs{q_{k_1}-q_{k_1+1}}}{(n+1)^2(n-k_1)(n-k_2)} + \frac{16\abs{q_{k_1}-q_{k_1+1}}}{(n+1)^2(n-k_1)(n-k_2)^2}\\
		&\qquad + \frac{16\abs{q_{k_2}-q_{k_2+1}}}{(n+1)^2(n-k_1)^2(n-k_2)} + \frac{16(1-q_{k_1})}{(n+1)^2(n-k_1)^2(n-k_2)^2}
	\end{align*}
	Here the latter inequality holds, since we have $\abs{q_{k}-q_{k+1}} \leq 1$ and $1-q_{k} \leq 1$ for all $k\leq n-1$. \\
	\\
	Recall that 
	\begin{align*}
	(Q_i^{n+1} - 1)^2 &\leq \sum_{k_1=1}^{i-1}\sum_{k_2=1}^{i-1}\abs{Q_{k_1+1}^{n+1} - Q_{k_1}^{n+1}}\abs{Q_{k_2+1}^{n+1} - Q_{k_2}^{n+1}}\\
	&\qquad + \frac{2}{n^2} \sum_{k_1=1}^{i-1}\abs{Q_{k_1+1}^{n+1} - Q_{k_1}^{n+1}} + \frac{1}{n^4} 
	\end{align*}
	%
	Taking expectations on each side yields
	\begin{align*}
		 \E[(Q_i^{n+1} - 1)^2] &\leq \sum_{k_1=1}^{i-1}\sum_{k_2=1}^{i-1}\E[\abs{Q_{k_1+1}^{n+1} - Q_{k_1}^{n+1}}\abs{Q_{k_2+1}^{n+1} - Q_{k_2}^{n+1}}]\\
		&\qquad + \frac{2}{n^2} \sum_{k_1=1}^{i-1}\E[|Q_{k_1+1}^{n+1} - Q_{k_1}^{n+1}|] + \frac{1}{n^4} \numberthis\label{eq:qiminusonesq_exp}
	\end{align*}
	%
	Consider the expectation under the double sum above. We have 
	\begin{align*}
		&\E\left[\abs{Q_{k_1+1}^{n+1} - Q_{k_1}^{n+1}}\abs{Q_{k_2+1}^{n+1} - Q_{k_2}^{n+1}}\right]\\ 
		&\leq \frac{16\E\left[\abs{q_{k_1}-q_{k_1+1}}\right]}{(n+1)^2(n-k_1)(n-k_2)} + \frac{16\E\left[\abs{q_{k_1}-q_{k_1+1}}\right]}{(n+1)^2(n-k_1)(n-k_2)^2}\\
		&\qquad + \frac{16\E\left[\abs{q_{k_2}-q_{k_2+1}}\right]}{(n+1)^2(n-k_1)^2(n-k_2)} + \frac{16\E\left[(1-q_{k_1})\right]}{(n+1)^2(n-k_1)^2(n-k_2)^2} \numberthis\label{eq:qiminusonesq}
	\end{align*}
	%
	We will now use Lemma \ref{lem:q_spacings} to establish an upper bound for the expectation above. Combining \eqref{eq:q_spacings_a} and \eqref{eq:q_spacings_lastk} above with \eqref{eq:qiminusonesq} yields
	\begin{align*}
		&\E[\abs{Q_{k_1+1}^{n+1} - Q_{k_1}^{n+1}}\abs{Q_{k_2+1}^{n+1} - Q_{k_2}^{n+1}}] \\
		&\leq \frac{16c_1}{(n+1)^3(n-k_1)(n-k_2)} + \frac{16c_1}{(n+1)^3(n-k_1)(n-k_2)^2}\\
		&\qquad + \frac{16c_1}{(n+1)^3(n-k_1)^2(n-k_2)} + \frac{16c_1(n-k_1) + 16c_1}{(n+1)^3(n-k_1)^2(n-k_2)^2}
	\end{align*}
	%
	Therefore we obtain
	\begin{align*}
		&\sum_{k_1=1}^{i-1}\sum_{k_2=1}^{i-1}\E[\abs{Q_{k_1+1}^{n+1} - Q_{k_1}^{n+1}}\abs{Q_{k_2+1}^{n+1} - Q_{k_2}^{n+1}}]\\
		&\leq \sum_{k_1=1}^{i-1}\sum_{k_2=1}^{i-1}\frac{16c_1}{(n+1)^3(n-k_1)(n-k_2)} + \sum_{k_1=1}^{i-1}\sum_{k_2=1}^{i-1}\frac{16c_1}{(n+1)^3(n-k_1)(n-k_2)^2}\\
		&\qquad + \sum_{k_1=1}^{i-1}\sum_{k_2=1}^{i-1}\frac{16c_1}{(n+1)^3(n-k_1)^2(n-k_2)} + \sum_{k_1=1}^{i-1}\sum_{k_2=1}^{i-1}\frac{16c_1(n-k_1) }{(n+1)^3(n-k_1)^2(n-k_2)^2}\\
		&\qquad + \sum_{k_1=1}^{i-1}\sum_{k_2=1}^{i-1}\frac{16c_1}{(n+1)^3(n-k_1)^2(n-k_2)^2}\\
		&= \sum_{k_1=1}^{i-1}\sum_{k_2=1}^{i-1}\frac{16c_1}{(n+1)^3(n-k_1)(n-k_2)} + \sum_{k_1=1}^{i-1}\sum_{k_2=1}^{i-1}\frac{32c_1}{(n+1)^3(n-k_1)(n-k_2)^2}\\
		&\qquad + \sum_{k_1=1}^{i-1}\sum_{k_2=1}^{i-1}\frac{16c_1}{(n+1)^3(n-k_1)^2(n-k_2)} + \sum_{k_1=1}^{i-1}\sum_{k_2=1}^{i-1}\frac{16c_1}{(n+1)^3(n-k_1)^2(n-k_2)^2}\\
		&\leq \frac{16c_1}{(n+1)^3}\sum_{k_1=1}^{i-1}\frac{1}{(n-k_1)}\sum_{k_2=1}^{i-1}\frac{1}{(n-k_2)} + \frac{32c_1}{(n+1)^3}\sum_{k_1=1}^{i-1}\frac{1}{n-k_1}\sum_{k_2=1}^{i-1}\frac{1}{(n-k_2)^2}\\
		&\qquad + \frac{16c_1}{(n+1)^3}\sum_{k_1=1}^{i-1}\frac{1}{(n-k_1)^2}\sum_{k_2=1}^{i-1}\frac{1}{n-k_2} + \frac{16c_1}{(n+1)^3}\sum_{k_1=1}^{i-1}\frac{1}{(n-k_1)^2}\sum_{k_2=1}^{i-1}\frac{1}{(n-k_2)^2}\\
		&\leq \frac{16c_1}{(n+1)^3}\sum_{k_1=n-i+1}^{n-1}\frac{1}{k_1}\sum_{k_2=n-i+1}^{n-1}\frac{1}{k_2} + \frac{32c_1}{(n+1)^3}\sum_{k_1=n-i+1}^{n-1}\frac{1}{k_1}\sum_{k_2=n-i+1}^{n-1}\frac{1}{k_2^2}\\
		&\qquad + \frac{16c_1}{(n+1)^3}\sum_{k_1=n-i+1}^{n-1}\frac{1}{k_1^2}\sum_{k_2=n-i+1}^{n-1}\frac{1}{k_2} + \frac{16c_1}{(n+1)^3}\sum_{k_1=n-i+1}^{n-1}\frac{1}{k_1^2}\sum_{k_2=n-i+1}^{n-1}\frac{1}{k_2^2} \numberthis \label{eq:sumqk}
	\end{align*}
	%
	Now using \eqref{eq:sum_ln} and \eqref{eq:ln_over_n_upperb} from Lemma \ref{lem:bounds} on inequality \eqref{eq:sumqk} yields 
	\begin{align*}
		&\sum_{k_1=1}^{i-1}\sum_{k_2=1}^{i-1}\E[\abs{Q_{k_1+1}^{n+1} - Q_{k_1}^{n+1}}\abs{Q_{k_2+1}^{n+1} - Q_{k_2}^{n+1}}]\\
		&\leq \frac{16c_1}{(n+1)^3}(\ln(n-1)+1)^2 + \frac{64c_1}{(n+1)^3}(\ln{(n-1)}+1)\\
		&\qquad + \frac{32c_1}{(n+1)^3}(\ln{(n-1)}+1) + \frac{64c_1}{(n+1)^3}\\
		&\leq \frac{144c_1}{(n+1)^\frac{7}{3}}+ \frac{288c_1}{(n+1)^\frac{8}{3}} + \frac{64c_1}{(n+1)^3}\\
		&\leq \frac{496c_1}{(n+1)^\frac{7}{3}} \numberthis \label{eq:first_sum}
	\end{align*}
	%
	We will now proceed with the second sum in \eqref{eq:qiminusonesq_exp}. We get from \eqref{eq:qi_diff}
	\begin{equation*}
		\E[\abs{Q_{i+1}^{n+1} - Q_i^{n+1}}] \leq \frac{4\E[\abs{q_i-q_{i+1}}]}{(n+1)(n-i)} + \frac{4\E[1-q_i]}{(n+1)(n-i)^2}
	\end{equation*}
	%
	Therefore we obtain
	\begin{equation*}
		\frac{2}{n^2} \sum_{k_1=1}^{i-1}\E[|Q_{k_1+1}^{n+1} - Q_{k_1}^{n+1}|] \leq \frac{8}{n^2(n+1)}\sum_{k_1=1}^{i-1}\frac{\E[\abs{q_{k_1}-q_{k_1+1}}]}{n-k_1} + \frac{\E[1-q_{k_1}]}{(n-k_1)^2}
	\end{equation*}
	%
	Again using \eqref{eq:q_spacings_a} and \eqref{eq:q_spacings_lastk} reveals
	\begin{align*}
		\frac{2}{n^2} \sum_{k_1=1}^{i-1}\E[|Q_{k_1+1}^{n+1} - Q_{k_1}^{n+1}|] &\leq  \frac{8}{n^2(n+1)^2}\left\{\sum_{k_1=1}^{i-1}\frac{c_1}{(n-k_1)} + \sum_{k_1=1}^{i-1}\frac{c_1(n-k_1+1)}{(n-k_1)^2}\right\}\\
		&=  \frac{8}{n^2(n+1)^2}\left\{2\sum_{k_1=1}^{i-1}\frac{c_1}{(n-k_1)} + \sum_{k_1=1}^{i-1}\frac{c_1}{(n-k_1)^2}\right\}\\
		&=  \frac{8}{n^2(n+1)^2}\left\{2\cdot\sum_{k_1=n-i+1}^{n-1}\frac{c_1}{k_1} + \sum_{k_1=n-i+1}^{n-1}\frac{c_1}{k_1^2}\right\}
	\end{align*}
	%
	By using \eqref{eq:sum_ln} and \eqref{eq:ln_over_n_upperb} again we obtain
	\begin{align*}
		\frac{2}{n^2} \sum_{k_1=1}^{i-1}\E[|Q_{k_1+1}^{n+1} - Q_{k_1}^{n+1}|] &\leq  \frac{8\cdot\{2c_1(\ln(n-1)+1)+2c_1\}}{n^2(n+1)^2}\\
		&=  \frac{16c_1(\ln(n-1)+1)}{n^2(n+1)^2} + \frac{16c_1}{n^2(n+1)^2}\\
		&\leq  \frac{48c_1}{n^2(n+1)^\frac{5}{3}} + \frac{16c_1}{n^2(n+1)^2}\\
		&\leq  \frac{64c_1}{n^2(n+1)^\frac{5}{3}} \numberthis\label{eq:second_sum}
	\end{align*}
	%
	Again recall the following fact 
	\begin{align*}
		\E[(Q_i^{n+1} - 1)^2] &= \sum_{k_1=1}^{i-1}\sum_{k_2=1}^{i-1}\E[\abs{Q_{k_1+1}^{n+1} - Q_{k_1}^{n+1}}\abs{Q_{k_2+1}^{n+1} - Q_{k_2}^{n+1}}]\\
		&\qquad + \frac{2}{n^2} \sum_{k_1=1}^{i-1}\E[|Q_{k_1+1}^{n+1} - Q_{k_1}^{n+1}|] + \frac{1}{n^4}
	\end{align*}
	%
	Combining the above with \eqref{eq:first_sum} and \eqref{eq:second_sum} yields
	\begin{align*}
		\E[(Q_i^{n+1} - 1)^2] &\leq \frac{496c_1}{(n+1)^\frac{7}{3}}+ \frac{64c_1}{n^2(n+1)^\frac{5}{3}} + \frac{1}{n^4}\\
		&\leq \frac{496c_1}{n^\frac{7}{3}} + \frac{64c_1}{n^\frac{11}{3}} + \frac{1}{n^4}\\
		&\leq \frac{1}{n^\frac{7}{3}}\left[496c_1 + \frac{64c_1}{n^\frac{4}{3}} + \frac{1}{n^\frac{5}{3}}\right]\\
		&\leq \frac{560c_1+1}{n^\frac{7}{3}}\\
		&= \frac{c_2}{n^\frac{7}{3}}
	\end{align*}
	with $c_2 := 560c_1+1$. Therefore
	\begin{equation*}
		\E[(Q_i^{n+1} - 1)^2]^\frac{1}{2} \leq \frac{\sqrt{c_2}}{n^\frac{7}{6}}
	\end{equation*}
	%
	Recall that we set $n=N-k$. Thus we can write 
	\begin{equation*}
		\E[(Q_i^{N-k+1} - 1)^2]^\frac{1}{2} \leq \frac{\sqrt{c_2}}{(N-k)^\frac{7}{6}}
	\end{equation*}
	%
	Now combining the latter with \eqref{eq:yn} yields 
	\begin{align*}
		\E[Y_N^N] &\leq  \E[\StN{N}] + \sum_{k=1}^{N-1} \doublesum\limits_{1\leq i<j\leq N-k+1}\E\left[\phi^2(Z_{i:N-k+1}, Z_{j:N-k+1}) W^2_{i:N-k+1} W^2_{j:N-k+1}\right]^{\frac{1}{2}}\\
		&\qquad\qquad\qquad\qquad\qquad\qquad \times \E\left[(Q_{i,j}^{N-k+1} - 1)^2\right]^{\frac{1}{2}}\\
		&\leq \E[\StN{N}] + \sum_{k=1}^{N-1} \doublesum\limits_{1\leq i<j\leq N-k+1}\E\left[\phi^2(Z_{i:N-k+1}, Z_{j:N-k+1}) W^2_{i:N-k+1} W^2_{j:N-k+1}\right]^{\frac{1}{2}}\\
		&\qquad\qquad\qquad\qquad\qquad\qquad \times \frac{\sqrt{c_2}}{(N-k)^\frac{7}{6}}
	\end{align*}
	%
	Thus it remains to show that 
	$$\doublesum\limits_{1\leq i<j\leq N-k+1}\E\left[\phi^2(Z_{(i)}, Z_{(j)}) W^2_{(i)} W^2_{(j)}\right]^{\frac{1}{2}} \leq c_3 < \infty$$
	is bounded above. Then we would have 
	\begin{align*}
		\lim\limits_{N\to\infty}\E[U_N[a,b]]
		&\leq \lim\limits_{N\to\infty}\E[Y_N^N]\\
		&\leq \lim\limits_{N\to\infty} \left\{\E[\StN{N}] + c_3\sum_{k=1}^{N-1}\frac{\sqrt{c_2}}{(N-k)^\frac{7}{6}}\right\}\\
		&\leq \sup\limits_{N}\E[\StN{N}] + \sqrt{c_2}c_3 \left\{\lim\limits_{N\to\infty}\sum_{k=1}^{N-1}\frac{1}{(N-k)^\frac{7}{6}}\right\}\\
		&< \infty
	\end{align*}
	%
	And therefore we may finally conclude that $S = \lim_{n\to\infty} S_n$ exists. Note that there is more argumentation about the relationship between $U_N[a,b]$ and $\lim\limits_{n\to\infty} S_n$ in my thesis.\\
\end{proof}
	\section{The missing bound}
	It remains to show that 
	$$\doublesum\limits_{1\leq i<j\leq N-k+1}\E\left[\phi^2(Z_{i:N-k+1}, Z_{j:N-k+1}) W^2_{i:N-k+1} W^2_{j:N-k+1}\right]^{\frac{1}{2}}$$
	is bounded above.	
	%
	For the sake of simplicity we will set $n=N-k+1$ again. The following lemma contains the result needed to prove Theorem \ref{thm:upcrossing}.
\begin{lemma}
	Suppose (A\ref{as:sup_sn}) holds. Then there exists $c_3 < \infty$ \st\ 
	$$\doublesum\limits_{1\leq i<j\leq n+1}\E\left[\phi^2(Z_{i:n}, Z_{j:n}) W^2_{i:n} W^2_{j:n}\right]^{\frac{1}{2}} \leq c_3 $$	
	\label{lem:expectation_sq}
\end{lemma}	
\noindent The prove of the lemma above will be given at the end of this section. In the following I will establish a few lemmas, that will be needed to prove lemma \ref{lem:expectation_sq} above. Define the following quantities for $n\geq 1$ and $s < t$:
\begin{align*}
	B_n(s) &:= \prod_{k=1}^{n}\left[1+\frac{1-q(Z_{k})}{n-R_{k,n}}\right]^{\I{Z_{k} < s}}\\
	C_n(s) &:= \sum_{i=1}^{n+1}\left[\frac{1-q(s)}{n-i+2}\right]\I{Z_{i-1:n} < s \leq Z_{i:n}}\\
	D_n(s,t) &:= \prod_{k=1}^{n} \left[1+\frac{1-q(Z_k)}{n-R_{k,n} +2}\right]^{2\I{Z_k<s}} \prod_{k=1}^{n}\left[1+\frac{1-q(Z_k)}{n-R_ {k,n}+1}\right]^{\I{s < Z_k < t}}\\
	\Delta_n(s,t) &:= \E\left[D_n(s,t) \right]\\
	\bar{\Delta}_n(s,t) &:= \E\left[C_n(s)D_n(s,t) \right]
\end{align*}
Here $Z_{0:n} := -\infty$ and $Z_{n+1:n} := \infty$.
%
\begin{lemma}
	Let $\tilde{\phi}: \R^2_+ \longrightarrow \R_+$ be a Borel-measurable function. Then we have for any $s<t$ and $n\geq 1$ 
	\begin{align*}
		&\E[\tilde{\phi}(Z_1,Z_2) B_n(Z_1) B_n(Z_2)]\\
		& = \E[2\tilde{\phi}(Z_1,Z_2) \{\Delta_{n-2}(Z_1,Z_2) + \bar{\Delta}_{n-2}(Z_1,Z_2)\}\I{Z_1<Z_2}]
	\end{align*}
	\label{lem:representation_bn}
	%
	\begin{proof}
		Consider the following
		\begin{align*}
			B_n(Z_1)B_n(Z_2) &= \prod_{k=1}^{n}\left[1+\frac{1-q(Z_{k})}{n-R_{k,n}}\right]^{\I{Z_{k} < Z_1}+\I{Z_{k} < Z_2}}\\
			&= \left[1+\frac{1-q(Z_{1})}{n-R_{1,n}}\right]^{\I{Z_{1} < Z_2}} \left[1+\frac{1-q(Z_{2})}{n-R_{2,n}}\right]^{\I{Z_{2} < Z_1}}\\
			&\qquad \times \prod_{k=3}^{n}\left[1+\frac{1-q(Z_{k})}{n-R_{k,n}}\right]^{\I{Z_{k} < Z_1}+\I{Z_{k} < Z_2}}\\
			&= \I{Z_1<Z_2}\left[1+\frac{1-q(Z_{1})}{n-R_{1,n}}\right] \\
			&\qquad\qquad \times \prod_{k=1}^{n-2}\left[1+\frac{1-q(Z_{k+2})}{n-R_{k+2,n}}\right]^{\I{Z_{k+2} < Z_1}+\I{Z_{k+2} < Z_2}}\\
			&\quad + \I{Z_1>Z_2}\left[1+\frac{1-q(Z_{2})}{n-R_{2,n}}\right] \\
			&\qquad\qquad \times \prod_{k=1}^{n-2}\left[1+\frac{1-q(Z_{k+2})}{n-R_{k+2,n}}\right]^{\I{Z_{k+2} < Z_1}+\I{Z_{k+2} < Z_2}}\\
			&\quad + \I{Z_1=Z_2}\prod_{k=1}^{n-2}\left[1+\frac{1-q(Z_{k+2})}{n-R_{k+2,n}}\right]^{2\I{Z_{k+2} < Z_1}} \numberthis\label{eq:bnbn1}
		\end{align*}
		%
		On $\{Z_1<Z_2\}$ we have 
		\begin{align*}
			\prod_{k=1}^{n-2}\left[1+\frac{1-q(Z_{k+2})}{n-R_{k+2,n}}\right]^{\I{Z_{k+2} < Z_2}} &= \prod_{k=1}^{n-2}\left[1+\frac{1-q(Z_{k+2})}{n-\tilde{R}_{k,n-2}}\right]^{\I{Z_{k+2} < Z_1}}\\
			&\quad \times  \prod_{k=1}^{n-2}\left[1+\frac{1-q(Z_{k+2})}{n-\tilde{R}_{k,n-2}-1}\right]^{\I{Z_1 < Z_{k+2} < Z_2}}
		\end{align*}
		where $\tilde{R}_{k,n-2}$ denotes the rank of the $Z_k$, $k=3,\dots, n$ among themselves. The above holds since 
		\[ R_{k+2,n} = \begin{cases} 
			\tilde{R}_{k,n-2} & \textrm{ if } Z_{k+2} < Z_1 \\
			\tilde{R}_{k, n-2} + 1 & \textrm{ if } Z_1 < Z_{k+2} < Z_2 
		\end{cases}
		\]
		for $k=1,\dots,n-2$. 
		% 
		Therefore \eqref{eq:bnbn1} yields
		\begin{align*}
			B_n(Z_1)B_n(Z_2) &= \I{Z_1<Z_2}\left[1+\frac{1-q(Z_{1})}{n-R_{1,n}}\right] \\
			&\qquad\qquad \times \prod_{k=1}^{n-2}\left[1+\frac{1-q(Z_{k+2})}{n-\tilde{R}_{k,n-2}}\right]^{2\I{Z_{k+2} < Z_1}}\\
			&\qquad\qquad \times \prod_{k=1}^{n-2}\left[1+\frac{1-q(Z_{k+2})}{n-\tilde{R}_{k,n-2}-1}\right]^{\I{Z_1 < Z_{k+2} < Z_2}}\\
			&\quad + \I{Z_2<Z_1}\left[1+\frac{1-q(Z_{2})}{n-R_{2,n}}\right] \\
			&\qquad\qquad \times \prod_{k=1}^{n-2}\left[1+\frac{1-q(Z_{k+2})}{n-\tilde{R}_{k,n-2}}\right]^{2\I{Z_{k+2} < Z_2}}\\
			&\qquad\qquad \times \prod_{k=1}^{n-2}\left[1+\frac{1-q(Z_{k+2})}{n-\tilde{R}_{k,n-2}-1}\right]^{\I{Z_2 < Z_{k+2} < Z_1}}\\
			&\quad + \I{Z_1=Z_2}\prod_{k=1}^{n-2}\left[1+\frac{1-q(Z_{k+2})}{n-\tilde{R}_{k,n-2}}\right]^{2\I{Z_{k+2} < Z_1}} \numberthis\label{eq:bnbn_rank}
		\end{align*}
		%
		Now let's denote for $k=1,\dots,n-2$, $Z_{k:n-2}$ the ordered $Z$-values among $Z_3,\dots, Z_n$. Consider that we can write 
		\begin{align*}
			\left[1+\frac{1-q(Z_{1})}{n-R_{1,n}}\right] = \sum_{i=1}^{n-1}\left[1+\frac{1-q(s)}{n-i}\right]\I{Z_{i-1:n-2} < Z_1 \leq Z_{i:n-2}}
		\end{align*}
		%
		Therefore we obtain the following, by conditioning \eqref{eq:bnbn_rank} on $Z_1,Z_2$:
		\begin{align*}
			&\E[B_n(Z_1)B_n(Z_2)|Z_1 = s, Z_2 = t]\\
			&= \I{s<t}\E\left[\left(\sum_{i=1}^{n-1}\left[1+\frac{1-q(s)}{n-i}\right]\I{Z_{i-1:n-2} < s \leq Z_{i:n-2}}\right)\right.\\
			&\qquad\qquad\qquad \times \prod_{k=1}^{n-2}\left[1+\frac{1-q(Z_{k:n-2})}{n-k}\right]^{2\I{Z_{k:n-2} < s}}\\
			&\qquad\qquad\qquad \times \left. \prod_{k=1}^{n-2}\left[1+\frac{1-q(Z_{k:n-2})}{n-k-1}\right]^{\I{s < Z_{k:n-2} < t}} \right]\\
			&\quad + \I{t<s}\E\left[\left(\sum_{i=1}^{n-1}\left[1+\frac{1-q(t)}{n-i}\right]\I{Z_{i-1:n-2} < t \leq Z_{i:n-2}}\right)\right. \\
			&\qquad\qquad\qquad \times \prod_{k=1}^{n-2}\left[1+\frac{1-q(Z_{k:n-2})}{n-k}\right]^{2\I{Z_{k:n-2} < t}}\\
			&\qquad\qquad\qquad \times \left. \prod_{k=1}^{n-2}\left[1+\frac{1-q(Z_{k:n-2})}{n-k-1}\right]^{\I{t < Z_{k:n-2} < s}}\right]\\
			&\quad + \I{s=t}\E\left[\prod_{k=1}^{n-2}\left[1+\frac{1-q(Z_{k:n-2})}{n-k}\right]^{2\I{Z_{k:n-2} < s}} \right]\\
			&= \alpha(s,t) + \alpha(t,s) + \beta(s,t)
		\end{align*}
		where 
		\begin{align*}
			\alpha(s,t) &:=\I{s<t}\E\left[\left(\sum_{i=1}^{n-1}\left[1+\frac{1-q(s)}{n-i}\right]\I{Z_{i-1:n-2} < s \leq Z_{i:n-2}}\right)\right.\\
			&\qquad\qquad\qquad \times \prod_{k=1}^{n-2}\left[1+\frac{1-q(Z_{k:n-2})}{n-k}\right]^{2\I{Z_{k:n-2} < s}}\\
			&\qquad\qquad\qquad \times \left. \prod_{k=1}^{n-2}\left[1+\frac{1-q(Z_{k:n-2})}{n-k-1}\right]^{\I{s < Z_{k:n-2} < t}} \right]
		\end{align*}
		and 
		\begin{align*}
			\beta(s,t) &:=\I{s=t}\E\left[\prod_{k=1}^{n-2}\left[1+\frac{1-q(Z_{k:n-2})}{n-k}\right]^{2\I{Z_{k:n-2} < s}} \right]
		\end{align*}		
		%
		Consider that we have
		$$\E[\alpha(Z_1,Z_2)] = \E[\alpha(Z_2,Z_1)]$$
		because $Z_1$ and $Z_2$ are \iid, and 
		$$\E[\beta(Z_1,Z_2)] = 0$$
		since $H$ is continuous. Therefore we get 
		\begin{align*}
			&\E[\tilde{\phi}(Z_1,Z_2)B_n(Z_1)B_n(Z_2)]\\
			&= \E[\tilde{\phi}(Z_1,Z_2)(\alpha(Z_1,Z_2) + \alpha(Z_2,Z_1) + \beta(Z_1,Z_2))]\\
			&= \E[2\tilde{\phi}(Z_1,Z_2)\alpha(Z_1,Z_2)] \numberthis\label{eq:expectationalpha}
		\end{align*}
		%
		Next consider that 
		\begin{align*}
			\alpha(s,t) &=\I{s<t}\E\left[(1+C_n(s)) D_{n-2}(s,t) \right]\\
						&= \I{s<t}(\Delta_{n-2}(s,t) + \bar{\Delta}_{n-2}(s,t))
		\end{align*}
		The latter equality holds, since
		\begin{align*}
			&\sum_{i=1}^{n-1}\left[1+\frac{1-q(s)}{n-i}\right]\I{Z_{i-1:n-2} < s \leq Z_{i:n-2}}\\
			&=\sum_{i=1}^{n-1}\I{Z_{i-1:n-2} < s \leq Z_{i:n-2}} + \sum_{i=1}^{n-1}\left[\frac{1-q(s)}{n-i}\right]\I{Z_{i-1:n-2} < s \leq Z_{i:n-2}}\\
			&= 1 + C_n(s)
		\end{align*}
		Now the statement of the lemma follows directly from \eqref{eq:expectationalpha}.
	\end{proof}
\end{lemma}
%
\noindent The next lemma identifies the $\P$-almost sure limit of $D_n$ for $n\to\infty$. Define for $s<t$
$$D(s,t) := \exp\left(2\int_{0}^{s} \frac{1-q(z)}{1-H(z)} H(dz) + \int_{s}^{t} \frac{1-q(z)}{1-H(z)} H(dz)\right)$$
\begin{lemma}
	For any $s < t$ \st\ $H(t)<1$, we have
	$$\lim\limits_{n\to\infty}D_n(s,t) = D(s,t)$$
	\label{lem:dn_limit}
	
	\begin{proof}
		First let for $s<t$ and $k=1,\dots,n$
		\begin{align*}
			x_k &:= \frac{1-q(Z_k)}{n(1- H_n(Z_k) + 2/n)}\\
			y_k &:= \frac{1-q(Z_k)}{n(1- H_n(Z_k) + 1/n)}\\
			s_k &:= \I{Z_k < s} \\
			t_k &:= \I{s < Z_k < t}
		\end{align*}
		%
		Next consider 
		\begin{align*}
			D_n(s,t) &= \prod_{k=1}^{n} \left[1+\frac{1-q(Z_k)}{n(1- H_n(Z_k) + 2/n)}\I{Z_k<s}\right]^{2}\\ 
			&\qquad \times \prod_{k=1}^{n}\left[1+\frac{1-q(Z_k)}{n(1-H_n(Z_k)+1/n)}\I{s < Z_k < t}\right]\\
			&= \prod_{k=1}^{n} \left[1+x_k s_k\right]^{2} \prod_{k=1}^{n}\left[1+y_k t_k\right]\\
			&= \exp\left(2\sum_{k=1}^{n}\ln\left[1+x_k s_k\right] + \sum_{k=1}^{n}\ln\left[1+y_k t_k\right]\right)\\
		\end{align*}
		Note that $0 \leq x_k s_k \leq 1$ and $0 \leq y_k t_k \leq 1$. Consider that the following inequality holds  
		$$-\frac{x^2}{2} \leq \ln(1+x) - x \leq 0$$ 
		for any $x \geq 0$ (cf.  \cite{stute1993strong}, p. 1603). This implies 
		$$-\frac{1}{2}\sum_{k=1}^{n}x_k^2 s_k \leq \sum_{k=1}^{n}\ln(1+x_k s_k) - \sum_{k=1}^{n}x_k s_k \leq 0$$ 
		But now 
		\begin{align*}
			\sum_{k=1}^{n} x_k^2 s_k &= \frac{1}{n^2} \sum_{k=1}^{n} \left(\frac{1-q(Z_k)}{1-H_n(Z_k)+\frac{2}{n}}\right)^2\I{Z_k<s}\\
			&\leq \frac{1}{n^2} \sum_{k=1}^{n} \left(\frac{1}{1-H_n(s)+\frac{1}{n}}\right)^2\\
			&= \frac{1}{n(1-H_n(s)+n^{-1})^2} \longrightarrow 0
		\end{align*}
		$\P$-almost surely as $n\to\infty$, since $H(s)<H(t)<1$. Therefore we have
		$$\abs{\sum_{k=1}^{n}\ln(1+x_k s_k) - \sum_{k=1}^{n}x_k s_k} \longrightarrow 0$$
		with probability 1 as $n\to\infty$. 
		%
		Similarly we obtain
		$$\abs{\sum_{k=1}^{n}\ln(1+y_k t_k) - \sum_{k=1}^{n}y_k t_k} \longrightarrow 0$$
		with probability 1 as $n\to\infty$. Hence 
		$$\lim\limits_{n\to\infty} D_n(s) = \lim\limits_{n\to\infty} \exp\left(2\sum_{k=1}^{n} x_k s_k + \sum_{k=1}^{n}y_k t_k\right)$$
		%
		Now consider 
		\begin{align*}
			\sum_{i=1}^{n} x_k s_k &= \frac{1}{n}\sum_{k=1}^{n} \frac{1-q(Z_k)}{1-H_n(Z_k)+\frac{2}{n}}\I{Z_k<s}\\
			&= \int_{0}^{s-} \frac{1-q(z)}{1-H_n(z)+\frac{2}{n}} H_n(dz)\\
			&\leq \int_{0}^{s-} \frac{1-q(z)}{1-H_n(z)} H_n(dz)\\
			&= \int_{0}^{s-} \frac{1-q(z)}{1-H(z)} H_n(dz) + \int_{0}^{s-} \frac{1-q(z)}{1-H_n(z)} - \frac{1-q(z)}{1-H(z)} H_n(dz)\\
			&= \int_{0}^{s-} \frac{1-q(z)}{1-H(z)} H_n(dz) + \int_{0}^{s-} \frac{(1-q(z))(H(z)-H_n(z))}{(1-H_n(z))(1-H(z))} H_n(dz) \numberthis \label{eq:xksk_int}
		\end{align*}
		%
		Note that the second term on the right hand side of the latter equation above tends to zero for  $n\to\infty$, because
		\begin{align*}
			& \int_{0}^{s-} \frac{(1-q(z))(H(z)-H_n(z)}{(1-H_n(z))(1-H(z))} H_n(dz)\\
			&\leq \frac{\sup_{z}|H(z)- H_n(z)|}{1-H(s)} \int_{0}^{s-}\frac{1}{1-H_n(z)} H_n(dz) \longrightarrow 0
		\end{align*}
		%
		$\P$-almost surely as $n\to\infty$, by the Glivenko-Cantelli Theorem and since $H(s)<1$. Now consider the first term in \eqref{eq:xksk_int}. We have
		\begin{align*}
			\int_{0}^{s-} \frac{1-q(z)}{1-H(z)} H_n(dz) \longrightarrow \int_{0}^{s} \frac{1-q(z)}{1-H(z)} H(dz)
		\end{align*}		
		by the SLLN. Therefore we obtain 
		$$\lim\limits_{n\to\infty} \sum_{i=1}^{n} x_k s_k = \int_{0}^{s} \frac{1-q(z)}{1-H(z)} H(dz)$$
		By the same arguments, we can show that 
		$$\lim\limits_{n\to\infty} \sum_{i=1}^{n} y_k t_k = \int_{s}^{t} \frac{1-q(z)}{1-H(z)} H(dz)$$
		Thus we finally conclude
		$$\lim\limits_{n\to\infty} D_n(s,t) = \exp\left(2\int_{0}^{s} \frac{1-q(z)}{1-H(z)} H(dz) + \int_{s}^{t} \frac{1-q(z)}{1-H(z)} H(dz)\right)$$
		$\P$-almost surely.
	\end{proof}
\end{lemma}

\begin{lemma}
	$\{D_n, \F_n\}_{n\geq 1}$ is a non-negative reverse supermartingale.
	\label{lem:dn_supermart}
	\begin{proof}
		Consider that for $s<t$ and $n\geq 1$
		\begin{align*}
			\E[D_n(s,t)| \F_{n+1}] &= \E\left[\prod_{k=1}^{n}\left(1+\frac{1-q(Z_{k:n})}{n-k+2}\right)^{2\I{Z_{k:n} <s}}\right.\\
			&\qquad \left. \times \prod_{k=1}^{n}\left(1+\frac{1-q(Z_{k:n})}{n-k+1}\right)^\I{s < Z_{k:n} < t} | \F_{n+1}\right]\\
			&= \sum_{i=1}^{n+1}\E\left[\I{Z_{n+1} = Z_{i:n+1}} \dots | \F_{n+1}\right]\\
			&= \sum_{i=1}^{n+1}\E\left[\I{Z_{n+1} = Z_{i:n+1}} \prod_{k=1}^{i-1}\left(1+\frac{1-q(Z_{k:n+1})}{n-k+2}\right)^{2\I{Z_{k:n+1} <s}} \right.\\
			&\qquad\qquad \times \prod_{k=i}^{n}\left(1+\frac{1-q(Z_{k+1:n+1})}{n-k+2}\right)^{2\I{Z_{k+1:n+1} <s}}\\
			&\qquad\qquad \times \prod_{k=1}^{i-1}\left(1+\frac{1-q(Z_{k:n+1})}{n-k+1}\right)^\I{s < Z_{k:n+1} < t}\\
			&\qquad\qquad \left. \times \prod_{k=i}^{n}\left(1+\frac{1-q(Z_{k+1:n+1})}{n-k+1}\right)^\I{s < Z_{k+1:n+1} < t}| \F_{n+1}\right]\\
			&= \sum_{i=1}^{n+1}\E\left[\I{Z_{n+1} = Z_{i:n+1}} \prod_{k=1}^{i-1}\left(1+\frac{1-q(Z_{k:n+1})}{n-k+2}\right)^{2\I{Z_{k:n+1} <s}} \right.\\
			&\qquad\qquad \times \prod_{k=i+1}^{n+1}\left(1+\frac{1-q(Z_{k:n+1})}{n-k+3}\right)^{2\I{Z_{k:n+1} <s}}\\
			&\qquad\qquad \times \prod_{k=1}^{i-1}\left(1+\frac{1-q(Z_{k:n+1})}{n-k+1}\right)^\I{s < Z_{k:n+1} < t}\\
			&\qquad\qquad \left. \times \prod_{k=i+1}^{n+1}\left(1+\frac{1-q(Z_{k:n+1})}{n-k+2}\right)^\I{s < Z_{k:n+1} < t}| \F_{n+1}\right]
	\end{align*}
	Now each product within the conditional expectation is measurable \wrt\ $\F_{n+1}$. Moreover we have for $i=1,\dots,n$ 
	\begin{align*}
		\E[\I{Z_{n+1}=Z_{i:n+1}}|\F_n+1] &= \P(Z_{n+1}=Z_{i:n+1}|\F_{n+1})\\
		&= \P(R_{n+1,n+1} = i)\\
		&= \frac{1}{n+1}
	\end{align*}
	%
	Thus we obtain
	\begin{align*}
			\E[D_n(s,t)| \F_{n+1}] &= \frac{1}{n+1} \sum_{i=1}^{n+1} \prod_{k=1}^{i-1}\left(1+\frac{1-q(Z_{k:n+1})}{n-k+2}\right)^{2\I{Z_{k:n+1} <s}}\\
			&\qquad\qquad\qquad \times \left(1+\frac{1-q(Z_{k:n+1})}{n-k+1}\right)^\I{s < Z_{k:n+1} < t}\\
			&\qquad\qquad \times \prod_{k=i+1}^{n+1}\left(1+\frac{1-q(Z_{k:n+1})}{n-k+3}\right)^{2\I{Z_{k:n+1} <s}}\\ &\qquad\qquad\qquad \times \left(1+\frac{1-q(Z_{k:n+1})}{n-k+2}\right)^\I{s < Z_{k:n+1} < t} \numberthis \label{eq:cond_exp_dnp1}
		\end{align*}
		%
		We will now proceed by induction on $n$. First let 
		$$x_k := 1-q(Z_{k:2}) \textrm{, } s_k := \I{Z_{k:2} < s} \textrm{ and } t_k := \I{s < Z_{k:2} < t}$$
		for $k=1,2$. Next consider
		\begin{align*}
			\E[D_1(s,t) | \F_2] &= \frac{1}{2}\left[\left(1+\frac{1-q(Z_{2:2})}{2}\right)^{2\I{Z_{2:2}<s}} \times \left(1+(1-q(Z_{2:2}))\right)^{\I{s < Z_{2:2} < t}} \right.\\
			&\qquad \left. + \left(1+\frac{1-q(Z_{1:2})}{2}\right)^{2\I{Z_{1:2}<s}} \times \left(1+(1-q(Z_{1:2}))\right)^{\I{s < Z_{1:2} < t}}\right]\\
			&= \frac{1}{2}\left[\left(1+\frac{x_2}{2}s_2\right)^{2} \times \left(1+x_2t_2\right) + \left(1+\frac{x_1}{2}s_1\right)^{2} \times \left(1+x_1t_1\right)\right]\\
		\end{align*}
		%
		Moreover we have
		\begin{align*}
			D_2(s,t) &= \left[1 + \frac{x_1}{3}s_1\right]^2 \times \left[1+\frac{x_1}{2}t_1\right] \times \left[1+\frac{x_2}{2}s_2\right]^2 \times \left[1+x_2t_2\right]\\
			&= \left[1 + \frac{x_1}{2}t_1 + \left(\frac{x_1^2}{9} + \frac{2}{3}x_1\right)s_1\right] \times \left[1 + x_2t_2 + \left(\frac{x_2^2}{4} + x_2\right)s_2\right]\\
		\end{align*}		
		%
		Therefore we obtain 
		\begin{align*}
			\E[D_1(s,t) | \F_2] - D_2(s,t) \leq \frac{x_1^2}{72} - \frac{x_1}{6} \leq 0
		\end{align*}
		since $0 \leq x_1 \leq 1$. Thus $\E[D_1(s,t) | \F_2] \leq D_2(s,t)$ for any $s<t$, as needed. Now assume that 
		$$\E[D_n(s,t) | \F_{n+1}] \leq D_{n+1}(s,t)$$
		holds for any $n\geq 1$. Note that the latter is equivalent to assuming
		\begin{align*}
			& \frac{1}{n+1} \sum_{i=1}^{n+1} \prod_{k=1}^{i-1}\left(1+\frac{1-q(y_k)}{n-k+2}\right)^{2\I{y_k <s}}  \left(1+\frac{1-q(y_k)}{n-k+1}\right)^\I{s < y_k < t}\\
			&\qquad\qquad \times \prod_{k=i+1}^{n+1}\left(1+\frac{1-q(y_k)}{n-k+3}\right)^{2\I{y_k <s}} \left(1+\frac{1-q(y_k)}{n-k+2}\right)^\I{s < y_k < t}\\
			&\leq \prod_{k=1}^{n+1}\left(1+\frac{1-q(y_k)}{n-k+3}\right)^{2\I{y_k <s}} \prod_{k=1}^{n+1}\left(1+\frac{1-q(y_k)}{n-k+2}\right)^\I{s < y_k < t} \numberthis \label{eq:supermart_yk}
		\end{align*}
		holds for arbitrary $y_k \geq 0$. Next define for $s<t$ and $n\geq 1$
		$$A_{n+2}(s,t) := \prod_{k=2}^{n+2} \left[1+\frac{1-q(Z_{k:n+2})}{n-k+4}\right]^{2\I{Z_{k:n+2} < s}} \times \left[1+\frac{1-q(Z_{k:n+2})}{n-k+3}\right]^\I{s < Z_{k:n+2} < t}$$
		Now consider that we get from \eqref{eq:cond_exp_dnp1}
		\begin{align*}
			&\E[D_{n+1}(s,t)| \F_{n+2}]	\\
			&= \frac{1}{n+2} \sum_{i=1}^{n+2} \prod_{k=1}^{i-1}\left(1+\frac{1-q(Z_{k:n+2})}{n-k+3}\right)^{2\I{Z_{k:n+2} <s}}  \left(1+\frac{1-q(Z_{k:n+2})}{n-k+2}\right)^\I{s < Z_{k:n+2} < t}\\
			&\qquad\qquad\quad \times \prod_{k=i+1}^{n+2}\left(1+\frac{1-q(Z_{k:n+2})}{n-k+4}\right)^{2\I{Z_{k:n+2} <s}} \left(1+\frac{1-q(Z_{k:n+2})}{n-k+3}\right)^\I{s < Z_{k:n+2} < t}\\
			&= \frac{A_{n+2}}{n+2} + \frac{1}{n+2}\sum_{i=2}^{n+2} \prod_{k=1}^{i-1} \dots \times \prod_{k=i+1}^{n+2}\dots \\
			&= \frac{A_{n+2}}{n+2} + \frac{1}{n+2}\sum_{i=1}^{n+1} \prod_{k=1}^{i} \dots \times \prod_{k=i+2}^{n+2}\dots\\
			&= \frac{A_{n+2}}{n+2} + \frac{1}{n+2}\left(1+\frac{1-q(Z_{1:n+2})}{n+2}\right)^{2\I{Z_{1:n+2} < s}} \left(1+\frac{1-q(Z_{1:n+2})}{n+1}\right)^\I{s < Z_{1:n+2} < t}\\
			&\qquad\qquad\qquad\times \sum_{i=1}^{n+1} \prod_{k=1}^{i-1} \left(1+\frac{1-q(Z_{k+1:n+2})}{n-k+2}\right)^{2\I{Z_{k+1:n+2} <s}}\\
			&\qquad\qquad\qquad\qquad\qquad \times \left(1+\frac{1-q(Z_{k+1:n+2})}{n-k+1}\right)^\I{s < Z_{k+1:n+2} < t}\\
			&\qquad\qquad\qquad\qquad \times \prod_{k=i+1}^{n+1}\left(1+\frac{1-q(Z_{k+1:n+2})}{n-k+3}\right)^{2\I{Z_{k+1:n+2} <s}}\\
			&\qquad\qquad\qquad\qquad\qquad \times \left(1+\frac{1-q(Z_{k+1:n+2})}{n-k+2}\right)^\I{s < Z_{k+1:n+2} < t}
		\end{align*}
		%
		Using \eqref{eq:supermart_yk} on the right hand side of the equation above yields
		\begin{align*}
			&\E[D_{n+1}(s,t)| \F_{n+2}]	\\
			&\leq \frac{A_{n+2}}{n+2} + \frac{n+1}{n+2}\left(1+\frac{1-q(Z_{1:n+2})}{n+2}\right)^{2\I{Z_{1:n+2} < s}} \left(1+\frac{1-q(Z_{1:n+2})}{n+1}\right)^\I{s < Z_{1:n+2} < t}\\
			&\qquad\qquad\qquad\times \prod_{k=1}^{n+1} \left(1+\frac{1-q(Z_{k+1:n+2})}{n-k+3}\right)^{2\I{Z_{k+1:n+2} <s}}\\
			&\qquad\qquad\qquad\qquad\qquad \times \left(1+\frac{1-q(Z_{k+1:n+2})}{n-k+2}\right)^\I{s < Z_{k+1:n+2} < t}\\
			&= A_{n+2} \left[\frac{1}{n+2} + \frac{n+1}{n+2}\left(1+\frac{1-q(Z_{1:n+2})}{n+2}\right)^{2\I{Z_{1:n+2} < s}} \right. \\
			&\qquad\qquad\qquad\qquad\qquad \left. \times \left(1+\frac{1-q(Z_{1:n+2})}{n+1}\right)^\I{s < Z_{1:n+2} < t}\right] \\
		\end{align*}
		%
		For the moment, let
		$$x_1 := 1-q(Z_{1:n+2}) \textrm{, } s_1 := \I{Z_{1:n+2} < s} \textrm{ and } t_1 := \I{s < Z_{1:n+2} < t} $$
		%
		Now we can rewrite the above as
		\begin{align*}
			\E[D_{n+1}(s,t)| \F_{n+2}]	&\leq A_{n+2} \left[\frac{1}{n+2} + \frac{n+1}{n+2}\left(1+\frac{x_1s_1}{n+2}\right)^{2} \left(1+\frac{x_1t_1}{n+1}\right)\right] \numberthis\label{eq:dn_supermart_an}
		\end{align*}
		Next consider 
		\begin{align*}
			\left(1+\frac{x_1t_1}{n+1}\right) &= \left(1+\frac{x_1t_1}{n+2}-\frac{1}{n+2}\right) \left(1+\frac{1}{n+1}\right)\\
			&=  \left(1+\frac{x_1t_1}{n+2}\right)+\frac{1}{n+1}\left(1+\frac{x_1t_1}{(n+2)}\right) - \frac{1}{n+1}\\
			&= \left(1+\frac{x_1t_1}{n+2}\right)+\frac{x_1t_1}{(n+1)(n+2)}
		\end{align*}
		%
		Thus we get
		\begin{align*}
			&\frac{n+1}{n+2}\left(1+\frac{x_1s_1}{n+2}\right)^{2} \left(1+\frac{x_1t_1}{n+1}\right) \\
			&= \frac{n+1}{n+2}\left(1+\frac{x_1s_1}{n+2}\right)^{2}\left(1+\frac{x_1t_1}{n+2}\right) + \left(1+\frac{x_1s_1}{n+2}\right)^{2}\frac{x_1t_1}{(n+2)^2}
		\end{align*}
		%
		But now 
		\begin{align*}
			\left(1+\frac{x_1s_1}{n+2}\right)^{2}\frac{x_1t_1}{(n+2)^2} &= \left(1+2\frac{x_1s_1}{n+2}+\frac{x^2_1s_1}{(n+2)^2}\right)\frac{x_1t_1}{(n+2)^2}\\
			&= \frac{x_1t_1}{(n+2)^2}
		\end{align*}
		since $s_1\cdot t_1=0$ for all $s<t$. Hence we can rewrite the term in brackets in \eqref{eq:dn_supermart_an} as 
		\begin{align*}
			&\frac{1}{n+2} + \frac{n+1}{n+2}\left(1+\frac{x_1s_1}{n+2}\right)^{2} \left(1+\frac{x_1t_1}{n+1}\right) \\
			&=\frac{1}{n+2} + \frac{x_1t_1}{(n+2)^2} + \frac{n+1}{n+2}\left(1+\frac{x_1s_1}{n+2}\right)^{2}\left(1+\frac{x_1t_1}{n+2}\right)\\
			&=\frac{1}{n+2}\left(1+\frac{x_1t_1}{n+2}\right) + \frac{n+1}{n+2}\left(1+\frac{x_1s_1}{n+2}\right)^{2}\left(1+\frac{x_1t_1}{n+2}\right)\\
			&=\left[\frac{1}{n+2} + \frac{n+1}{n+2}\left(1+\frac{x_1}{n+2}\right)^{2s_1}\right]\left(1+\frac{x_1}{n+2}\right)^{t_1}\\
			&\leq \left(1+\frac{x_1}{n+3}\right)^{2s_1}\left(1+\frac{x_1}{n+2}\right)^{t_1}
		\end{align*}
		The latter inequality above holds, since 
		$$\left[\frac{1}{n+2} + \frac{n+1}{n+2}\left(1+\frac{x}{n+2}\right)^{2}\right] \leq \left(1+\frac{x}{n+3}\right)^{2}$$
		for any $0\leq x\leq 1$. (\todo{prove?})
		Therefore we can rewrite \eqref{eq:dn_supermart_an} as
		\begin{align*}
			\E[D_{n+1}(s,t)| \F_{n+2}]	&\leq A_{n+2} \left(1+\frac{1-q(Z_{1:n+2})}{n+3}\right)^{2\I{Z_{1:n+2}<s}} \\
			&\qquad\quad \times \left(1+\frac{1-q(Z_{1:n+2})}{n+2}\right)^{\I{s<Z_{1:n+2} <t}}\\
			&= D_{n+2}(s,t)
		\end{align*}		
		This concludes the proof.
	\end{proof}
\end{lemma}
%
\begin{lemma}
	Let $\F_\infty = \bigcap_{n\geq 2} \F_n$. Then we have each $A\in \F_\infty$ that $\P(A)\in \{0,1\}$.
	\label{lem:hewitt_savage}
	\begin{proof}
		Define 
		\begin{equation*}
		\Pi_n := \{\pi | \pi \textrm{ is permutation of } 1,\dots, n\} \nonumber
		\end{equation*}	
		and
		\begin{equation*}
		\Pi := \bigcup\limits_{n\in\mathbb{N}} \Pi_n \nonumber\\	
		\end{equation*}
		We will use the Hewitt-Savage zero-one law xin order to show the statement of the lemma. Thus we need to show that for all $A\in\F_\infty$ and all $\pi\in\Pi$ there exists $B\in\mathcal{B}_\mathbb{N}^*$ \st\ 
		\begin{equation}
		A = \{\omega| (Z_i(\omega))_{i\in\mathbb{N}} \in B\} = \{\omega| (Z_{\pi(i)}(\omega))_{i\in\mathbb{N}} \in B\}
		\label{eq:hewitt_savage}
		\end{equation} 
		Let $A\in\F_\infty$, then $A\in\F_n$ for all $n\in\mathbb{N}$. Let $n\in\mathbb{N}$ fixed but arbitrary and $A\in\F_n$. Then, because of measurability \todo{wording}, there must exist $\tilde{B}\in\mathcal{B}_\mathbb{N}^*$ such that
		\begin{equation}
		A = (Z_{1:n}, \dots, Z_{n:n}, Z_{n+1}, Z_{n+2}, \dots)^{-1}(\tilde{B})
		\label{eq:A_Z_inv}
		\end{equation}
		For fixed $\omega\in\Omega$ define the map 
		$$T: (\R^\mathbb{N}, \mathcal{B}_\mathbb{N}^*)\ni(Z_i(\omega))_{i\in\mathbb{N}} \longrightarrow T((Z_i(\omega))_{i\in\mathbb{N}})\in (\R^\mathbb{N}, \mathcal{B}_\mathbb{N}^*)$$
		with
		$$T((Z_i(\omega))_{i\in\mathbb{N}}):=(Z_{1:n}, \dots, Z_{n:n}, Z_{n+1}, Z_{n+2}, \dots)(\omega)$$
		Note that for any $\pi\in\Pi_n$ we have 
		\begin{align}
		T((Z_i(\omega))_{i\in\mathbb{N}}) &= T((Z_{\pi(i)}(\omega))_{i\in\mathbb{N}}) \label{eq:T_permutable}\\
		&= (Z_{1:n}, \dots, Z_{n:n}, Z_{n+1}, Z_{n+2}, \dots)(\omega) \nonumber
		\end{align}
		Hence on the one hand, we get from \eqref{eq:A_Z_inv}
		\begin{align*}
		A &= (T((Z_i)_{i\in\mathbb{N}}))^{-1}(\tilde{B}) \\
		&= ((Z_i)_{i\in\mathbb{N}})^{-1}(T^{-1}(\tilde{B}))\\
		&= \{\omega | (Z_i(\omega))_{i\in\mathbb{N}} \in B\}
		\end{align*}
		where $B = T^{-1}(\tilde{B})$. On the other hand we get from \eqref{eq:T_permutable} and again by \eqref{eq:A_Z_inv} that 
		\begin{align*}
		A &= (T((Z_{\pi(i)})_{i\in\mathbb{N}}))^{-1}(\tilde{B}) \\
		&= ((Z_{\pi(i)})_{i\in\mathbb{N}})^{-1}(T^{-1}(\tilde{B}))\\
		&= \{\omega | (Z_{\pi(i)}(\omega))_{i\in\mathbb{N}} \in B\}
		\end{align*}
		Now since $n\in\mathbb{N}$ was chosen arbitrarily, the above statement holds for all $n\in\mathbb{N}$ and hence for all $\pi\in\Pi$. Whence establishing \eqref{eq:hewitt_savage}.
	\end{proof}
\end{lemma}
\begin{proof}[\textbf{Proof of lemma \ref{lem:expectation_sq}}]
	Suppose (A\ref{as:sup_sn}) is satisfied. Consider the following
	\begin{align*}
		& \doublesum\limits_{1\leq i<j\leq n}\E\left[\phi^2(Z_{i:n}, Z_{j:n}) W^2_{i:n} W^2_{j:n}\right]^{\frac{1}{2}}\\
		&= \doublesum\limits_{1\leq i<j\leq n}\E\left[\phi^2(Z_{i:n}, Z_{j:n}) \frac{q^2(Z_{i:n})}{(n-i+1)^2}\prod_{k=1}^{i-1}\left[1-\frac{q(Z_{k:n})}{n-k+1}\right]^2\right.\\
		&\qquad\qquad\qquad \times \left. \frac{q^2(Z_{j:n})}{(n-j+1)^2}\prod_{l=1}^{j-1}\left[1-\frac{q(Z_{l:n})}{n-l+1}\right]^2\right]^\frac{1}{2}\\
		&\leq \doublesum\limits_{1\leq i<j\leq n}\E\left[\phi^2(Z_{i:n}, Z_{j:n}) \frac{q^2(Z_{i:n})}{(n-i+1)^2}\prod_{k=1}^{i-1}\left[1-\frac{q(Z_{k:n})}{n-k+1}\right]\right.\\
		&\qquad\qquad\qquad \times \left. \frac{q^2(Z_{j:n})}{(n-j+1)^2}\prod_{l=1}^{j-1}\left[1-\frac{q(Z_{l:n})}{n-l+1}\right]\right]^\frac{1}{2}\numberthis\label{eq:E_n}
	\end{align*}
	%
	Next we will modify the products above. Recall the following definition 
	$$B_n(s) := \prod_{k=1}^{n}\left[1+\frac{1-q(Z_{k})}{n-R_{k,n}}\right]^{\I{Z_{k} < s}}$$
	and note that for $i=1,\dots,n$
	\begin{align*}
		B_n(Z_{i:n}) &= \prod_{k=1}^{n}\left[1+\frac{1-q(Z_{k})}{n-R_{k,n}}\right]^{\I{Z_{k} < Z_{i:n}}}\\
		&= \prod_{k=1}^{n}\left[1+\frac{1-q(Z_{k:n})}{n-k}\right]^{\I{Z_{k:n} < Z_{i:n}}}\\
		&=  \prod_{k=1}^{i-1}\left[1+\frac{1-q(Z_{k:n})}{n-k}\right]
	\end{align*}
	%
	Moreover consider that for $i=1,\dots,n$
	\begin{align*}
		\frac{1}{n-i+1}\prod_{k=1}^{i-1}\left[1-\frac{q(Z_{k:n})}{n-k+1}\right]
		&=  \frac{1}{n-i+1}\prod_{k=1}^{i-1}\left[\frac{n-k+1-q(Z_{k:n})}{n-k+1}\right] \\
		&=  \frac{1}{n-i+1}\prod_{k=1}^{i-1}\left[\frac{n-k+1-q(Z_{k:n})}{n-k} \cdot \frac{n-k}{n-k+1}\right]\\
		&=  \frac{1}{n}\prod_{k=1}^{i-1}\left[1+\frac{1-q(Z_{k:n})}{n-k}\right]\\
		&=  \frac{B_n(Z_{i:n})}{n}\\
	\end{align*}
	%
	Thus we get, according to \eqref{eq:E_n}
	\begin{align*}
		&\doublesum\limits_{1\leq i<j\leq n}\E\left[\phi^2(Z_{i:n}, Z_{j:n}) W^2_{i:n} W^2_{j:n}\right]^{\frac{1}{2}}\\
		&\leq \doublesum\limits_{1\leq i<j\leq n}\E\left[\phi^2(Z_{i:n}, Z_{j:n}) \frac{q^2(Z_{i:n})}{n(n-i+1)}\frac{q^2(Z_{j:n})}{n(n-j+1)} B_n(Z_{i:n})B_n(Z_{j:n})\right]^\frac{1}{2}\\
		&\leq \frac{1}{n}\sum_{i=1}^{n}\sum_{j=1}^{n}\frac{1}{(n-i+1)^\frac{1}{2}(n-j+1)^\frac{1}{2}}\\
		&\qquad\qquad\times \E\left[\phi^2(Z_{i:n}, Z_{j:n}) q^2(Z_{i:n})q^2(Z_{j:n}) B_n(Z_{i:n})B_n(Z_{j:n})\right]^\frac{1}{2}\\
		&= \frac{1}{n}\sum_{i=1}^{n}\sum_{j=1}^{n} \frac{1}{(n-R_{i,n}+1)^{\frac{1}{2}} (n-R_{j,n}+1)^{\frac{1}{2}}}\\
		&\qquad\qquad \times \E\left[\phi^2(Z_{i}, Z_{j}) q^2(Z_{i})q^2(Z_{j})B_n(Z_{i})B_n(Z_{j})\right]^{\frac{1}{2}}
	\end{align*}
	%
	Note that for $1\leq i,j \leq n$
	\begin{align*}
		&\E\left[\phi^2(Z_{i}, Z_{j}) q^2(Z_{i})q^2(Z_{j})B_n(Z_{i})B_n(Z_{j})\right]\\
		&= \int_{0}^{\infty} \int_{0}^{\infty} \phi^2(s,t) q^2(s) q^2(t) B_n(s)B_n(t) H(ds)H(dt)\\
		&= \E\left[\phi^2(Z_{1}, Z_{2}) q^2(Z_{1})q^2(Z_{2})B_n(Z_{1})B_n(Z_{2})\right]
	\end{align*}
	since $Z_i \sim H$ for $i=1,\dots,n$.
	%
%	\begin{align*}
%		&\E\left[\phi^2(Z_{i}, Z_{j}) q^2(Z_{i})q^2(Z_{j})B_n(Z_{i})B_n(Z_{j})\right]\\ 
%		&= \E\left[\phi^2(Z_{1}, Z_{2}) q^2(Z_{1})q^2(Z_{2})B_n(Z_{1})B_n(Z_{2})\right]
%	\end{align*}
%	%
%	This is true, because we have
%	\begin{align*}
%	&\E\left[\phi^2(Z_{i}, Z_{j}) q^2(Z_{i})q^2(Z_{j})B_n(Z_{i})B_n(Z_{j})\right]\\
%	&= \E\left[\phi^2(Z_{i}, Z_{j}) q^2(Z_{i})\prod_{k=1}^{n}\left[1+\frac{1-q(Z_{k})}{n-R_{k,n}}\right]^{\I{Z_k < Z_{i}}}\right.\\
%	&\qquad\qquad \times \left. q^2(Z_{j}) \prod_{l=1}^{n}\left[1+\frac{1-q(Z_{l})}{n-R_{l,n}}\right]^{\I{Z_l < Z_{j}}} \right]\\
%	&= \E\left[\phi^2(Z_{i}, Z_{j}) q^2(Z_{i})\prod_{k=1}^{n}\left[\frac{n-R_{k,n}+1-q(Z_{k})}{n-R_{k,n}}\right]^{\I{Z_k < Z_{i}}}\right.\\
%	&\qquad\qquad \times \left. q^2(Z_{j}) \prod_{l=1}^{n}\left[\frac{n-R_{l,n}+1-q(Z_{l})}{n-R_{l,n}}\right]^{\I{Z_l < Z_{j}}} \right]\\
%	&= \E\left[\phi^2(Z_i,Z_j) q^2(Z_i) \exp\left(\sum_{k=1}^{n}\I{Z_k < Z_i}  \ln\left\{\frac{n(1-H_n(Z_k))+1-q(Z_k)}{n(1-H_n(Z_k))}\right\} \right)\right.\\
%	&\qquad\qquad\qquad \left. \times q^2(Z_j) \exp\left(\sum_{l=1}^{n} \ln\left\{\frac{n(1-H_n(Z_l))+1-q(Z_l)}{n(1-H_n(Z_l))}\right\} \right)\right]\\
%	&= \E\left[\phi^2(Z_i,Z_j) q^2(Z_i) \exp\left(\int_{0}^{Z_i-} \ln\left\{1+\frac{n(1-H_n(z))+1-q(z)}{1-H_n(z)}\right\}H_n(dz) \right)\right.\\
%	&\qquad\qquad\qquad \left. \times q^2(Z_j) \exp\left(\int_{0}^{Z_j-} \ln\left\{1+\frac{n(1-H_n(z))+1-q(z)}{1-H_n(z)}\right\}H_n(dz) \right)\right]\\
%	&= \int_{0}^{\infty} \int_{0}^{\infty} \phi^2(s,t) q^2(s) \exp\left(\int_{0}^{s-} \ln\left\{\frac{n(1-H_n(z))+1-q(z)}{1-H_n(z)}\right\}H_n(dz) \right)\\
%	&\qquad\qquad\qquad \times q^2(t) \exp\left(\int_{0}^{t-} \ln\left\{\frac{n(1-H_n(z))+1-q(z)}{1-H_n(z)}\right\}H_n(dz) \right)\\
%	&\qquad\qquad\qquad \times H(ds)H(dt)\\
%	&= \E\left[\phi^2(Z_{1}, Z_{2}) q^2(Z_{1})\prod_{k=1}^{n}\left[1+\frac{1-q(Z_{k})}{n-R_{k,n}}\right]^{\I{Z_k < Z_{1}}}\right.\\
%	&\qquad\qquad \times \left. q^2(Z_{2}) \prod_{l=1}^{n}\left[1+\frac{1-q(Z_{l})}{n-R_{l,n}}\right]^{\I{Z_l < Z_{2}}} \right]\\
%	\end{align*}
	%
	Hence we get 
	\begin{align*}
		&\doublesum\limits_{1\leq i<j\leq n}\E\left[\phi^2(Z_{i:n}, Z_{j:n}) W^2_{i:n} W^2_{j:n}\right]^{\frac{1}{2}}\\
		&\leq \frac{1}{n}\sum_{i=1}^{n}\sum_{j=1}^{n} \frac{1}{(n-R_{i,n}+1)^{\frac{1}{2}} (n-R_{j,n}+1)^{\frac{1}{2}}}\\
		&\qquad\qquad \times \E\left[\phi^2(Z_{1}, Z_{2}) q^2(Z_{1})q^2(Z_{2})B_n(Z_{1})B_n(Z_{2})\right]^{\frac{1}{2}}
	\end{align*}
	%
	Next consider that we have
	\begin{align*}
		\sum_{j=1}^{n} \frac{1}{(n-R_{j,n}+1)^{\frac{1}{2}}} &= \sum_{j=1}^{n} \frac{1}{j^{\frac{1}{2}}}\\
		&= 1 + \sum_{j=2}^{n} \int_{j-1}^{j} \frac{1}{\sqrt{j}} dx\\
		&\leq 1 + \sum_{j=2}^{n} \int_{j-1}^{j} \frac{1}{\sqrt{x}} dx\\
		&= 1 + 2\sum_{j=2}^{n} (\sqrt{j} - \sqrt{j-1})\\
		&= 2\sqrt{n} - 1\\
		&\leq 2\sqrt{n}
	\end{align*}
	for all $n\geq 1$.
	%
	We therefore obtain
	\begin{align*}
		&\doublesum\limits_{1\leq i<j\leq n}\E\left[\phi^2(Z_{i:n}, Z_{j:n}) W^2_{i:n} W^2_{j:n}\right]^{\frac{1}{2}}\\
		&\leq 4\cdot\E\left[\phi^2(Z_{1}, Z_{2}) q^2(Z_{1})q^2(Z_{2})B_n(Z_{1})B_n(Z_{2})\right]^{\frac{1}{2}} \numberthis \label{eq:expectation_phi_sq}
	\end{align*}
	%
	Since $q$ and $\phi$ are Borel-measurable, we can apply Lemma \ref{lem:representation_bn} to obtain
	\begin{align*}
		&\doublesum\limits_{1\leq i<j\leq n}\E\left[\phi^2(Z_{i:n}, Z_{j:n}) W^2_{i:n} W^2_{j:n}\right]^{\frac{1}{2}}\\
		&\leq 8\cdot\E\left[\phi^2(Z_{1}, Z_{2}) q^2(Z_{1})q^2(Z_{2})(\Delta_{n-2}(Z_1,Z_2) + \bar{\Delta}_{n-2}(Z_1,Z_2))\right]^{\frac{1}{2}}
	\end{align*}
	%
	Note that $0\leq C_n(s)\leq 1$ for all $n\geq 1$ and $s\in \R_+$. Thus 
	$$\bar\Delta_n(s,t) = \E[C_n(s)D_n(s,t)] \leq \Delta_{n}(s,t)$$ 
	for all $n\geq 1$ and $s<t$.
	%
	Therefore we get 
	\begin{align*}
		&\doublesum\limits_{1\leq i<j\leq n}\E\left[\phi^2(Z_{i:n}, Z_{j:n}) W^2_{i:n} W^2_{j:n}\right]^{\frac{1}{2}}\\
		&\leq 16\cdot\E\left[\phi^2(Z_{1}, Z_{2}) q^2(Z_{1})q^2(Z_{2})\Delta_{n-2}(Z_1,Z_2)\right]^{\frac{1}{2}}
	\end{align*}	
	%
	According to Lemma \ref{lem:dn_limit}, $D_n(s,t) \to D(s,t)$ $\P$-almost surely. Moreover we get from Lemma \ref{lem:dn_supermart}, that $\{D_n,\F_n\}_{n\geq 1}$ is a reverse supermartingale. Now this together with Proposition 5-3-11 of \cite{neveu1975discrete} and Lemma \ref{lem:hewitt_savage} yields
	$$\Delta_{n}(s,t) = \E[D_n(s,t)] = \E[D_n(s,t)|\F_\infty] \nearrow D(s,t)$$
	But this implies in particular that $\E[D_n(s,t)] \leq D(s,t)$ for all $n\geq 1$. Hence 
	\begin{align*}
		&\doublesum\limits_{1\leq i<j\leq n}\E\left[\phi^2(Z_{i:n}, Z_{j:n}) W^2_{i:n} W^2_{j:n}\right]^{\frac{1}{2}}\\
		&\leq 16\cdot\E\left[\phi^2(Z_{1}, Z_{2}) q^2(Z_{1})q^2(Z_{2})D(Z_1,Z_2)\right]^{\frac{1}{2}}
	\end{align*}		
	%
	Next consider that for $s<t$ \st\ $H(t)<1$
	\begin{align*}
		D(s,t) &= \exp\left(2\int_{0}^{s} \frac{1-q(z)}{1-H(z)} H(dz) + \int_{s}^{t} \frac{1-q(z)}{1-H(z)} H(dz)\right)\\
		&\leq \exp\left(2\int_{0}^{s} \frac{1}{1-H(z)} H(dz) + \int_{s}^{t} \frac{1}{1-H(z)} H(dz)\right)\\
		&= \exp\left(\int_{0}^{s} \frac{1}{1-H(z)} H(dz) + \int_{0}^{t} \frac{1}{1-H(z)} H(dz)\right)\\
		&= \exp\left(-\ln(1-H(s)) -\ln(1-H(t))\right)\\
		&= \frac{1}{(1-H(s))(1-H(t))} < \infty
	\end{align*}
	since $H(s)<H(t)<1$. Thus there must exist $c<\infty$ \st\ $D(s,t)\leq c$ for all $s<t$ and therefore
	\begin{align*}
		&\doublesum\limits_{1\leq i<j\leq n}\E\left[\phi^2(Z_{i:n}, Z_{j:n}) W^2_{i:n} W^2_{j:n}\right]^{\frac{1}{2}}\\
		&\leq 16c\cdot\E\left[\phi^2(Z_{1}, Z_{2}) \right]^{\frac{1}{2}}
	\end{align*}	
	since $0\leq q(s)\leq 1$ for all $s\in\R_+$. But now, under assumption (A\ref{as:sup_sn}), the expectation above is finite.
\end{proof}



	\bibliographystyle{./ametsoc}
	%\bibliographystyle{alphanum}
	
	% Generate the bibliography using entries from the following .BIB file
	\bibliography{./thesis}
\end{document}