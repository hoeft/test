\chapter{Discussion}
%
The SLLN has been established during this thesis under proper condition, stated in Chapter \ref{ch:notation}. In addition to the assumptions made in \cite{dikta2000strong} and \cite{bose1999strong}, we assumed that the censoring model, \ie\ conditional expectation of the censoring indicator given the observation, is a monotone non-decreasing function. However Chapter \ref{ch:model} shows a variety of examples, which are relevant in the field of survival analysis, for which this additional condition is satisfied. These examples include, among others, the proportional hazards model. The product limit estimator on which the U-Statistics is based in this example has the same asymptotic properties as the \cite{cheng1987mle} estimator (\cf\ \cite{dikta2000strong}, page 3). In Chapter \ref{ch:simulation}, we conducted simulation studies for different scenarios. The simulation studies verify the SLLN result in Theorem \ref{thm:snmn_limit}. Moreover the studies show that the semiparametric estimator outperforms the Kaplan-Meier estimate, especially in terms of variance, in all setups. This was expected because of \cite{dikta2005central} and \cite{dikta2014efficient}. The gain in efficiency was especially large for smaller sample sizes. The results of Section \ref{sec:sim_exppar} indicate, that the semiparametric estimator might still be consistent, even if the censoring model is not monotone non-decreasing.\\
\\
There are some obvious options to extend the results of this thesis in the future. Firstly one could try to establish the SLLN for the semiparametric estimator under weaker assumptions. In the appendix section, the interested reader may find thoughts on how to work around the additional restriction for the censoring model by modifying Doob's Upcrossing Theorem. Furthermore a CLT statement for the the semiparametric estimator could possibly derived from \cite{dikta2005central} and \cite{bose2002asymptotic}. As another option for future work, based on this thesis, one could transfer the result of Theorem \ref{thm:snmn_limit} to the estimator derived in \cite{dikta2016volterra}, using stochastic equivalence. 