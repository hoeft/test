\chapter{Basic results}

Within this chapter we will establish basic properties of $\E[S_n]$. In Section \ref{sec:representation} a representation is derived for $\E[S_n|\F_{n+1}]$, which is similar to the result established in \cite{bose1999strong}, Lemma 1. Later on in this section we will derive properties of the process above based on this representation. In \cite{bose1999strong} and \cite{dikta1998semiparametric}, the proof of existence of the limit of the considered estimator was based on a reverse supermartingale argument. We will not be able to establish the reverse supermartingale property for $\snse{2}{n}$. This will be discussed in more detail within Section \ref{sec:not_supermart}. Consequently, we will derive a generalized version of Doob's Upcrossing Theorem within Chapter \ref{ch:ex_limit} in order to show that the limit of $\snse{2}{n}$ exists with probability one.
%
%
%
\section{Basic results about $\E[\sn{n}|\F_{n+1}]$} \label{sec:representation}
%
We will first derive an explicit representation for $\E[\sn{n}|\F_{n+1}]$, which is similar to the one established in the proof of \cite{bose1999strong}, Lemma 1.
\begin{lemma} \label{lem:qi}
Define for $1\leq i<j\leq n$
\[Q_{ij}^{n+1} = \begin{cases} 
      Q_i^{n+1} & j\leq n \\
      Q_i^{n+1} - \frac{(n+1)\pi_i \pi_n (1-q(Z_{n:n+1}))}{(n-i+1)(2-q(Z_{n:n+1}))} & j=n+1
   \end{cases}
\]
%
where
\begin{equation}
	Q_i^{n+1} = (n+1) \left\{ \sum\limits_{r=1}^{i-1} \left[ \frac{\pi_r}{n-r+2-q(Z_{r:n+1})} \right]^2 + \frac{\pi_i\pi_{i+1}}{n-i+1} \right\}
	\label{eq:qi}
\end{equation}
and 
$$\pi_i = \prod\limits_{k=1}^{i-1} \frac{n-k+1-q(Z_{k:n+1})}{n-k+2-q(Z_{k:n+1})}$$
Then we have
$$\E[\sn{n}|\F_{n+1}] = \doublesum\limits_{1\leq i<j \leq n+1} \phi(Z_{i:n+1},Z_{j:n+1}) \wn{i}{n+1}\wn{j}{n+1}Q_{ij}^{n+1}$$
\end{lemma}

\begin{proof}
We will need the following result for the proof of lemma \ref{lem:qi}. Let 
$$A_i = \pi_i + \sum\limits_{r=1}^{i-1}\left[\frac{\pi_r}{n-r+2-q(Z_{r:n+1})} \right]$$
for $1\leq i\leq n$ with $\pi_i$ as defined above. Note that $\pi_1 = 1$, since the product is empty and hence taken as 1. Therefore we have $A_1=\pi_1=1$. Now for any $1\leq i\leq n-1$ 
\begin{align*}
	A_{i+1} &= \pi_{i+1} + \sum\limits_{r=1}^i\left[\frac{\pi_r}{n-r+2-q(Z_{r:n+1})} \right]\\
	&= \pi_i \left[ \frac{n-i+1-q(Z_{i:n+1})}{n-i+2-q(Z_{i:n+1})}\right] + \sum\limits_{r=1}^{i-1}\left[\frac{\pi_r}{n-r+2-q(Z_{r:n+1})}\right] + \left[\frac{\pi_i}{n-i+2-q(Z_{i:n+1})}  \right]\\
	&= \pi_i + \sum\limits_{r=1}^{i-1}\left[\frac{\pi_r}{n-r+2-q(Z_{r:n+1})}\right]\\
	&= A_i
\end{align*}
And therefore 
\begin{equation}
1=A_1=A_2=\dots=A_{n-1}=A_n
\label{eq:Ai}
\end{equation}
%
Now let's establish lemma \ref{lem:qi}. Let $F_n^q$ denote the measure that assigns mass to $Z_{1:n},\dots,Z_{n:n}$, then
\begin{align*}
\E[\sn{n}|\F_{n+1}] &= \E[\doublesum\limits_{1\leq i<j\leq n} \phi(Z_{i:n},Z_{j:n}) \wn{i}{n}\wn{j}{n}|\F_{n+1}]\\
    &= \E[\doublesum\limits_{1\leq i<j\leq n+1} \phi(Z_{i:n+1},Z_{j:n+1}) F_n^q\{Z_{i:n+1}\}F_n^q\{Z_{j:n+1}\}|\F_{n+1}]\\
		&= \doublesum\limits_{1\leq i<j\leq n+1} \phi(Z_{i:n+1},Z_{j:n+1}) \E[F_n^q\{Z_{i:n+1}\}F_n^q\{Z_{j:n+1}\}|\F_{n+1}]
%\label{eq:}
\end{align*}
%
Consider for $1\leq i<j\leq n$ 
\begin{align*}
& \E[F_n^q\{Z_{i:n+1}\}F_n^q\{Z_{j:n+1}\}|\F_{n+1}] \\
&=\E\left[\sum\limits_{r=1}^{n+1}F_n^q\{Z_{i:n+1}\}F_n^q\{Z_{j:n+1}\}I_{\{Z_{n+1}=Z_{r:n+1}\}}|\F_{n+1}\right]
%\label{eq:}
\end{align*}
%
Define the set $A_{rn} := \{Z_{n+1}=Z_{r:n+1}\}$. Note that on $A_{rn}$ we have for $1\leq l\leq n+1$
\begin{equation}
Z_{l:n+1} = \begin{cases} 
Z_{l:n} & l<r \\
Z_{l-1:n} & l>r
\end{cases}
\label{eq:change_order_delta}
\end{equation}
and therefore 
\begin{equation}
F_n^q\{Z_{l:n+1}\} = \begin{cases} 
W_{l:n} & l<r \\
0 & l=r\\
W_{l-1:n} & l>r
\end{cases}
\label{eq:change_order_w}
\end{equation}
%
Now we have
\begin{align}
&  \sum\limits_{r=1}^{n+1}F_n^q\{Z_{i:n+1}\}F_n^q\{Z_{j:n+1}\}I_{\{Z_{n+1}=Z_{r:n+1}\}}\nonumber\\
&= \sum\limits_{r=1}^{n+1}F_n^q\{Z_{i:n+1}\}F_n^q\{Z_{j:n+1}\}I_{A_{rn}}\nonumber\\
&= \sum\limits_{r=1}^{i-1}\wn{i-1}{n}\wn{j-1}{n}I_{A_{rn}} + \sum\limits_{r=i+1}^{j-1}\wn{i}{n}\wn{j-1}{n}I_{A_{rn}} + \sum\limits_{r=j+1}^{n+1}\wn{i}{n}\wn{j}{n}I_{A_{rn}}\nonumber\\
&=: T_1 + T_2 + T_3
\label{eq:sum_t}
\end{align}
%
Let's now consider each of the sums $T_1$, $T_2$, and $T_3$ in the above equation individually. First consider $T_1$. We have
\begin{align*}
T_1 &= \sum\limits_{r=1}^{i-1} \frac{q(Z_{i-1:n})}{n-i+2} \prod\limits_{k=1}^{i-2}\left[1-\frac{q(Z_{k:n})}{n-k+1}\right]\\
    &\qquad \times \frac{q(Z_{j-1:n})}{n-j+2} \prod\limits_{k=1}^{j-2}\left[1-\frac{q(Z_{k:n})}{n-k+1}\right]I_{A_{rn}}\\
	&= \sum\limits_{r=1}^{i-1} \frac{q(Z_{i:n+1})}{n-i+2} \prod_{k=1}^{r-1}\left[1-\frac{q(Z_{k:n+1})}{n-k+1}\right]\prod_{k=r}^{i-2}\left[1-\frac{q(Z_{k+1:n+1})}{n-k+1}\right]\\
	&\qquad \times \frac{q(Z_{j:n+1})}{n-j+2} \prod_{k=1}^{r-1}\left[1-\frac{q(Z_{k:n+1})}{n-k+1}\right]\prod_{k=r}^{j-2}\left[1-\frac{q(Z_{k+1:n+1})}{n-k+1}\right]I_{A_{rn}}
\end{align*}
using \eqref{eq:change_order_delta}. We will now continue to find an expression for $T_1$ in terms of $\wn{i}{n+1}$ and $\wn{j}{n+1}$. We have
\begin{align*}		
	T_1	&= \sum\limits_{r=1}^{i-1} \frac{q(Z_{i:n+1})}{n-i+2} \prod_{k=1}^{r-1}\left[1-\frac{q(Z_{k:n+1})}{n-k+1}\right]\prod_{k=r}^{i-2}\left[1-\frac{q(Z_{k+1:n+1})}{n-k+1}\right]\\
		& \qquad \times \frac{q(Z_{j:n+1})}{n-j+2} \prod_{k=1}^{r-1}\left[1-\frac{q(Z_{k:n+1})}{n-k+1}\right]\prod_{k=r}^{j-2}\left[1-\frac{q(Z_{k+1:n+1})}{n-k+1}\right]I_{A_{rn}}\\
		&= \sum\limits_{r=1}^{i-1} \frac{q(Z_{i:n+1})}{n-i+2} \prod_{k=1}^{r-1}\left[1-\frac{q(Z_{k:n+1})}{n-k+2}\right]\prod_{k=r}^{i-2}\left[1-\frac{q(Z_{k+1:n+1})}{n-k+1}\right]\\
		& \qquad \times \frac{q(Z_{j:n+1})}{n-j+2} \prod_{k=1}^{r-1}\left[1-\frac{q(Z_{k:n+1})}{n-k+2}\right]\prod_{k=r}^{j-2}\left[1-\frac{q(Z_{k+1:n+1})}{n-k+1}\right]I_{A_{rn}}\\
		& \qquad \times \left[\frac{\prod_{k=1}^{r-1}\left[1-\frac{q(Z_{k:n+1})}{n-k+1}\right]}{\prod_{k=1}^{r-1}\left[1-\frac{q(Z_{k:n+1})}{n-k+2}\right]}\right]^2\\
		&= \sum\limits_{r=1}^{i-1} \frac{q(Z_{i:n+1})}{n-i+2} \prod_{k=1}^{r-1}\left[1-\frac{q(Z_{k:n+1})}{n-k+2}\right]\prod_{k=r}^{i-2}\left[1-\frac{q(Z_{k+1:n+1})}{n-k+1}\right]\\
		& \qquad \times \frac{q(Z_{j:n+1})}{n-j+2} \prod_{k=1}^{r-1}\left[1-\frac{q(Z_{k:n+1})}{n-k+2}\right]\prod_{k=r}^{j-2}\left[1-\frac{q(Z_{k+1:n+1})}{n-k+1}\right]I_{A_{rn}}\\
		& \qquad \times \prod_{k=1}^{r-1}\left[ \frac{n-k+1-q(Z_{k:n+1})}{n-k+2-q(Z_{k:n+1})}\right]^2\prod_{k=1}^{r-1}\left[ \frac{n-k+2}{n-k+1}\right]^2\\
%\label{eq:}
\end{align*}
%
Now using index transformation on the products $\prod_{k=r}^{i-2}[\ldots]$ and $\prod_{k=r}^{j-2}[\ldots]$ yields
\begin{align*}		
	T_1 &= \sum\limits_{r=1}^{i-1} \frac{q(Z_{i:n+1})}{n-i+2} \prod_{k=1}^{r-1}\left[1-\frac{q(Z_{k:n+1})}{n-k+2}\right]\prod_{k=r+1}^{i-1}\left[1-\frac{q(Z_{k:n+1})}{n-k+2}\right]\\
	& \qquad \times \frac{q(Z_{j:n+1})}{n-j+2} \prod_{k=1}^{r-1}\left[1-\frac{q(Z_{k:n+1})}{n-k+2}\right]\prod_{k=r+1}^{j-1}\left[1-\frac{q(Z_{k:n+1})}{n-k+2}\right]I_{A_{rn}}\\
	& \qquad \times \prod_{k=1}^{r-1}\left[ \frac{n-k+1-q(Z_{k:n+1})}{n-k+2-q(Z_{k:n+1})}\right]^2\prod_{k=1}^{r-1}\left[ \frac{n-k+2}{n-k+1}\right]^2\\
    &= \sum\limits_{r=1}^{i-1} \frac{q(Z_{i:n+1})}{n-i+2} \prod_{k=1}^{i-1}\left[1-\frac{q(Z_{k:n+1})}{n-k+2}\right]\left[1-\frac{q(Z_{r:n+1})}{n-r+2}\right]^{-1}\\
	& \qquad \times \frac{q(Z_{j:n+1})}{n-j+2} \prod_{k=1}^{j-1}\left[1-\frac{q(Z_{k:n+1})}{n-k+2}\right]\left[1-\frac{q(Z_{r:n+1})}{n-r+2}\right]^{-1}I_{A_{rn}}\\
	& \qquad \times \prod_{k=1}^{r-1}\left[ \frac{n-k+1-q(Z_{k:n+1})}{n-k+2-q(Z_{k:n+1})}\right]^2\prod_{k=1}^{r-1}\left[ \frac{n-k+2}{n-k+1}\right]^2\\
	&= \wn{i}{n+1}\wn{j}{n+1}\sum\limits_{r=1}^{i-1}\prod_{k=1}^{r-1} \left[ \frac{n-k+1-q(Z_{k:n+1})}{n-k+2-q(Z_{k:n+1})}\right]^2\prod_{k=1}^{r-1} \left[ \frac{n-k+2}{n-k+1}\right]^2\\
	& \qquad \times \left[ \frac{n-r+2}{n-r+2-q(Z_{r:n+1})}\right]^2I_{A_{rn}}\\
\end{align*}
%
Note that 
\begin{align}
\prod_{k=1}^{r-1}\left[\frac{n-k+2}{n-k+1}\right] &= \frac{n+1}{n} \cdot \frac{n}{n-1} \cdots \frac{n-r+4}{n-r+3} \cdot \frac{n-r+3}{n-r+2}\nonumber\\
&= \frac{n+1}{n-r+2}
\label{eq:telescope_prod}
\end{align}
%
and recall the following definition
\begin{equation*}
	\pi_r = \prod_{k=1}^{r-1} \left[ \frac{n-k+1-q(Z_{k:n+1})}{n-k+2-q(Z_{k:n+1})}\right]
\end{equation*}
Now we finally get	
\begin{align*}		
	T_1 &= \wn{i}{n+1}\wn{j}{n+1}\sum\limits_{r=1}^{i-1}\prod_{k=1}^{r-1} \left[ \frac{n-k+1-q(Z_{k:n+1})}{n-k+2-q(Z_{k:n+1})}\right]^2\\
	& \qquad \times\left[\frac{n+1}{n-r+2}\right]^2 \left[ \frac{n-r+2}{n-r+2-q(Z_{r:n+1})}\right]^2I_{A_{rn}}\\
	&= \wn{i}{n+1}\wn{j}{n+1}\sum\limits_{r=1}^{i-1}\pi_r^2 \left[ \frac{n+1}{n-r+2-q(Z_{r:n+1})}\right]^2I_{A_{rn}}\\
%\label{eq:}
\end{align*}
%
Now let's consider $T_2$. We will, again, firstly express $T_2$ completely in terms of the ordered $Z$ values \wrt\ order $n+1$ using \eqref{eq:change_order_delta}. Consider
\begin{align*}
T_2 &= \sum\limits_{r=i+1}^{j-1} \frac{q(Z_{i:n})}{n-i+1} \prod\limits_{k=1}^{i-1}\left[1-\frac{q(Z_{k:n})}{n-k+1}\right]\\
    & \qquad \times \frac{q(Z_{j-1:n})}{n-j+2} \prod\limits_{k=1}^{j-2}\left[1-\frac{q(Z_{k:n})}{n-k+1}\right]I_{A_{rn}}\\
	&= \sum\limits_{r=i+1}^{j-1} \frac{q(Z_{i:n+1})}{n-i+1} \prod_{k=1}^{i-1}\left[1-\frac{q(Z_{k:n+1})}{n-k+1}\right]\\
	& \qquad \times \frac{q(Z_{j:n+1})}{n-j+2} \prod_{k=1}^{r-1}\left[1-\frac{q(Z_{k:n+1})}{n-k+1}\right]\prod_{k=r}^{j-2}\left[1-\frac{q(Z_{k+1:n+1})}{n-k+1}\right]I_{A_{rn}}
%\label{eq:}
\end{align*}
%
Now let's find a representation of $T_2$ which relies on $\wn{i}{n+1}$ and $\wn{j}{n+1}$ only. Consider
\begin{align*}
	T_2 &= \sum\limits_{r=i+1}^{j-1} \left[\frac{n-i+2}{n-i+1}\right]\left[\frac{q(Z_{i:n+1})}{n-i+2}\right] \prod_{k=1}^{i-1}\left[1-\frac{q(Z_{k:n+1})}{n-k+2}\right] \\
	& \qquad \times \frac{q(Z_{j:n+1})}{n-j+2} \prod_{k=1}^{r-1}\left[1-\frac{q(Z_{k:n+1})}{n-k+2}\right]\prod_{k=r}^{j-2}\left[1-\frac{q(Z_{k+1:n+1})}{n-k+1}\right]I_{A_{rn}}\\
	& \qquad \times \prod_{k=1}^{i-1}\left[\frac{n-k+1-q(Z_{k:n+1})}{n-k+2-q(Z_{k:n+1})}\right] \prod_{k=1}^{i-1}\left[\frac{n-k+2}{n-k+1}\right]\\
	& \qquad \times \prod_{k=1}^{r-1}\left[\frac{n-k+1-q(Z_{k:n+1})}{n-k+2-q(Z_{k:n+1})}\right] \prod_{k=1}^{r-1}\left[\frac{n-k+2}{n-k+1}\right]\\
	&= \left[\frac{n-i+2}{n-i+1}\right]\left[\frac{q(Z_{i:n+1})}{n-i+2}\right] \prod_{k=1}^{i-1}\left[1-\frac{q(Z_{k:n+1})}{n-k+2}\right] \\
	& \qquad \times \prod_{k=1}^{i-1}\left[\frac{n-k+1-q(Z_{k:n+1})}{n-k+2-q(Z_{k:n+1})}\right] \prod_{k=1}^{i-1}\left[\frac{n-k+2}{n-k+1}\right]\\
	& \qquad \times \sum\limits_{r=i+1}^{j-1} \frac{q(Z_{j:n+1})}{n-j+2} \prod_{k=1}^{r-1}\left[1-\frac{q(Z_{k:n+1})}{n-k+2}\right]\prod_{k=r}^{j-2}\left[1-\frac{q(Z_{k+1:n+1})}{n-k+1}\right]I_{A_{rn}}\\
	& \qquad \times \prod_{k=1}^{r-1}\left[\frac{n-k+1-q(Z_{k:n+1})}{n-k+2-q(Z_{k:n+1})}\right] \prod_{k=1}^{r-1}\left[\frac{n-k+2}{n-k+1}\right]\\
\end{align*}
%
Now using \eqref{eq:telescope_prod} on $\prod_{k=1}^{i-1}[\ldots]$ yields
\begin{align*}
&= \left[\frac{n+1}{n-i+1}\right]\left[\frac{q(Z_{i:n+1})}{n-i+2}\right] \prod_{k=1}^{i-1}\left[1-\frac{q(Z_{k:n+1})}{n-k+2}\right] \\
& \qquad \times \prod_{k=1}^{i-1}\left[\frac{n-k+1-q(Z_{k:n+1})}{n-k+2-q(Z_{k:n+1})}\right]\\
& \qquad \times \sum\limits_{r=i+1}^{j-1} \frac{q(Z_{j:n+1})}{n-j+2} \prod_{k=1}^{r-1}\left[1-\frac{q(Z_{k:n+1})}{n-k+2}\right]\prod_{k=r}^{j-2}\left[1-\frac{q(Z_{k+1:n+1})}{n-k+1}\right]I_{A_{rn}}\\
& \qquad \times \prod_{k=1}^{r-1}\left[\frac{n-k+1-q(Z_{k:n+1})}{n-k+2-q(Z_{k:n+1})}\right] \prod_{k=1}^{r-1}\left[\frac{n-k+2}{n-k+1}\right]\\
&= \left[\frac{n+1}{n-i+1}\right]\wn{i}{n+1}\pi_i\\
& \qquad \times \sum\limits_{r=i+1}^{j-1} \frac{q(Z_{j:n+1})}{n-j+2} \prod_{k=1}^{r-1}\left[1-\frac{q(Z_{k:n+1})}{n-k+2}\right]\prod_{k=r}^{j-2}\left[1-\frac{q(Z_{k+1:n+1})}{n-k+1}\right]I_{A_{rn}}\\
& \qquad \times \prod_{k=1}^{r-1}\left[\frac{n-k+1-q(Z_{k:n+1})}{n-k+2-q(Z_{k:n+1})}\right] \prod_{k=1}^{r-1}\left[\frac{n-k+2}{n-k+1}\right]\\
\end{align*}
%
Again doing an index transformation on $\prod_{k=r}^{j-2}[\ldots]$ yields
\begin{align*}
	&= \left[\frac{n+1}{n-i+1}\right]\wn{i}{n+1}\pi_i\\
	& \qquad \times \sum\limits_{r=i+1}^{j-1} \frac{q(Z_{j:n+1})}{n-j+2} \prod_{k=1}^{r-1}\left[1-\frac{q(Z_{k:n+1})}{n-k+2}\right]\prod_{k=r+1}^{j-1}\left[1-\frac{q(Z_{k:n+1})}{n-k+2}\right]I_{A_{rn}}\\
	& \qquad \times \prod_{k=1}^{r-1}\left[\frac{n-k+1-q(Z_{k:n+1})}{n-k+2-q(Z_{k:n+1})}\right] \prod_{k=1}^{r-1}\left[\frac{n-k+2}{n-k+1}\right]I_{A_{rn}}\\	
	&= \wn{i}{n+1} \pi_i \frac{n+1}{n-i+1} \sum\limits_{r=i+1}^{j-1}\frac{q(Z_{j:n+1})}{n-j+2} \prod_{k=1}^{j-1}\left[1-\frac{q(Z_{k:n+1})}{n-k+2}\right] \left[ 1 - \frac{q(Z_{r:n+1})}{n-r+2}\right]^{-1}\\
	& \qquad \times \prod_{k=1}^{r-1} \left[\frac{n-k+1-q(Z_{k:n+1})}{n-k+2-q(Z_{k:n+1})}\right] \prod_{k=1}^{r-1}\left[\frac{n-k+2}{n-k+1}\right]I_{A_{rn}}\\
	&= \wn{i}{n+1}\wn{j}{n+1} \pi_i \frac{n+1}{n-i+1} \\
	& \qquad \times \sum\limits_{r=i+1}^{j-1} \prod_{k=1}^{r-1} \left[\frac{n-k+1-q(Z_{k:n+1})}{n-k+2-q(Z_{k:n+1})}\right] \prod_{k=1}^{r-1}\left[\frac{n-k+2}{n-k+1}\right]\\
	& \qquad \times \frac{n-r+2}{n-r+2-q(Z_{r:n+1})} I_{A_{rn}}\\
\end{align*}
%
Now applying \eqref{eq:telescope_prod} to the latter product yields
\begin{align*}
	T_2	&= \wn{i}{n+1}\wn{j}{n+1} \pi_i \frac{n+1}{n-i+1} \sum\limits_{r=i+1}^{j-1} \pi_r \frac{n+1}{n-r+2-q(Z_{r:n+1})} I_{A_{rn}}	
\end{align*}
%
We will now proceed similarly for $T_3$. Consider
$$T_3 = \sum\limits_{r=j+1}^{n+1} \wn{i}{n}\wn{j}{n}\I{A_{rn}}$$
Note that for $j=n+1$ the sum above is empty and hence zero. Now consider for $j\leq n$
\begin{align*}
	T_3 &= \sum\limits_{r=j+1}^{n+1} \frac{q(Z_{i:n})}{n-i+1} \prod\limits_{k=1}^{i-1} \left[1-\frac{q(Z_{k:n})}{n-k+1}\right]\\
	&\qquad \times \frac{q(Z_{j:n})}{n-j+1} \prod\limits_{k=1}^{j-1} \left[1-\frac{q(Z_{k:n})}{n-k+1}\right]\I{A_{rn}}\\
	&= \sum\limits_{r=j+1}^{n+1} \frac{q(Z_{i:n+1})}{n-i+1} \prod\limits_{k=1}^{i-1} \left[1-\frac{q(Z_{k:n+1})}{n-k+1}\right]\\
	&\qquad \times \frac{q(Z_{j:n+1})}{n-j+1} \prod\limits_{k=1}^{j-1} \left[1-\frac{q(Z_{k:n+1})}{n-k+1}\right]\I{A_{rn}}\\	
	&= \sum\limits_{r=j+1}^{n+1} \frac{n-i+2}{n-i+1}\frac{q(Z_{i:n+1})}{n-i+2} \prod\limits_{k=1}^{i-1} \left[1-\frac{q(Z_{k:n+1})}{n-k+2}\right]\\
	&\qquad \times \frac{n-j+2}{n-j+1}\frac{q(Z_{j:n+1})}{n-j+2} \prod\limits_{k=1}^{j-1} \left[1-\frac{q(Z_{k:n+1})}{n-k+2}\right]\\	
	&\qquad \times \prod\limits_{k=1}^{i-1} \left[\frac{n-k+1-q(Z_{k:n+1})}{n-k+2-q(Z_{k:n+1})}\right]\prod\limits_{k=1}^{i-1} \left[\frac{n-k+2}{n-k+1}\right]\\
	&\qquad \times \prod\limits_{k=1}^{j-1} \left[\frac{n-k+1-q(Z_{k:n+1})}{n-k+2-q(Z_{k:n+1})}\right]\prod\limits_{k=1}^{j-1} \left[\frac{n-k+2}{n-k+1}\right]\I{A_{rn}}\\
	&= \sum\limits_{r=j+1}^{n+1} \frac{n-i+2}{n-i+1}\frac{n-j+2}{n-j+1}\pi_i\pi_j \wn{i}{n+1}\wn{j}{n+1}\\	
	&\qquad \times \prod\limits_{k=1}^{i-1} \left[\frac{n-k+2}{n-k+1}\right] \prod\limits_{k=1}^{j-1} \left[\frac{n-k+2}{n-k+1}\right]\I{A_{rn}}\\
\end{align*}
%
Again, by \eqref{eq:telescope_prod}, we have
\begin{align*}
T_3 &= \sum\limits_{r=j+1}^{n+1} \frac{(n+1)^2\pi_i\pi_j}{(n-i+1)(n-j+1)}\wn{i}{n+1}\wn{j}{n+1}\I{A_{rn}}\\	
\end{align*}
%
Therefore
\[T_3 = \begin{cases} 
      \wn{i}{n+1}\wn{j}{n+1}\pi_i\pi_j\left[\frac{(n+1)^2}{(n-i+1)(n-j+1)}\right]\sum\limits_{i=j+1}^{n+1}\I{A_{rn}} & j\leq n \\
			0 & j=n+1
   \end{cases}
\]
for $1\leq i<j\leq n$.
%
Now using these expressions for $T_1$, $T_2$ and $T_3$ in equation (\ref{eq:sum_t}) together with the fact that 
$$\E[I_{A_{rn}}|\F_{n+1}] = \frac{1}{n+1}$$
yields
\begin{align*}
&  \E[F_n^q\{Z_{i:n+1}\}F_n^q\{Z_{j:n+1}\}|\F_{n+1}]\\
&= \E[T_1 + T_2 + T_3|\F_{n+1}]\\
&= \wn{i}{n+1}\wn{j}{n+1} \times \left\{\sum\limits_{r=1}^{i-1}\pi_r^2\left[\frac{n+1}{n-r+2-q(Z_{r:n+1})} \right]^2 \E[I_{A_{rn}}|\F_{n+1}]\right.\\
& \qquad + \sum\limits_{r=i+1}^{j-1}\pi_i\pi_r \left[\frac{n+1}{n-i+1}\right]\left[\frac{n+1}{n-r+2-q(Z_{r:n+1})}\right]\E[I_{A_{rn}}|\F_{n+1}]\\
& \qquad \left. + \pi_i\pi_j \frac{(n+1)^2}{(n-i+1)(n-j+1)}[1-I_{\{j=n+1\}}]\sum\limits_{i=j+1}^{n+1}\E[I_{A_{rn}}|\F_{n+1}]\right\}\\
&= \wn{i}{n+1}\wn{j}{n+1}\left[\frac{1}{n+1}\right] \times \left\{\sum\limits_{r=1}^{i-1}\pi_r^2\left[\frac{n+1}{n-r+2-q(Z_{r:n+1})} \right]^2\right.\\
& \qquad + \sum\limits_{r=i+1}^{j-1}\pi_i\pi_r \left[\frac{n+1}{n-i+1}\right]\left[\frac{n+1}{n-r+2-q(Z_{r:n+1})}\right]\\
& \qquad \left. + \pi_i\pi_j \frac{(n+1)^2}{n-i+1}[1-I_{\{j=n+1\}}]\right\}
%\label{eq:}
\end{align*}
%
Next consider that we have  
\begin{align*}
&  \E[F_n^q\{Z_{i:n+1}\}F_n^q\{Z_{j:n+1}\}|\F_{n+1}]\\
&= \wn{i}{n+1}\wn{j}{n+1}(n+1) \left\{\sum\limits_{r=1}^{i-1}\left[\frac{\pi_r}{n-r+2-q(Z_{r:n+1})} \right]^2\right.\\
& \qquad + \left. \frac{\pi_i}{n-i+1} \left[\sum\limits_{r=i+1}^{j-1} \left[\frac{\pi_r}{n-r+2-q(Z_{r:n+1})}\right] + \pi_j \right]\right\}\\
\end{align*}
for $1\leq i<j\leq n$. 
%
Now applying \eqref{eq:Ai} yields
\begin{align*}
&= \wn{i}{n+1}\wn{j}{n+1}(n+1) \left\{\sum\limits_{r=1}^{i-1}\left[\frac{\pi_r}{n-r+2-q(Z_{r:n+1})} \right]^2\right.\\
& \qquad + \left. \frac{\pi_i}{n-i+1} (A_j - A_{i+1} + \pi_{i+1})\right\}\\
&= \wn{i}{n+1}\wn{j}{n+1}(n+1) \left\{\sum\limits_{r=1}^{i-1}\left[\frac{\pi_r}{n-r+2-q(Z_{r:n+1})} \right]^2\right.\\
& \qquad + \left. \frac{\pi_i\pi_{i+1}}{n-i+1}\right\}\\
&= \wn{i}{n+1}\wn{j}{n+1}Q_i^{n+1}
%\label{eq:}
\end{align*}
%
It remains to consider the case $j=n+1$. We have
\begin{align*}
&  \E[F_n^q\{Z_{i:n+1}\}F_n^q\{Z_{j:n+1}\}|\F_{n+1}]\\
&= \wn{i}{n+1}W_{n+1:n+1}(n+1) \left\{\sum\limits_{r=1}^{i-1}\left[\frac{\pi_r}{n-r+2-q(Z_{r:n+1})} \right]^2\right.\\
& \qquad + \left. \frac{\pi_i}{n-i+1}\sum\limits_{r=i+1}^{n} \left[\frac{\pi_r}{n-r+2-q(Z_{r:n+1})}\right]\right\}\\
&= \wn{i}{n+1}W_{n+1:n+1}(n+1) \left\{\sum\limits_{r=1}^{i-1}\left[\frac{\pi_r}{n-r+2-q(Z_{r:n+1})} \right]^2\right.\\
& \qquad + \left. \frac{\pi_i}{n-i+1}\left[\sum\limits_{r=1}^{n} \left[\frac{\pi_r}{n-r+2-q(Z_{r:n+1})}\right] - \sum\limits_{r=1}^{i} \left[\frac{\pi_r}{n-r+2-q(Z_{r:n+1})}\right]\right]\right\}\\
&= \wn{i}{n+1}W_{n+1:n+1}(n+1) \left\{\frac{Q_i^{n+1}}{n+1} - \frac{\pi_i\pi_{i+1}}{n-i+1}\right.\\
& \qquad + \left. \frac{\pi_i}{n-i+1}\left[\sum\limits_{r=1}^{n} \left[\frac{\pi_r}{n-r+2-q(Z_{r:n+1})}\right] - \sum\limits_{r=1}^{i}  \left[\frac{\pi_r}{n-r+2-q(Z_{r:n+1})}\right]\right]\right\}
\end{align*}
Now using \eqref{eq:Ai} again yields
\begin{align*}
&= \wn{i}{n+1}W_{n+1:n+1}(n+1) \left\{\frac{Q_i^{n+1}}{n+1} - \frac{\pi_i\pi_{i+1}}{n-i+1}\right.\\
& \qquad \left. + \frac{\pi_i}{n-i+1}\left[A_{n+1} - \pi_{n+1} - (A_{i+1}-\pi_{i+1}) \right] \right\}\\
&= \wn{i}{n+1}W_{n+1:n+1}(n+1) \left\{\frac{Q_i^{n+1}}{n+1} - \frac{\pi_i\pi_{i+1}}{n-i+1}\right.\\
& \qquad \left. + \frac{\pi_i}{n-i+1}\left[  \pi_{i+1} - \pi_{n+1} \right] \right\}\hphantom{asdfasdfasfsadfasdfasdfasfsadfasdfasdfasfsa}
\end{align*}
Note that for $1\leq i\leq n$ we have
$$\pi_{i+1} = \frac{\pi_i (1-q(Z_{i:n+1}))}{2-q(Z_{i:n+1})}$$
Thus we obtain
\begin{align*}
&  \E[F_n^q\{Z_{i:n+1}\}F_n^q\{Z_{j:n+1}\}|\F_{n+1}]\\
&= \wn{i}{n+1}W_{n+1:n+1}(n+1) \left\{\frac{Q_i^{n+1}}{n+1} - \frac{\pi_i\pi_{i+1}}{n-i+1}\right.\\
& \qquad\qquad\qquad \left. + \frac{\pi_i}{n-i+1}\left[\pi_{i+1} - \frac{\pi_n(1-q(Z_{n:n+1}))}{2-q(z_{n:n+1})}   \right] \right\}\\
&= \wn{i}{n+1}W_{n+1:n+1}(n+1) \left\{\frac{Q_i^{n+1}}{n+1} - \frac{\pi_i\pi_n(1-q(Z_{n:n+1}))}{(n-i+1)(2-q(Z_{n:n+1}))} \right\}\\
&= \wn{i}{n+1}W_{n+1:n+1} \left\{Q_i^{n+1} - \frac{\pi_i\pi_n(n+1)(1-q(Z_{n:n+1}))}{(n-i+1)(2-q(Z_{n:n+1}))} \right\}
%\label{eq:}
\end{align*}
\end{proof}
%
The following result on the increases of $Q_i^{n+1}$ will be useful for the proof of Lemma \ref{lem:qisquare_upper_bound}.
\begin{lemma}
	Let $Q_i^{n+1}$ be defined as in lemma \ref{lem:qi} for $1\leq i\leq n$. Moreover define 
	$$\tilde{\pi}_i := \prod\limits_{k=1}^{i-1} \left[\frac{n-k+1-q(Z_{k:n+1})}{n-k+2-q(Z_{k:n+1})}\right]\prod\limits_{k=1}^{i-1} \left[\frac{n-k+2}{n-k+1}\right]$$
	%
	Then we have
	\begin{align*}
		Q_{i+1}^{n+1} - Q_{i}^{n+1} &= \frac{(q_i-q_{i+1})(n-i)(n-i+1)-q_{i+1}(1-q_i)(n-i+1-q_i) }{(n-i)(n-i+1)(n-i+2-q_i)^2(n-i+1-q_{i+1})}\\
		&\qquad\times \frac{\tilde{\pi}_i(n-i+2)^2}{n+1}
	\end{align*}
	\label{lem:qi_increas}
%
	\begin{proof}
		For the sake of simplicity we will write $q_{i}\equiv q(Z_{i:n+1})$ during this proof. From equation (\ref{eq:qi}) we get
		\begin{align}
		\frac{Q_{i+1}^{n+1}-Q_i^{n+1}}{n+1} &= \left\{ \sum\limits_{r=1}^{i} \left[ \frac{\pi_r}{n-r+2-q_r} \right]^2 + \frac{\pi_{i+1}\pi_{i+2}}{n-i} \right\}\nonumber\\
		& - \left\{ \sum\limits_{r=1}^{i-1} \left[ \frac{\pi_r}{n-r+2-q_r} \right]^2 + \frac{\pi_i\pi_{i+1}}{n-i+1} \right\}\nonumber\\
		&= \frac{\pi_i^2}{(n-i+2-q_i)^2}+\frac{\pi_{i+1}\pi_{i+2}}{n-i}-\frac{\pi_{i}\pi_{i+1}}{n-i+1} \nonumber\\
		&= \frac{\pi_i^2}{(n-i+2-q_i)^2} + \frac{\pi_{i}^2(n-i+1-q_i)^2(n-i-q_{i+1})}{(n-i)(n-i+2-q_i)^2(n-i+1-q_{i+1})}\nonumber\\
		&  - \frac{\pi_{i}^2(n-i+1-q_i)}{(n-i+1)(n-i+2-q_i)} \nonumber\\
		&= \pi_i^2\left\{\frac{1}{(n-i+2-q_i)^2} + \frac{(n-i+1-q_i)^2(n-i-q_{i+1})}{(n-i)(n-i+2-q_i)^2(n-i+1-q_{i+1})}\right.\nonumber\\
		&  \left.\qquad\quad - \frac{n-i+1-q_i}{(n-i+1)(n-i+2-q_i)} \right\}\nonumber\\
		&=: \pi_i^2\left\{a(n,i) + b(n,i) - c(n,i) \right\}
		\label{eq:qi_diffaa}
		\end{align}
		%
		Now consider 
		\begin{align}
		& b(n,i) - c(n,i) \nonumber\\
		&= (n-i+1-q_i)\left[\frac{(n-i+1-q_i)(n-i-q_{i+1})}{(n-i)(n-i+2-q_i)^2(n-i+1-q_{i+1})}\right.\nonumber \\
		& \left. \qquad\qquad\qquad\qquad - \frac{1}{(n-i+1)(n-i+2-q_i)}\right]\nonumber\\
		&= (n-i+1-q_i)\left[\frac{(n-i+1-q_i)(n-i-q_{i+1})(n-i+1)}{(n-i)(n-i+1)(n-i+2-q_i)^2(n-i+1-q_{i+1})}\right.\nonumber\\
		& \left. \qquad\qquad\qquad\qquad - \frac{(n-i+2-q_i)(n-i+1-q_{i+1})(n-i)}{(n-i)(n-i+1)(n-i+2-q_i)^2(n-i+1-q_{i+1})}\right]
		\label{eq:large_numerator}
		\end{align}
		%
		Next we will simplify the difference of the numerators above. We have 
		\begin{align}
		&\quad (n-i+1-q_i)(n-i-q_{i+1})(n-i+1)\nonumber\\
		&\quad- (n-i+2-q_i)(n-i+1-q_{i+1})(n-i)\nonumber\\
		&= (n-i+1-q_i)(n-i)(n-i+1)- q_{i+1}(n-i+1-q_i)(n-i+1)\nonumber\\
		&\quad- (n-i+2-q_i)(n-i+1-q_{i+1})(n-i)\nonumber\\
		&= (n-i+1-q_i)(n-i)(n-i+1)- q_{i+1}(n-i+1-q_i)(n-i+1)\nonumber\\
		&\quad- (n-i+1-q_i)(n-i+1-q_{i+1})(n-i)- (n-i+1-q_{i+1})(n-i)\nonumber\\
		&= (n-i+1-q_i)(n-i)(n-i+1)- q_{i+1}(n-i+1-q_i)(n-i+1)\nonumber\\
		&\quad- (n-i+1-q_i)(n-i+1)(n-i)+ q_{i+1}(n-i+1-q_i)(n-i)\nonumber\\
		&\quad- (n-i+1-q_{i+1})(n-i)\nonumber\\
		&= - q_{i+1}(n-i+1-q_i) - (n-i+1-q_{i+1})(n-i)\nonumber
		\end{align}
		%
		Hence we get, according to \eqref{eq:large_numerator} 
		\begin{align*}
		& b(n,i) - c(n,i) \nonumber\\
		&= -(n-i+1-q_i)\left[\frac{ q_{i+1}(n-i+1-q_i) + (n-i+1-q_{i+1})(n-i)}{(n-i)(n-i+1)(n-i+2-q_i)^2(n-i+1-q_{i+1})}\right]
		\end{align*}
		%
		Therefore we have 
		\begin{align}
		& a(n,i) + b(n,i) - c(n,i) \nonumber\\
		&= \frac{1}{(n-i+2-q_i)^2}\nonumber\\
		&\quad - \frac{q_{i+1}(n-i+1-q_i)^2 + (n-i+1-q_i)(n-i+1-q_{i+1})(n-i)}{(n-i)(n-i+1)(n-i+2-q_i)^2(n-i+1-q_{i+1})}\nonumber\\
		&= \frac{(n-i)(n-i+1)(n-i+1-q_{i+1})}{(n-i)(n-i+1)(n-i+2-q_i)^2(n-i+1-q_{i+1})}\nonumber\\
		&\quad - \frac{q_{i+1}(n-i+1-q_i)^2 + (n-i+1-q_i)(n-i+1-q_{i+1})(n-i)}{(n-i)(n-i+1)(n-i+2-q_i)^2(n-i+1-q_{i+1})}\nonumber
		\end{align}
		%
		Consider again the numerator of the latter expression. We have
		\begin{align}
		&= (n-i)(n-i+1)(n-i+1-q_{i+1}) - q_{i+1}(n-i+1-q_i)^2 \nonumber\\
		& \quad - (n-i)(n-i+1-q_i)(n-i+1-q_{i+1})\nonumber\\
		&= q_i(n-i)(n-i+1-q_{i+1}) - q_{i+1}(n-i+1-q_i)^2 \nonumber\\
		&= q_i(n-i)^2 + q_i(1-q_{i+1})(n-i) -q_{i+1}(n-i)^2 \nonumber\\
		&\quad - 2q_{i+1}(1-q_i)(n-i) - q_{i+1}(1-q_i)^2 \nonumber\\	
		&= (q_i-q_{i+1})(n-i)^2 + q_i(n-i) - q_iq_{i+1}(n-i) \nonumber\\
		&\quad - 2q_{i+1}(n-i) + 2q_iq_{i+1}(n-i) - q_{i+1}(1-q_i)^2 \nonumber\\
		&= (q_i-q_{i+1})(n-i)^2 + (q_i + q_iq_{i+1} - 2q_{i+1})(n-i) - q_{i+1}(1-q_i)^2 \nonumber
		\end{align}
		%
		Note that $q_i - q_{i+1} \geq -1$ and $q_i + q_iq_{i+1} - 2q_{i+1} \geq -2$, since $0\leq q_i \leq 1$ for $1\leq i \leq n$. Similarly we have $q_{i+1}(1-q_i)^2 \leq 1$. Thus we get 
		\begin{align}
		& a(n,i) + b(n,i) - c(n,i) \nonumber\\
		&= \frac{(q_i-q_{i+1})(n-i)^2 + (q_i + q_iq_{i+1} - 2q_{i+1})(n-i) - q_{i+1}(1-q_i)^2}{(n-i)(n-i+1)(n-i+2-q_i)^2(n-i+1-q_{i+1})}\nonumber\\
		&= \frac{(q_i-q_{i+1})(n-i)^2 + [(q_i - q_{i+1}) - q_{i+1}(1-q_i))(n-i) - q_{i+1}(1-q_i)^2}{(n-i)(n-i+1)(n-i+2-q_i)^2(n-i+1-q_{i+1})}\nonumber\\
		&= \frac{(q_i-q_{i+1})(n-i)(n-i+1) - q_{i+1}(1-q_i)(n-i+1-q_i)}{(n-i)(n-i+1)(n-i+2-q_i)^2(n-i+1-q_{i+1})}\label{eq:abc}
		\end{align}
		%
		Finally note that 
		\begin{align}
		\tilde{\pi}_i &= \frac{n+1}{n-i+2}\prod\limits_{k=1}^{i-1} \left[\frac{n-k+1-q(Z_{k:n+1})}{n-k+2-q(Z_{k:n+1})}\right]\nonumber\\
		&= \pi_i \cdot \frac{n+1}{n-i+2}
		\label{eq:pi_tilde_pi}
		\end{align}
		with $\pi_i$ as defined in Lemma \ref{lem:qi}. Now the statement of the lemma follows directly from \eqref{eq:qi_diffaa}, \eqref{eq:abc} and \eqref{eq:pi_tilde_pi}
	\end{proof}
\end{lemma}
%
%
%
\section{$S_n$ is not necessarily a reverse supermartingale} \label{sec:not_supermart}
As discussed in Chapter \ref{ch:introduction}, the Strong Law of Large Numbers for Kaplan-Meier U-Statistics of degree $2$ was established by \cite{bose1999strong}. Recall definition of said estimator: 
\begin{equation*}
	S_n^{km} = \doublesum\limits_{1\leq i<j\leq n} \phi(Z_{i:n}, Z_{j:n}) W_{i:n}^{km} W_{j:n}^{km}
\end{equation*}
with
\begin{equation*}
	W_{i:n}^{km} = \frac{\delta_{[i:n]}}{n-i+1}\prod\limits_{k=1}^{i-1}\left[1-\frac{\delta_{[k:n]}}{n-k+1}\right]
\end{equation*}
The prove of existence of the limit $S = \lim_{n\to\infty} S_n^{km}$ was here essentially based on a supermartingale argument together with \cite{neveu1975discrete}, proposition 5-3-11. In Lemma 1 of \cite{bose1999strong} a representation for $\E[S_n^{km}| \F_{n+1}]$ was derived, which is similar to our lemma \ref{lem:qi}. It was shown that for $1\leq i<j\leq n$
$$\E[S_n^{km}|\F_{n+1}] = \doublesum\limits_{1\leq i<j \leq n+1} \phi(Z_{i:n+1},Z_{j:n+1}) W_{i:n+1}^{km}W_{j:n+1}^{km}Q_{ij}^{km}$$
where 
\[Q_{ij}^{km} = \begin{cases} 
Q_i^{km} &\textrm{if } j\leq n \\
Q_i^{km} - \pi_i \pi_n (1-\delta_{[n:n+1]}))\frac{n-i+2}{(n+1)(n-i+1)} &\textrm{if } j=n+1
\end{cases}
\]
and
\begin{align*}
Q_i^{km} &= \frac{1}{n+1} \left\{ \sum\limits_{r=1}^{i-1} \pi_r^2 \left[ \frac{n-r+2}{n-r+1} \right]^{2\delta_{[r:n+1]}} \right.\\
		 & \quad \left. + \pi_i^2 (n-i+2) \left[\frac{(n-i)(n-i+2)}{(n-i+1)^2}\right]^{\delta_{[i:n+1]}} \right\}
\end{align*}
Then \cite{bose1999strong} show that $Q_{ij}^{km}\leq 1$ for $1\leq i<j\leq n$, in order to establish the reverse time supermartingale property for $(S_n^{km}, \F_n)$. However their prove relies on the fact that 
\begin{align*}
	W_{i:n}^{km} &= \frac{\delta_{[i:n]}}{n-i+1}\prod\limits_{k=1}^{i-1}\left[1-\frac{\delta_{[k:n]}}{n-k+1}\right] \\
				 &= \frac{\delta_{[i:n]}}{n-i+1}\prod\limits_{k=1}^{i-1}\left[1-\frac{1}{n-k+1}\right]^{\delta_{[k:n]}}
\end{align*}
%
But the corresponding statement is not true for $W_{i:n}$, since we have in general that
\begin{align*}
W_{i:n} &= \frac{q(Z_{i:n})}{n-i+1}\prod\limits_{k=1}^{i-1}\left[1-\frac{q(Z_{k:n})}{n-k+1}\right] \\
&\neq \frac{q(Z_{i:n})}{n-i+1}\prod\limits_{k=1}^{i-1}\left[1-\frac{1}{n-k+1}\right]^{q(Z_{k:n})}
\end{align*}
%
In \cite{dikta2000strong}, the following estimator was considered
$$S_n^{se}(q) = \sum_{i=1}^n \phi(Z_{i:n}) W_{i:n}^{se}$$
The proof of existence of the limit $S^{se} = \lim\limits_{n\to\infty} S_n^{se}$ shows a similar structure, as the one in \cite{bose1999strong}. In Lemma 2.1 of \cite{dikta2000strong}, it was shown that $\E[\mu_n\{Z_{1:n+1}\} | \F_{n+1}] = W_{1:n}^{se}$ and for $2\leq i\leq n$
$$\E[\mu_n\{Z_{i:n+1}\} | \F_{n+1}] = W_{i:n}^{se}Q_i^{se}$$
where $\mu_n$ is the measure assigning mass $W_{i:n}$ to  $Z_{i:n}$ and
$$Q_i^{se} = \pi_i + \sum_{k=1}^{i-1} \frac{\pi_k}{n-k+2-q(Z_{k:n+2})}$$
Here $\pi_i$ is defined as in Lemma \ref{lem:qi}. Furthermore it was shown that $Q_i^{se} = Q_{i+1}^{se} = 1$ for all $2\leq i\leq n$, which, among other arguments, implies the reverse supermartingale property for $S_n^{se}$.\\
\\
Within the framework of this thesis, we were neither able to show $Q_i = Q_{i+1} = 1$, nor to show that $Q_i \leq 1$ for all $1\leq i\leq n$. Therefore it is not clear, if $\{\snse{2}{n}, \F_{n+1}\}$ is a reverse supermartingale. However, in Lemma \ref{lem:qi_increas} we derived a representation for the increases $Q_{i+1} - Q_i$, which we will use in Chapter 4 in order to  generalize Doob's Upcrossing Theorem to our framework. Afterwards we will derive conditions under which the expected number of upcrossings is finite in Chapter 4. This will then imply the almost sure existence of the limit of $\snse{2}{n}$.

