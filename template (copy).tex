%%%%%%%%%%%%%%%%%%%%%%%%%%%%%%%%%%%%%%%%%%%%%%%%%%%%%%%%%%%%%%%%%%%%%%%%
% Template for a Master's Thesis or Ph.D. dissertation
% at the University of Wisconsin-Milwaukee.
%
% Designed for LaTeX version 2e
%
% Updated by Adam J. Smith
% December, 2007
%
% This thesis template requires the file "UWMthesis.sty", available
% on the UWM's Atmospheric Science Club website, and possible in other
% locations.
%
% A LaTeX primer is not provided here.  For instructions on how to use commands for
% figures, table, equations, and bibliography citations, please see the documentation
% listed later in this document.
% 
% This template follows the "Fall 2007" version of the 
% University of Wisconsin-Milwaukee standards for the Master's thesis
% and Ph.D. dissertation.  Please feel free to update this template
% as needed to comply with these standards.
%
% For current thesis and dissertation formatting information, visit the UWM graduate school
% website at the following address:
% http://www.graduateschool.uwm.edu/students/current/thesis-and-dissertation-formatting/
%
% IMPORTANT: Be sure to meet with the appropriate Graduate School personnel to verify
% whether your final document meets the UWM standards.  If not, the document may not
% be accepted until any problems are corrected.
%%%%%%%%%%%%%%%%%%%%%%%%%%%%%%%%%%%%%%%%%%%%%%%%%%%%%%%%%%%%%%%%%%%%%%%%

% If you are writing a master's thesis, use the first option (master).
% If you are writing a phd dissertation, use the second option (phd).
%\documentclass[master]{UWMThesis}
\documentclass[phd]{UWMThesis}

\usepackage{fancyhdr}
\usepackage{xcolor}
\usepackage{amsmath}
\usepackage{amsthm}
\usepackage{bbm}
\usepackage{amssymb}
\usepackage{enumerate}
\usepackage{kantlipsum}
\usepackage{tabularx}

%\usepackage{todonotes}
%\usepackage{todo}
\usepackage{fixmetodonotes}
\defnote{Comment}{inline}{\hspace{10pt}}
\defnote{Note}{inline}{\hspace{10pt}\fbox}
%------------------------------------

% Uncomment the following "usepackage" line if you wish to use BiBTeX to create the
% bibliography.  This package is necessary to use the \citep or \citet commands, which are
% commonly used in publications like the Journal of Geophysical Research.
% If the style file "natbib.sty" is not provided with your release of MiKTeX, it is available
% on the Internet.  One example web source is:
% http://ads.harvard.edu/pubs/bibtex/astronat/natbib.sty
\usepackage{natbib}
\usepackage[ngerman, english]{babel} 
%------------------------------------

%Absolute value nice definition
\usepackage{mathtools}

\DeclarePairedDelimiter\abs{\lvert}{\rvert}%
\DeclarePairedDelimiter\norm{\lVert}{\rVert}%

% Swap the definition of \abs* and \norm*, so that \abs
% and \norm resizes the size of the brackets, and the 
% starred version does not.
\makeatletter
\let\oldabs\abs
\def\abs{\@ifstar{\oldabs}{\oldabs*}}
%
\let\oldnorm\norm
\def\norm{\@ifstar{\oldnorm}{\oldnorm*}}
\makeatother
%------------------------------------

% Uncomment this package if you want to use graphics files, such as .eps files
\usepackage{graphicx}

% Other packages go here as needed...
%\usepackage{mathabx}
\usepackage[colorlinks=true, linkcolor=blue, citecolor=blue]{hyperref}

% make todo note lines fancy af and display at left margin
%\let\tmptodo\todo
%\renewcommand{\todo}[1]{\tmptodo[fancyline]{#1}}
%\reversemarginpar

% Why not..
\allowdisplaybreaks 
%------------------------------------

%%%%%%%%%%%%%%%%%%%%%%%%%%%%%%%%%%%%%%%
%%%%%%%%%%%% Customization %%%%%%%%%%%%
%%%%%%%%%%%%%%%%%%%%%%%%%%%%%%%%%%%%%%%

% ----- my commands -----
\newcommand{\comment}[1]{\fbox{\begin{minipage}{\textwidth}\Comment{\textcolor{blue}{#1}}\end{minipage}}\\}
\newcommand\numberthis{\addtocounter{equation}{1}\tag{\theequation}}
\newcommand{\intrange}[3]{$#1 = #2, \dots, #3$}
\newcommand{\itref}[2]{(#1\ref{#2})}

\newenvironment{myarray}{\begin{center}$\begin{array}{ll} }{\end{array}$ \end{center}}

\renewcommand{\P}{\mathbb{P}}
\newcommand{\E}{\mathbb{E}}
\newcommand{\R}{\mathbb{R}}
\newcommand{\F}{\mathcal{F}}
\newcommand{\A}{\mathcal{A}}
\newcommand{\G}{\mathcal{G}}
\newcommand{\N}{\mathcal{N}}

\newcommand{\fnkm}{F_n^{km}}
\newcommand{\fnse}{F_n^{se}}
\newcommand{\wn}[2]{W_{#1:#2}}
\newcommand{\wnkm}[2]{W_{#1:#2}^{km}}
\newcommand{\wnse}[2]{W_{#1:#2}^{se}}
\newcommand{\wnseb}[2]{\bar{W}_{#1:#2}^{se}}
\newcommand{\sn}[1]{S_{#1}}
\newcommand{\snkm}[2]{S_{#1,#2}^{km}}
\newcommand{\snse}[2]{S_{#1,#2}^{se}}
\newcommand{\stnse}[1]{S_{2,#1}^{se}}
\newcommand{\snseb}[2]{\bar{S}_{#1,#2}^{se}}
\newcommand{\unkm}[2]{U_{#1,#2}^{km}}
\newcommand{\unse}[2]{U_{#1,#2}^{se}}

\newcommand{\StN}[1]{\tilde{S}_{#1}^N}
\newcommand{\FtN}[1]{\tilde{\F}_{#1}^N}
\newcommand{\YN}[1]{Y_{#1}^N}
\newcommand{\xitN}[1]{\tilde{\xi}_{#1}^N}
\newcommand{\UNab}[1]{U_{#1}^N[a,b]}

\newcommand{\I}[1]{{\mathbbm{I}\{#1\}}}
%\newcommand{\I}[1]{{\mathbbm{1}\{#1\}}}

\newcommand{\df}{d.\,f.}
\newcommand{\ie}{i.\,e.}
\newcommand{\iid}{i.\,i.\,d.}
\newcommand{\rv}{r.\,v.}
%\newcommand{\st}{s.\,t.}
\newcommand{\wpo}{w.\,p.\,1}
\newcommand{\as}{a.\,s.}
\newcommand{\st}{s.\,t.}
\newcommand{\wrt}{w.\,r.\,t.}

\newcommand{\cfbox}[2]{%
	\colorlet{currentcolor}{.}%
	{\color{#1}%
		\fbox{\color{currentcolor}#2}}%
}
\renewcommand{\.}{\textrm{ .}}


% \newtheorem{lemma}{Lemma}
% \newtheorem{theorem}{Theorem}

% ----- my mathoperators -----
\newcommand{\doublesum}{\mathop{\sum\sum}}
\newcommand{\qeq}{\mathop{\stackrel{?}{=}}}
\newcommand{\qleq}{\mathop{\stackrel{?}{\leq}}}
\newcommand{\qgeq}{\mathop{\stackrel{?}{\geq}}}

\newcommand{\cf}{c.\,f.}
\newcommand{\mdot}{\textrm{ .}}
\newcommand{\mcomma}{\textrm{ ,}}

% ----- UWM stylings -----
%\renewcommand{\evensidemargin}{.875in}  
%\renewcommand{\oddsidemargin}{.875in}   
%\renewcommand{\topmargin}{-.3in}
%\renewcommand{\headheight}{0.2in}
%\renewcommand{\marginparwidth}{1.4in}
%\renewcommand{\marginparsep}{1.4pt}
%\renewcommand{\headsep}{.65in}
%\renewcommand{\footskip}{0.3in}
%\renewcommand{\textheight}{8in}
%\renewcommand{\textwidth}{5.9in} % crashes todonotes

\newcommand{\ls}{\vspace{.1in}}

\newtheorem{thm}{Theorem}
\newtheorem{lemma}[thm]{Lemma}
\newtheorem{cor}[thm]{Corollary}
\newtheorem{prop}[thm]{Proposition}
\theoremstyle{definition}
\newtheorem{example}[thm]{Example}
\newtheorem{remark}[thm]{Remark}
\newtheorem{defn}[thm]{Definition}
\newtheorem{ques}[thm]{Question}
\newtheorem{exer}[thm]{Exercise}
\numberwithin{thm}{chapter}

\newcommand{\todo}{\TODO}

% \renewcommand{\phi}{\varphi}

%%%%%%%%%%%%%%%%%%%%%%%%%%%%%%%%%%%%%%%%%
%%%%%%%%%%%% Personalization %%%%%%%%%%%%
%%%%%%%%%%%%%%%%%%%%%%%%%%%%%%%%%%%%%%%%%

% Insert your full name in the brackets.
\renewcommand{\ThesisAuthor}{Jan Hoft}

% If you are graduating in Spring, insert May.  If you are graduating in Fall, insert December.
\renewcommand{\ThesisMonth}{May}

% Insert the year of your graduation here
\renewcommand{\ThesisYear}{2018}

% Your thesis title goes here.  It will automatically be formatted to use multiple lines
% (if needed)
\renewcommand{\ThesisTitle}{Large sample properties of U-Statistics under semiparametric Random Censorship}

% Insert your advising professor's name here.  DO NOT include a prefix of "Prof." or "Dr."
% here!  The prefix will be inserted automatically.
\renewcommand{\ThesisAdvisor}{Gerhard Dikta and Professor Jugal Ghorai}

%------------------------------------

% If your thesis or dissertation has multiple volumes, set the argument to true.
% If not, set the argument to false.
\setboolean{multvolumes}{false}

% If your thesis or dissertation has multiple appendices, set the argument to true.
% If not, set the argument to false.
\setboolean{singleappendix}{false}

%---------------------------------------

% Creating new commands for displaying derivatives
% Add additional new commands as needed...
\newcommand{\ptlder}[2]{\frac{\partial #1}{\partial #2}}
\newcommand{\totder}[2]{\frac{d #1}{d #2}}

%---------------------------------------

%%%%%%%%%%%%%%%%%%%%%%%%%%%%%%%%%%%%%%%%%%%%%%%%%%%%%%%%%%%%%%%%%%%%%%%%%%%%%%%%%%%%%%%%%%%
%%%%%%%%%%%%%%%%%%%%%%%%%%%%%%%%%% BEGINNING OF DOCUMENT %%%%%%%%%%%%%%%%%%%%%%%%%%%%%%%%%%
%%%%%%%%%%%%%%%%%%%%%%%%%%%%%%%%%%%%%%%%%%%%%%%%%%%%%%%%%%%%%%%%%%%%%%%%%%%%%%%%%%%%%%%%%%%
\begin{document}
	
	% \listofnotes
	
	\tableofcontents

	\chapter{Notation and assumptions} \label{ch:notation}
In this chapter we will state the main definitions and assumptions used throughout this work. We will start by defining the estimator to be considered and introduce all necessary notation for the remaining chapters.
%
\section{Definitions and notation}
Recall the following definitions for $n\geq2$
$$\wnse{i}{n} = \frac{m(Z_{i:n}, \hat{\theta}_n)}{n-i+1} \prod\limits_{j=1}^{i-1}\left(1-\frac{m(Z_{j:n}, \hat{\theta}_n)}{n-j+1}\right)$$
%
and
$$S_{2,n}^{se} = \doublesum\limits_{1\leq i<j\leq n}\phi(Z_{i:n}, Z_{j:n})W_{i:n}^{se}W_{j:n}^{se}$$
%
Furthermore define
$$W_{i:n}(q) = \frac{q(Z_{i:n})}{n-i+1}\prod_{k=1}^{i-1}\left[1-\frac{q(Z_{k:n})}{n-k+1}\right]$$
and 
$$S_{n}(q) = \doublesum\limits_{1\leq i<j\leq n}\phi(Z_{i:n}, Z_{j:n})W_{i:n}(q)W_{j:n}(q)$$
for some measurable function $q$ \st\ $q(t)\in[0,1]$ for all $t\in\R_+$.
%
Next define
$$\F_n = \sigma\{Z_{1:n}, \dots, Z_{n:n}, Z_{n+1}, Z_{n+2}, \dots\}$$
%
The following quantities will be needed in section \ref{sec:supermart}. Define for $n\geq 2$ and $s < t$
\begin{align*}
B_n(s,q) &:= \prod_{k=1}^{n}\left[1+\frac{1-q(Z_{k})}{n-R_{k,n}}\right]^{\I{Z_{k} < s}}\\
C_n(s,q) &:= \sum_{i=1}^{n+1}\left[\frac{1-q(s)}{n-i+2}\right]\I{Z_{i-1:n} < s \leq Z_{i:n}}\\
D_n(s,t,q) &:= \prod_{k=1}^{n} \left[1+\frac{1-q(Z_k)}{n-R_{k,n} +2}\right]^{2\I{Z_k<s}} \prod_{k=1}^{n}\left[1+\frac{1-q(Z_k)}{n-R_ {k,n}+1}\right]^{\I{s < Z_k < t}}\\
\Delta_n(s,t,q) &:= \E\left[D_n(s,t,q) \right]\\
\bar{\Delta}_n(s,t,q) &:= \E\left[C_n(s,q)D_n(s,t,q) \right]
\end{align*}
and
$$D(s,t,q) := \exp\left(2\int_{0}^{s} \frac{1-q(z)}{1-H(z)} H(dz) + \int_{s}^{t} \frac{1-q(z)}{1-H(z)} H(dz)\right)\mdot$$
We will write $B_n(s) \equiv B_n(s,q)$, $C_n(s) \equiv C_n(s,q)$, $D_n(s,t) \equiv D_n(s,t,q)$, $\Delta_n(s,t) \equiv \Delta_n(s,t,q)$, $\bar\Delta_n(s,t) \equiv \bar\Delta_n(s,t,q)$ and $D(s,t) \equiv D(s,t,q)$. Next let
$$\bar{S}_n(q) := \doublesum\limits_{1\leq i < j \leq n} \phi(Z_{i:n},Z_{j:n}) \bar{W}_{i:n}(q) \bar{W}_{j:n}(q)$$
where 
$$\bar{W}_{i:n}(q) := \frac{1}{n-i+1}\prod_{k=1}^{n}\left(1-\frac{q(Z_{k:n})}{n-k+1}\right)\mdot$$
Moreover define for $s<t$
\begin{align*}
S(q) &:= \frac{1}{2}\int_{0}^{\infty} \int_{0}^{\infty} \phi(s,t) q(s)q(t) \exp\left(\int_{0}^{s} \frac{1-q(x)}{1-H(x)} H(dx)\right)\\
&\qquad\qquad\qquad \times \exp\left(\int_{0}^{t} \frac{1-q(x)}{1-H(x)} H(dx) \right)H(ds)H(dt)
\end{align*}
and 
\begin{align*}
\bar{S}(q) &:= \frac{1}{2}\int_{0}^{\infty} \int_{0}^{\infty} \phi(s,t)  \exp\left(\int_{0}^{s} \frac{1-q(x)}{1-H(x)} H(dx)\right)\\
&\qquad\qquad\qquad \times \exp\left(\int_{0}^{t} \frac{1-q(x)}{1-H(x)} H(dx) \right)H(ds)H(dt)\mdot
\end{align*}
We will write $\sn{n} \equiv \sn{n}(q)$, $\wn{i}{n} \equiv \wn{i}{n}(q)$, $S\equiv S(q)$ and $\bar S\equiv \bar S(q)$ throughout this thesis.
%
\section{Assumptions}
The following assumptions will be needed throughout this thesis:
\begin{enumerate}[({A}1)]
	\item \label{ass:kernel_gen} The kernel $\phi: \R^2 \longrightarrow \R$ is measurable, non-negative and symmetric in its arguments. In effect $\phi(s,t) = \phi(t,s)$ for all $s,t \in \R_+$. 
	\item \label{ass:H_nonneg} $H$ is continuous and concentrated on the non-negative real line.
	\item \label{ass:intgral_phi_q} The following statement holds true
	$$\int_{0}^{\tau_H} \int_{0}^{\tau_H} \frac{\phi(s,t)}{m(s, \theta_0)m(t,\theta_0)(1-H(s))^\epsilon(1-H(t))^{\epsilon}} F(dt)F(ds) < \infty$$
	for some $0<\epsilon\leq 1$.
	\item $m(z,\theta)$ is non-decreasing in $z$. \label{ass:m_increas}
\end{enumerate}
Here condition (A\ref{ass:kernel_gen}) is a the standard assumption for U-Statistics (\cf\ \cite{lee1990u}). Assumptions (A\ref{ass:H_nonneg}) is the same as in \cite{dikta2000strong}. (A\ref{ass:intgral_phi_q}) is here the 2-dimensional equivalent to the condition in Theorem 1.1 of \cite{dikta2000strong}. Condition (A\ref{ass:m_increas}) poses an additional restriction on the censoring model $m$ here. We will discuss the restrictions imposed by (A\ref{ass:m_increas}) and see examples of different models for $m$, which satisfy this condition in Chapter \ref{ch:model}. Moreover, Chapter \ref{ch:simulation} shows simulation studies under different choices for $m$.\\
\\
%
%\clearpage
%
We will need the following assumptions about the Censoring Model $m$ and the Maximum Likelihood estimate $\hat\theta_n$:
\begin{enumerate}[({M}1)]
	\item \label{ass:m_consistency} $\hat{\theta}_n$ is measurable and tends to $\theta_0$
	\item \label{ass:m_nbhd} For any $\epsilon>0$ there exists a neighborhood $V(\epsilon, \theta_0)\subset \Theta$ of $\theta_0$ \st\ for all $\theta\in V(\epsilon, \theta_0)$ 
	$$\sup\limits_{x\geq 0} |m(x, \theta) - m(x, \theta_0)| < \epsilon$$
\end{enumerate}
Condition (M\ref{ass:m_consistency}) above guarantees the strong consistency of the MLE. (M\ref{ass:m_consistency}) and (M\ref{ass:m_nbhd}) are identical to (A1) and (A2) in \cite{dikta2000strong}.

	
	\chapter{Modifying the Martingale Convergence Theorem}
	Define $\Pi_n := \{\pi=(\pi_1,\dots, \pi_n) | \pi \textrm{ is permutation of }\{1,\dots,n\}\}$. Now let for $\pi\in\Pi_n$ 
$$I_n^\pi := \I{R_n(Z_l) = \pi_l, 1\leq l\leq n} = \prod_{l=1}^n \I{R_n(Z_l)=\pi_l}$$

\begin{lemma} \label{lem:expectation_sq}
	Suppose condition (A\ref{ass:intgral_phi_q}) is satisfied. Then the following statement holds true
	$$\lim\limits_{n\to\infty}\doublesum\limits_{1\leq i<j\leq n+1}\E\left[\phi^2(Z_{i:n}, Z_{j:n}) W^2_{i:n} W^2_{j:n}\right]^{\frac{1}{2}} < \infty $$	
	%
	\begin{proof} %[\textbf{Proof of lemma \ref{lem:expectation_sq}}]
		Let (A\ref{ass:intgral_phi_q}) be satisfied. Consider the following
		\begin{align*}
		& \doublesum\limits_{1\leq i<j\leq n}\E\left[\phi^2(Z_{i:n}, Z_{j:n}) W^2_{i:n} W^2_{j:n}\right]^{\frac{1}{2}}\\
		&= \doublesum\limits_{1\leq i<j\leq n}\E\left[\phi^2(Z_{i:n}, Z_{j:n}) \frac{q^2(Z_{i:n})}{(n-i+1)^2}\prod_{k=1}^{i-1}\left[1-\frac{q(Z_{k:n})}{n-k+1}\right]^2\right.\\
		&\qquad\qquad\qquad \times \left. \frac{q^2(Z_{j:n})}{(n-j+1)^2}\prod_{l=1}^{j-1}\left[1-\frac{q(Z_{l:n})}{n-l+1}\right]^2\right]^\frac{1}{2}\\
		&\leq \doublesum\limits_{1\leq i<j\leq n}\E\left[\phi^2(Z_{i:n}, Z_{j:n}) \frac{q^2(Z_{i:n})}{(n-i+1)^2}\prod_{k=1}^{i-1}\left[1-\frac{q(Z_{k:n})}{n-k+1}\right]\right.\\
		&\qquad\qquad\qquad \times \left. \frac{q^2(Z_{j:n})}{(n-j+1)^2}\prod_{l=1}^{j-1}\left[1-\frac{q(Z_{l:n})}{n-l+1}\right]\right]^\frac{1}{2}\mdot \numberthis\label{eq:E_na}
		\end{align*}
		%
		Next we will modify the products above. Recall the following definition 
		$$B_n(s) := \prod_{k=1}^{n}\left[1+\frac{1-q(Z_{k})}{n-R_{k,n}}\right]^{\I{Z_{k} < s}}$$
		and note that for $i=1,\dots,n$
		\begin{align*}
		B_n(Z_{i:n}) &= \prod_{k=1}^{n}\left[1+\frac{1-q(Z_{k})}{n-R_{k,n}}\right]^{\I{Z_{k} < Z_{i:n}}}\\
		&= \prod_{k=1}^{n}\left[1+\frac{1-q(Z_{k:n})}{n-k}\right]^{\I{Z_{k:n} < Z_{i:n}}}\\
		&=  \prod_{k=1}^{i-1}\left[1+\frac{1-q(Z_{k:n})}{n-k}\right]\mdot
		\end{align*}
		%
		Moreover consider that for $i=1,\dots,n$
		\begin{align*}
		\frac{1}{n-i+1}\prod_{k=1}^{i-1}\left[1-\frac{q(Z_{k:n})}{n-k+1}\right]
		&=  \frac{1}{n-i+1}\prod_{k=1}^{i-1}\left[\frac{n-k+1-q(Z_{k:n})}{n-k+1}\right] \\
		&=  \frac{1}{n-i+1}\prod_{k=1}^{i-1}\left[\frac{n-k+1-q(Z_{k:n})}{n-k} \cdot \frac{n-k}{n-k+1}\right]\\
		&=  \frac{1}{n}\prod_{k=1}^{i-1}\left[1+\frac{1-q(Z_{k:n})}{n-k}\right]\\
		&=  \frac{B_n(Z_{i:n})}{n}\mdot
		\end{align*}
		%
		Now combining the above with \eqref{eq:E_na} yields 
		\begin{align*}
		&\doublesum\limits_{1\leq i<j\leq n}\E\left[\phi^2(Z_{i:n}, Z_{j:n}) W^2_{i:n} W^2_{j:n}\right]^{\frac{1}{2}}\\
		&\leq \doublesum\limits_{1\leq i<j\leq n}\E\left[\phi^2(Z_{i:n}, Z_{j:n}) \frac{q^2(Z_{i:n})}{n(n-i+1)}\frac{q^2(Z_{j:n})}{n(n-j+1)} B_n(Z_{i:n})B_n(Z_{j:n})\right]^\frac{1}{2}\\
		&\leq \frac{1}{n}\sum_{i=1}^{n}\sum_{j=1}^{n}\frac{\E\left[\phi^2(Z_{i:n}, Z_{j:n}) q^2(Z_{i:n})q^2(Z_{j:n}) B_n(Z_{i:n})B_n(Z_{j:n})\right]^\frac{1}{2}}{(n-i+1)^\frac{1}{2}(n-j+1)^\frac{1}{2}} \mdot
		\end{align*}
		%
		Consider that 
		\begin{align*}
		& \E\left[\phi^2(Z_{i:n}, Z_{j:n}) q^2(Z_{i:n})q^2(Z_{j:n}) B_n(Z_{i:n})B_n(Z_{j:n})\right]^\frac{1}{2}\\
		&\leq \max\left(1, \E\left[\phi^2(Z_{i:n}, Z_{j:n}) q^2(Z_{i:n})q^2(Z_{j:n}) B_n(Z_{i:n})B_n(Z_{j:n})\right]^\frac{1}{2}\right)\\
		&\leq \max\left(1, \E\left[\phi^2(Z_{i:n}, Z_{j:n}) q^2(Z_{i:n})q^2(Z_{j:n}) B_n(Z_{i:n})B_n(Z_{j:n})\right]\right)\\
		&\leq 1 + \E\left[\phi^2(Z_{i:n}, Z_{j:n}) q^2(Z_{i:n})q^2(Z_{j:n}) B_n(Z_{i:n})B_n(Z_{j:n})\right]\\
		\end{align*}
		%
		Hence we obtain
		\begin{align*}
		&\doublesum\limits_{1\leq i<j\leq n}\E\left[\phi^2(Z_{i:n}, Z_{j:n}) W^2_{i:n} W^2_{j:n}\right]^{\frac{1}{2}}\\
		&\leq \frac{1}{n}\sum_{i=1}^{n}\sum_{j=1}^{n}\frac{1 + \E\left[\phi^2(Z_{i:n}, Z_{j:n}) q^2(Z_{i:n})q^2(Z_{j:n}) B_n(Z_{i:n})B_n(Z_{j:n})\right]}{(n-i+1)^\frac{1}{2}(n-j+1)^\frac{1}{2}}\\
		&= \frac{1}{n}\sum_{i=1}^{n}\sum_{j=1}^{n}\E\left[\frac{\phi^2(Z_{i:n}, Z_{j:n}) q^2(Z_{i:n})q^2(Z_{j:n}) B_n(Z_{i:n})B_n(Z_{j:n})}{(n-i+1)^\frac{1}{2}(n-j+1)^\frac{1}{2}}\right]\\
		&\qquad + \frac{1}{n}\sum_{i=1}^{n}\sum_{j=1}^{n}\frac{1}{(n-i+1)^\frac{1}{2}(n-j+1)^\frac{1}{2}} \mdot
		\end{align*}
		%
		Next consider that we have
		\begin{align*}
		\sum_{j=1}^{n} \frac{1}{(n-j+1)^{\frac{1}{2}}} &= \sum_{j=1}^{n} \frac{1}{j^{\frac{1}{2}}}\\
		&= 1 + \sum_{j=2}^{n} \int_{j-1}^{j} \frac{1}{\sqrt{j}} dx\\
		&\leq 1 + \sum_{j=2}^{n} \int_{j-1}^{j} \frac{1}{\sqrt{x}} dx\\
		&\leq 2\sqrt{n} \numberthis \label{eq:upper_bnd_sum_sq}
		\end{align*}
		for all $n\geq 1$.
		%
		Therefore we get
		\begin{align*}
		&\doublesum\limits_{1\leq i<j\leq n}\E\left[\phi^2(Z_{i:n}, Z_{j:n}) W^2_{i:n} W^2_{j:n}\right]^{\frac{1}{2}}\\
		&\leq \frac{1}{n}\sum_{i=1}^{n}\sum_{j=1}^{n}\E\left[\frac{\phi^2(Z_{i:n}, Z_{j:n}) q^2(Z_{i:n})q^2(Z_{j:n}) B_n(Z_{i:n})B_n(Z_{j:n})}{(n-i+1)^\frac{1}{2}(n-j+1)^\frac{1}{2}}\right] + 4\mdot
		\end{align*}
		%
		Let's now consider the double sum above. We have
		\begin{align*}
		& \frac{1}{n}\sum_{i=1}^{n}\sum_{j=1}^{n}\E\left[\frac{\phi^2(Z_{i:n}, Z_{j:n}) q^2(Z_{i:n})q^2(Z_{j:n}) B_n(Z_{i:n})B_n(Z_{j:n})}{(n-i+1)^\frac{1}{2}(n-j+1)^\frac{1}{2}}\right]\\
		&= \frac{1}{n}\E\left[\sum_{\pi \in\Pi_n} I_n^\pi \sum_{i=1}^{n}\sum_{j=1}^{n}\frac{\phi^2(Z_{i:n}, Z_{j:n}) q^2(Z_{i:n})q^2(Z_{j:n}) B_n(Z_{i:n})B_n(Z_{j:n})}{(n-i+1)^\frac{1}{2}(n-j+1)^\frac{1}{2}}\right]\\
		&= \frac{1}{n}\E\left[\sum_{\pi \in\Pi_n} I_n^\pi \sum_{i=1}^{n}\sum_{j=1}^{n}\frac{\phi^2(Z_{i}, Z_{j}) q^2(Z_{i})q^2(Z_{j}) B_n(Z_{i})B_n(Z_{j})}{(n-\pi_i+1)^\frac{1}{2}(n-\pi_j+1)^\frac{1}{2}}\right]\\
		&= \frac{1}{n}\sum_{\pi \in\Pi_n} \sum_{i=1}^{n}\sum_{j=1}^{n}\frac{\E\left[I_n^\pi\phi^2(Z_{i}, Z_{j}) q^2(Z_{i})q^2(Z_{j}) B_n(Z_{i})B_n(Z_{j})\right]}{(n-\pi_i+1)^\frac{1}{2}(n-\pi_j+1)^\frac{1}{2}}
		\end{align*}
		According to Lemma \ref{lem:zizone}, we have
		\begin{align*}
		&\E\left[I_n^\pi\phi^2(Z_{i}, Z_{j}) q^2(Z_{i})q^2(Z_{j})B_n(Z_{i})B_n(Z_{j})\right]\\
		&= \E\left[I_n^\pi\phi^2(Z_{1}, Z_{2}) q^2(Z_{1})q^2(Z_{2})B_n(Z_{1})B_n(Z_{2})\right]\mdot
		\end{align*}
		%
		Therefore we get
		\begin{align*}
		& \frac{1}{n}\sum_{i=1}^{n}\sum_{j=1}^{n}\E\left[\frac{\phi^2(Z_{i:n}, Z_{j:n}) q^2(Z_{i:n})q^2(Z_{j:n}) B_n(Z_{i:n})B_n(Z_{j:n})}{(n-i+1)^\frac{1}{2}(n-j+1)^\frac{1}{2}}\right]\\
		&= \frac{1}{n}\sum_{\pi \in\Pi_n} \sum_{i=1}^{n}\sum_{j=1}^{n}\frac{\E\left[I_n^\pi\phi^2(Z_{1}, Z_{2}) q^2(Z_{1})q^2(Z_{2}) B_n(Z_{1})B_n(Z_{2})\right]}{(n-\pi_i+1)^\frac{1}{2}(n-\pi_j+1)^\frac{1}{2}}\\
		&= \frac{1}{n}\sum_{\pi \in\Pi_n} \E\left[I_n^\pi\phi^2(Z_{1}, Z_{2}) q^2(Z_{1})q^2(Z_{2}) B_n(Z_{1})B_n(Z_{2})\right]\\
		&\qquad + \sum_{i=1}^{n}\sum_{j=1}^{n}\frac{1}{(n-\pi_i+1)^\frac{1}{2}(n-\pi_j+1)^\frac{1}{2}}\\
		&\leq 4\cdot \sum_{\pi \in\Pi_n} \E\left[I_n^\pi\phi^2(Z_{1}, Z_{2}) q^2(Z_{1})q^2(Z_{2}) B_n(Z_{1})B_n(Z_{2})\right]\\
		&= 4\cdot\E\left[\phi^2(Z_{1}, Z_{2}) q^2(Z_{1})q^2(Z_{2}) B_n(Z_{1})B_n(Z_{2})\right]\\
		\end{align*}
		%
		Since $q$ and $\phi$ are Borel-measurable, we can apply Lemma \ref{lem:representation_bn} to obtain
		\begin{align*}
		& \frac{1}{n}\sum_{i=1}^{n}\sum_{j=1}^{n}\E\left[\frac{\phi^2(Z_{i:n}, Z_{j:n}) q^2(Z_{i:n})q^2(Z_{j:n}) B_n(Z_{i:n})B_n(Z_{j:n})}{(n-i+1)^\frac{1}{2}(n-j+1)^\frac{1}{2}}\right]\\
		&\leq 8\cdot\E\left[\phi^2(Z_{1}, Z_{2}) q^2(Z_{1})q^2(Z_{2})(\Delta_{n-2}(Z_1,Z_2) + \bar{\Delta}_{n-2}(Z_1,Z_2))\right]\mdot
		\end{align*}
		%
		Note that $0\leq C_n(s)\leq 1$ for all $n\geq 1$ and $s\in \R_+$. Thus 
		$$\bar\Delta_n(s,t) = \E[C_n(s)D_n(s,t)] \leq \Delta_{n}(s,t)$$ 
		for all $n\geq 1$ and $s<t$.
		%
		Therefore we get 
		\begin{align*}
		&\frac{1}{n}\sum_{i=1}^{n}\sum_{j=1}^{n}\E\left[\frac{\phi^2(Z_{i:n}, Z_{j:n}) q^2(Z_{i:n})q^2(Z_{j:n}) B_n(Z_{i:n})B_n(Z_{j:n})}{(n-i+1)^\frac{1}{2}(n-j+1)^\frac{1}{2}}\right]\\
		&\leq 16\cdot\E\left[\phi^2(Z_{1}, Z_{2}) q^2(Z_{1})q^2(Z_{2})\Delta_{n-2}(Z_1,Z_2)\right]\mdot
		\end{align*}	
		%
		By virtue of Lemma \ref{lem:neveu}, we have
		$$\Delta_{n}(s,t) = \E[D_n(s,t)] = \E[D_n(s,t)|\F_\infty] \nearrow D(s,t)\mdot$$
		%
		But this implies in particular that $\E[D_n(s,t)] \leq D(s,t)$ for all $n\geq 1$. Hence 
		\begin{align*}
		&\frac{1}{n}\sum_{i=1}^{n}\sum_{j=1}^{n}\E\left[\frac{\phi^2(Z_{i:n}, Z_{j:n}) q^2(Z_{i:n})q^2(Z_{j:n}) B_n(Z_{i:n})B_n(Z_{j:n})}{(n-i+1)^\frac{1}{2}(n-j+1)^\frac{1}{2}}\right]\\
		&\leq 16\cdot\E\left[\phi^2(Z_{1}, Z_{2}) q^2(Z_{1})q^2(Z_{2})D(Z_1,Z_2)\right]\mdot
		\end{align*}		
		by the Monotone Convergence Theorem.
		%
		Next consider that for each $s<t$ \st\ $H(t)<1$
		\begin{align*}
		D(s,t) &= \exp\left(2\int_{0}^{s} \frac{1-q(z)}{1-H(z)} H(dz) + \int_{s}^{t} \frac{1-q(z)}{1-H(z)} H(dz)\right)\\
		&\leq \exp\left(2\int_{0}^{s} \frac{1}{1-H(z)} H(dz) + \int_{s}^{t} \frac{1}{1-H(z)} H(dz)\right)\\
		&= \exp\left(\int_{0}^{s} \frac{1}{1-H(z)} H(dz) + \int_{0}^{t} \frac{1}{1-H(z)} H(dz)\right)\\
		&= \exp\left(-\ln(1-H(s)) -\ln(1-H(t))\right)\\
		&= \frac{1}{(1-H(s))(1-H(t))}\mdot
		\end{align*}
		%
		Therefore we have
		\begin{align*}
		&\frac{1}{n}\sum_{i=1}^{n}\sum_{j=1}^{n}\E\left[\frac{\phi^2(Z_{i:n}, Z_{j:n}) q^2(Z_{i:n})q^2(Z_{j:n}) B_n(Z_{i:n})B_n(Z_{j:n})}{(n-i+1)^\frac{1}{2}(n-j+1)^\frac{1}{2}}\right]\\
		&\leq 16\cdot\E\left[\frac{\phi^2(Z_{1}, Z_{2})}{(1-H(Z_1))(1-H(Z_2))}\right]\\
		&\leq 16\cdot\left\{\int_{0}^{Z_1} \int_{0}^{Z_2}\frac{\phi^2(s, t)}{(1-H(s))(1-H(t))} H(ds)H(dt)\right\}\mdot
		\end{align*}		
		%
		Now taking into consideration the Radon-Nikodym derivatives (\cf\ \cite{dikta2000strong}, page 8)
		$$\frac{H^1(dt)}{H(dt)} = m(t,\theta_0) \textrm{ and } \frac{H^1(dt)}{F(dt)} = 1-G(t) \textrm{,}$$
		yields
		\begin{align*}
		&\frac{1}{n}\sum_{i=1}^{n}\sum_{j=1}^{n}\E\left[\frac{\phi^2(Z_{i:n}, Z_{j:n}) q^2(Z_{i:n})q^2(Z_{j:n}) B_n(Z_{i:n})B_n(Z_{j:n})}{(n-i+1)^\frac{1}{2}(n-j+1)^\frac{1}{2}}\right]\\
		&\leq 16\cdot\left\{\int_{0}^{Z_1} \int_{0}^{Z_2}\frac{\phi^2(s, t)}{m(s,\theta_0)m(t, \theta_0)(1-H(s))(1-H(t))} H^1(ds)H^1(dt)\right\}^{\frac{1}{2}}\\
		&= 16\cdot\left\{\int_{0}^{Z_1} \int_{0}^{Z_2}\frac{\phi^2(s, t)}{m(s,\theta_0)m(t, \theta_0)(1-F(s))(1-F(t))} F(ds)F(dt)\right\}^{\frac{1}{2}}
		\end{align*}	
		since $1-H(x) = (1-F(x))(1-G(x))$ for all $x\in \R_+$. But now the integral above is finite under \todo{adjust condition (A\ref{ass:intgral_phi_q})}.\\
		\\
		Therefore can finally conclude
		$$\lim\limits_{n\to\infty}\doublesum\limits_{1\leq i<j\leq n}\E\left[\phi^2(Z_{i:n}, Z_{j:n}) W^2_{i:n} W^2_{j:n}\right]^{\frac{1}{2}} < \infty\mdot$$
	\end{proof}
\end{lemma}	
%
%
%
\section{Modified Lemmas from my thesis}
The following is an adapted version of Lemma 4.6 of my thesis.
\begin{lemma}
	\label{lem:bnbn_change_order}
	Let $s\neq t$. Then the conditional expectation 
	$$\E[I_n^\pi B_n(s)B_n(t) | Z_i=s, Z_j=t]$$ 
	is independent of $i,j$ and hence
	$$\E[I_n^\pi B_n(s)B_n(t) | Z_i=s, Z_j=t] = \E[I_n^\pi B_n(s)B_n(t) | Z_1=s, Z_2=t]$$
	holds almost surely.
	%
	\begin{proof}
		First let $s<t$. For the sake of notational simplicity denote $s_k^n := \I{Z_{k:n} < s}$ and $t_k^n := \I{s\leq Z_{k:n} < t}$ and consider
		\begin{align*}
		&\E\left[I_n^\pi B_n(s)B_n(t)|Z_i=s, Z_j=t\right]\\
		&= \E\left[I_n^\pi\prod_{k=1}^{n}\left(1+\frac{1-q(Z_{k:n})}{n-k}\right)^{2s_k^n + t_k^n} | Z_i=s, Z_j=t\right]\\
		&= \E\left[\sum_{\alpha=1}^{n-1} \sum_{\beta=2}^{n} \I{Z_{\alpha:n} = Z_i} \I{Z_{\beta:n} = Z_j} I_n^\pi \left(1+\frac{1-q(Z_i)}{n-\alpha}\right)\right.\\ 
		&\qquad\qquad \times \prod_{k=1}^{\alpha-1}\left(1+\frac{1-q(Z_{k:n})}{n-k}\right)^{2s_k^n + t_k^n} \\
		&\qquad\qquad \times \prod_{k=\alpha+1}^{\beta-1}\left(1+\frac{1-q(Z_{k:n})}{n-k}\right)^{2s_k^n + t_k^n} \\
		&\qquad\qquad \times  \left.\prod_{k=\beta+1}^{n}\left(1+\frac{1-q(Z_{k:n})}{n-k}\right)^{2s_k^n + t_k^n} | Z_i=s, Z_j=t\right]
		\end{align*}
		since $s_{\alpha}^n = 0$, $t_{\alpha}^n=1$, $s_{\beta}^n = 0$ and $t_{\beta}^n=0$. Moreover we have 
		\[ \begin{cases} 
		s_{k}^n = 1 \textrm{ and } t_{k}^n = 0 & \textrm{ if } k < \alpha  \\
		s_{k}^n = 0 \textrm{ and } t_{k}^n = 1 & \textrm{ if } \alpha < k < \beta\\
		s_{k}^n = 0 \textrm{ and } t_{k}^n = 0 & \textrm{ if } \beta < k \\
		\end{cases}\mdot
		\]		
		Therefore we obtain
		\begin{align*}
		&\E\left[I_n^\pi B_n(s)B_n(t)|Z_i=s, Z_j=t\right]\\
		&= \E\left[\sum_{\alpha=1}^{n-1} \sum_{\beta=2}^{n} \I{Z_{\alpha:n} = s} \I{Z_{\beta:n} = t} I_n^\pi \left(1+\frac{1-q(Z_i)}{n-\alpha}\right)\right.\\ 
		&\qquad\qquad \times \prod_{k=1}^{\alpha-1}\left(1+\frac{1-q(Z_{k:n})}{n-k}\right)^{2s_k^n} \\
		&\qquad\qquad \times \left. \prod_{k=\alpha+1}^{\beta-1}\left(1+\frac{1-q(Z_{k:n})}{n-k}\right)^{t_k^n} | Z_i=s, Z_j=t\right]\mdot\numberthis \label{eq:indicator_transform}
		\end{align*}
		%
		Next define 
		$$\tilde{I}_n^\pi(\alpha, \beta) := \prod_{l=1}^{i-1} \I{Z_{\pi_l:n}=Z_l} \prod_{l=i + 1}^{j-1} \I{Z_{\pi_l:n}=Z_l} \prod_{l=j + 1}^{n}\I{Z_{\pi_l:n}=Z_l}$$
		and note that 
		\begin{align*}
			\sum_{\alpha=1}^{n-1} \sum_{\beta=2}^{n} \I{Z_{\alpha:n} = Z_i} \I{Z_{\beta:n} = Z_j} I_n^\pi &= \sum_{\alpha=1}^{n-1} \sum_{\beta=2}^{n} \I{Z_{\alpha:n} = Z_i} \I{Z_{\beta:n} = Z_j} \tilde{I}_n^\pi(\alpha, \beta)\mcomma
		\end{align*}
		since $\I{Z_{\alpha:n}=Z_i}\I{Z_{\pi_i:n}=Z_i} = 1$ and $\I{Z_{\beta:n}=Z_j}\I{Z_{\pi_j:n}=Z_j} = 1$ if and only if $\pi_i=\alpha$ and $\pi_j=\beta$ respectively.
		%
		Now we can rewrite \eqref{eq:indicator_transform} as
		\begin{align*}
			&\E\left[I_n^\pi B_n(s)B_n(t)|Z_i=s, Z_j=t\right]\\
			&= \E\left[\sum_{\alpha=1}^{n-1} \sum_{\beta=2}^{n} \I{Z_{\alpha:n} = Z_i} \I{Z_{\beta:n} = Z_j} \tilde{I}_n^\pi(\alpha, \beta) \left(1+\frac{1-q(Z_i)}{n-\alpha}\right)\right.\\ 
			&\qquad\qquad \times \prod_{k=1}^{\alpha-1}\left(1+\frac{1-q(Z_{k:n})}{n-k}\right)^{2s_k^n} \\
			&\qquad\qquad \times \left. \prod_{k=\alpha+1}^{\beta-1}\left(1+\frac{1-q(Z_{k:n})}{n-k}\right)^{t_k^n} | Z_i=s, Z_j=t\right]\\
			&= \E\left[\sum_{\alpha=1}^{n-1} \sum_{\beta=2}^{n} \I{Z_{\alpha:n} = s} \I{Z_{\beta:n} = t} \tilde{I}_n^\pi(\alpha, \beta) \left(1+\frac{1-q(s)}{n-\alpha}\right)\right.\\ 
			&\qquad\qquad \times \prod_{k=1}^{\alpha-1}\left(1+\frac{1-q(Z_{k:n})}{n-k}\right)^{2s_k^n} \\
			&\qquad\qquad \times \left. \prod_{k=\alpha+1}^{\beta-1}\left(1+\frac{1-q(Z_{k:n})}{n-k}\right)^{t_k^n} | Z_i=s, Z_j=t\right]\mdot
		\end{align*}		
		%
		Next we need to introduce some more notation. For $1\leq i,j\leq n$ and $n\geq 2$, let $\{Z_{k:n-2}\}_{k\leq n-2}$ denote the ordered $Z$-values among $Z_1,\dots, Z_n$ with $Z_i$ and $Z_j$ removed from the sample. Note that
		\[ Z_{k:n} = \begin{cases} 
				Z_{k:n-2} & k < \alpha  \\
				Z_{k-1:n-2} & \alpha < k < \beta\\
				Z_{k-2:n-2} & k > \beta
			\end{cases}\numberthis\label{eq:zkn_zknminustwo}\mdot
		\]	
		Moreover let $Z_{0:n-2} := 0$ and $Z_{n-1:n-2} = \infty$. Furthermore consider $\pi\in\Pi_n$ as a mapping
		$$\pi: (1,\dots, n) \longrightarrow  (\pi_1,\dots, \pi_n)$$		
		Note that $\pi$ is a permutation of $\{1,\dots, n\}$ and hence bijective. We will denote its inverse as $\pi^{-1}$. Here $\pi^{-1}(i)=k$ whenever $\pi_k = i$. Now we can rewrite $\tilde{I}_n^\pi(\alpha,\beta)$ as follows
		\begin{align*}
			\tilde{I}_n^\pi(\alpha, \beta) &= \prod_{l=1}^{i-1} \I{Z_{\pi_l:n}=Z_l} \prod_{l=i + 1}^{j-1} \I{Z_{\pi_l:n}=Z_l} \prod_{l=j + 1}^{n}\I{Z_{\pi_l:n}=Z_l}\\
			&= \prod_{l=1}^{\alpha-1} \I{Z_{l:n}=Z_{\pi_l^{-1}}} \prod_{l=\alpha + 1}^{\beta-1} \I{Z_{l:n}=Z_{\pi_l^{-1}}} \prod_{l=\beta + 1}^{n} \I{Z_{l:n}=Z_{\pi_l^{-1}}}\\
			&= \prod_{l=1}^{\alpha-1} \I{Z_{l:n-2}=Z_{\pi_l^{-1}}} \prod_{l=\alpha + 1}^{\beta-1} \I{Z_{l-1:n-2}=Z_{\pi_l^{-1}}} \prod_{l=\beta + 1}^{n} \I{Z_{l-2:n-2}=Z_{\pi_l^{-1}}}
		\end{align*}
		by equation \eqref{eq:zkn_zknminustwo}. Now $\tilde{I}_n^\pi(\alpha, \beta)$ is independent of $Z_i$ and $Z_j$, since $\pi_{\alpha}^{-1} = i$ and $\pi_\beta^{-1}=j$.
		%
		Moreover we have (\cf\ \cite{bose1999strong}, page 193)
		\begin{align*}
			&\E\left[I_n^\pi B_n(s)B_n(t)|Z_i=s, Z_j=t\right]\\
			&= \E\left[\sum_{\alpha=1}^{n-1} \sum_{\beta=2}^{n} \I{Z_{\alpha-1:n-2} < s \leq Z_{\alpha:n-2}} \I{Z_{\beta-2:n-2} < t \leq Z_{\beta-1:n-2}} \tilde{I}_n^\pi(\alpha, \beta)\right.\\ 
			&\qquad\qquad \times \left(1+\frac{1-q(s)}{n-\alpha}\right)\prod_{k=1}^{\alpha-1}\left(1+\frac{1-q(Z_{k:n-2})}{n-k}\right)^{2s_k^{n-2}} \\
			&\qquad\qquad \times \left. \prod_{k=\alpha+1}^{\beta-1}\left(1+\frac{1-q(Z_{k-1:n-2})}{n-k}\right)^{t_{k-1}^{n-2}} | Z_i=s, Z_j=t\right]\\
			&= \E\left[\sum_{\alpha=1}^{n-1} \sum_{\beta=2}^{n} \I{Z_{\alpha-1:n-2} < s \leq Z_{\alpha:n-2}} \I{Z_{\beta-2:n-2} < t \leq Z_{\beta-1:n-2}}\tilde{I}_n^\pi(\alpha, \beta)\right.\\ 
			&\qquad\qquad \times \left(1+\frac{1-q(s)}{n-\alpha}\right)\prod_{k=1}^{\alpha-1}\left(1+\frac{1-q(Z_{k:n-2})}{n-k}\right)^{2s_k^{n-2}} \\
			&\qquad\qquad \times \left. \prod_{k=\alpha}^{\beta-2}\left(1+\frac{1-q(Z_{k:n-2})}{n-k-1}\right)^{t_{k}^{n-2}}\right]\\
			&= \E\left[\sum_{\alpha=1}^{n-1} \I{Z_{\alpha-1:n-2} < s \leq Z_{\alpha:n-2}} \tilde{I}_n^\pi(\alpha, \beta) \left(1+\frac{1-q(s)}{n-\alpha}\right) \right.\\ 
			&\qquad\qquad \times \prod_{k=1}^{n-2}\left(1+\frac{1-q(Z_{k:n-2})}{n-k}\right)^{2s_k^{n-2}} \\
			&\qquad\qquad \times \left. \prod_{k=1}^{n-2}\left(1+\frac{1-q(Z_{k:n-2})}{n-k-1}\right)^{t_{k}^{n-2}}\right]
		\end{align*}		
		which is independent of $i,j$. 
		%
		Next consider the case $t<s$. Define $\tilde{t}_k^n := \I{Z_{k:n} < t}$ and $\tilde{s}_k^n := \I{t\leq Z_{k:n} < s}$. Using similar arguments we can show that in this case 
		\begin{align*}
		&\E\left[I_n^\pi B_n(s)B_n(t)|Z_i=s, Z_j=t\right]\\
		&= \E\left[\sum_{\alpha=1}^{n-1} \I{Z_{\alpha-1:n-2} < t \leq Z_{\alpha:n-2}} \tilde{I}_n^\pi(\alpha, \beta) \left(1+\frac{1-q(t)}{n-\alpha}\right) \right.\\ 
		&\qquad\qquad \times \prod_{k=1}^{n-2}\left(1+\frac{1-q(Z_{k:n-2})}{n-k}\right)^{2\tilde{t}_k^{n-2}} \\
		&\qquad\qquad \times \left. \prod_{k=\alpha}^{n-2}\left(1+\frac{1-q(Z_{k:n-2})}{n-k-1}\right)^{\tilde{s}_{k}^{n-2}}\right]
		\end{align*}	
		which is independent of $i,j$ as well. Thus we have on $\{s\neq t\}$ that $\E\left[B_n(s)B_n(t)|Z_i=s, Z_j=t\right]$ is independent of $i,j$ and hence
		$$\E\left[I_n^\pi B_n(s)B_n(t)|Z_i=s, Z_j=t\right] = \E\left[I_n^\pi B_n(s)B_n(t)|Z_1=s, Z_2=t\right]\mdot$$
	\end{proof}
\end{lemma}
%
The following is an adapted version of Lemma 4.7 of my thesis.
\begin{lemma} \label{lem:zizone}
	Let $\tilde{\phi}: \R^2_+ \longrightarrow \R_+$ be a Borel-measurable function. Then we have for any $n\geq 2$ 
	\begin{align*}
	&\E[I_n^\pi \tilde{\phi}(Z_{i},Z_{j}) B_n(Z_{i}) B_n(Z_{j})]\\
	& = \E[I_n^\pi \tilde{\phi}(Z_1,Z_2) B_n(Z_1) B_n(Z_2)]\mdot
	\end{align*}
	%
	\begin{proof}
		Consider that $\{Z_i=Z_j\}$ is a measure zero set, since $H$ is continuous. Therefore the following holds for $1\leq i,j \leq n$ 
		\begin{align*}
			& \E[I_n^\pi \tilde{\phi}(Z_{i},Z_{j}) B_n(Z_{i}) B_n(Z_{j})]\\
			&= \E\left[\I{Z_i\neq Z_j}I_n^\pi\tilde{\phi}(Z_{i},Z_{j}) B_n(Z_{i}) B_n(Z_{j})\right]\\
			&= \E\left[\I{Z_i\neq Z_j}\tilde{\phi}(Z_{i},Z_{j}) \E\left[I_n^\pi B_n(Z_{i}) B_n(Z_{j})| Z_i,Z_j\right]\right]\\
			&= \int_{0}^{\infty}\int_{0}^{\infty}\I{s\neq t}\tilde{\phi}(s,t) \E\left[I_n^\pi B_n(s) B_n(t)| Z_i=s,Z_j=t\right]H(ds)H(dt)
		\end{align*}
		%
		Applying Lemma \ref{lem:bnbn_change_order} we obtain for $1\leq i,j\leq n$
		\begin{align*}
			&\I{s\neq t} \E[I_n^\pi B_n(s)B_n(t)|Z_i=s, Z_j=t]\\
			&= \I{s\neq t} \E[I_n^\pi B_n(s)B_n(t)|Z_1=s, Z_2=t]
		\end{align*}
		%
		Therefore we conclude
		\begin{align*}
			\E\left[I_n^\pi \tilde{\phi}(Z_{i},Z_{j}) B_n(Z_{i}) B_n(Z_{j})\right] &= \E\left[\tilde{\phi}(Z_{i},Z_{j}) \E\left[I_n^\pi B_n(Z_{i}) B_n(Z_{j})| Z_i,Z_j\right]\right]\\
			&= \E\left[I_n^\pi \tilde{\phi}(Z_{1},Z_{2}) B_n(Z_{1}) B_n(Z_{2})\right]\mdot
		\end{align*}
	\end{proof}
\end{lemma}

\section{Lemmas from my thesis (unchanged)}
\begin{lemma} \label{lem:representation_bn}
	Let $\tilde{\phi}: \R^2_+ \longrightarrow \R_+$ be a Borel-measurable function. Then we have for any $s<t$ and $n\geq 2$ 
	\begin{align*}
	&\E[\tilde{\phi}(Z_1,Z_2) B_n(Z_1) B_n(Z_2)]\\
	& = \E[2\tilde{\phi}(Z_1,Z_2) \{\Delta_{n-2}(Z_1,Z_2) + \bar{\Delta}_{n-2}(Z_1,Z_2)\}\I{Z_1<Z_2}]\mdot
	\end{align*}
	%
	\begin{proof}
		Consider the following
		\begin{align*}
		B_n(Z_1)B_n(Z_2) &= \prod_{k=1}^{n}\left[1+\frac{1-q(Z_{k})}{n-R_{k,n}}\right]^{\I{Z_{k} < Z_1}+\I{Z_{k} < Z_2}}\\
		&= \left[1+\frac{1-q(Z_{1})}{n-R_{1,n}}\right]^{\I{Z_{1} < Z_2}} \left[1+\frac{1-q(Z_{2})}{n-R_{2,n}}\right]^{\I{Z_{2} < Z_1}}\\
		&\qquad \times \prod_{k=3}^{n}\left[1+\frac{1-q(Z_{k})}{n-R_{k,n}}\right]^{\I{Z_{k} < Z_1}+\I{Z_{k} < Z_2}}\\
		&= \I{Z_1<Z_2}\left[1+\frac{1-q(Z_{1})}{n-R_{1,n}}\right] \\
		&\qquad\qquad \times \prod_{k=1}^{n-2}\left[1+\frac{1-q(Z_{k+2})}{n-R_{k+2,n}}\right]^{\I{Z_{k+2} < Z_1}+\I{Z_{k+2} < Z_2}}\\
		&\quad + \I{Z_1>Z_2}\left[1+\frac{1-q(Z_{2})}{n-R_{2,n}}\right] \\
		&\qquad\qquad \times \prod_{k=1}^{n-2}\left[1+\frac{1-q(Z_{k+2})}{n-R_{k+2,n}}\right]^{\I{Z_{k+2} < Z_1}+\I{Z_{k+2} < Z_2}}\\
		&\quad + \I{Z_1=Z_2}\prod_{k=1}^{n-2}\left[1+\frac{1-q(Z_{k+2})}{n-R_{k+2,n}}\right]^{2\I{Z_{k+2} < Z_1}}\mdot \numberthis\label{eq:bnbn1}
		\end{align*}
		%
		On $\{Z_1<Z_2\}$ we have 
		\begin{align*}
		\prod_{k=1}^{n-2}\left[1+\frac{1-q(Z_{k+2})}{n-R_{k+2,n}}\right]^{\I{Z_{k+2} < Z_2}} &= \prod_{k=1}^{n-2}\left[1+\frac{1-q(Z_{k+2})}{n-\tilde{R}_{k,n-2}}\right]^{\I{Z_{k+2} < Z_1}}\\
		&\quad \times  \prod_{k=1}^{n-2}\left[1+\frac{1-q(Z_{k+2})}{n-\tilde{R}_{k,n-2}-1}\right]^{\I{Z_1 < Z_{k+2} < Z_2}}
		\end{align*}
		where $\tilde{R}_{k,n-2}$ denotes the rank of the $Z_k$, $k=3,\dots, n$ among themselves. The above holds since 
		\[ R_{k+2,n} = \begin{cases} 
		\tilde{R}_{k,n-2} & \textrm{ if } Z_{k+2} < Z_1 \\
		\tilde{R}_{k, n-2} + 1 & \textrm{ if } Z_1 < Z_{k+2} < Z_2 
		\end{cases}
		\]
		for $k=1,\dots,n-2$. 
		% 
		Therefore \eqref{eq:bnbn1} yields
		\begin{align*}
		B_n(Z_1)B_n(Z_2) &= \I{Z_1<Z_2}\left[1+\frac{1-q(Z_{1})}{n-R_{1,n}}\right] \\
		&\qquad\qquad \times \prod_{k=1}^{n-2}\left[1+\frac{1-q(Z_{k+2})}{n-\tilde{R}_{k,n-2}}\right]^{2\I{Z_{k+2} < Z_1}}\\
		&\qquad\qquad \times \prod_{k=1}^{n-2}\left[1+\frac{1-q(Z_{k+2})}{n-\tilde{R}_{k,n-2}-1}\right]^{\I{Z_1 < Z_{k+2} < Z_2}}\\
		&\quad + \I{Z_2<Z_1}\left[1+\frac{1-q(Z_{2})}{n-R_{2,n}}\right] \\
		&\qquad\qquad \times \prod_{k=1}^{n-2}\left[1+\frac{1-q(Z_{k+2})}{n-\tilde{R}_{k,n-2}}\right]^{2\I{Z_{k+2} < Z_2}}\\
		&\qquad\qquad \times \prod_{k=1}^{n-2}\left[1+\frac{1-q(Z_{k+2})}{n-\tilde{R}_{k,n-2}-1}\right]^{\I{Z_2 < Z_{k+2} < Z_1}}\\
		&\quad + \I{Z_1=Z_2}\prod_{k=1}^{n-2}\left[1+\frac{1-q(Z_{k+2})}{n-\tilde{R}_{k,n-2}}\right]^{2\I{Z_{k+2} < Z_1}}\mdot \numberthis\label{eq:bnbn_rank}
		\end{align*}
		%
		Now let's denote $Z_{k:n-2}$ the ordered $Z$-values among $Z_3,\dots, Z_n$ for $k=1,\dots,n-2$. Consider that we can write 
		\begin{align*}
		\left[1+\frac{1-q(Z_{1})}{n-R_{1,n}}\right] = \sum_{i=1}^{n-1}\left[1+\frac{1-q(s)}{n-i}\right]\I{Z_{i-1:n-2} < Z_1 \leq Z_{i:n-2}}\mdot
		\end{align*}
		%
		Note that $Z_{k:n-2}$ is independent of $Z_1$ and $Z_2$ for $k=1,\dots,n-2$. Therefore we obtain the following, by conditioning \eqref{eq:bnbn_rank} on $Z_1,Z_2$:
		\begin{align*}
		&\E[B_n(Z_1)B_n(Z_2)|Z_1 = s, Z_2 = t]\\
		&= \I{s<t}\E\left[\left(\sum_{i=1}^{n-1}\left[1+\frac{1-q(s)}{n-i}\right]\I{Z_{i-1:n-2} < s \leq Z_{i:n-2}}\right)\right.\\
		&\qquad\qquad\qquad \times \prod_{k=1}^{n-2}\left[1+\frac{1-q(Z_{k:n-2})}{n-k}\right]^{2\I{Z_{k:n-2} < s}}\\
		&\qquad\qquad\qquad \times \left. \prod_{k=1}^{n-2}\left[1+\frac{1-q(Z_{k:n-2})}{n-k-1}\right]^{\I{s < Z_{k:n-2} < t}} \right]\\
		&\quad + \I{t<s}\E\left[\left(\sum_{i=1}^{n-1}\left[1+\frac{1-q(t)}{n-i}\right]\I{Z_{i-1:n-2} < t \leq Z_{i:n-2}}\right)\right. \\
		&\qquad\qquad\qquad \times \prod_{k=1}^{n-2}\left[1+\frac{1-q(Z_{k:n-2})}{n-k}\right]^{2\I{Z_{k:n-2} < t}}\\
		&\qquad\qquad\qquad \times \left. \prod_{k=1}^{n-2}\left[1+\frac{1-q(Z_{k:n-2})}{n-k-1}\right]^{\I{t < Z_{k:n-2} < s}}\right]\\
		&\quad + \I{s=t}\E\left[\prod_{k=1}^{n-2}\left[1+\frac{1-q(Z_{k:n-2})}{n-k}\right]^{2\I{Z_{k:n-2} < s}} \right]\\
		&= \alpha(s,t) + \alpha(t,s) + \beta(s,t)
		\end{align*}
		where 
		\begin{align*}
		\alpha(s,t) &:=\I{s<t}\E\left[\left(\sum_{i=1}^{n-1}\left[1+\frac{1-q(s)}{n-i}\right]\I{Z_{i-1:n-2} < s \leq Z_{i:n-2}}\right)\right.\\
		&\qquad\qquad\qquad \times \prod_{k=1}^{n-2}\left[1+\frac{1-q(Z_{k:n-2})}{n-k}\right]^{2\I{Z_{k:n-2} < s}}\\
		&\qquad\qquad\qquad \times \left. \prod_{k=1}^{n-2}\left[1+\frac{1-q(Z_{k:n-2})}{n-k-1}\right]^{\I{s < Z_{k:n-2} < t}} \right]
		\end{align*}
		and 
		\begin{align*}
		\beta(s,t) &:=\I{s=t}\E\left[\prod_{k=1}^{n-2}\left[1+\frac{1-q(Z_{k:n-2})}{n-k}\right]^{2\I{Z_{k:n-2} < s}} \right]\mdot
		\end{align*}		
		%
		Consider that we have
		$$\E[\alpha(Z_1,Z_2)] = \E[\alpha(Z_2,Z_1)]$$
		under (A\ref{ass:kernel_gen}), because $Z_1$ and $Z_2$ are \iid\ and 
		$$\E[\beta(Z_1,Z_2)] = 0$$
		since $H$ is continuous. Therefore we get 
		\begin{align*}
		&\E[\tilde{\phi}(Z_1,Z_2)B_n(Z_1)B_n(Z_2)]\\
		&= \E[\tilde{\phi}(Z_1,Z_2)(\alpha(Z_1,Z_2) + \alpha(Z_2,Z_1) + \beta(Z_1,Z_2))]\\
		&= \E[2\tilde{\phi}(Z_1,Z_2)\alpha(Z_1,Z_2)]\mdot \numberthis\label{eq:expectationalpha}
		\end{align*}
		%
		Next consider that 
		\begin{align*}
		\alpha(s,t) &=\I{s<t}\E\left[(1+C_n(s)) D_{n-2}(s,t) \right]\\
		&= \I{s<t}(\Delta_{n-2}(s,t) + \bar{\Delta}_{n-2}(s,t))\mdot
		\end{align*}
		The latter equality holds, since
		\begin{align*}
		&\sum_{i=1}^{n-1}\left[1+\frac{1-q(s)}{n-i}\right]\I{Z_{i-1:n-2} < s \leq Z_{i:n-2}}\\
		&=\sum_{i=1}^{n-1}\I{Z_{i-1:n-2} < s \leq Z_{i:n-2}} + \sum_{i=1}^{n-1}\left[\frac{1-q(s)}{n-i}\right]\I{Z_{i-1:n-2} < s \leq Z_{i:n-2}}\\
		&= 1 + C_n(s)\mdot
		\end{align*}
		Now the statement of the lemma follows directly from \eqref{eq:expectationalpha}.
	\end{proof}
\end{lemma}
%
The next lemma identifies the almost sure limit of $D_n$ for $n\to\infty$. Define for $s<t$
$$D(s,t) := \exp\left(2\int_{0}^{s} \frac{1-q(z)}{1-H(z)} H(dz) + \int_{s}^{t} \frac{1-q(z)}{1-H(z)} H(dz)\right)$$
\begin{lemma} \label{lem:dn_limit}
	For any $s < t$ \st\ $H(t)<1$, we have
	$$\lim\limits_{n\to\infty}D_n(s,t) = D(s,t)\mdot$$
	%	
	\begin{proof}
		First recall the following definition
		\begin{align*}
		D_n(s,t) &:= \prod_{k=1}^{n} \left[1+\frac{1-q(Z_k)}{n-R_{k,n} +2}\right]^{2\I{Z_k<s}} \prod_{k=1}^{n}\left[1+\frac{1-q(Z_k)}{n-R_ {k,n}+1}\right]^{\I{s < Z_k < t}} \mdot
		\end{align*}
		%
		Next let 
		\begin{align*}
		x_k &:= \frac{1-q(Z_k)}{n(1- H_n(Z_k) + 2/n)}\\
		y_k &:= \frac{1-q(Z_k)}{n(1- H_n(Z_k) + 1/n)}\\
		s_k &:= \I{Z_k < s} \\
		t_k &:= \I{s < Z_k < t}
		\end{align*}
		for $s<t$ and $k=1,\dots,n$.
		%
		Now consider 
		\begin{align*}
		D_n(s,t) &= \prod_{k=1}^{n} \left[1+\frac{1-q(Z_k)}{n(1- H_n(Z_k) + 2/n)}\I{Z_k<s}\right]^{2}\\ 
		&\qquad \times \prod_{k=1}^{n}\left[1+\frac{1-q(Z_k)}{n(1-H_n(Z_k)+1/n)}\I{s < Z_k < t}\right]\\
		&= \prod_{k=1}^{n} \left[1+x_k s_k\right]^{2} \prod_{k=1}^{n}\left[1+y_k t_k\right]\\
		&= \exp\left(2\sum_{k=1}^{n}\ln\left[1+x_k s_k\right] + \sum_{k=1}^{n}\ln\left[1+y_k t_k\right]\right)\mdot
		\end{align*}
		%
		Note that $0 \leq x_k s_k \leq 1$ and $0 \leq y_k t_k \leq 1$. Consider that the following inequality holds  
		$$-\frac{x^2}{2} \leq \ln(1+x) - x \leq 0$$ 
		for any $x \geq 0$ (cf.  \cite{stute1993strong}, p. 1603). This implies 
		$$-\frac{1}{2}\sum_{k=1}^{n}x_k^2 s_k \leq \sum_{k=1}^{n}\ln(1+x_k s_k) - \sum_{k=1}^{n}x_k s_k \leq 0\mdot$$ 
		But now 
		\begin{align*}
		\sum_{k=1}^{n} x_k^2 s_k &= \frac{1}{n^2} \sum_{k=1}^{n} \left(\frac{1-q(Z_k)}{1-H_n(Z_k)+\frac{2}{n}}\right)^2\I{Z_k<s}\\
		&\leq \frac{1}{n^2} \sum_{k=1}^{n} \left(\frac{1}{1-H_n(s)+\frac{1}{n}}\right)^2\\
		&= \frac{1}{n(1-H_n(s)+n^{-1})^2} \longrightarrow 0
		\end{align*}
		almost surely as $n\to\infty$, since $H(s)<H(t)<1$ (\cf\ \cite{stute1993strong}, p. 1603). Therefore we have
		$$\abs{\sum_{k=1}^{n}\ln(1+x_k s_k) - \sum_{k=1}^{n}x_k s_k} \longrightarrow 0$$
		with probability 1 as $n\to\infty$. 
		%
		Similarly we obtain
		$$\abs{\sum_{k=1}^{n}\ln(1+y_k t_k) - \sum_{k=1}^{n}y_k t_k} \longrightarrow 0$$
		with probability 1 as $n\to\infty$. Hence 
		$$\lim\limits_{n\to\infty} D_n(s) = \lim\limits_{n\to\infty} \exp\left(2\sum_{k=1}^{n} x_k s_k + \sum_{k=1}^{n}y_k t_k\right)\mdot$$
		%
		Now consider 
		\begin{align*}
		\sum_{k=1}^{n} x_k s_k &= \frac{1}{n}\sum_{k=1}^{n} \frac{1-q(Z_k)}{1-H_n(Z_k)+\frac{2}{n}}\I{Z_k<s}\\
		&= \int_{0}^{s-} \frac{1-q(z)}{1-H_n(z)+\frac{2}{n}} H_n(dz)\\
		&= \int_{0}^{s-} \frac{1-q(z)}{1-H(z)} H_n(dz) + \int_{0}^{s-} \frac{1-q(z)}{1-H_n(z)+\frac{2}{n}} - \frac{1-q(z)}{1-H(z)} H_n(dz)\\
		&= \int_{0}^{s-} \frac{1-q(z)}{1-H(z)} H_n(dz) + \int_{0}^{s-} \frac{(1-q(z))(H_n(z)-H(z)-\frac{2}{n})}{(1-H_n(z)+\frac{2}{n})(1-H(z))} H_n(dz)\mdot \numberthis \label{eq:xksk_int}
		\end{align*}
		%
		Note that the second term on the right hand side of the latter equation above tends to zero for  $n\to\infty$, because
		\begin{align*}
		& \int_{0}^{s-} \frac{(1-q(z))(H_n(z)-H(z)-\frac{2}{n})}{(1-H_n(z)+\frac{2}{n})(1-H(z))} H_n(dz)\\
		&\leq \frac{\sup_{z}|H_n(z)- H(z) -\frac{2}{n}|}{1-H(s)} \int_{0}^{s-}\frac{1}{1-H_n(z)} H_n(dz) \longrightarrow 0
		\end{align*}
		%
		almost surely as $n\to\infty$, by the Glivenko-Cantelli Theorem and since $H(s)<1$. Moreover we have
		\begin{align*}
		\int_{0}^{s-} \frac{1-q(z)}{1-H(z)} H_n(dz) \longrightarrow \int_{0}^{s} \frac{1-q(z)}{1-H(z)} H(dz)
		\end{align*}		
		by the SLLN. Therefore we obtain 
		$$\lim\limits_{n\to\infty} \sum_{k=1}^{n} x_k s_k = \int_{0}^{s} \frac{1-q(z)}{1-H(z)} H(dz)\mdot$$
		By the same arguments, we can show that 
		$$\lim\limits_{n\to\infty} \sum_{k=1}^{n} y_k t_k = \int_{s}^{t} \frac{1-q(z)}{1-H(z)} H(dz)\mdot$$
		Thus we finally conclude
		$$\lim\limits_{n\to\infty} D_n(s,t) = \exp\left(2\int_{0}^{s} \frac{1-q(z)}{1-H(z)} H(dz) + \int_{s}^{t} \frac{1-q(z)}{1-H(z)} H(dz)\right)$$
		almost surely.
	\end{proof}
\end{lemma}
%
\begin{lemma} \label{lem:dn_supermart}
	$\{D_n, \F_n\}_{n\geq 1}$ is a non-negative reverse supermartingale.
	%
	\begin{proof}
		Consider that for $s<t$ and $n\geq 1$
		\begin{align*}
		\E[D_n(s,t)| \F_{n+1}] &= \E\left[\prod_{k=1}^{n}\left(1+\frac{1-q(Z_{k:n})}{n-k+2}\right)^{2\I{Z_{k:n} <s}}\right.\\
		&\qquad \left. \times \prod_{k=1}^{n}\left(1+\frac{1-q(Z_{k:n})}{n-k+1}\right)^\I{s < Z_{k:n} < t} | \F_{n+1}\right]\\
		&= \sum_{i=1}^{n+1}\E\left[\I{Z_{n+1} = Z_{i:n+1}} \prod_{k=1}^{n}\dots | \F_{n+1}\right]\\
		&= \sum_{i=1}^{n+1}\E\left[\I{Z_{n+1} = Z_{i:n+1}} \prod_{k=1}^{i-1}\left(1+\frac{1-q(Z_{k:n+1})}{n-k+2}\right)^{2\I{Z_{k:n+1} <s}} \right.\\
		&\qquad\qquad \times \prod_{k=i}^{n}\left(1+\frac{1-q(Z_{k+1:n+1})}{n-k+2}\right)^{2\I{Z_{k+1:n+1} <s}}\\
		&\qquad\qquad \times \prod_{k=1}^{i-1}\left(1+\frac{1-q(Z_{k:n+1})}{n-k+1}\right)^\I{s < Z_{k:n+1} < t}\\
		&\qquad\qquad \left. \times \prod_{k=i}^{n}\left(1+\frac{1-q(Z_{k+1:n+1})}{n-k+1}\right)^\I{s < Z_{k+1:n+1} < t}| \F_{n+1}\right]\\
		&= \sum_{i=1}^{n+1}\E\left[\I{Z_{n+1} = Z_{i:n+1}} \prod_{k=1}^{i-1}\left(1+\frac{1-q(Z_{k:n+1})}{n-k+2}\right)^{2\I{Z_{k:n+1} <s}} \right.\\
		&\qquad\qquad \times \prod_{k=i+1}^{n+1}\left(1+\frac{1-q(Z_{k:n+1})}{n-k+3}\right)^{2\I{Z_{k:n+1} <s}}\\
		&\qquad\qquad \times \prod_{k=1}^{i-1}\left(1+\frac{1-q(Z_{k:n+1})}{n-k+1}\right)^\I{s < Z_{k:n+1} < t}\\
		&\qquad\qquad \left. \times \prod_{k=i+1}^{n+1}\left(1+\frac{1-q(Z_{k:n+1})}{n-k+2}\right)^\I{s < Z_{k:n+1} < t}| \F_{n+1}\right]\mdot
		\end{align*}
		%
		Now each product within the conditional expectation is measurable \wrt\ $\F_{n+1}$. Moreover we have for $i=1,\dots,n$ 
		\begin{align*}
		\E[\I{Z_{n+1}=Z_{i:n+1}}|\F_n+1] &= \P(Z_{n+1}=Z_{i:n+1}|\F_{n+1})\\
		&= \P(R_{n+1,n+1} = i)\\
		&= \frac{1}{n+1}\mdot
		\end{align*}
		%
		Thus we obtain
		\begin{align*}
		\E[D_n(s,t)| \F_{n+1}] &= \frac{1}{n+1} \sum_{i=1}^{n+1} \prod_{k=1}^{i-1}\left(1+\frac{1-q(Z_{k:n+1})}{n-k+2}\right)^{2\I{Z_{k:n+1} <s}}\\
		&\qquad\qquad\qquad \times \left(1+\frac{1-q(Z_{k:n+1})}{n-k+1}\right)^\I{s < Z_{k:n+1} < t}\\
		&\qquad\qquad \times \prod_{k=i+1}^{n+1}\left(1+\frac{1-q(Z_{k:n+1})}{n-k+3}\right)^{2\I{Z_{k:n+1} <s}}\\ &\qquad\qquad\qquad \times \left(1+\frac{1-q(Z_{k:n+1})}{n-k+2}\right)^\I{s < Z_{k:n+1} < t}\mdot \numberthis \label{eq:cond_exp_dnp1}
		\end{align*}
		%
		We will now proceed by induction on $n$. First let 
		$$x_k := 1-q(Z_{k:2}) \textrm{, } s_k := \I{Z_{k:2} < s} \textrm{ and } t_k := \I{s < Z_{k:2} < t}$$
		for $k=1,2$. Note that that $x_k$ and $y_k$ are different, compared to the corresponding definitions in lemma \ref{lem:dn_limit}, as they involves the ordered $Z$-values here. 
		%
		Next consider
		\begin{align*}
		\E[D_1(s,t) | \F_2] &= \frac{1}{2}\left[\left(1+\frac{1-q(Z_{2:2})}{2}\right)^{2\I{Z_{2:2}<s}} \times \left(1+(1-q(Z_{2:2}))\right)^{\I{s < Z_{2:2} < t}} \right.\\
		&\qquad \left. + \left(1+\frac{1-q(Z_{1:2})}{2}\right)^{2\I{Z_{1:2}<s}} \times \left(1+(1-q(Z_{1:2}))\right)^{\I{s < Z_{1:2} < t}}\right]\\
		&= \frac{1}{2}\left[\left(1+\frac{x_2}{2}s_2\right)^{2} \times \left(1+x_2t_2\right) + \left(1+\frac{x_1}{2}s_1\right)^{2} \times \left(1+x_1t_1\right)\right]\mdot
		\end{align*}
		%
		Moreover we have
		\begin{align*}
		D_2(s,t) &= \prod_{k=1}^{2} \left[1+\frac{1-q(Z_{k:2})}{4-k}\right]^{2\I{Z_{k:2}<s}} \prod_{k=1}^{2}\left[1+\frac{1-q(Z_{k:2})}{3-k}\right]^{\I{s < Z_{k:2} < t}}\\
		&= \left[1 + \frac{x_1}{3}s_1\right]^2 \times \left[1+\frac{x_1}{2}t_1\right] \times \left[1+\frac{x_2}{2}s_2\right]^2 \times \left[1+x_2t_2\right]\\
		&= \left[1 + \frac{x_1}{2}t_1 + \left(\frac{x_1^2}{9} + \frac{2}{3}x_1\right)s_1\right] \times \left[1 + x_2t_2 + \left(\frac{x_2^2}{4} + x_2\right)s_2\right]\mdot
		\end{align*}		
		%
		Therefore we obtain 
		\begin{align*}
		\E[D_1(s,t) | \F_2] - D_2(s,t) \leq \frac{x_1^2}{72} - \frac{x_1}{6} \leq 0\mdot 
		\end{align*}
		since $0 \leq x_1 \leq 1$. Thus $\E[D_1(s,t) | \F_2] \leq D_2(s,t)$ for any $s<t$, as needed. Now assume that 
		$$\E[D_n(s,t) | \F_{n+1}] \leq D_{n+1}(s,t)$$
		holds for any $n\geq 1$. 
		%
		Note that the latter is equivalent to assuming
		\begin{align*}
		& \frac{1}{n+1} \sum_{i=1}^{n+1} \prod_{k=1}^{i-1}\left(1+\frac{1-q(y_k)}{n-k+2}\right)^{2\I{y_k <s}}  \left(1+\frac{1-q(y_k)}{n-k+1}\right)^\I{s < y_k < t}\\
		&\qquad\qquad \times \prod_{k=i+1}^{n+1}\left(1+\frac{1-q(y_k)}{n-k+3}\right)^{2\I{y_k <s}} \left(1+\frac{1-q(y_k)}{n-k+2}\right)^\I{s < y_k < t}\\
		&\leq \prod_{k=1}^{n+1}\left(1+\frac{1-q(y_k)}{n-k+3}\right)^{2\I{y_k <s}} \prod_{k=1}^{n+1}\left(1+\frac{1-q(y_k)}{n-k+2}\right)^\I{s < y_k < t} \numberthis \label{eq:supermart_yk}
		\end{align*}
		holds for arbitrary $y_k \geq 0$. Next define for $s<t$ and $n\geq 1$
		$$A_{n+2}(s,t) := \prod_{k=2}^{n+2} \left[1+\frac{1-q(Z_{k:n+2})}{n-k+4}\right]^{2\I{Z_{k:n+2} < s}} \times \left[1+\frac{1-q(Z_{k:n+2})}{n-k+3}\right]^\I{s < Z_{k:n+2} < t}\mdot $$
		%
		Now consider that we get from \eqref{eq:cond_exp_dnp1}
		\begin{align*}
		&\E[D_{n+1}(s,t)| \F_{n+2}]	\\
		&= \frac{1}{n+2} \sum_{i=1}^{n+2} \prod_{k=1}^{i-1}\left(1+\frac{1-q(Z_{k:n+2})}{n-k+3}\right)^{2\I{Z_{k:n+2} <s}}  \left(1+\frac{1-q(Z_{k:n+2})}{n-k+2}\right)^\I{s < Z_{k:n+2} < t}\\
		&\qquad\qquad\quad \times \prod_{k=i+1}^{n+2}\left(1+\frac{1-q(Z_{k:n+2})}{n-k+4}\right)^{2\I{Z_{k:n+2} <s}} \left(1+\frac{1-q(Z_{k:n+2})}{n-k+3}\right)^\I{s < Z_{k:n+2} < t}\\
		&= \frac{A_{n+2}}{n+2} + \frac{1}{n+2}\sum_{i=2}^{n+2} \prod_{k=1}^{i-1} \dots \times \prod_{k=i+1}^{n+2}\dots \\
		&= \frac{A_{n+2}}{n+2} + \frac{1}{n+2}\sum_{i=1}^{n+1} \prod_{k=1}^{i} \dots \times \prod_{k=i+2}^{n+2}\dots\\
		&= \frac{A_{n+2}}{n+2} + \frac{1}{n+2}\left(1+\frac{1-q(Z_{1:n+2})}{n+2}\right)^{2\I{Z_{1:n+2} < s}} \left(1+\frac{1-q(Z_{1:n+2})}{n+1}\right)^\I{s < Z_{1:n+2} < t}\\
		&\qquad\qquad\qquad\times \sum_{i=1}^{n+1} \prod_{k=1}^{i-1} \left(1+\frac{1-q(Z_{k+1:n+2})}{n-k+2}\right)^{2\I{Z_{k+1:n+2} <s}}\\
		&\qquad\qquad\qquad\qquad\qquad \times \left(1+\frac{1-q(Z_{k+1:n+2})}{n-k+1}\right)^\I{s < Z_{k+1:n+2} < t}\\
		&\qquad\qquad\qquad\qquad \times \prod_{k=i+1}^{n+1}\left(1+\frac{1-q(Z_{k+1:n+2})}{n-k+3}\right)^{2\I{Z_{k+1:n+2} <s}}\\
		&\qquad\qquad\qquad\qquad\qquad \times \left(1+\frac{1-q(Z_{k+1:n+2})}{n-k+2}\right)^\I{s < Z_{k+1:n+2} < t} \mdot 
		\end{align*}
		%
		Using \eqref{eq:supermart_yk} on the right hand side of the equation above yields
		\begin{align*}
		&\E[D_{n+1}(s,t)| \F_{n+2}]	\\
		&\leq \frac{A_{n+2}}{n+2} + \frac{n+1}{n+2}\left(1+\frac{1-q(Z_{1:n+2})}{n+2}\right)^{2\I{Z_{1:n+2} < s}} \left(1+\frac{1-q(Z_{1:n+2})}{n+1}\right)^\I{s < Z_{1:n+2} < t}\\
		&\qquad\qquad\qquad\times \prod_{k=1}^{n+1} \left(1+\frac{1-q(Z_{k+1:n+2})}{n-k+3}\right)^{2\I{Z_{k+1:n+2} <s}}\\
		&\qquad\qquad\qquad\qquad\qquad \times \left(1+\frac{1-q(Z_{k+1:n+2})}{n-k+2}\right)^\I{s < Z_{k+1:n+2} < t}\\
		&= A_{n+2} \left[\frac{1}{n+2} + \frac{n+1}{n+2}\left(1+\frac{1-q(Z_{1:n+2})}{n+2}\right)^{2\I{Z_{1:n+2} < s}} \right. \\
		&\qquad\qquad\qquad\qquad\qquad \left. \times \left(1+\frac{1-q(Z_{1:n+2})}{n+1}\right)^\I{s < Z_{1:n+2} < t}\right] \mdot 
		\end{align*}
		%
		For the moment, let
		$$x_1 := 1-q(Z_{1:n+2}) \textrm{, } s_1 := \I{Z_{1:n+2} < s} \textrm{ and } t_1 := \I{s < Z_{1:n+2} < t} $$
		%
		Now we can rewrite the above as
		\begin{align*}
		\E[D_{n+1}(s,t)| \F_{n+2}]	&\leq A_{n+2} \left[\frac{1}{n+2} + \frac{n+1}{n+2}\left(1+\frac{x_1s_1}{n+2}\right)^{2} \left(1+\frac{x_1t_1}{n+1}\right)\right]\mdot  \numberthis\label{eq:dn_supermart_an}
		\end{align*}
		%
		Next consider 
		\begin{align*}
		\left(1+\frac{x_1t_1}{n+1}\right) &= \left(1+\frac{x_1t_1}{n+2}-\frac{1}{n+2}\right) \left(1+\frac{1}{n+1}\right)\\
		&=  \left(1+\frac{x_1t_1}{n+2}\right)+\frac{1}{n+1}\left(1+\frac{x_1t_1}{(n+2)}\right) - \frac{1}{n+1}\\
		&= \left(1+\frac{x_1t_1}{n+2}\right)+\frac{x_1t_1}{(n+1)(n+2)}\mdot 
		\end{align*}
		%
		Thus we get
		\begin{align*}
		&\frac{n+1}{n+2}\left(1+\frac{x_1s_1}{n+2}\right)^{2} \left(1+\frac{x_1t_1}{n+1}\right) \\
		&= \frac{n+1}{n+2}\left(1+\frac{x_1s_1}{n+2}\right)^{2}\left(1+\frac{x_1t_1}{n+2}\right) + \left(1+\frac{x_1s_1}{n+2}\right)^{2}\frac{x_1t_1}{(n+2)^2}\mdot 
		\end{align*}
		%
		But now 
		\begin{align*}
		\left(1+\frac{x_1s_1}{n+2}\right)^{2}\frac{x_1t_1}{(n+2)^2} &= \left(1+2\frac{x_1s_1}{n+2}+\frac{x^2_1s_1}{(n+2)^2}\right)\frac{x_1t_1}{(n+2)^2}\\
		&= \frac{x_1t_1}{(n+2)^2}
		\end{align*}
		since $s_1\cdot t_1=0$ for all $s<t$. Hence we can rewrite the term in brackets in \eqref{eq:dn_supermart_an} as 
		\begin{align*}
		&\frac{1}{n+2} + \frac{n+1}{n+2}\left(1+\frac{x_1s_1}{n+2}\right)^{2} \left(1+\frac{x_1t_1}{n+1}\right) \\
		&=\frac{1}{n+2} + \frac{x_1t_1}{(n+2)^2} + \frac{n+1}{n+2}\left(1+\frac{x_1s_1}{n+2}\right)^{2}\left(1+\frac{x_1t_1}{n+2}\right)\\
		&=\frac{1}{n+2}\left(1+\frac{x_1t_1}{n+2}\right) + \frac{n+1}{n+2}\left(1+\frac{x_1s_1}{n+2}\right)^{2}\left(1+\frac{x_1t_1}{n+2}\right)\\
		&=\left[\frac{1}{n+2} + \frac{n+1}{n+2}\left(1+\frac{x_1}{n+2}\right)^{2s_1}\right]\left(1+\frac{x_1}{n+2}\right)^{t_1}\\
		&\leq \left(1+\frac{x_1}{n+3}\right)^{2s_1}\left(1+\frac{x_1}{n+2}\right)^{t_1}\mdot 
		\end{align*}
		The latter inequality above holds, since 
		$$\left[\frac{1}{n+2} + \frac{n+1}{n+2}\left(1+\frac{x}{n+2}\right)^{2}\right] \leq \left(1+\frac{x}{n+3}\right)^{2}$$
		for any $0\leq x\leq 1$. (\cf\ \cite{bose1999strong}, page 197). Therefore we can rewrite \eqref{eq:dn_supermart_an} as
		\begin{align*}
		\E[D_{n+1}(s,t)| \F_{n+2}]	&\leq A_{n+2} \left(1+\frac{1-q(Z_{1:n+2})}{n+3}\right)^{2\I{Z_{1:n+2}<s}} \\
		&\qquad\quad \times \left(1+\frac{1-q(Z_{1:n+2})}{n+2}\right)^{\I{s<Z_{1:n+2} <t}}\\
		&= D_{n+2}(s,t)\mdot
		\end{align*}		
		This concludes the proof.
	\end{proof}
\end{lemma}
%
\begin{lemma} \label{lem:hewitt_savage}
	Let $\F_\infty = \bigcap_{n\geq 2} \F_n$. Then we have for each $A\in \F_\infty$ that $\P(A)\in \{0,1\}$.
	%
	\begin{proof}
		Denote $\tilde{Z}:=(Z_1, Z_2,\dots)\in\R^\infty$ and let $1\leq n<\infty$ fixed but arbitrary. Moreover define \todo{phrasing}
		\begin{equation*}
		\Pi_n := \{\pi | \pi \textrm{ is permutation of } 1,\dots, n\} \mcomma
		\end{equation*}	
		\ie\ for each $\pi\in \Pi_n$ we have
		\begin{align*}
		\pi: (\R^\infty, \mathcal{B}(\R^\infty)) &\longrightarrow (\R^\infty, \mathcal{B}(\R^\infty))\\
		(Z_1, Z_2,\dots,Z_n, Z_{n+1}, \dots) &\longmapsto (Z_{\pi(1)}, Z_{\pi(2)},\dots,Z_{\pi(n)}, Z_{n+1}, \dots)\mdot
		\end{align*}
		%
		%		\begin{equation*}
		%			\Pi := \bigcup\limits_{n\in\mathbb{N}} \Pi_n \nonumber\mdot 
		%		\end{equation*}
		We will now use the Hewitt-Savage zero-one law to prove the statement of this lemma. We need to show that for all $A\in\F_\infty$ and for all $\pi_0\in\Pi$ there exists $B\in\mathcal{B}(\R^\infty)$ \st\ 
		\begin{equation}
		A = \{\omega| \tilde{Z}(\omega) \in B\} = \{\omega| \pi_0(\tilde{Z}(\omega)) \in B\} \mdot
		\label{eq:hewitt_savage}
		\end{equation} 	
		%
		Let $A\in\F_\infty$, then $A\in\F_n$ for all $n\in\mathbb{N}$. Since the map $(Z_{1:n}, \dots, Z_{n:n}, Z_{n+1}, Z_{n+2}, \dots)$ is measurable, there must exist $\tilde{B}\in\mathcal{B}(\R^\infty)$ such that
		\begin{equation*}
		A = \{\omega | (Z_{1:n}(\omega), \dots, Z_{n:n}(\omega), Z_{n+1}(\omega), Z_{n+2}(\omega), \dots) \in \tilde{B}\}\mdot 
		\label{eq:A_Z_inv}
		\end{equation*}
		%
		Note that each of the maps $\pi\in\Pi_n$ is measurable. Hence we can write $A$ as
		\begin{align*}
		A &= \bigcup\limits_{\pi\in\Pi_n}\left \{\omega | \pi(\tilde{Z}) \in \tilde{B}\right\}\\
		&=  \bigcup\limits_{\pi\in\Pi_n} \left\{\omega | \tilde{Z} \in \pi^{-1}(\tilde{B})\right\}\\
		&=  \left\{\omega | \tilde{Z} \in \bigcup\limits_{\pi\in\Pi_n}\pi^{-1}(\tilde{B})\right\}\\
		&= \left\{\omega | \tilde{Z} \in B\right\}\mcomma
		\end{align*}		
		with 
		$$B:=\bigcup\limits_{\pi\in\Pi_n}\pi^{-1}(\tilde{B})\mdot$$
		%
		Clearly $B\in\mathcal{B}(\R^\infty)$, since it is constructed as countable union of sets in $\mathcal{B}(\R^\infty)$.
		%
		Moreover note that 
		$$\bigcup\limits_{\pi\in\Pi_n}\pi^{-1}(\tilde{B}) = \bigcup\limits_{\pi\in\Pi_n}(\pi_0\circ\pi)^{-1}(\tilde{B})\mcomma$$
		since the union is iterating over all $\pi\in\Pi_n$. Thus we can write
		\begin{align*}
		A &= \left\{\omega | \tilde{Z} \in \bigcup\limits_{\pi\in\Pi_n}(\pi_0\circ\pi)^{-1}(\tilde{B})\right\}\\
		&= \bigcup\limits_{\pi\in\Pi_n}\left\{\omega | \tilde{Z} \in (\pi_0\circ\pi)^{-1}(\tilde{B})\right\}\\
		&= \bigcup\limits_{\pi\in\Pi_n}\left\{\omega | \pi_0(\tilde{Z}) \in \pi^{-1}(\tilde{B})\right\}\\
		&= \left\{\omega | \pi_0(\tilde{Z}) \in B\right\}\mdot
		\end{align*}		
		Whence establishing \eqref{eq:hewitt_savage}.
	\end{proof}
\end{lemma}
%
\begin{lemma}For any $s<t$ \st\ $H(t)<1$ the following statement holds true
	$$\Delta_{n}(s,t) = \E[D_n(s,t)] = \E[D_n(s,t)|\F_\infty] \nearrow D(s,t)\mdot$$
	\label{lem:neveu}
	\begin{proof}
		Consider that we have for $n\geq 2$
		$$\Delta_{n}(s,t) = \E[D_n(s,t)] = \E[D_n(s,t)|\F_\infty] $$
		by definition of $\Delta_{n}(s,t)$ and Lemma \ref{lem:hewitt_savage}. Next note that we have $D_n(s,t) \to D(s,t)$ almost surely, according to Lemma \ref{lem:dn_limit}. Moreover we get from Lemma \ref{lem:dn_supermart}, that $\{D_n,\F_n\}_{n\geq 1}$ is a reverse supermartingale. Now this together with Proposition 5-3-11 of \cite{neveu1975discrete} yields
		$$\E[D_n(s,t)|\F_\infty] \nearrow D(s,t)\mdot$$
		This proves the lemma.
	\end{proof}
\end{lemma}



	\bibliographystyle{./ametsoc}
	%\bibliographystyle{alphanum}
	
	% Generate the bibliography using entries from the following .BIB file
	\bibliography{./thesis}
\end{document}