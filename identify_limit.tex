\chapter{Identifying the limit}
\label{ch:identify_limit}
%
In the previous chapter we established the existence of the limit \\
$$\lim\limits_{n\to\infty} S_n = S_\infty\mdot$$
%
We will now continue to identify the limit $S(m(\cdot,\hat{\theta}_n))$ throughout this chapter. The interdependence structure of the proofs within this chapter in figure \ref{fig:structure_identify} below.\\
\begin{figure}[h!]
	\begin{center}
		\fbox{\begin{forest}
			for tree={
				font=\sffamily\bfseries,
				line width=1pt,
				draw=black,
				rounded corners,
				align=center,
				child anchor=north,
				parent anchor=south,
				drop shadow,
				grow = south,
				l sep+=5pt,
				edge path={
					\noexpand\path[color=black, rounded corners=5pt, >={Stealth[length=1pt]}, line width=1pt, \forestoption{edge}]
					(!u.parent anchor) -- +(0,-5pt) -|
					(.child anchor)\forestoption{edge label};
				},
				where level={3}{tier=tier3}{}
			}
			[Theorem \ref{thm:snmn_limit}, inner color=col1in, outer color=col1out
				%[Lemma \ref{lem:connector}, inner color=white, outer color=white
				%	[Lemma \ref{lem:hewitt_savage}, inner color=white, outer color=white]
				%]
				[Lemma \ref{lem:sn_limit}, inner color=white, outer color=white
					[Lem. \ref{lem:dn_limit}, inner color=white, outer color=white]
					[Lem. \ref{lem:neveu}, inner color=white, outer color=white]
					[Lem. \ref{lem:Sn_Delta}, inner color=white, outer color=white
						[L \ref{lem:zizone}, inner color=white, outer color=white]
						[L \ref{lem:representation_bn}, inner color=white, outer color=white]	
					]
					[Lem. \ref{lem:Cn_bounds_and_limit}, inner color=white, outer color=white]			
				]
				[Corollary \ref{lem:sandwich}, inner color=white, outer color=white
					[Lem. \ref{lem:Mm}, inner color=white, outer color=white]
				]
			]]
		\end{forest}}
	\end{center}
	\caption{Interdependence Structure of the lemmas and theorems within this chapter.}
	\label{fig:structure_identify}
\end{figure}
%
\section{The reverse supermartingale $D_n$}
Let's first define the following quantities for $n\geq 1$ and $s < t$:
\begin{align*}
B_n(s) &:= \prod_{k=1}^{n}\left[1+\frac{1-q(Z_{k})}{n-R_{k,n}}\right]^{\I{Z_{k} < s}}\\
C_n(s) &:= \sum_{i=1}^{n+1}\left[\frac{1-q(s)}{n-i+2}\right]\I{Z_{i-1:n} < s \leq Z_{i:n}}\\
D_n(s,t) &:= \prod_{k=1}^{n} \left[1+\frac{1-q(Z_k)}{n-R_{k,n} +2}\right]^{2\I{Z_k<s}} \prod_{k=1}^{n}\left[1+\frac{1-q(Z_k)}{n-R_ {k,n}+1}\right]^{\I{s < Z_k < t}}\\
\Delta_n(s,t) &:= \E\left[D_n(s,t) \right]\\
\bar{\Delta}_n(s,t) &:= \E\left[C_n(s)D_n(s,t) \right]\mdot
\end{align*}
Here $Z_{0:n} := 0$ and $Z_{n+1:n} := \infty$.\\
\\
During this section, we will derive a representation of $\E[S_n]$ which involves the process $D_n$. This will be done in Lemma \ref{lem:zizone} and Lemma \ref{lem:representation_bn}. We will then show that $\{D_n,\F_n\}$ is a reverse supermartingale in Lemma \ref{lem:dn_supermart} and identify the limit of $D_n$ in Lemma \ref{lem:dn_limit}. Moreover Lemmas \ref{lem:zizone}, \ref{lem:representation_bn} and \ref{lem:neveu} will play a central role in identifying the limit $S$ in Chapter \ref{ch:identify_limit}.\\
\\
The lemma below contains a basic result needed to prove Lemma \ref{lem:representation_bn}.
\begin{lemma}
	\label{lem:bnbn_change_order}
	Let $i\neq j$. Then the conditional expectation 
	$$\E[B_n(s)B_n(t) | Z_i=s, Z_j=t]$$ 
	is independent of $i,j$ and hence
	$$\E[B_n(s)B_n(t) | Z_i=s, Z_j=t] = \E[B_n(s)B_n(t) | Z_1=s, Z_2=t]$$
	holds almost surely.
	%
	\begin{proof}
		For the sake of notational simplicity denote for $s<t$ $s_k^n := \I{Z_{k:n} < s}$ and $t_k^n := \I{s\leq Z_{k:n} < t}$. Note that $i\neq j$ implies $s\neq t$, since the $(Z_i)_{i\leq n}$ are pairwise distinct. Now consider on $\{s<t\}$ 
		\begin{align*}
		&\E\left[B_n(s)B_n(t)|Z_i=s, Z_j=t\right]\\
		&= \E\left[\prod_{k=1}^{n}\left(1+\frac{1-q(Z_{k:n})}{n-k}\right)^{2s_k^n + t_k^n} | Z_i=s, Z_j=t\right]\\
		&= \E\left[\sum_{k_1=1}^{n-1} \sum_{k_2=2}^{n} \I{Z_{k_1:n} = s} \I{Z_{k_2:n} = t}\left(1+\frac{1-q(s)}{n-k_1}\right)\right.\\ 
		&\qquad\qquad \times \prod_{k=1}^{k_1-1}\left(1+\frac{1-q(Z_{k:n})}{n-k}\right)^{2s_k^n + t_k^n} \\
		&\qquad\qquad \times \prod_{k=k_1+1}^{k_2-1}\left(1+\frac{1-q(Z_{k:n})}{n-k}\right)^{2s_k^n + t_k^n} \\
		&\qquad\qquad \times  \left.\prod_{k=k_2+1}^{n}\left(1+\frac{1-q(Z_{k:n})}{n-k}\right)^{2s_k^n + t_k^n} | Z_i=s, Z_j=t\right]
		\end{align*}
		since $s_{k_1}^n = 0$, $t_{k_1}^n=1$, $s_{k_2}^n = 0$ and $t_{k_2}^n=0$. Moreover we have 
		\[ \begin{cases} 
		s_{k}^n = 1 \textrm{ and } t_{k}^n = 0 & \textrm{ if } k < k_1  \\
		s_{k}^n = 0 \textrm{ and } t_{k}^n = 1 & \textrm{ if } k_1 < k < k_2\\
		s_{k}^n = 0 \textrm{ and } t_{k}^n = 0 & \textrm{ if } k_2 < k \\
		\end{cases}\mdot
		\]		
		Therefore we obtain
		\begin{align*}
		&\E\left[B_n(s)B_n(t)|Z_i=s, Z_j=t\right]\\
		&= \E\left[\sum_{k_1=1}^{n-1} \sum_{k_2=2}^{n} \I{Z_{k_1:n} = s} \I{Z_{k_2:n} = t}\left(1+\frac{1-q(s)}{n-k_1}\right)\right.\\ 
		&\qquad\qquad \times \prod_{k=1}^{k_1-1}\left(1+\frac{1-q(Z_{k:n})}{n-k}\right)^{2s_k^n} \\
		&\qquad\qquad \times \left. \prod_{k=k_1+1}^{k_2-1}\left(1+\frac{1-q(Z_{k:n})}{n-k}\right)^{t_k^n} | Z_i=s, Z_j=t\right]\mdot
		\end{align*}
		%
		Next we need to introduce some more notation. For $1\leq i,j\leq n$ and $n\geq 2$, let $\{Z_{k:n-2}\}_{k\leq n-2}$ denote the ordered $Z$-values among $Z_1,\dots, Z_n$ with $Z_i$ and $Z_j$ removed from the sample. Note that
		\[ Z_{k:n} = \begin{cases} 
		Z_{k:n-2} & k < k_1  \\
		Z_{k-1:n-2} & k_1 < k < k_2
		\end{cases}\numberthis\label{eq:zkn_zknminustwo}\mdot
		\]	
		Thus we have
		\begin{align*}
		&\E\left[B_n(s)B_n(t)|Z_i=s, Z_j=t\right]\\
		&= \E\left[\sum_{k_1=1}^{n} \sum_{k_2=1}^{n} \I{Z_{k_1-1:n-2} < s \leq Z_{k_1:n-2}} \I{Z_{k_2-2:n-2} < t \leq Z_{k_2-1:n-2}}\right.\\ 
		&\qquad\qquad \times \left(1+\frac{1-q(s)}{n-k_1}\right)\prod_{k=1}^{k_1-1}\left(1+\frac{1-q(Z_{k:n-2})}{n-k}\right)^{2s_k^{n-2}} \\
		&\qquad\qquad \times \left. \prod_{k=k_1+1}^{k_2-1}\left(1+\frac{1-q(Z_{k-1:n-2})}{n-k}\right)^{t_{k-1}^{n-2}} | Z_i=s, Z_j=t\right]\\
		&= \E\left[\sum_{k_1=1}^{n} \sum_{k_2=1}^{n} \I{Z_{k_1-1:n-2} < s \leq Z_{k_1:n-2}} \I{Z_{k_2-2:n-2} < t \leq Z_{k_2-1:n-2}}\right.\\ 
		&\qquad\qquad \times \left(1+\frac{1-q(s)}{n-k_1}\right)\prod_{k=1}^{k_1-1}\left(1+\frac{1-q(Z_{k:n-2})}{n-k}\right)^{2s_k^{n-2}} \\
		&\qquad\qquad \times \left. \prod_{k=k_1}^{k_2-2}\left(1+\frac{1-q(Z_{k:n-2})}{n-k-1}\right)^{t_{k}^{n-2}}\right]\\
		&= \E\left[\sum_{k_1=1}^{n} \I{Z_{k_1-1:n-2} < s \leq Z_{k_1:n-2}} \left(1+\frac{1-q(s)}{n-k_1}\right) \right.\\ 
		&\qquad\qquad \times \prod_{k=1}^{n-2}\left(1+\frac{1-q(Z_{k:n-2})}{n-k}\right)^{2s_k^{n-2}} \\
		&\qquad\qquad \times \left. \prod_{k=k_1}^{n-2}\left(1+\frac{1-q(Z_{k:n-2})}{n-k-1}\right)^{t_{k}^{n-2}}\right]
		\end{align*}		
		which is independent of $i,j$. Next consider the case $t<s$. Define $\tilde{t}_k^n := \I{Z_{k:n} < t}$ and $\tilde{s}_k^n := \I{t\leq Z_{k:n} < s}$. Using similar arguments we can show that in this case 
		\begin{align*}
		&\E\left[B_n(s)B_n(t)|Z_i=s, Z_j=t\right]\\
		&= \E\left[\sum_{k_1=1}^{n} \I{Z_{k_1-1:n-2} < t \leq Z_{k_1:n-2}} \left(1+\frac{1-q(t)}{n-k_1}\right) \right.\\ 
		&\qquad\qquad \times \prod_{k=1}^{n-2}\left(1+\frac{1-q(Z_{k:n-2})}{n-k}\right)^{2\tilde{t}_k^{n-2}} \\
		&\qquad\qquad \times \left. \prod_{k=k_1}^{n-2}\left(1+\frac{1-q(Z_{k:n-2})}{n-k-1}\right)^{\tilde{s}_{k}^{n-2}}\right]
		\end{align*}	
		which is independent of $i,j$ as well. Thus we have on $\{s\neq t\}$ that $\E\left[B_n(s)B_n(t)|Z_i=s, Z_j=t\right]$ is independent of $i,j$ and hence
		$$\E\left[B_n(s)B_n(t)|Z_i=s, Z_j=t\right] = \E\left[B_n(s)B_n(t)|Z_1=s, Z_2=t\right]\mdot$$
	\end{proof}
\end{lemma}
%
\begin{lemma} \label{lem:zizone}
	Let $\tilde{\phi}: \R^2_+ \longrightarrow \R_+$ be a Borel-measurable function. Then we have for any $n\geq 2$ 
	\begin{align*}
	&\E[\tilde{\phi}(Z_{i},Z_{j}) B_n(Z_{i}) B_n(Z_{j})]\\
	& = \E[\tilde{\phi}(Z_1,Z_2) B_n(Z_1) B_n(Z_2)]\mdot
	\end{align*}
	%
	\begin{proof}
		Consider that $\{Z_i=Z_j\}$ is a measure zero set, since $H$ is continuous. Therefore the following holds for $1\leq i,j \leq n$ 
		\begin{align*}
		& \E\left[\tilde{\phi}(Z_{i},Z_{j}) \E\left[B_n(Z_{i}) B_n(Z_{j})| Z_i,Z_j\right]\right]\\
		&= \E\left[\I{Z_i\neq Z_j}\tilde{\phi}(Z_{i},Z_{j}) \E\left[B_n(Z_{i}) B_n(Z_{j})| Z_i,Z_j\right]\right]\\
		&= \E\left[\I{i\neq j}\tilde{\phi}(Z_{i},Z_{j}) \E\left[B_n(Z_{i}) B_n(Z_{j})| Z_i,Z_j\right]\right]\\
		&= \int_{0}^{\infty}\int_{0}^{\infty}\I{i\neq j}\tilde{\phi}(s,t) \E\left[B_n(s) B_n(t)| Z_i=s,Z_j=t\right]H(ds)H(dt)\mdot \numberthis\label{eq:ioneitwo}
		\end{align*}
		%
		According to Lemma \ref{lem:bnbn_change_order} we have for $1\leq i\neq j\leq n$
		\begin{align*}
		\E[B_n(s)B_n(t)|Z_i=s, Z_j=t] = \E[B_n(s)B_n(t)|Z_1=s, Z_2=t]
		\end{align*}
		%
		Therefore we obtain, according to \eqref{eq:ioneitwo}, that 
		\begin{align*}
		\E\left[\tilde{\phi}(Z_{i},Z_{j}) B_n(Z_{i}) B_n(Z_{j})\right] &= \E\left[\tilde{\phi}(Z_{i},Z_{j}) \E\left[B_n(Z_{i}) B_n(Z_{j})| Z_i,Z_j\right]\right]\\
		&= \E\left[\tilde{\phi}(Z_{1},Z_{2}) B_n(Z_{1}) B_n(Z_{2})\right]\mdot
		\end{align*}
	\end{proof}
\end{lemma}
%
\begin{lemma} \label{lem:representation_bn}
	Let $\tilde{\phi}: \R^2_+ \longrightarrow \R_+$ be a Borel-measurable function. Then we have for any $s<t$ and $n\geq 2$ 
	\begin{align*}
	&\E[\tilde{\phi}(Z_1,Z_2) B_n(Z_1) B_n(Z_2)]\\
	& = \E[2\tilde{\phi}(Z_1,Z_2) \{\Delta_{n-2}(Z_1,Z_2) + \bar{\Delta}_{n-2}(Z_1,Z_2)\}\I{Z_1<Z_2}]\mdot
	\end{align*}
	%
	\begin{proof}
		Note that w.l.o.g. we can assume that the $(Z_i)_{i\leq n}$ are pairwise distinct, since $H$ is continuous. Consider the following
		\begin{align*}
		B_n(Z_1)B_n(Z_2) &= \prod_{k=1}^{n}\left[1+\frac{1-q(Z_{k})}{n-R_{k,n}}\right]^{\I{Z_{k} < Z_1}+\I{Z_{k} < Z_2}}\\
		&= \left[1+\frac{1-q(Z_{1})}{n-R_{1,n}}\right]^{\I{Z_{1} < Z_2}} \left[1+\frac{1-q(Z_{2})}{n-R_{2,n}}\right]^{\I{Z_{2} < Z_1}}\\
		&\qquad \times \prod_{k=3}^{n}\left[1+\frac{1-q(Z_{k})}{n-R_{k,n}}\right]^{\I{Z_{k} < Z_1}+\I{Z_{k} < Z_2}}\\
		&= \I{Z_1<Z_2}\left[1+\frac{1-q(Z_{1})}{n-R_{1,n}}\right] \\
		&\qquad\qquad \times \prod_{k=1}^{n-2}\left[1+\frac{1-q(Z_{k+2})}{n-R_{k+2,n}}\right]^{\I{Z_{k+2} < Z_1}+\I{Z_{k+2} < Z_2}}\\
		&\quad + \I{Z_1>Z_2}\left[1+\frac{1-q(Z_{2})}{n-R_{2,n}}\right] \\
		&\qquad\qquad \times \prod_{k=1}^{n-2}\left[1+\frac{1-q(Z_{k+2})}{n-R_{k+2,n}}\right]^{\I{Z_{k+2} < Z_1}+\I{Z_{k+2} < Z_2}}\\
		&\quad + \I{Z_1=Z_2}\prod_{k=1}^{n-2}\left[1+\frac{1-q(Z_{k+2})}{n-R_{k+2,n}}\right]^{2\I{Z_{k+2} < Z_1}}\mdot \numberthis\label{eq:bnbn1}
		\end{align*}
		%
		On $\{Z_1<Z_2\}$ we have 
		\begin{align*}
		\prod_{k=1}^{n-2}\left[1+\frac{1-q(Z_{k+2})}{n-R_{k+2,n}}\right]^{\I{Z_{k+2} < Z_2}} &= \prod_{k=1}^{n-2}\left[1+\frac{1-q(Z_{k+2})}{n-\tilde{R}_{k,n-2}}\right]^{\I{Z_{k+2} < Z_1}}\\
		&\quad \times  \prod_{k=1}^{n-2}\left[1+\frac{1-q(Z_{k+2})}{n-\tilde{R}_{k,n-2}-1}\right]^{\I{Z_1 < Z_{k+2} < Z_2}}
		\end{align*}
		where $\tilde{R}_{k,n-2}$ denotes the rank of the $Z_k$, $k=3,\dots, n$ among themselves. The above holds since 
		\[ R_{k+2,n} = \begin{cases} 
		\tilde{R}_{k,n-2} & \textrm{ if } Z_{k+2} < Z_1 \\
		\tilde{R}_{k, n-2} + 1 & \textrm{ if } Z_1 < Z_{k+2} < Z_2 
		\end{cases}
		\]
		for $k=1,\dots,n-2$. 
		% 
		Therefore \eqref{eq:bnbn1} yields
		\begin{align*}
		B_n(Z_1)B_n(Z_2) &= \I{Z_1<Z_2}\left[1+\frac{1-q(Z_{1})}{n-R_{1,n}}\right] \\
		&\qquad\qquad \times \prod_{k=1}^{n-2}\left[1+\frac{1-q(Z_{k+2})}{n-\tilde{R}_{k,n-2}}\right]^{2\I{Z_{k+2} < Z_1}}\\
		&\qquad\qquad \times \prod_{k=1}^{n-2}\left[1+\frac{1-q(Z_{k+2})}{n-\tilde{R}_{k,n-2}-1}\right]^{\I{Z_1 < Z_{k+2} < Z_2}}\\
		&\quad + \I{Z_2<Z_1}\left[1+\frac{1-q(Z_{2})}{n-R_{2,n}}\right] \\
		&\qquad\qquad \times \prod_{k=1}^{n-2}\left[1+\frac{1-q(Z_{k+2})}{n-\tilde{R}_{k,n-2}}\right]^{2\I{Z_{k+2} < Z_2}}\\
		&\qquad\qquad \times \prod_{k=1}^{n-2}\left[1+\frac{1-q(Z_{k+2})}{n-\tilde{R}_{k,n-2}-1}\right]^{\I{Z_2 < Z_{k+2} < Z_1}}\\
		&\quad + \I{Z_1=Z_2}\prod_{k=1}^{n-2}\left[1+\frac{1-q(Z_{k+2})}{n-\tilde{R}_{k,n-2}}\right]^{2\I{Z_{k+2} < Z_1}}\mdot \numberthis\label{eq:bnbn_rank}
		\end{align*}
		%
		Now let's denote $Z_{k:n-2}$ the ordered $Z$-values among $Z_3,\dots, Z_n$ for $k=1,\dots,n-2$. Consider that we can write 
		\begin{align*}
		\left[1+\frac{1-q(Z_{1})}{n-R_{1,n}}\right] = \sum_{i=1}^{n-1}\left[1+\frac{1-q(s)}{n-i}\right]\I{Z_{i-1:n-2} < Z_1 \leq Z_{i:n-2}}\mdot
		\end{align*}
		%
		Recall that we set $Z_{0:n}=0$ and $Z_{n-1:n-2}=\infty$. Now note that $Z_{k:n-2}$ is independent of $Z_1$ and $Z_2$ for $k=1,\dots,n-2$. Therefore we obtain the following, by conditioning \eqref{eq:bnbn_rank} on $Z_1,Z_2$:
		\begin{align*}
		&\E[B_n(Z_1)B_n(Z_2)|Z_1 = s, Z_2 = t]\\
		&= \I{s<t}\E\left[\left(\sum_{i=1}^{n-1}\left[1+\frac{1-q(s)}{n-i}\right]\I{Z_{i-1:n-2} < s \leq Z_{i:n-2}}\right)\right.\\
		&\qquad\qquad\qquad \times \prod_{k=1}^{n-2}\left[1+\frac{1-q(Z_{k:n-2})}{n-k}\right]^{2\I{Z_{k:n-2} < s}}\\
		&\qquad\qquad\qquad \times \left. \prod_{k=1}^{n-2}\left[1+\frac{1-q(Z_{k:n-2})}{n-k-1}\right]^{\I{s < Z_{k:n-2} < t}} \right]\\
		&\quad + \I{t<s}\E\left[\left(\sum_{i=1}^{n-1}\left[1+\frac{1-q(t)}{n-i}\right]\I{Z_{i-1:n-2} < t \leq Z_{i:n-2}}\right)\right. \\
		&\qquad\qquad\qquad \times \prod_{k=1}^{n-2}\left[1+\frac{1-q(Z_{k:n-2})}{n-k}\right]^{2\I{Z_{k:n-2} < t}}\\
		&\qquad\qquad\qquad \times \left. \prod_{k=1}^{n-2}\left[1+\frac{1-q(Z_{k:n-2})}{n-k-1}\right]^{\I{t < Z_{k:n-2} < s}}\right]\\
		&\quad + \I{s=t}\E\left[\prod_{k=1}^{n-2}\left[1+\frac{1-q(Z_{k:n-2})}{n-k}\right]^{2\I{Z_{k:n-2} < s}} \right]\\
		&= \alpha(s,t) + \alpha(t,s) + \beta(s,t)
		\end{align*}
		where 
		\begin{align*}
		\alpha(s,t) &:=\I{s<t}\E\left[\left(\sum_{i=1}^{n-1}\left[1+\frac{1-q(s)}{n-i}\right]\I{Z_{i-1:n-2} < s \leq Z_{i:n-2}}\right)\right.\\
		&\qquad\qquad\qquad \times \prod_{k=1}^{n-2}\left[1+\frac{1-q(Z_{k:n-2})}{n-k}\right]^{2\I{Z_{k:n-2} < s}}\\
		&\qquad\qquad\qquad \times \left. \prod_{k=1}^{n-2}\left[1+\frac{1-q(Z_{k:n-2})}{n-k-1}\right]^{\I{s < Z_{k:n-2} < t}} \right]
		\end{align*}
		and 
		\begin{align*}
		\beta(s,t) &:=\I{s=t}\E\left[\prod_{k=1}^{n-2}\left[1+\frac{1-q(Z_{k:n-2})}{n-k}\right]^{2\I{Z_{k:n-2} < s}} \right]\mdot
		\end{align*}		
		%
		Consider that we have
		$$\E[\alpha(Z_1,Z_2)] = \E[\alpha(Z_2,Z_1)]\mcomma$$
		because $Z_1$ and $Z_2$ are \iid\ and $\alpha$ is symmetric in its arguments. Moreover
		$$\E[\beta(Z_1,Z_2)] = 0$$
		since $H$ is continuous. Therefore we get 
		\begin{align*}
		&\E[\tilde{\phi}(Z_1,Z_2)B_n(Z_1)B_n(Z_2)]\\
		&= \E[\tilde{\phi}(Z_1,Z_2)(\alpha(Z_1,Z_2) + \alpha(Z_2,Z_1) + \beta(Z_1,Z_2))]\\
		&= \E[2\tilde{\phi}(Z_1,Z_2)\alpha(Z_1,Z_2)]\mdot \numberthis\label{eq:expectationalpha}
		\end{align*}
		under (A\ref{ass:kernel_gen}). 
		%
		Next consider that 
		\begin{align*}
		\alpha(s,t) &=\I{s<t}\E\left[(1+C_{n-2}(s)) D_{n-2}(s,t) \right]\\
		&= \I{s<t}(\Delta_{n-2}(s,t) + \bar{\Delta}_{n-2}(s,t))\mdot
		\end{align*}
		The latter equality holds, since
		\begin{align*}
		&\sum_{i=1}^{n-1}\left[1+\frac{1-q(s)}{n-i}\right]\I{Z_{i-1:n-2} < s \leq Z_{i:n-2}}\\
		&=\sum_{i=1}^{n-1}\I{Z_{i-1:n-2} < s \leq Z_{i:n-2}} + \sum_{i=1}^{n-1}\left[\frac{1-q(s)}{n-i}\right]\I{Z_{i-1:n-2} < s \leq Z_{i:n-2}}\\
		&= 1 + C_{n-2}(s)\mdot
		\end{align*}
		Now the statement of the lemma follows directly from \eqref{eq:expectationalpha}.
	\end{proof}
\end{lemma}
%
The next lemma identifies the almost sure limit of $D_n$ for $n\to\infty$. Define for $s<t$
$$D(s,t) := \exp\left(2\int_{0}^{s} \frac{1-q(z)}{1-H(z)} H(dz) + \int_{s}^{t} \frac{1-q(z)}{1-H(z)} H(dz)\right)$$
\begin{lemma} \label{lem:dn_limit}
	For any $s < t$ \st\ $H(t)<1$, we have
	$$\lim\limits_{n\to\infty}D_n(s,t) = D(s,t)\mdot$$
	%	
	\begin{proof}
		First recall the following definition
		\begin{align*}
		D_n(s,t) &:= \prod_{k=1}^{n} \left[1+\frac{1-q(Z_k)}{n-R_{k,n} +2}\right]^{2\I{Z_k<s}} \prod_{k=1}^{n}\left[1+\frac{1-q(Z_k)}{n-R_ {k,n}+1}\right]^{\I{s < Z_k < t}} \mdot
		\end{align*}
		%
		Next let 
		\begin{align*}
		x_k &:= \frac{1-q(Z_k)}{n(1- H_n(Z_k) + 2/n)}\\
		y_k &:= \frac{1-q(Z_k)}{n(1- H_n(Z_k) + 1/n)}\\
		s_k &:= \I{Z_k < s} \\
		t_k &:= \I{s < Z_k < t}
		\end{align*}
		for $s<t$ and $k=1,\dots,n$.
		%
		Now consider 
		\begin{align*}
		D_n(s,t) &= \prod_{k=1}^{n} \left[1+\frac{1-q(Z_k)}{n(1- H_n(Z_k) + 2/n)}\I{Z_k<s}\right]^{2}\\ 
		&\qquad \times \prod_{k=1}^{n}\left[1+\frac{1-q(Z_k)}{n(1-H_n(Z_k)+1/n)}\I{s < Z_k < t}\right]\\
		&= \prod_{k=1}^{n} \left[1+x_k s_k\right]^{2} \prod_{k=1}^{n}\left[1+y_k t_k\right]\\
		&= \exp\left(2\sum_{k=1}^{n}\ln\left[1+x_k s_k\right] + \sum_{k=1}^{n}\ln\left[1+y_k t_k\right]\right)\mdot
		\end{align*}
		%
		Note that $0 \leq x_k s_k \leq 1$ and $0 \leq y_k t_k \leq 1$. Consider that the following inequality holds  
		$$-\frac{x^2}{2} \leq \ln(1+x) - x \leq 0$$ 
		for any $x \geq 0$ (cf.  \cite{stute1993strong}, p. 1603). This implies 
		$$-\frac{1}{2}\sum_{k=1}^{n}x_k^2 s_k \leq \sum_{k=1}^{n}\ln(1+x_k s_k) - \sum_{k=1}^{n}x_k s_k \leq 0\mdot$$ 
		But now 
		\begin{align*}
		\sum_{k=1}^{n} x_k^2 s_k &= \frac{1}{n^2} \sum_{k=1}^{n} \left(\frac{1-q(Z_k)}{1-H_n(Z_k)+\frac{2}{n}}\right)^2\I{Z_k<s}\\
		&\leq \frac{1}{n^2} \sum_{k=1}^{n} \left(\frac{1}{1-H_n(s)+\frac{1}{n}}\right)^2\\
		&= \frac{1}{n(1-H_n(s)+n^{-1})^2} \longrightarrow 0
		\end{align*}
		almost surely as $n\to\infty$, since $H(s)<H(t)<1$ (\cf\ \cite{stute1993strong}, p. 1603). Therefore we have
		$$\abs{\sum_{k=1}^{n}\ln(1+x_k s_k) - \sum_{k=1}^{n}x_k s_k} \longrightarrow 0$$
		with probability 1 as $n\to\infty$. 
		%
		Similarly we obtain
		$$\abs{\sum_{k=1}^{n}\ln(1+y_k t_k) - \sum_{k=1}^{n}y_k t_k} \longrightarrow 0$$
		with probability 1 as $n\to\infty$. Hence 
		$$\lim\limits_{n\to\infty} D_n(s) = \lim\limits_{n\to\infty} \exp\left(2\sum_{k=1}^{n} x_k s_k + \sum_{k=1}^{n}y_k t_k\right)\mdot$$
		%
		Now consider 
		\begin{align*}
		\sum_{k=1}^{n} x_k s_k &= \frac{1}{n}\sum_{k=1}^{n} \frac{1-q(Z_k)}{1-H_n(Z_k)+\frac{2}{n}}\I{Z_k<s}\\
		&= \int_{0}^{s-} \frac{1-q(z)}{1-H_n(z)+\frac{2}{n}} H_n(dz)\\
		&= \int_{0}^{s-} \frac{1-q(z)}{1-H(z)} H_n(dz) + \int_{0}^{s-} \frac{1-q(z)}{1-H_n(z)+\frac{2}{n}} - \frac{1-q(z)}{1-H(z)} H_n(dz)\\
		&= \int_{0}^{s-} \frac{1-q(z)}{1-H(z)} H_n(dz) + \int_{0}^{s-} \frac{(1-q(z))(H_n(z)-H(z)-\frac{2}{n})}{(1-H_n(z)+\frac{2}{n})(1-H(z))} H_n(dz)\mdot \numberthis \label{eq:xksk_int}
		\end{align*}
		%
		Note that the second term on the right hand side of the latter equation above tends to zero for  $n\to\infty$, because
		\begin{align*}
		& \left|\int_{0}^{s-} \frac{(1-q(z))(H_n(z)-H(z)-\frac{2}{n})}{(1-H_n(z)+\frac{2}{n})(1-H(z))} H_n(dz)\right|\\
		&\leq \frac{\sup_{z}|H_n(z)- H(z) -\frac{2}{n}|}{1-H(s)} \int_{0}^{s-}\frac{1}{1-H_n(z)} H_n(dz) \longrightarrow 0
		\end{align*}
		%
		almost surely as $n\to\infty$, by the Glivenko-Cantelli Theorem and since $H(s)<1$. Moreover we have
		\begin{align*}
		\int_{0}^{s-} \frac{1-q(z)}{1-H(z)} H_n(dz) \longrightarrow \int_{0}^{s} \frac{1-q(z)}{1-H(z)} H(dz)
		\end{align*}		
		by the SLLN. Therefore we obtain 
		$$\lim\limits_{n\to\infty} \sum_{k=1}^{n} x_k s_k = \int_{0}^{s} \frac{1-q(z)}{1-H(z)} H(dz)\mdot$$
		By the same arguments, we can show that 
		$$\lim\limits_{n\to\infty} \sum_{k=1}^{n} y_k t_k = \int_{s}^{t} \frac{1-q(z)}{1-H(z)} H(dz)\mdot$$
		Thus we finally conclude
		$$\lim\limits_{n\to\infty} D_n(s,t) = \exp\left(2\int_{0}^{s} \frac{1-q(z)}{1-H(z)} H(dz) + \int_{s}^{t} \frac{1-q(z)}{1-H(z)} H(dz)\right)$$
		almost surely.
	\end{proof}
\end{lemma}

\begin{remark}
	The measure zero set $\{\omega| D_n(s,t;\omega) \to D(s,t) \textrm{ as } n\to\infty\}$ is independent of $s$ and $t$.
\end{remark}
%
\begin{lemma} \label{lem:dn_supermart}
	$\{D_n, \F_n\}_{n\geq 1}$ is a non-negative reverse supermartingale.
	%
	\begin{proof}
		Consider that for $s<t$ and $n\geq 1$
		\begin{align*}
		\E[D_n(s,t)| \F_{n+1}] &= \E\left[\prod_{k=1}^{n}\left(1+\frac{1-q(Z_{k:n})}{n-k+2}\right)^{2\I{Z_{k:n} <s}}\right.\\
		&\qquad \left. \times \prod_{k=1}^{n}\left(1+\frac{1-q(Z_{k:n})}{n-k+1}\right)^\I{s < Z_{k:n} < t} | \F_{n+1}\right]\\
		&= \sum_{i=1}^{n+1}\E\left[\I{Z_{n+1} = Z_{i:n+1}} \prod_{k=1}^{n}\dots | \F_{n+1}\right]\\
		&= \sum_{i=1}^{n+1}\E\left[\I{Z_{n+1} = Z_{i:n+1}} \prod_{k=1}^{i-1}\left(1+\frac{1-q(Z_{k:n+1})}{n-k+2}\right)^{2\I{Z_{k:n+1} <s}} \right.\\
		&\qquad\qquad \times \prod_{k=i}^{n}\left(1+\frac{1-q(Z_{k+1:n+1})}{n-k+2}\right)^{2\I{Z_{k+1:n+1} <s}}\\
		&\qquad\qquad \times \prod_{k=1}^{i-1}\left(1+\frac{1-q(Z_{k:n+1})}{n-k+1}\right)^\I{s < Z_{k:n+1} < t}\\
		&\qquad\qquad \left. \times \prod_{k=i}^{n}\left(1+\frac{1-q(Z_{k+1:n+1})}{n-k+1}\right)^\I{s < Z_{k+1:n+1} < t}| \F_{n+1}\right]\\
		&= \sum_{i=1}^{n+1}\E\left[\I{Z_{n+1} = Z_{i:n+1}} \prod_{k=1}^{i-1}\left(1+\frac{1-q(Z_{k:n+1})}{n-k+2}\right)^{2\I{Z_{k:n+1} <s}} \right.\\
		&\qquad\qquad \times \prod_{k=i+1}^{n+1}\left(1+\frac{1-q(Z_{k:n+1})}{n-k+3}\right)^{2\I{Z_{k:n+1} <s}}\\
		&\qquad\qquad \times \prod_{k=1}^{i-1}\left(1+\frac{1-q(Z_{k:n+1})}{n-k+1}\right)^\I{s < Z_{k:n+1} < t}\\
		&\qquad\qquad \left. \times \prod_{k=i+1}^{n+1}\left(1+\frac{1-q(Z_{k:n+1})}{n-k+2}\right)^\I{s < Z_{k:n+1} < t}| \F_{n+1}\right]\mdot
		\end{align*}
		%
		Now each product within the conditional expectation is measurable \wrt\ $\F_{n+1}$. Moreover we have for $i=1,\dots,n$ 
		\begin{align*}
		\E[\I{Z_{n+1}=Z_{i:n+1}}|\F_{n+1}] &= \P(Z_{n+1}=Z_{i:n+1}|\F_{n+1})\\
		&= \P(R_{n+1,n+1} = i)\\
		&= \frac{1}{n+1}\mdot
		\end{align*}
		%
		Thus we obtain
		\begin{align*}
		\E[D_n(s,t)| \F_{n+1}] &= \frac{1}{n+1} \sum_{i=1}^{n+1} \prod_{k=1}^{i-1}\left(1+\frac{1-q(Z_{k:n+1})}{n-k+2}\right)^{2\I{Z_{k:n+1} <s}}\\
		&\qquad\qquad\qquad \times \left(1+\frac{1-q(Z_{k:n+1})}{n-k+1}\right)^\I{s < Z_{k:n+1} < t}\\
		&\qquad\qquad \times \prod_{k=i+1}^{n+1}\left(1+\frac{1-q(Z_{k:n+1})}{n-k+3}\right)^{2\I{Z_{k:n+1} <s}}\\ &\qquad\qquad\qquad \times \left(1+\frac{1-q(Z_{k:n+1})}{n-k+2}\right)^\I{s < Z_{k:n+1} < t}\mdot \numberthis \label{eq:cond_exp_dnp1}
		\end{align*}
		%
		We will now proceed by induction on $n$. First let 
		$$x_k := 1-q(Z_{k:2}) \textrm{, } s_k := \I{Z_{k:2} < s} \textrm{ and } t_k := \I{s < Z_{k:2} < t}$$
		for $k=1,2$. Note that that $x_k$ and $y_k$ are different, compared to the corresponding definitions in lemma \ref{lem:dn_limit}, as they involves the ordered $Z$-values here. 
		%
		Next consider
		\begin{align*}
		\E[D_1(s,t) | \F_2] &= \frac{1}{2}\left[\left(1+\frac{1-q(Z_{2:2})}{2}\right)^{2\I{Z_{2:2}<s}} \times \left(1+(1-q(Z_{2:2}))\right)^{\I{s < Z_{2:2} < t}} \right.\\
		&\qquad \left. + \left(1+\frac{1-q(Z_{1:2})}{2}\right)^{2\I{Z_{1:2}<s}} \times \left(1+(1-q(Z_{1:2}))\right)^{\I{s < Z_{1:2} < t}}\right]\\
		&= \frac{1}{2}\left[\left(1+\frac{x_2}{2}s_2\right)^{2} \times \left(1+x_2t_2\right) + \left(1+\frac{x_1}{2}s_1\right)^{2} \times \left(1+x_1t_1\right)\right]\mdot
		\end{align*}
		%
		Moreover we have
		\begin{align*}
		D_2(s,t) &= \prod_{k=1}^{2} \left[1+\frac{1-q(Z_{k:2})}{4-k}\right]^{2\I{Z_{k:2}<s}} \prod_{k=1}^{2}\left[1+\frac{1-q(Z_{k:2})}{3-k}\right]^{\I{s < Z_{k:2} < t}}\\
		&= \left[1 + \frac{x_1}{3}s_1\right]^2 \times \left[1+\frac{x_1}{2}t_1\right] \times \left[1+\frac{x_2}{2}s_2\right]^2 \times \left[1+x_2t_2\right]\\
		&= \left[1 + \frac{x_1}{2}t_1 + \left(\frac{x_1^2}{9} + \frac{2}{3}x_1\right)s_1\right] \times \left[1 + x_2t_2 + \left(\frac{x_2^2}{4} + x_2\right)s_2\right]\mdot
		\end{align*}		
		%
		Therefore we obtain 
		\begin{align*}
		\E[D_1(s,t) | \F_2] - D_2(s,t) \leq \frac{x_1^2}{72} - \frac{x_1}{6} \leq 0\mdot 
		\end{align*}
		since $0 \leq x_1 \leq 1$. Thus $\E[D_1(s,t) | \F_2] \leq D_2(s,t)$ for any $s<t$, as needed. Now assume that 
		$$\E[D_n(s,t) | \F_{n+1}] \leq D_{n+1}(s,t)$$
		holds for any $n\geq 1$. 
		%
		Note that the latter is equivalent to assuming
		\begin{align*}
		& \frac{1}{n+1} \sum_{i=1}^{n+1} \prod_{k=1}^{i-1}\left(1+\frac{1-q(y_k)}{n-k+2}\right)^{2\I{y_k <s}}  \left(1+\frac{1-q(y_k)}{n-k+1}\right)^\I{s < y_k < t}\\
		&\qquad\qquad \times \prod_{k=i+1}^{n+1}\left(1+\frac{1-q(y_k)}{n-k+3}\right)^{2\I{y_k <s}} \left(1+\frac{1-q(y_k)}{n-k+2}\right)^\I{s < y_k < t}\\
		&\leq \prod_{k=1}^{n+1}\left(1+\frac{1-q(y_k)}{n-k+3}\right)^{2\I{y_k <s}} \prod_{k=1}^{n+1}\left(1+\frac{1-q(y_k)}{n-k+2}\right)^\I{s < y_k < t} \numberthis \label{eq:supermart_yk}
		\end{align*}
		holds for arbitrary $y_k \geq 0$. Next define for $s<t$ and $n\geq 1$
		$$A_{n+2}(s,t) := \prod_{k=2}^{n+2} \left[1+\frac{1-q(Z_{k:n+2})}{n-k+4}\right]^{2\I{Z_{k:n+2} < s}} \times \left[1+\frac{1-q(Z_{k:n+2})}{n-k+3}\right]^\I{s < Z_{k:n+2} < t}\mdot $$
		%
		Now consider that we get from \eqref{eq:cond_exp_dnp1}
		\begin{align*}
		&\E[D_{n+1}(s,t)| \F_{n+2}]	\\
		&= \frac{1}{n+2} \sum_{i=1}^{n+2} \prod_{k=1}^{i-1}\left(1+\frac{1-q(Z_{k:n+2})}{n-k+3}\right)^{2\I{Z_{k:n+2} <s}}  \left(1+\frac{1-q(Z_{k:n+2})}{n-k+2}\right)^\I{s < Z_{k:n+2} < t}\\
		&\qquad\qquad\quad \times \prod_{k=i+1}^{n+2}\left(1+\frac{1-q(Z_{k:n+2})}{n-k+4}\right)^{2\I{Z_{k:n+2} <s}} \left(1+\frac{1-q(Z_{k:n+2})}{n-k+3}\right)^\I{s < Z_{k:n+2} < t}\\
		&= \frac{A_{n+2}}{n+2} + \frac{1}{n+2}\sum_{i=2}^{n+2} \prod_{k=1}^{i-1} \dots \times \prod_{k=i+1}^{n+2}\dots \\
		&= \frac{A_{n+2}}{n+2} + \frac{1}{n+2}\sum_{i=1}^{n+1} \prod_{k=1}^{i} \dots \times \prod_{k=i+2}^{n+2}\dots\\
		&= \frac{A_{n+2}}{n+2} + \frac{1}{n+2}\left(1+\frac{1-q(Z_{1:n+2})}{n+2}\right)^{2\I{Z_{1:n+2} < s}} \left(1+\frac{1-q(Z_{1:n+2})}{n+1}\right)^\I{s < Z_{1:n+2} < t}\\
		&\qquad\qquad\qquad\times \sum_{i=1}^{n+1} \prod_{k=1}^{i-1} \left(1+\frac{1-q(Z_{k+1:n+2})}{n-k+2}\right)^{2\I{Z_{k+1:n+2} <s}}\\
		&\qquad\qquad\qquad\qquad\qquad \times \left(1+\frac{1-q(Z_{k+1:n+2})}{n-k+1}\right)^\I{s < Z_{k+1:n+2} < t}\\
		&\qquad\qquad\qquad\qquad \times \prod_{k=i+1}^{n+1}\left(1+\frac{1-q(Z_{k+1:n+2})}{n-k+3}\right)^{2\I{Z_{k+1:n+2} <s}}\\
		&\qquad\qquad\qquad\qquad\qquad \times \left(1+\frac{1-q(Z_{k+1:n+2})}{n-k+2}\right)^\I{s < Z_{k+1:n+2} < t} \mdot 
		\end{align*}
		%
		Using \eqref{eq:supermart_yk} on the right hand side of the equation above yields
		\begin{align*}
		&\E[D_{n+1}(s,t)| \F_{n+2}]	\\
		&\leq \frac{A_{n+2}}{n+2} + \frac{n+1}{n+2}\left(1+\frac{1-q(Z_{1:n+2})}{n+2}\right)^{2\I{Z_{1:n+2} < s}} \left(1+\frac{1-q(Z_{1:n+2})}{n+1}\right)^\I{s < Z_{1:n+2} < t}\\
		&\qquad\qquad\qquad\times \prod_{k=1}^{n+1} \left(1+\frac{1-q(Z_{k+1:n+2})}{n-k+3}\right)^{2\I{Z_{k+1:n+2} <s}}\\
		&\qquad\qquad\qquad\qquad\qquad \times \left(1+\frac{1-q(Z_{k+1:n+2})}{n-k+2}\right)^\I{s < Z_{k+1:n+2} < t}\\
		&= A_{n+2} \left[\frac{1}{n+2} + \frac{n+1}{n+2}\left(1+\frac{1-q(Z_{1:n+2})}{n+2}\right)^{2\I{Z_{1:n+2} < s}} \right. \\
		&\qquad\qquad\qquad\qquad\qquad \left. \times \left(1+\frac{1-q(Z_{1:n+2})}{n+1}\right)^\I{s < Z_{1:n+2} < t}\right] \mdot 
		\end{align*}
		%
		For the moment, let
		$$x_1 := 1-q(Z_{1:n+2}) \textrm{, } s_1 := \I{Z_{1:n+2} < s} \textrm{ and } t_1 := \I{s < Z_{1:n+2} < t} $$
		%
		Now we can rewrite the above as
		\begin{align*}
		\E[D_{n+1}(s,t)| \F_{n+2}]	&\leq A_{n+2} \left[\frac{1}{n+2} + \frac{n+1}{n+2}\left(1+\frac{x_1s_1}{n+2}\right)^{2} \left(1+\frac{x_1t_1}{n+1}\right)\right]\mdot  \numberthis\label{eq:dn_supermart_an}
		\end{align*}
		%
		Next consider 
		\begin{align*}
		\left(1+\frac{x_1t_1}{n+1}\right) &= \left(1+\frac{x_1t_1}{n+2}-\frac{1}{n+2}\right) \left(1+\frac{1}{n+1}\right)\\
		&=  \left(1+\frac{x_1t_1}{n+2}\right)+\frac{1}{n+1}\left(1+\frac{x_1t_1}{(n+2)}\right) - \frac{1}{n+1}\\
		&= \left(1+\frac{x_1t_1}{n+2}\right)+\frac{x_1t_1}{(n+1)(n+2)}\mdot 
		\end{align*}
		%
		Thus we get
		\begin{align*}
		&\frac{n+1}{n+2}\left(1+\frac{x_1s_1}{n+2}\right)^{2} \left(1+\frac{x_1t_1}{n+1}\right) \\
		&= \frac{n+1}{n+2}\left(1+\frac{x_1s_1}{n+2}\right)^{2}\left(1+\frac{x_1t_1}{n+2}\right) + \left(1+\frac{x_1s_1}{n+2}\right)^{2}\frac{x_1t_1}{(n+2)^2}\mdot 
		\end{align*}
		%
		But now 
		\begin{align*}
		\left(1+\frac{x_1s_1}{n+2}\right)^{2}\frac{x_1t_1}{(n+2)^2} &= \left(1+2\frac{x_1s_1}{n+2}+\frac{x^2_1s_1}{(n+2)^2}\right)\frac{x_1t_1}{(n+2)^2}\\
		&= \frac{x_1t_1}{(n+2)^2}
		\end{align*}
		since $s_1\cdot t_1=0$ for all $s<t$. Hence we can rewrite the term in brackets in \eqref{eq:dn_supermart_an} as 
		\begin{align*}
		&\frac{1}{n+2} + \frac{n+1}{n+2}\left(1+\frac{x_1s_1}{n+2}\right)^{2} \left(1+\frac{x_1t_1}{n+1}\right) \\
		&=\frac{1}{n+2} + \frac{x_1t_1}{(n+2)^2} + \frac{n+1}{n+2}\left(1+\frac{x_1s_1}{n+2}\right)^{2}\left(1+\frac{x_1t_1}{n+2}\right)\\
		&=\frac{1}{n+2}\left(1+\frac{x_1t_1}{n+2}\right) + \frac{n+1}{n+2}\left(1+\frac{x_1s_1}{n+2}\right)^{2}\left(1+\frac{x_1t_1}{n+2}\right)\\
		&=\left[\frac{1}{n+2} + \frac{n+1}{n+2}\left(1+\frac{x_1}{n+2}\right)^{2s_1}\right]\left(1+\frac{x_1}{n+2}\right)^{t_1}\\
		&\leq \left(1+\frac{x_1}{n+3}\right)^{2s_1}\left(1+\frac{x_1}{n+2}\right)^{t_1}\mdot 
		\end{align*}
		The latter inequality above holds, since 
		$$\left[\frac{1}{n+2} + \frac{n+1}{n+2}\left(1+\frac{x}{n+2}\right)^{2}\right] \leq \left(1+\frac{x}{n+3}\right)^{2}$$
		for any $0\leq x\leq 1$. (\cf\ \cite{bose1999strong}, page 197). Therefore we can rewrite \eqref{eq:dn_supermart_an} as
		\begin{align*}
		\E[D_{n+1}(s,t)| \F_{n+2}]	&\leq A_{n+2} \left(1+\frac{1-q(Z_{1:n+2})}{n+3}\right)^{2\I{Z_{1:n+2}<s}} \\
		&\qquad\quad \times \left(1+\frac{1-q(Z_{1:n+2})}{n+2}\right)^{\I{s<Z_{1:n+2} <t}}\\
		&= D_{n+2}(s,t)\mdot
		\end{align*}		
		This concludes the proof.
	\end{proof}
\end{lemma}
%
\begin{lemma}Let $s<t$ \st\ $H(t)<1$. Then $\Delta_{n}(s,t) \nearrow D(s,t)$.
	\label{lem:neveu}
	\begin{proof}
		Consider that we have for $n\geq 2$
		$$\Delta_{n}(s,t) = \E[D_n(s,t)] = \E[D_n(s,t)|\F_\infty] $$
		by definition of $\Delta_{n}(s,t)$ and Lemma \ref{lem:hewitt_savage}. Next note that we have $D_n(s,t) \to D(s,t)$ almost surely, according to Lemma \ref{lem:dn_limit}. Moreover we get from Lemma \ref{lem:dn_supermart}, that $\{D_n,\F_n\}_{n\geq 1}$ is a reverse supermartingale. Now this together with Proposition 5-3-11 of \cite{neveu1975discrete} yields
		$$\E[D_n(s,t)|\F_\infty] \nearrow D(s,t)\mdot$$
		This proves the lemma.
	\end{proof}
\end{lemma}
%
We will now proceed to find an explicit representation for $\E[S_n]$ in terms of the reverse supermartingale $D_n$ (see Section \ref{sec:dn}) to identify the limit $S = S(q)$. Consider the following lemma.
\begin{lemma}
	For continuous $H(\cdot)$, we have
	$$\E[S_n(q)] = \frac{n-1}{n} \E[\phi(Z_1,Z_2)q(Z_1)q(Z_2)\{\Delta_{n-2}(Z_1, Z_2) + \bar\Delta_{n-2}(Z_1, Z_2)\} \I{Z_1<Z_2}]\mdot$$
	\label{lem:Sn_Delta}
	
	\begin{proof}
		The proof of the lemma above is similar to the proof of Lemma \ref{lem:expectation_sq}. Consider
		\begin{align*}
			\E[S_n(q)] &= \doublesum\limits_{1\leq i<j\leq n}\E\left[\phi(Z_{i:n}, Z_{j:n}) \frac{q(Z_{i:n})}{n-i+1)}\prod_{k=1}^{i-1}\left[1-\frac{q(Z_{k:n})}{n-k+1}\right]\right.\\
			&\qquad\qquad\qquad \times \left. \frac{q(Z_{j:n})}{n-j+1}\prod_{l=1}^{j-1}\left[1-\frac{q(Z_{l:n})}{n-l+1}\right]\right]\\
			&= \frac{1}{n^2}\doublesum\limits_{1\leq i<j\leq n}\E\left[\phi(Z_{i:n}, Z_{j:n}) q(Z_{i:n})\prod_{k=1}^{i-1}\left[1+\frac{1-q(Z_{k:n})}{n-k+1}\right]\right.\\
			&\qquad\qquad\qquad \times \left. q(Z_{j:n})\prod_{l=1}^{j-1}\left[1+\frac{1-q(Z_{l:n})}{n-l+1}\right]\right]\\
			&= \frac{1}{n^2}\doublesum\limits_{1\leq i<j\leq n}\E\left[\phi(Z_{i:n}, Z_{j:n}) q(Z_{i:n})q(Z_{j:n})B_n(Z_{i:n})B_n(Z_{j:n})\right]\\
			&= \frac{1}{n^2}\sum_{i=1}^{n}\sum_{j=1}^{n}\I{R_{i,n} < R_{j,n}}\E\left[\phi(Z_{i}, Z_{j}) q(Z_{i})q(Z_{j})B_n(Z_{i})B_n(Z_{j})\right]\mdot
			\numberthis\label{eq:E_n}
		\end{align*}
		%
		According to Lemma \ref{lem:zizone} we obtain
		\begin{align*}
			\E[S_n(q)] &= \frac{n-1}{2n}\E\left[\phi(Z_{1}, Z_{2}) q(Z_{1})q(Z_{2})B_n(Z_{1})B_n(Z_{2})\right]\mdot
		\end{align*}	
		Now, since $\phi$ and $q$ are measurable, we can apply Lemma \ref{lem:representation_bn} to obtain the result.		
	\end{proof}
\end{lemma}
%
The following result is necessary for the proof of Lemma \ref{lem:sn_limit}.
\begin{lemma}
	For continuous $H(\cdot)$ and $t<s$, we have $C_n(t) \to 0$ as $n \to \infty$ \wpo, and $C_n(t) \in [0,1]$ for all $n\geq 1$ and $t\geq 0$.
	\label{lem:Cn_bounds_and_limit}
	%
	\begin{proof}
		It is easy to see that $0\leq C_n(t) \leq 1$ for any $t\geq 0$ and $n\geq 2$, since $0\leq q(t)\leq 1$ and $\I{Z_{i-1:n} < t \leq Z_{i:n}} = 1$ for exactly one $i \in \{1,\dots,n\}$. Let's now consider 
		\begin{align*}
		C_n(t) &= \sum_{i=1}^{n+1} \frac{1-q(t)}{n-i+2} [\I{Z_{i-1:n}<t} - \I{Z_{i:n}<t}]\\
		&= \sum_{i=1}^{n+1} \frac{1-q(t)}{n-i+2} \I{Z_{i-1:n}<t} - \sum_{i=1}^{n+1} \frac{1-q(t)}{n-i+2} \I{Z_{i:n}<t}\\
		&= \sum_{i=0}^{n} \frac{1-q(t)}{n-i+1} \I{Z_{i:n}<t} - \sum_{i=1}^{n} \frac{1-q(t)}{n-i+2} \I{Z_{i:n}<t}\\
		&= \sum_{i=1}^{n} \frac{1-q(t)}{n-i+1} \I{Z_{i:n}<t} + \frac{(1-q(t))}{n+1}  - \sum_{i=1}^{n} \frac{1-q(t)}{n-i+2} \I{Z_{i:n}<t}\\
		&= (1-q(t))\left\{\frac{1}{n+1} + \sum_{i=1}^{n} \left[\frac{1}{n-i+1} - \frac{1}{n-i+2}\right]\I{Z_{i:n}<t}\right\}\\
		&= (1-q(t))\sum_{i=1}^{n} \left[\frac{1}{n-nH_n(Z_{i:n})+1} \frac{1}{n-nH_n(Z_{i:n})+2}\right]\I{Z_{i:n}<t}\\
		&\qquad + \frac{1-q(t)}{n+1}\\
		&= (1-q(t))\int_{0}^{t} \left[\frac{1}{1-H_n(x)+\frac{1}{n}} - \frac{1}{1-H_n(x)+\frac{2}{n}}\right]H_n(dx)\\
		&\qquad + \frac{1-q(t)}{n+1}\mdot \numberthis \label{eq:cn}
		\end{align*}
		In Lemma \ref{lem:dn_limit} we have seen that
		$$\int_{0}^{t} \frac{1}{1-H_n(x)+\frac{2}{n}} H_n(dx) \to \int_{0}^{t} \frac{1}{1-H(x)} H(dx)\mdot$$
		By the same arguments we obtain 
		$$\int_{0}^{t} \frac{1}{1-H_n(x)+\frac{1}{n}}H_n(dx) \to \int_{0}^{t} \frac{1}{1-H(x)} H(dx)\mdot$$
		Therefore the right hand side of \eqref{eq:cn} converges to zero. 
	\end{proof}
\end{lemma}
%
We will now identify the almost sure limits of $S_n(q)$ and $\bar{S}_n(q)$ in Lemma \ref{lem:sn_limit}. Define for $n\geq 2$
$$\bar{S}_n(q) := \doublesum\limits_{1\leq i < j \leq n} \phi(Z_{i:n},Z_{j:n}) \bar{W}_{i:n}(q) \bar{W}_{j:n}(q)$$
where 
$$\bar{W}_{i:n}(q) := \prod_{k=1}^{n}\left(1-\frac{q(Z_{k:n})}{n-k+1}\right)\mdot$$
Moreover let
\begin{align*}
	S(q) &:= \frac{1}{2}\int_{0}^{\infty} \int_{0}^{\infty} \phi(s,t) q(s)q(t) \exp\left(\int_{0}^{s} \frac{1-q(x)}{1-H(x)} H(dx)\right)\\
	&\qquad\qquad\qquad \times \exp\left(\int_{0}^{t} \frac{1-q(x)}{1-H(x)} H(dx) \right)H(ds)H(dt)
\end{align*}
and 
\begin{align*}
	\bar{S}(q) &:= \frac{1}{2}\int_{0}^{\infty} \int_{0}^{\infty} \phi(s,t)  \exp\left(\int_{0}^{s} \frac{1-q(x)}{1-H(x)} H(dx)\right)\\
	&\qquad\qquad\qquad \times \exp\left(\int_{0}^{t} \frac{1-q(x)}{1-H(x)} H(dx) \right)H(ds)H(dt)\mdot
\end{align*}

\begin{lemma}
	Let $H$ be continuous and $q$ be increasing. Then the following statements hold
	$$\lim\limits_{n\to\infty} S_n(q) = S(q)$$
	and 
	$$\lim\limits_{n\to\infty} \bar{S}_n(q) = \bar{S}(q)$$
	with probability one, if the limits above exist.
	\label{lem:sn_limit}
	%
	\begin{proof}
		Suppose $H$ is continuous and $q$ is increasing. First consider that $S$ exists almost surely and we have
		$$\lim\limits_{n\to\infty} S_n = \lim\limits_{n\to\infty} \E[S_n] = S$$
		according to Theorem \ref{thm:ex_limit_sn}. Next consider
		\begin{align*}
			\E[S_n(q)] &= \frac{n-1}{n} \E[\phi(Z_1,Z_2)q(Z_1)q(Z_2)\{\Delta_{n-2}(Z_1, Z_2) + \bar\Delta_{n-2}(Z_1, Z_2)\} \I{Z_1<Z_2}]	\\
			&= \frac{n-1}{n} \E[\phi(Z_1,Z_2)q(Z_1)q(Z_2)\Delta_{n-2}(Z_1, Z_2)\I{Z_1<Z_2}]\\
			&\qquad + \frac{n-1}{n} \E[\phi(Z_1,Z_2)q(Z_1)q(Z_2) \bar\Delta_{n-2}(Z_1, Z_2) \I{Z_1<Z_2}]
		\end{align*}
		by Lemma \ref{lem:Sn_Delta}. 
		%
		Note that we get from Lemma \ref{lem:dn_limit} that $D_n(s,t) \to D(s,t)$ \wpo, and according to Lemma \ref{lem:Cn_bounds_and_limit} $C_n(s) \to 0$ \wpo. Thus $\bar{\Delta}_n(s,t) \to 0$ as $n\to\infty$ for each $s<t$.
		%
		Moreover, the fact that $0\leq C_n(s)\leq 1$ and Lemma \ref{lem:neveu} imply that $\bar{\Delta}_n(s,t) \leq \Delta_n(s,t) \leq D(s,t)$ for all $n\geq 2$. Now note that $D(s,t)$ is integrable, since on $\{Z_1<Z_2\}$ we have
		\begin{align*}
			\E[D(Z_1,Z_2)] &= \E\left[\int_{0}^{s} \frac{1-q(x)}{1-H(x)} H(dx) + \int_{0}^{t} \frac{1-q(x)}{1-H(x)} H(dx)\right]\\
			&\leq \E\left[\int_{0}^{Z_{n:n}} \frac{1}{1-H(x)} H(dx) + \int_{0}^{Z_{n:n}} \frac{1}{1-H(x)} H(dx)\right]\\
			&\leq \E\left[-2\ln(1-H(Z_{n:n}))\right]\\
			& < \infty\mdot
		\end{align*}
		Therefore applying the Dominated Convergence Theorem yields
		$$\lim\limits_{n\to\infty}\E[2\phi(Z_1,Z_2)q(Z_1)q(Z_2) \bar\Delta_{n-2}(Z_1, Z_2) \I{Z_1<Z_2}] = 0\mdot$$
		%
		Furthermore, according to Lemma \ref{lem:neveu}, we have $\Delta_{n}(s,t) \nearrow D(s,t)$ for $s<t$ and $H(t)<1$. Hence we have 
		\begin{align*}
			&\lim\limits_{n\to\infty}\E[\phi(Z_1,Z_2)q(Z_1)q(Z_2) \Delta_{n-2}(Z_1, Z_2) \I{Z_1<Z_2}] \\
			&= \E[\phi(Z_1,Z_2)q(Z_1)q(Z_2) D(Z_1, Z_2) \I{Z_1<Z_2}]\mdot
		\end{align*}
		%
		Therefore we obtain 
		\begin{align*}
		\lim\limits_{n\to\infty}\E[S_n(q)]&= \E[\phi(Z_1,Z_2)q(Z_1)q(Z_2) D(Z_1, Z_2) \I{Z_1<Z_2}]\\
		&= {\int\int}_{\{Z_1<Z_2\}} \phi(s,t)q(s)\exp\left(\int_{0}^{s} \frac{1-q(z)}{1-H(z)} H(dz)\right) \\
		&\qquad\qquad \times q(t) \exp\left(\int_{0}^{t} \frac{1-q(z)}{1-H(z)} H(dz)\right) H(ds)H(dt)\\
		&= \frac{1}{2}\int_{0}^{\infty}\int_{0}^{\infty}\phi(s,t)q(s) \exp\left(\int_{0}^{s} \frac{1-q(z)}{1-H(z)} H(dz)\right)\\
		&\qquad\qquad \times q(t)\exp\left(\int_{0}^{t} \frac{1-q(z)}{1-H(z)} H(dz)\right) H(ds)H(dt)
		\end{align*}
		almost surely, since $\phi(s,t)q(s)q(t)D(s,t)$ is symmetric by (A\ref{ass:kernel_gen}), and $Z_1$ and $Z_2$ are \iid. This concludes the argument for $S_n$. By similar arguments, we obtain $\bar{S}_n \to \bar{S}$ \wpo. 
	\end{proof}
\end{lemma}
%
%
%
%
\section{On the censoring model $m$}
\todo{Relationship of model $m$  and hazard rates}. However, we can show that common choices for $m$ (\cf\ \todo{cite}) will work within this framework, though some under restrictions. Consider the following examples:
\begin{example}
	Let $\theta=(\theta_1, \theta_2)$. For $\theta_1>0$ and $\theta_2<0$ let
	$$m(t, \theta) := \frac{\theta_1}{\theta_1 + t^\theta_2}\mdot$$
	%
	Clearly $m(t,\theta)$ is increasing in $t$. \todo{More Weibull framework. Conclude that $\theta_2>0$ is restriction.}
\end{example}
%
\begin{example}
	For $\theta>0$ let
	$$m(t, \theta) := \frac{1}{1 + + \exp(-\theta t)}\mdot$$
	%
	The logistic model works without restrictions, since $m(t,\theta)$ is increasing whenever $\theta>0$.
\end{example}
%
\begin{example}
	For $\theta>0$ let
	$$m(t, \theta) := \frac{1}{1 - \exp(-\exp(\theta t))}\mdot$$
	%
	The complementary log-log model is increasing if $\theta>0$.
\end{example}

\section{Calculating the limit} \label{sec:model}

In order to identify the limit of $\snse{2}{n} = S_n(m(\cdot, \hat\theta_n))$ we need the statement of Corollary \ref{lem:sandwich}, which is based upon the following lemma. Define for any $\epsilon >0$ let
$$M_{1,\epsilon}(x) := \max(0, m(x, \theta_0) - \epsilon)) \textrm{ and } M_{2,\epsilon}(x) := \min(1, m(x, \theta_0) + \epsilon))$$
\begin{lemma} 
	Suppose (M\ref{ass:m_consistency}) and (M\ref{ass:m_nbhd}) hold. Then the following statements hold for each $0 < \epsilon \leq 1$ and $n$ large enough
	\begin{enumerate}[(i)]
		\item $M_{1,\epsilon}(x) \leq m(x,\hat\theta_n) \leq M_{2,\epsilon}(x)$
		\item $M_{2,\epsilon}(x)M_{2,\epsilon}(y) - 4\epsilon \leq m(x,\hat\theta_n)m(y,\hat\theta_n) \leq M_{1,\epsilon}(x)M_{1,\epsilon}(y) + 4\epsilon$.
	\end{enumerate}
	\label{lem:Mm}
	%
	\begin{proof}
		First we will introduce some notation. We will write $m_n(x) := m(x,\theta_n)$ and $m(x):= m(x,\theta_0)$. Let's start with part \textit{(i)}. Suppose $M_{1,\epsilon}(x) = 0$, then the condition above is trivially satisfied since $m_n(x) \geq 0$. Now suppose $M_{1,\epsilon}(x) = m(x)-\epsilon$. 
		\begin{align*}
			m_n(x) &= (m_n(x) - m(x)) + m(x)\\
			&\geq m(x) - \abs{m_n(x) - m(x)} \mdot
		\end{align*}
		%
		Now using assumption (M\ref{ass:m_consistency}), we have for $n$ large enough that for some $\epsilon > 0$ $\theta_n \in V(\epsilon, \theta_0)$. Now we get, according to (M\ref{ass:m_nbhd}), that 
		$$\sup_{x\geq 0}\abs{m_n(x) - m(x)} < \epsilon$$
		Therefore we obtain $m_n(x) \geq m(x) - \epsilon= M_{1,\epsilon}(x)$. Let's now consider $M_{2,\epsilon}$. The case $M_{2,\epsilon} = 1$ is trivial again, since $m_n(x) \leq 1$. Now suppose $M_{2,\epsilon} = m(x) + \epsilon$. Then we obtain, for $n$ large enough
		\begin{align*}
			m_n(x) &= (m_n(x) - m(x)) + m(x)\\
			&\leq m(x) + \abs{m_n(x) - m(x)}\\
			&\leq m(x) + \epsilon\\
			&= M_{2,\epsilon}(x)\mdot
		\end{align*}
		%
		This concludes the proof of part \textit{(i)}. Now note that, according to (M\ref{ass:m_consistency})  and (M\ref{ass:m_nbhd}), the following holds for $n$ large enough and $\epsilon>0$
		\begin{align*}
			m_n(x) &= (m_n(x) - m(x)) + m(x)\\
			&\leq \abs{m_n(x) - m (x)} + m(x) \\
			&\leq m(x) + \epsilon\mdot \numberthis\label{eq:mn_m_eps}
		\end{align*}		
		%
		Moreover consider that 
		\begin{align*}
			m_n(x)m_n(y) &= (m_n(x) - m(x))(m_n(y) - m(y))\\
			&\qquad + m(x)m_n(y) + m_n(x)m(y) - m(x)m(y)\\
			&\leq \epsilon^2 + m(x)m_n(y) + m_n(x)m(y) - m(x)m(y)\mdot
		\end{align*}
		%
		Using on the right hand side of the latter inequality \eqref{eq:mn_m_eps} yields
		\begin{align*}
			m_n(x)m_n(y) &\leq \epsilon^2 + m(x)(m(y) + \epsilon) + (m(x) + \epsilon)m(y) - m(x)m(y)\\
			&= m(x)m(y) + \epsilon(m(x) + m(y)) + \epsilon^2\mdot \numberthis \label{eq:mnmn_mm_eps}
		\end{align*}
		%
		Now suppose $M_{1,\epsilon}(x) = 0$ and $M_{1,\epsilon}(y) = 0$ for $x,y \in \R_+$. Then $m(x) \leq \epsilon$ and $m(y) \leq \epsilon$. Hence, using \eqref{eq:mnmn_mm_eps} yields
		\begin{align*}
			m_n(x)m_n(y) &\leq 4\epsilon^2\mdot
		\end{align*}
		%
		Next suppose $M_{1,\epsilon}(x) = 0$ and $M_{1,\epsilon}(y) = m(y) -\epsilon$. Using \eqref{eq:mnmn_mm_eps} again, we obtain
		\begin{align*}
			m_n(x)m_n(y) &\leq m(x)m(y) + \epsilon(m(x) + m(y)) + \epsilon^2\\
			&\leq \epsilon + \epsilon(1+\epsilon) + \epsilon^2\\
			&= 2\epsilon(1+\epsilon)\mcomma
		\end{align*}
		since $m(x)\leq \epsilon$ and $m(y) \leq 1$. 
		%
		By similar calculations, we obtain the exact same result for the case $M_{1,\epsilon}(x) = m(x) -\epsilon$ and $M_{1,\epsilon}(y) = 0$. Now suppose $M_{1,\epsilon}(x) = m(x) -\epsilon$ and $M_{1,\epsilon}(y) = m(y) -\epsilon$ and note that 
		\begin{align*}
			M_{1,\epsilon}(x)M_{1,\epsilon}(y) &= (m(x)-\epsilon)(m(y)-\epsilon)\\
			&= m(x)m(y) - \epsilon(m(x) + m(y)) + \epsilon^2\mdot
		\end{align*}
		%
		Now \eqref{eq:mnmn_mm_eps} implies 
		\begin{align*}
			m_n(x)m_n(y) &\leq m(x)m(y) + \epsilon(m(x) + m(y)) + \epsilon^2\\
			&= M_{1,\epsilon}(x)M_{1,\epsilon}(y) + 2\epsilon(m(x) + m(y))\\
			&\leq M_{1,\epsilon}(x)M_{1,\epsilon}(y) + 4\epsilon\mdot
		\end{align*}
		%
		Thus we have for $0\leq \epsilon\leq 1$ that 
		$$m_n(x)m_n(y) \leq M_{1,\epsilon}(x)M_{1,\epsilon}(y) + 4\epsilon$$
		as claimed in the statement of this lemma. It remains to show that $M_{2,\epsilon}(x)M_{2,\epsilon}(y) - 4\epsilon \leq m_n(x)m_n(y)$. By calculations similar to those, that lead to \eqref{eq:mn_m_eps} and \eqref{eq:mnmn_mm_eps} we obtain
		$$m_n(x) \geq m(x) - \epsilon$$
		and
		\begin{equation}
			m_n(x)m_n(y) \geq m(x)m(y) - \epsilon(m(x) + m(y)) - \epsilon^2 \mdot \label{eq:mnmn_mm_eps_a}
		\end{equation}
		%
		Now we will continue and look at $M_{2,\epsilon}$ case by case. Suppose $M_{2,\epsilon}(x) = 1$ and $M_{2,\epsilon}(y) = 1$. This is equivalent to $m(x) \geq 1 - \epsilon$ and $m(y) \geq 1 - \epsilon$. Therefore \eqref{eq:mnmn_mm_eps_a} implies 
		\begin{align*}
			m_n(x)m_n(y) &\geq (1-\epsilon)^2 - 2\epsilon - \epsilon^2\\
			&= 1 - 4\epsilon\\
			&= M_{2,\epsilon}(x)M_{2,\epsilon}(y) - 4\epsilon\mdot
		\end{align*}
		%
		Next consider the case $M_{2,\epsilon}(x) = 1$ and $M_{2,\epsilon}(y) = m(y) + \epsilon$. Then we have $m(x) \geq 1-\epsilon$ and  $m(y) \leq 1-\epsilon$. Moreover we have $M_{2,\epsilon}(x)M_{2,\epsilon}(y) = m(y) + \epsilon$. Hence we obtain the following, according to \eqref{eq:mnmn_mm_eps_a}
		\begin{align*}
			m_n(x)m_n(y) &\geq (1-\epsilon)m(y) - \epsilon((1 + (1-\epsilon)) - \epsilon^2\\
			&= m(y)-\epsilon m(y) - 2\epsilon\\
			&\geq m(y)-\epsilon (1-\epsilon) - 2\epsilon\\
			&\geq m(y)-3\epsilon \\
			&= M_{2,\epsilon}(x)M_{2,\epsilon}(y) - 4\epsilon\mdot
		\end{align*} 
		%
		By similar calculations we obtain the same result, if $M_{2,\epsilon}(x) = m(x) + \epsilon$ and $M_{2,\epsilon}(y) = 1$. Finally consider the case $M_{2,\epsilon}(x) = m(x) + \epsilon$ and $M_{2,\epsilon}(y) = m(y) + \epsilon$. Then we have $m(x) \geq 1-\epsilon$ and $m(y) \geq 1-\epsilon$. Furthermore we have 
		\begin{align*}
			M_{2,\epsilon}(x)M_{2,\epsilon}(y) &= (m(x) +\epsilon)(m(y) + \epsilon)\\
			&= m(x)m(y) + \epsilon(m(x) + m(y)) + \epsilon^2\mdot
		\end{align*}
		%
		Therefore, using \eqref{eq:mnmn_mm_eps_a} again, yields
		\begin{align*}
			m_n(x)m_n(y) &\geq m(x)m(y) - \epsilon(m(x) + m(y)) - \epsilon^2\\
			&= M_{2,\epsilon}(x) M_{2,\epsilon}(y) - 2\epsilon(m(x) + m(y)) - 2\epsilon^2\\
			&= M_{2,\epsilon}(x) M_{2,\epsilon}(y) - 4\epsilon(1-\epsilon) - 2\epsilon^2\\
			&\geq M_{2,\epsilon}(x) M_{2,\epsilon}(y) - 4\epsilon \mdot
		\end{align*}
		%
		This concludes the proof.
	\end{proof}
\end{lemma}

\begin{cor}
	Suppose (M\ref{ass:m_consistency}) and (M\ref{ass:m_nbhd}) hold and $H$ is continuous. Then we have for each $0 < \epsilon \leq 1$ and $n$ large enough
	$$S_n(M_{2,\epsilon}) - 4\epsilon \bar{S}_n(M_{2,\epsilon}) \leq S_n(m(\cdot, \hat\theta_n)) \leq S_n(M_{1,\epsilon}) + 4\epsilon \bar{S}_n(M_{1,\epsilon})\textrm{.}$$
	\label{lem:sandwich}
	%
	\begin{proof}
		Consider that we have the following for any $n\geq 1$
		\begin{align*}
			S_n(M_{2,\epsilon}) - 4\epsilon \bar{S}_n(M_{2,\epsilon}) &= \doublesum\limits_{1\leq i<j\leq n} \phi(Z_{i:n}, Z_{j:n}) (M_{2,\epsilon}(Z_{i:n}) M_{2,\epsilon}(Z_{j:n}) - 4\epsilon) \\
			&\qquad \times \prod_{k=1}^{i-1}\left[1-\frac{M_{2,\epsilon}(Z_{k:n})}{n-k+1}\right] \prod_{k=1}^{j-1}\left[1-\frac{M_{2,\epsilon}(Z_{k:n})}{n-k+1}\right]\mdot
		\end{align*}
		But according to Lemma \ref{lem:Mm} we have 
		$$m(x,\hat\theta_n) \leq M_{2,\epsilon}(x) \textrm{ and } M_{2,\epsilon}(x)M_{2,\epsilon}(y) \leq m(x,\hat\theta_n)m(y,\hat\theta_n)$$
		for all $x,y \in \R_+$. Hence we obtain
		\begin{align*}
		S_n(M_{2,\epsilon}) - 4\epsilon \bar{S}_n(M_{2,\epsilon}) &\leq \doublesum\limits_{1\leq i<j\leq n} \phi(Z_{i:n}, Z_{j:n}) m(Z_{i:n},\hat\theta_n)m(Z_{j:n},\hat\theta_n) \\
		&\qquad \times \prod_{k=1}^{i-1}\left[1-\frac{m(Z_{k:n},\hat\theta_n)}{n-k+1}\right] \prod_{k=1}^{j-1}\left[1-\frac{m(Z_{k:n},\hat\theta_n)}{n-k+1}\right]\\
		&= S_n(m(\cdot, \hat\theta_n))\textrm{.}
		\end{align*}
		%
		Similarly we obtain 
		$$S_n(M_{1,\epsilon}) + 4\epsilon \bar{S}_n(M_{1,\epsilon}) \geq S_n(m(\cdot, \hat\theta_n)) \textrm{.}$$
	\end{proof}
\end{cor}

Now we are in a position, to identify $S = \lim_{n\to\infty} \snse{2}{n}$. The following theorem gives the main statement of this thesis.
\begin{thm}
	Suppose (A\ref{ass:kernel_gen}) through (A\ref{ass:m_increas}), (M\ref{ass:m_consistency}) and (M\ref{ass:m_nbhd}) hold. Then we have
	$$\lim\limits_{n\to\infty} S_n(m(\cdot, \hat{\theta}_n)) = \frac{1}{2}\int_{0}^{\tau_H}\int_{0}^{\tau_H} \phi(s,t) F(ds)F(dt)\mdot$$
	\label{thm:snmn_limit}
	\begin{proof}
		Consider that we have
		$$S_n(M_{2,\epsilon}) - 4\epsilon \bar{S}_n(M_{2,\epsilon}) \leq S_n(m(\cdot, \hat\theta_n)) \leq S_n(M_{1,\epsilon}) + 4\epsilon \bar{S}_n(M_{1,\epsilon})$$
		by Corollary \ref{lem:sandwich} under (M\ref{ass:m_consistency}) and (M\ref{ass:m_nbhd}). Next take note of the Radon-Nikodym derivatives (\cf \cite{dikta2000strong}, page 8)
		$$m(s,\theta_0) = \frac{H^1(ds)}{H(ds)} \textrm{ and } (1-G(s)) = \frac{H^1(ds)}{F(ds)}\mdot$$
		Moreover consider that we have
		$$\int_{0}^{s} \frac{1-m(x,\theta_0)}{1-H(x)} = -\ln(1-G(s))$$
		and 
		$$\int_{0}^{s} \frac{\epsilon}{1-H(x)} = -\ln((1-H(s))^\epsilon)$$
		according to \cite{dikta2000strong}.
		%
		Consider that we have 
		\begin{align*}
			M_{1,\epsilon}(x) &= \I{ m(x,\theta_0)>\epsilon}(m(x,\theta_0)-\epsilon)\\
			&\leq m(x,\theta_0)-\epsilon \mdot
		\end{align*}
		Therefore, we obtain
		\begin{align*}
			\bar{S}(M_{1,\epsilon}) &\leq \frac{1}{2}\int_{0}^{\infty} \int_{0}^{\infty} \phi(s,t)  \exp\left(\int_{0}^{s} \frac{1-m(x,\theta_0)}{1-H(x)} + \frac{\epsilon}{1-H(x)} H(dx)\right)\\
			&\qquad\qquad\qquad \times \exp\left(\int_{0}^{t} \frac{1-m(x,\theta_0)}{1-H(x)} + \frac{\epsilon}{1-H(x)} H(dx) \right)H(ds)H(dt)\\
			&= \frac{1}{2}\int_{0}^{\infty} \int_{0}^{\infty} \frac{\phi(s,t)}{(1-G(s))(1-G(t))(1-H(s))^\epsilon(1-H(t))^\epsilon} H(ds)H(dt)\\
			&= \frac{1}{2}\int_{0}^{\infty} \int_{0}^{\infty} \frac{\phi(s,t)}{m(s,\theta_0)m(t,\theta_0)(1-H(s))^\epsilon(1-H(t))^\epsilon} F(ds)F(dt)\mdot
		\end{align*}
		%
		But by condition (A\ref{ass:intgral_phi_q}), the integral above is finite. Moreover $M_{1,\epsilon}(x)$ is increasing in $x$, since $m$ is increasing under (A\ref{ass:m_increas}). Therefore $S(M_{1,\epsilon})$ exists almost surely under (A\ref{ass:kernel_gen}) through (A\ref{ass:m_increas}), by Theorem \ref{thm:ex_limit_sn}. Hence we have that for each $0<\epsilon\leq1$ we have $S_n(M_{1,\epsilon}) + 4\epsilon \bar{S}_n(M_{1,\epsilon}) \to S(M_{1,\epsilon}) + 4\epsilon \bar{S}(M_{1,\epsilon})$ \wpo\ as $n\to\infty$, according to Lemma \ref{lem:sn_limit} . 
		%
		Next consider that 
		\begin{align*}
			S(M_{1,\epsilon}) + 4\epsilon \bar{S}(M_{1,\epsilon}) &\leq \frac{1}{2}\int_{0}^{\infty} \int_{0}^{\infty} \frac{\phi(s,t)}{(1-H(s))^\epsilon(1-H(t))^\epsilon}\\
			&\qquad\qquad \times \frac{m(s,\theta_0)m(t,\theta_0) + 4\epsilon}{(1-G(s))(1-G(t))} H(ds)H(dt)\mdot
		\end{align*}
		%
		By similar arguments we can show that $S_n(M_{2,\epsilon}) - 4\epsilon \bar{S}_n(M_{2,\epsilon}) \to S(M_{2,\epsilon}) - 4\epsilon \bar{S}(M_{2,\epsilon})$ \wpo\ as $n\to\infty$ and 
		\begin{align*}
			S(M_{2,\epsilon}) - 4\epsilon \bar{S}(M_{2,\epsilon}) &\geq \frac{1}{2}\int_{0}^{\infty} \int_{0}^{\infty} \phi(s,t)(1-H(s))^\epsilon(1-H(t))^\epsilon\\
			&\qquad\qquad \times \frac{m(s,\theta_0)m(t,\theta_0) - 4\epsilon}{(1-G(s))(1-G(t))} H(ds)H(dt)\mdot
		\end{align*}
		%
		Let's summarize the above. So far we have shown, that for $0<\epsilon\leq 1$ small enough
		\begin{align*}
			&\frac{1}{2}\int_{0}^{\infty} \int_{0}^{\infty} \phi(s,t)(1-H(s))^\epsilon(1-H(t))^\epsilon\\
			&\qquad\qquad \times \frac{m(s,\theta_0)m(t,\theta_0) - 4\epsilon}{(1-G(s))(1-G(t))} H(ds)H(dt)\\
			&\leq \liminf_{n\to\infty} \int_{0}^{\infty} \phi F_n^{se} F_n^{se}\\
			&\leq \limsup_{n\to\infty} \int_{0}^{\infty} \phi F_n^{se} F_n^{se}\\
			&\leq \frac{1}{2}\int_{0}^{\infty} \int_{0}^{\infty} \frac{\phi(s,t)}{(1-H(s))^\epsilon(1-H(t))^\epsilon}\\
			&\qquad\qquad \times \frac{m(s,\theta_0)m(t,\theta_0) + 4\epsilon}{(1-G(s))(1-G(t))} H(ds)H(dt)\mdot
		\end{align*}
		Finally let $\epsilon \searrow 0$ and apply the Monotone Convergence Theorem to obtain that the upper and lower bound converge both to the same limit. In effect
		\begin{align*}
			&\lim\limits_{\epsilon\searrow 0}\frac{1}{2}\int_{0}^{\infty} \int_{0}^{\infty} \frac{\phi(s,t)(1-H(s))^\epsilon(1-H(t))^\epsilon}{(1-G(s))(1-G(t))} H(ds)H(dt)\\
			&= \frac{1}{2}\int_{0}^{\infty} \int_{0}^{\infty} \frac{\phi(s,t)m(s,\theta_0)m(t,\theta_0)}{(1-G(s))(1-G(t))} H(ds)H(dt)\\
			&= \frac{1}{2}\int_{0}^{\infty} \int_{0}^{\infty} \phi(s,t)F(ds)F(dt)\\
			&= \lim\limits_{\epsilon\searrow 0} \frac{1}{2}\int_{0}^{\infty} \int_{0}^{\infty} \frac{\phi(s,t)}{(1-G(s))(1-G(t))}\\
			&\qquad\qquad \times \frac{1}{(1-H(s))^\epsilon(1-H(t))^\epsilon} H(ds)H(dt)\mdot
		\end{align*}
		Hereby the proof of Theorem \ref{thm:snmn_limit} is concluded. 
	\end{proof}
\end{thm}
%
\begin{remark}
	Note that according to Theorem \ref{thm:snmn_limit} 
	\begin{equation*}
		S_n(1) = \doublesum\limits_{1\leq i<j \leq n} W_{i:n} W_{j:n} \to  \frac{1}{2}\int_{0}^{\tau_H} \int_{0}^{\tau_H} F(ds)F(dt) = \frac{1}{2} F^{2}(\tau_H)\mdot
	\end{equation*}
	Therefore we have 
	\begin{equation*}
		\lim\limits_{n\to\infty}\frac{S_n(\phi)}{S_n(1)} = F^{-2}(\tau_H) \int_{0}^{\tau_H} \int_{0}^{\tau_H} \phi(s,t)F(ds)F(dt)\mdot
	\end{equation*}
	which is the normalized version of $S_n$.
\end{remark}


