\chapter{Identifying the limit}
\label{ch:identify_limit}
%
In the previous chapter we established the existence of the limit \\
$$\lim\limits_{n\to\infty} S_n = S_\infty\mdot$$
%
We will now continue to identify the limit $S(m(\cdot,\hat{\theta}_n))$ throughout this chapter. The interdependence structure of the proofs within this chapter in figure \ref{fig:structure_identify} below.\\
\begin{figure}[h!]
	\begin{center}
		\fbox{\begin{forest}
			for tree={
				font=\sffamily\bfseries,
				line width=1pt,
				draw=black,
				rounded corners,
				align=center,
				child anchor=north,
				parent anchor=south,
				drop shadow,
				grow = south,
				l sep+=5pt,
				edge path={
					\noexpand\path[color=black, rounded corners=5pt, >={Stealth[length=1pt]}, line width=1pt, \forestoption{edge}]
					(!u.parent anchor) -- +(0,-5pt) -|
					(.child anchor)\forestoption{edge label};
				},
				where level={3}{tier=tier3}{}
			}
			[Theorem \ref{thm:snmn_limit}, inner color=col1in, outer color=col1out
				[Lemma \ref{lem:connector}, inner color=white, outer color=white
					[Lemma \ref{lem:hewitt_savage}, inner color=white, outer color=white]
				]
				[Lemma \ref{lem:sn_limit}, inner color=white, outer color=white
					[Lem. \ref{lem:dn_limit}, inner color=white, outer color=white]
					[Lem. \ref{lem:neveu}, inner color=white, outer color=white]
					[Lem. \ref{lem:Sn_Delta}, inner color=white, outer color=white
						[L \ref{lem:zizone}, inner color=white, outer color=white]
						[L \ref{lem:representation_bn}, inner color=white, outer color=white]	
					]
					[Lem. \ref{lem:Cn_bounds_and_limit}, inner color=white, outer color=white]			
				]
				[Corollary \ref{lem:sandwich}, inner color=white, outer color=white
					[Lem. \ref{lem:Mm}, inner color=white, outer color=white]
				]
			]]
		\end{forest}}
	\end{center}
	\caption{Interdependence Structure of the lemmas and theorems within this chapter.}
	\label{fig:structure_identify}
\end{figure}
%
\\
The following lemma is similar to \cite{neveu1975discrete}, Proposition 5-3-11, and it will be essential to identify the limit of $S_n$.
\begin{lemma}
	The following statement holds true:
	$$S_\infty = \lim\limits_{n\to\infty} \E[S_n| \F_\infty] = \lim\limits_{n\to\infty}\E[S_n]$$
	almost surely, if the limits above exist.
	\label{lem:connector}
	
	\begin{proof}
		Let $a>0$ and note that, since $S_n\to S$ almost surely as $n\to\infty$, we have
		$$\lim\limits_{n\to\infty}\min(S_n, a) = \min(S,a)$$ 
		almost surely, since $\min(\cdot, a)$ is continuous (see \cite{van2000asymptotic}, Theorem 2.3). Now $\min(S_n,a)$ is bounded by $a$. Hence applying the Dominated Convergence Theorem yields
		\begin{align*}
		\lim\limits_{n\to\infty} \E[\min(S_n,a)|\F_\infty] &= \E[\lim\limits_{n\to\infty} \min(S_n,a)|\F_\infty] \nonumber\\
		&= \E[\min(S_\infty,a)|\F_\infty]\mdot
		%&= \min(S_\infty,a) \numberthis	\label{eq:dom_conv_1}\\
		\end{align*}
		%
		Note that $S_k$ is measurable with respect to $\F_n$  whenever $k\geq n$, therefore $S_\infty$ must be $\F_n$-measurable for all $n\in\mathbb{N}$. Consequently $S_\infty$ must be $F_\infty$-measurable. Moreover, for $a\in\R$, $\min(\cdot,a)$ is a continuous function. Thus $\min(S_\infty,a)$ is $\F_\infty$-measurable as well. Hence
		$$\lim\limits_{n\to\infty}\E[min(S_n,a)|\F_\infty] = \min(S_\infty,a) $$
		almost surely. Thus we have
		\begin{align*}
		\lim\limits_{n\to\infty} \E[S_n|\F_\infty] &=  \lim\limits_{n\to\infty}\lim\limits_{a\to\infty}\E[\min(S_n,a)|\F_\infty]\\
		&= \lim\limits_{a\to\infty}\lim\limits_{n\to\infty}\E[\min(S_n,a)|\F_\infty]\\
		&= \lim\limits_{a\to\infty}\min(S_\infty,a)\\
		&= S_\infty\mdot \numberthis \label{eq:neveu}
		\end{align*}
		almost surely. Moreover we obtain
		$$\E[S_n| \F_\infty] = \E[S_n] $$
		for all $n$, by applying Lemma \ref{lem:hewitt_savage}. Now the latter together with \eqref{eq:neveu} implies the statement of the lemma.
	\end{proof}
\end{lemma}
%
We will now proceed to find an explicit representation for $\E[S_n]$ in terms of the reverse supermartingale $D_n$ (see Section \ref{sec:dn}) to identify the limit $S = S(q)$. Consider the following lemma.
\begin{lemma}
	For continuous $H(\cdot)$, we have
	$$\E[S_n(q)] = \frac{n-1}{n} \E[\phi(Z_1,Z_2)q(Z_1)q(Z_2)\{\Delta_{n-2}(Z_1, Z_2) + \bar\Delta_{n-2}(Z_1, Z_2)\} \I{Z_1<Z_2}]\mdot$$
	\label{lem:Sn_Delta}
	
	\begin{proof}
		The proof of the lemma above is similar to the proof of Lemma \ref{lem:expectation_sq}. Consider
		\begin{align*}
			\E[S_n(q)] &= \doublesum\limits_{1\leq i<j\leq n}\E\left[\phi(Z_{i:n}, Z_{j:n}) \frac{q(Z_{i:n})}{n-i+1)}\prod_{k=1}^{i-1}\left[1-\frac{q(Z_{k:n})}{n-k+1}\right]\right.\\
			&\qquad\qquad\qquad \times \left. \frac{q(Z_{j:n})}{n-j+1}\prod_{l=1}^{j-1}\left[1-\frac{q(Z_{l:n})}{n-l+1}\right]\right]\\
			&= \frac{1}{n^2}\doublesum\limits_{1\leq i<j\leq n}\E\left[\phi(Z_{i:n}, Z_{j:n}) q(Z_{i:n})\prod_{k=1}^{i-1}\left[1+\frac{1-q(Z_{k:n})}{n-k+1}\right]\right.\\
			&\qquad\qquad\qquad \times \left. q(Z_{j:n})\prod_{l=1}^{j-1}\left[1+\frac{1-q(Z_{l:n})}{n-l+1}\right]\right]\\
			&= \frac{1}{n^2}\doublesum\limits_{1\leq i<j\leq n}\E\left[\phi(Z_{i:n}, Z_{j:n}) q(Z_{i:n})q(Z_{j:n})B_n(Z_{i:n})B_n(Z_{j:n})\right]\\
			&= \frac{1}{n^2}\sum_{i=1}^{n}\sum_{j=1}^{n}\I{R_{i,n} < R_{j,n}}\E\left[\phi(Z_{i}, Z_{j}) q(Z_{i})q(Z_{j})B_n(Z_{i})B_n(Z_{j})\right]\mdot
			\numberthis\label{eq:E_n}
		\end{align*}
		%
		According to Lemma \ref{lem:zizone} we obtain
		\begin{align*}
			\E[S_n(q)] &= \frac{n-1}{2n}\E\left[\phi(Z_{1}, Z_{2}) q(Z_{1})q(Z_{2})B_n(Z_{1})B_n(Z_{2})\right]\mdot
		\end{align*}	
		Now, since $\phi$ and $q$ are measurable, we can apply Lemma \ref{lem:representation_bn} to obtain the result.		
	\end{proof}
\end{lemma}
%
The following result is necessary for the proof of Lemma \ref{lem:sn_limit}.
\begin{lemma}
	For continuous $H(\cdot)$ and $t<s$, we have $C_n(t) \to 0$ as $n \to \infty$ \wpo, and $C_n(t) \in [0,1]$ for all $n\geq 1$ and $t\geq 0$.
	\label{lem:Cn_bounds_and_limit}
	%
	\begin{proof}
		It is easy to see that $0\leq C_n(t) \leq 1$ for any $t\geq 0$ and $n\geq 2$, since $0\leq q(t)\leq 1$ and $\I{Z_{i-1:n} < t \leq Z_{i:n}} = 1$ for exactly one $i \in \{1,\dots,n\}$. Let's now consider 
		\begin{align*}
		C_n(t) &= \sum_{i=1}^{n+1} \frac{1-q(t)}{n-i+2} [\I{Z_{i-1:n}<t} - \I{Z_{i:n}<t}]\\
		&= \sum_{i=1}^{n+1} \frac{1-q(t)}{n-i+2} \I{Z_{i-1:n}<t} - \sum_{i=1}^{n+1} \frac{1-q(t)}{n-i+2} \I{Z_{i:n}<t}\\
		&= \sum_{i=0}^{n} \frac{1-q(t)}{n-i+1} \I{Z_{i:n}<t} - \sum_{i=1}^{n} \frac{1-q(t)}{n-i+2} \I{Z_{i:n}<t}\\
		&= \sum_{i=1}^{n} \frac{1-q(t)}{n-i+1} \I{Z_{i:n}<t} + \frac{(1-q(t))}{n+1}  - \sum_{i=1}^{n} \frac{1-q(t)}{n-i+2} \I{Z_{i:n}<t}\\
		&= (1-q(t))\left\{\frac{1}{n+1} + \sum_{i=1}^{n} \left[\frac{1}{n-i+1} - \frac{1}{n-i+2}\right]\I{Z_{i:n}<t}\right\}\\
		&= (1-q(t))\sum_{i=1}^{n} \left[\frac{1}{n-nH_n(Z_{i:n})+1} \frac{1}{n-nH_n(Z_{i:n})+2}\right]\I{Z_{i:n}<t}\\
		&\qquad + \frac{1-q(t)}{n+1}\\
		&= (1-q(t))\int_{0}^{t} \left[\frac{1}{1-H_n(x)+\frac{1}{n}} - \frac{1}{1-H_n(x)+\frac{2}{n}}\right]H_n(dx)\\
		&\qquad + \frac{1-q(t)}{n+1}\mdot \numberthis \label{eq:cn}
		\end{align*}
		In Lemma \ref{lem:dn_limit} we have seen that
		$$\int_{0}^{t} \frac{1}{1-H_n(x)+\frac{2}{n}} H_n(dx) \to \int_{0}^{t} \frac{1}{1-H(x)} H(dx)\mdot$$
		By the same arguments we obtain 
		$$\int_{0}^{t} \frac{1}{1-H_n(x)+\frac{1}{n}}H_n(dx) \to \int_{0}^{t} \frac{1}{1-H(x)} H(dx)\mdot$$
		Therefore the right hand side of \eqref{eq:cn} converges to zero. 
	\end{proof}
\end{lemma}
%
We will now identify the almost sure limits of $S_n(q)$ and $\bar{S}_n(q)$ in Lemma \ref{lem:sn_limit}. Define for $n\geq 2$
$$\bar{S}_n(q) := \doublesum\limits_{1\leq i < j \leq n} \phi(Z_{i:n},Z_{j:n}) \bar{W}_{i:n}(q) \bar{W}_{j:n}(q)$$
where 
$$\bar{W}_{i:n}(q) := \prod_{k=1}^{n}\left(1-\frac{q(Z_{k:n})}{n-k+1}\right)\mdot$$
Moreover let
\begin{align*}
	S(q) &:= \frac{1}{2}\int_{0}^{\infty} \int_{0}^{\infty} \phi(s,t) q(s)q(t) \exp\left(\int_{0}^{s} \frac{1-q(x)}{1-H(x)} H(dx)\right)\\
	&\qquad\qquad\qquad \times \exp\left(\int_{0}^{t} \frac{1-q(x)}{1-H(x)} H(dx) \right)H(ds)H(dt)
\end{align*}
and 
\begin{align*}
	\bar{S}(q) &:= \frac{1}{2}\int_{0}^{\infty} \int_{0}^{\infty} \phi(s,t)  \exp\left(\int_{0}^{s} \frac{1-q(x)}{1-H(x)} H(dx)\right)\\
	&\qquad\qquad\qquad \times \exp\left(\int_{0}^{t} \frac{1-q(x)}{1-H(x)} H(dx) \right)H(ds)H(dt)\mdot
\end{align*}

\begin{lemma}
	For continuous $H$ the following statements hold
	$$\lim\limits_{n\to\infty} S_n(q) = S(q)$$
	and 
	$$\lim\limits_{n\to\infty} \bar{S}_n(q) = \bar{S}(q)$$
	with probability one, if the limits above exist.
	\label{lem:sn_limit}
	%
	\begin{proof}
		Suppose $H$ is continuous. First consider that we have
		\begin{align*}
			\E[S_n(q)] &= \frac{n-1}{n} \E[\phi(Z_1,Z_2)q(Z_1)q(Z_2)\{\Delta_{n-2}(Z_1, Z_2) + \bar\Delta_{n-2}(Z_1, Z_2)\} \I{Z_1<Z_2}]	\\
			&= \frac{n-1}{n} \E[\phi(Z_1,Z_2)q(Z_1)q(Z_2)\Delta_{n-2}(Z_1, Z_2)\I{Z_1<Z_2}]\\
			&\qquad + \frac{n-1}{n} \E[\phi(Z_1,Z_2)q(Z_1)q(Z_2) \bar\Delta_{n-2}(Z_1, Z_2) \I{Z_1<Z_2}]
		\end{align*}
		by Lemma \ref{lem:Sn_Delta}. 
		%
		Next note that we get from Lemma \ref{lem:dn_limit} that $D_n(s,t) \to D(s,t)$ \wpo, and according to Lemma \ref{lem:Cn_bounds_and_limit} $C_n(s) \to 0$ \wpo. Thus $\bar{\Delta}_n(s,t) \to 0$ as $n\to\infty$ for each $s<t$.
		%
		Moreover, the fact that $0\leq C_n(s)\leq 1$ and Lemma \ref{lem:neveu} imply that $\bar{\Delta}_n(s,t) \leq \Delta_n(s,t) \leq D(s,t)$ for all $n\geq 2$. Now note that $D(s,t)$ is integrable, since on $\{Z_1<Z_2\}$ we have
		\begin{align*}
			\E[D(Z_1,Z_2)] &= \E\left[\int_{0}^{s} \frac{1-q(x)}{1-H(x)} H(dx) + \int_{0}^{t} \frac{1-q(x)}{1-H(x)} H(dx)\right]\\
			&\leq \E\left[\int_{0}^{Z_{n:n}} \frac{1}{1-H(x)} H(dx) + \int_{0}^{Z_{n:n}} \frac{1}{1-H(x)} H(dx)\right]\\
			&\leq \E\left[-2\ln(1-H(Z_{n:n}))\right]\\
			& < \infty\mdot
		\end{align*}
		Therefore applying the Dominated Convergence Theorem yields
		$$\lim\limits_{n\to\infty}\E[2\phi(Z_1,Z_2)q(Z_1)q(Z_2) \bar\Delta_{n-2}(Z_1, Z_2) \I{Z_1<Z_2}] = 0\mdot$$
		%
		Furthermore, according to Lemma \ref{lem:neveu}, the following holds
		$$\Delta_{n}(s,t) = \E[D_n(s,t)] = \E[D_n(s,t)|\F_\infty] \nearrow D(s,t)\mdot$$
		Hence we have 
		\begin{align*}
			&\lim\limits_{n\to\infty}\E[\phi(Z_1,Z_2)q(Z_1)q(Z_2) \Delta_{n-2}(Z_1, Z_2) \I{Z_1<Z_2}] \\
			&= \E[\phi(Z_1,Z_2)q(Z_1)q(Z_2) D(Z_1, Z_2) \I{Z_1<Z_2}]\mdot
		\end{align*}
		%
		Therefore we obtain 
		\begin{align*}
		\lim\limits_{n\to\infty}\E[S_n(q)]&= \E[\phi(Z_1,Z_2)q(Z_1)q(Z_2) D(Z_1, Z_2) \I{Z_1<Z_2}]\\
		&= {\int\int}_{\{Z_1<Z_2\}} \phi(s,t)q(s)\exp\left(\int_{0}^{s} \frac{1-q(z)}{1-H(z)} H(dz)\right) \\
		&\qquad\qquad \times q(t) \exp\left(\int_{0}^{t} \frac{1-q(z)}{1-H(z)} H(dz)\right) H(ds)H(dt)\\
		&= \frac{1}{2}\int_{0}^{\infty}\int_{0}^{\infty}\phi(s,t)q(s) \exp\left(\int_{0}^{s} \frac{1-q(z)}{1-H(z)} H(dz)\right)\\
		&\qquad\qquad \times q(t)\exp\left(\int_{0}^{t} \frac{1-q(z)}{1-H(z)} H(dz)\right) H(ds)H(dt)
		\end{align*}
		almost surely, since $\phi(s,t)q(s)q(t)D(s,t)$ is symmetric in $s$ and $t$ by (A\ref{ass:kernel_gen}), and $Z_1$ and $Z_2$ are \iid. By similar arguments, we obtain $\bar{S}_n \to \bar{S}$ \wpo. 
	\end{proof}
\end{lemma}
%
In order to identify the limit of $\snse{2}{n} = S_n(m(\cdot, \hat\theta_n))$ we need the statement of Corollary \ref{lem:sandwich}, which is based upon the following lemma. Define for any $\epsilon >0$ let
$$M_{1,\epsilon}(x) := \max(0, m(x, \theta_0) - \epsilon)) \textrm{ and } M_{2,\epsilon}(x) := \min(1, m(x, \theta_0) + \epsilon))$$
\begin{lemma}
	Suppose (M\ref{ass:m_consistency}) and (M\ref{ass:m_nbhd}) hold. Then the following statements hold for each $0 < \epsilon \leq 1$ and $n$ large enough
	\begin{enumerate}[(i)]
		\item $M_{1,\epsilon}(x) \leq m(x,\hat\theta_n) \leq M_{2,\epsilon}(x)$
		\item $M_{2,\epsilon}(x)M_{2,\epsilon}(y) - 4\epsilon \leq m(x,\hat\theta_n)m(y,\hat\theta_n) \leq M_{1,\epsilon}(x)M_{1,\epsilon}(y) + 4\epsilon$.
	\end{enumerate}
	\label{lem:Mm}
	%
	\begin{proof}
		First we will introduce some notation. We will write $m_n(x) := m(x,\theta_n)$ and $m(x):= m(x,\theta_0)$. Let's start with part \textit{(i)}. Suppose $M_{1,\epsilon}(x) = 0$, then the condition above is trivially satisfied since $m_n(x) \geq 0$. Now suppose $M_{1,\epsilon}(x) = m(x)-\epsilon$. 
		\begin{align*}
			m_n(x) &= (m_n(x) - m(x)) + m(x)\\
			&\geq m(x) - \abs{m_n(x) - m(x)} \mdot
		\end{align*}
		%
		Now using assumption (M\ref{ass:m_consistency}), we have for $n$ large enough that for some $\epsilon > 0$ $\theta_n \in V(\epsilon, \theta_0)$. Now we get, according to (M\ref{ass:m_nbhd}), that 
		$$\sup_{x\geq 0}\abs{m_n(x) - m(x)} < \epsilon$$
		Therefore we obtain $m_n(x) \geq m(x) - \epsilon= M_{1,\epsilon}(x)$. Let's now consider $M_{2,\epsilon}$. The case $M_{2,\epsilon} = 1$ is trivial again, since $m_n(x) \leq 1$. Now suppose $M_{2,\epsilon} = m(x) + \epsilon$. Then we obtain, for $n$ large enough
		\begin{align*}
			m_n(x) &= (m_n(x) - m(x)) + m(x)\\
			&\leq m(x) + \abs{m_n(x) - m(x)}\\
			&\leq m(x) + \epsilon\\
			&= M_{2,\epsilon}(x)\mdot
		\end{align*}
		%
		This concludes the proof of part \textit{(i)}. Now note that, according to (M\ref{ass:m_consistency})  and (M\ref{ass:m_nbhd}), the following holds for $n$ large enough and $\epsilon>0$
		\begin{align*}
			m_n(x) &= (m_n(x) - m(x)) + m(x)\\
			&\leq \abs{m_n(x) - m (x)} + m(x) \\
			&\leq m(x) + \epsilon\mdot \numberthis\label{eq:mn_m_eps}
		\end{align*}		
		%
		Moreover consider that 
		\begin{align*}
			m_n(x)m_n(y) &= (m_n(x) - m(x))(m_n(y) - m(y))\\
			&\qquad + m(x)m_n(y) + m_n(x)m(y) - m(x)m(y)\\
			&\leq \epsilon^2 + m(x)m_n(y) + m_n(x)m(y) - m(x)m(y)\mdot
		\end{align*}
		%
		Using on the right hand side of the latter inequality \eqref{eq:mn_m_eps} yields
		\begin{align*}
			m_n(x)m_n(y) &\leq \epsilon^2 + m(x)(m(y) + \epsilon) + (m(x) + \epsilon)m(y) - m(x)m(y)\\
			&= m(x)m(y) + \epsilon(m(x) + m(y)) + \epsilon^2\mdot \numberthis \label{eq:mnmn_mm_eps}
		\end{align*}
		%
		Now suppose $M_{1,\epsilon}(x) = 0$ and $M_{1,\epsilon}(y) = 0$ for $x,y \in \R_+$. Then $m(x) \leq \epsilon$ and $m(y) \leq \epsilon$. Hence, using \eqref{eq:mnmn_mm_eps} yields
		\begin{align*}
			m_n(x)m_n(y) &\leq 4\epsilon^2\mdot
		\end{align*}
		%
		Next suppose $M_{1,\epsilon}(x) = 0$ and $M_{1,\epsilon}(y) = m(y) -\epsilon$. Using \eqref{eq:mnmn_mm_eps} again, we obtain
		\begin{align*}
			m_n(x)m_n(y) &\leq m(x)m(y) + \epsilon(m(x) + m(y)) + \epsilon^2\\
			&\leq \epsilon + \epsilon(1+\epsilon) + \epsilon^2\\
			&= 2\epsilon(1+\epsilon)\mcomma
		\end{align*}
		since $m(x)\leq \epsilon$ and $m(y) \leq 1$. 
		%
		By similar calculations, we obtain the exact same result for the case $M_{1,\epsilon}(x) = m(x) -\epsilon$ and $M_{1,\epsilon}(y) = 0$. Now suppose $M_{1,\epsilon}(x) = m(x) -\epsilon$ and $M_{1,\epsilon}(y) = m(y) -\epsilon$ and note that 
		\begin{align*}
			M_{1,\epsilon}(x)M_{1,\epsilon}(y) &= (m(x)-\epsilon)(m(y)-\epsilon)\\
			&= m(x)m(y) - \epsilon(m(x) + m(y)) + \epsilon^2\mdot
		\end{align*}
		%
		Now \eqref{eq:mnmn_mm_eps} implies 
		\begin{align*}
			m_n(x)m_n(y) &\leq m(x)m(y) + \epsilon(m(x) + m(y)) + \epsilon^2\\
			&= M_{1,\epsilon}(x)M_{1,\epsilon}(y) + 2\epsilon(m(x) + m(y))\\
			&\leq M_{1,\epsilon}(x)M_{1,\epsilon}(y) + 4\epsilon\mdot
		\end{align*}
		%
		Thus we have for $0\leq \epsilon\leq 1$ that 
		$$m_n(x)m_n(y) \leq M_{1,\epsilon}(x)M_{1,\epsilon}(y) + 4\epsilon$$
		as claimed in the statement of this lemma. It remains to show that $M_{2,\epsilon}(x)M_{2,\epsilon}(y) - 4\epsilon \leq m_n(x)m_n(y)$. By calculations similar to those, that lead to \eqref{eq:mn_m_eps} and \eqref{eq:mnmn_mm_eps} we obtain
		$$m_n(x) \geq m(x) - \epsilon$$
		and
		\begin{equation}
			m_n(x)m_n(y) \geq m(x)m(y) - \epsilon(m(x) + m(y)) - \epsilon^2 \mdot \label{eq:mnmn_mm_eps_a}
		\end{equation}
		%
		Now we will continue and look at $M_{2,\epsilon}$ case by case. Suppose $M_{2,\epsilon}(x) = 1$ and $M_{2,\epsilon}(y) = 1$. This is equivalent to $m(x) \geq 1 - \epsilon$ and $m(y) \geq 1 - \epsilon$. Therefore \eqref{eq:mnmn_mm_eps_a} implies 
		\begin{align*}
			m_n(x)m_n(y) &\geq (1-\epsilon)^2 - 2\epsilon - \epsilon^2\\
			&= 1 - 4\epsilon\\
			&= M_{2,\epsilon}(x)M_{2,\epsilon}(y) - 4\epsilon\mdot
		\end{align*}
		%
		Next consider the case $M_{2,\epsilon}(x) = 1$ and $M_{2,\epsilon}(y) = m(y) + \epsilon$. Then we have $m(x) \geq 1-\epsilon$ and  $m(y) \leq 1-\epsilon$. Moreover we have $M_{2,\epsilon}(x)M_{2,\epsilon}(y) = m(y) + \epsilon$. Hence we obtain the following, according to \eqref{eq:mnmn_mm_eps_a}
		\begin{align*}
			m_n(x)m_n(y) &\geq (1-\epsilon)m(y) - \epsilon((1 + (1-\epsilon)) - \epsilon^2\\
			&= m(y)-\epsilon m(y) - 2\epsilon\\
			&\geq m(y)-\epsilon (1-\epsilon) - 2\epsilon\\
			&\geq m(y)-3\epsilon \\
			&= M_{2,\epsilon}(x)M_{2,\epsilon}(y) - 4\epsilon\mdot
		\end{align*} 
		%
		By similar calculations we obtain the same result, if $M_{2,\epsilon}(x) = m(x) + \epsilon$ and $M_{2,\epsilon}(y) = 1$. Finally consider the case $M_{2,\epsilon}(x) = m(x) + \epsilon$ and $M_{2,\epsilon}(y) = m(y) + \epsilon$. Then we have $m(x) \geq 1-\epsilon$ and $m(y) \geq 1-\epsilon$. Furthermore we have 
		\begin{align*}
			M_{2,\epsilon}(x)M_{2,\epsilon}(y) &= (m(x) +\epsilon)(m(y) + \epsilon)\\
			&= m(x)m(y) + \epsilon(m(x) + m(y)) + \epsilon^2\mdot
		\end{align*}
		%
		Therefore, using \eqref{eq:mnmn_mm_eps_a} again, yields
		\begin{align*}
			m_n(x)m_n(y) &\geq m(x)m(y) - \epsilon(m(x) + m(y)) - \epsilon^2\\
			&= M_{2,\epsilon}(x) M_{2,\epsilon}(y) - 2\epsilon(m(x) + m(y)) - 2\epsilon^2\\
			&= M_{2,\epsilon}(x) M_{2,\epsilon}(y) - 4\epsilon(1-\epsilon) - 2\epsilon^2\\
			&\geq M_{2,\epsilon}(x) M_{2,\epsilon}(y) - 4\epsilon \mdot
		\end{align*}
		%
		This concludes the proof.
	\end{proof}
\end{lemma}

\begin{cor}
	Suppose (M\ref{ass:m_consistency}) and (M\ref{ass:m_nbhd}) hold and $H$ is continuous. Then we have for each $0 < \epsilon \leq 1$ and $n$ large enough
	$$S_n(M_{2,\epsilon}) - 4\epsilon \bar{S}_n(M_{2,\epsilon}) \leq S_n(m(\cdot, \hat\theta_n)) \leq S_n(M_{1,\epsilon}) + 4\epsilon \bar{S}_n(M_{1,\epsilon})\textrm{.}$$
	\label{lem:sandwich}
	%
	\begin{proof}
		Consider that we have the following for any $n\geq 1$
		\begin{align*}
			S_n(M_{2,\epsilon}) - 4\epsilon \bar{S}_n(M_{2,\epsilon}) &= \doublesum\limits_{1\leq i<j\leq n} \phi(Z_{i:n}, Z_{j:n}) (M_{2,\epsilon}(Z_{i:n}) M_{2,\epsilon}(Z_{j:n}) - 4\epsilon) \\
			&\qquad \times \prod_{k=1}^{i-1}\left[1-\frac{M_{2,\epsilon}(Z_{k:n})}{n-k+1}\right] \prod_{k=1}^{j-1}\left[1-\frac{M_{2,\epsilon}(Z_{k:n})}{n-k+1}\right]\mdot
		\end{align*}
		But according to Lemma \ref{lem:Mm} we have 
		$$m(x,\hat\theta_n) \leq M_{2,\epsilon}(x) \textrm{ and } M_{2,\epsilon}(x)M_{2,\epsilon}(y) \leq m(x,\hat\theta_n)m(y,\hat\theta_n)$$
		for all $x,y \in \R_+$. Hence we obtain
		\begin{align*}
		S_n(M_{2,\epsilon}) - 4\epsilon \bar{S}_n(M_{2,\epsilon}) &\leq \doublesum\limits_{1\leq i<j\leq n} \phi(Z_{i:n}, Z_{j:n}) m(Z_{i:n},\hat\theta_n)m(Z_{j:n},\hat\theta_n) \\
		&\qquad \times \prod_{k=1}^{i-1}\left[1-\frac{m(Z_{k:n},\hat\theta_n)}{n-k+1}\right] \prod_{k=1}^{j-1}\left[1-\frac{m(Z_{k:n},\hat\theta_n)}{n-k+1}\right]\\
		&= S_n(m(\cdot, \hat\theta_n))\textrm{.}
		\end{align*}
		%
		Similarly we obtain 
		$$S_n(M_{1,\epsilon}) + 4\epsilon \bar{S}_n(M_{1,\epsilon}) \geq S_n(m(\cdot, \hat\theta_n)) \textrm{.}$$
	\end{proof}
\end{cor}

Now we are in a position, to identify $S = \lim_{n\to\infty} \snse{2}{n}$. The following theorem gives the main statement of this thesis.
\begin{thm}
	Suppose (A\ref{ass:kernel_gen}) through (A\ref{ass:m_H_one}), (M\ref{ass:m_consistency}) and (M\ref{ass:m_nbhd}) hold. Then we have
	$$\lim\limits_{n\to\infty} S_n(m(\cdot, \hat{\theta}_n)) = \frac{1}{2}\int_{0}^{\tau_H}\int_{0}^{\tau_H} \phi(s,t) F(ds)F(dt)\mdot$$
	\label{thm:snmn_limit}
	\begin{proof}
		Consider that we have
		$$S_n(M_{2,\epsilon}) - 4\epsilon \bar{S}_n(M_{2,\epsilon}) \leq S_n(m(\cdot, \hat\theta_n)) \leq S_n(M_{1,\epsilon}) + 4\epsilon \bar{S}_n(M_{1,\epsilon})$$
		by Corollary \ref{lem:sandwich} under (M\ref{ass:m_consistency}) and (M\ref{ass:m_nbhd}). Next take note of the Radon-Nikodym derivatives (\cf \cite{dikta2000strong}, page 8)
		$$m(s,\theta_0) = \frac{H^1(ds)}{H(ds)} \textrm{ and } (1-G(s)) = \frac{H^1(ds)}{F(ds)}\mdot$$
		Moreover consider that we have
		$$\int_{0}^{s} \frac{1-m(x,\theta_0)}{1-H(x)} = -\ln(1-G(s))$$
		and 
		$$\int_{0}^{s} \frac{\epsilon}{1-H(x)} = -\ln((1-H(s))^\epsilon)$$
		according to \cite{dikta2000strong}.
		%
		Consider that we have 
		\begin{align*}
			M_{1,\epsilon}(x) &= \I{ m(x,\theta_0)>\epsilon}(m(x,\theta_0)-\epsilon)\\
			&\leq m(x,\theta_0)-\epsilon \mdot
		\end{align*}
		Therefore, we obtain
		\begin{align*}
			\bar{S}(M_{1,\epsilon}) &\leq \frac{1}{2}\int_{0}^{\infty} \int_{0}^{\infty} \phi(s,t)  \exp\left(\int_{0}^{s} \frac{1-m(x,\theta_0)}{1-H(x)} + \frac{\epsilon}{1-H(x)} H(dx)\right)\\
			&\qquad\qquad\qquad \times \exp\left(\int_{0}^{t} \frac{1-m(x,\theta_0)}{1-H(x)} + \frac{\epsilon}{1-H(x)} H(dx) \right)H(ds)H(dt)\\
			&= \frac{1}{2}\int_{0}^{\infty} \int_{0}^{\infty} \frac{\phi(s,t)}{(1-G(s))(1-G(t))(1-H(s))^\epsilon(1-H(t))^\epsilon} H(ds)H(dt)\\
			&= \frac{1}{2}\int_{0}^{\infty} \int_{0}^{\infty} \frac{\phi(s,t)}{m(s,\theta_0)m(t,\theta_0)(1-H(s))^\epsilon(1-H(t))^\epsilon} F(ds)F(dt)\mdot
		\end{align*}
		%
		But by condition (A\ref{ass:intgral_phi_q}), the integral above is finite. Moreover we know that $S(M_{1,\epsilon})$ exists almost surely under (A\ref{ass:kernel_gen}) through (A\ref{ass:m_H_one}), by Theorem \ref{thm:existence_limit}. Therefore we get by Lemma \ref{lem:sn_limit} that for each $0<\epsilon\leq1$ we have $S_n(M_{1,\epsilon}) + 4\epsilon \bar{S}_n(M_{1,\epsilon}) \to S(M_{1,\epsilon}) + 4\epsilon \bar{S}(M_{1,\epsilon})$ \wpo\ as $n\to\infty$. 
		%
		Next consider that 
		\begin{align*}
			S(M_{1,\epsilon}) + 4\epsilon \bar{S}(M_{1,\epsilon}) &\leq \frac{1}{2}\int_{0}^{\infty} \int_{0}^{\infty} \frac{\phi(s,t)}{(1-H(s))^\epsilon(1-H(t))^\epsilon}\\
			&\qquad\qquad \times \frac{m(s,\theta_0)m(t,\theta_0) + 4\epsilon}{(1-G(s))(1-G(t))} H(ds)H(dt)\mdot
		\end{align*}
		%
		By similar arguments we can show that $S_n(M_{2,\epsilon}) - 4\epsilon \bar{S}_n(M_{2,\epsilon}) \to S(M_{2,\epsilon}) - 4\epsilon \bar{S}(M_{2,\epsilon})$ \wpo\ as $n\to\infty$ and 
		\begin{align*}
			S(M_{2,\epsilon}) - 4\epsilon \bar{S}(M_{2,\epsilon}) &\geq \frac{1}{2}\int_{0}^{\infty} \int_{0}^{\infty} \phi(s,t)(1-H(s))^\epsilon(1-H(t))^\epsilon\\
			&\qquad\qquad \times \frac{m(s,\theta_0)m(t,\theta_0) - 4\epsilon}{(1-G(s))(1-G(t))} H(ds)H(dt)\mdot
		\end{align*}
		%
		Let's summarize the above. So far we have shown, that for $0<\epsilon\leq 1$ small enough
		\begin{align*}
			&\frac{1}{2}\int_{0}^{\infty} \int_{0}^{\infty} \phi(s,t)(1-H(s))^\epsilon(1-H(t))^\epsilon\\
			&\qquad\qquad \times \frac{m(s,\theta_0)m(t,\theta_0) - 4\epsilon}{(1-G(s))(1-G(t))} H(ds)H(dt)\\
			&\leq \liminf_{n\to\infty} \int_{0}^{\infty} \phi F_n^{se} F_n^{se}\\
			&\leq \limsup_{n\to\infty} \int_{0}^{\infty} \phi F_n^{se} F_n^{se}\\
			&\leq \frac{1}{2}\int_{0}^{\infty} \int_{0}^{\infty} \frac{\phi(s,t)}{(1-H(s))^\epsilon(1-H(t))^\epsilon}\\
			&\qquad\qquad \times \frac{m(s,\theta_0)m(t,\theta_0) + 4\epsilon}{(1-G(s))(1-G(t))} H(ds)H(dt)\mdot
		\end{align*}
		Finally let $\epsilon \searrow 0$ and apply the Monotone Convergence Theorem to obtain that the upper and lower bound converge both to the same limit. In effect
		\begin{align*}
			&\lim\limits_{\epsilon\searrow 0}\frac{1}{2}\int_{0}^{\infty} \int_{0}^{\infty} \frac{\phi(s,t)(1-H(s))^\epsilon(1-H(t))^\epsilon}{(1-G(s))(1-G(t))} H(ds)H(dt)\\
			&= \frac{1}{2}\int_{0}^{\infty} \int_{0}^{\infty} \frac{\phi(s,t)m(s,\theta_0)m(t,\theta_0)}{(1-G(s))(1-G(t))} H(ds)H(dt)\\
			&= \frac{1}{2}\int_{0}^{\infty} \int_{0}^{\infty} \phi(s,t)F(ds)F(dt)\\
			&= \lim\limits_{\epsilon\searrow 0} \frac{1}{2}\int_{0}^{\infty} \int_{0}^{\infty} \frac{\phi(s,t)}{(1-G(s))(1-G(t))}\\
			&\qquad\qquad \times \frac{1}{(1-H(s))^\epsilon(1-H(t))^\epsilon} H(ds)H(dt)\mdot
		\end{align*}
		Hereby the proof of Theorem \ref{thm:snmn_limit} is concluded. 
	\end{proof}
\end{thm}
%
\begin{remark}
	Note that according to Theorem \ref{thm:snmn_limit} 
	\begin{equation*}
		S_n(1) = \doublesum\limits_{1\leq i<j \leq n} W_{i:n} W_{j:n} \to  \frac{1}{2}\int_{0}^{\tau_H} \int_{0}^{\tau_H} F(ds)F(dt) = \frac{1}{2} F^{2}(\tau_H)\mdot
	\end{equation*}
	Therefore we have 
	\begin{equation*}
		\lim\limits_{n\to\infty}\frac{S_n(\phi)}{S_n(1)} = F^{-2}(\tau_H) \int_{0}^{\tau_H} \int_{0}^{\tau_H} \phi(s,t)F(ds)F(dt)\mdot
	\end{equation*}
	which is the normalized version of $S_n$.
\end{remark}


